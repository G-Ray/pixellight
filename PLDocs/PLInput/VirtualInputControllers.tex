\chapter{Virtual Input Controllers}




\section{Overview}
Using physical input devices directly, as seen within chapter~\ref{Chapter:PhysicalInputDevices}, is quite comfortable and also an obvious way to deal with input devices - maybe that's the reason why most input systems out there just support this approach. While someone may be happy with this \emph{direct} solution for some time, there will probably come the point were the need arises to support multiple input devices to, for example, moving around a camera by the keyboard, the mouse, the joystick, the space mouse or by using all devices at once in a combination. A trivial solution for multi-input-devices-support would be, to just build in the support right at the place were it's required like
\begin{quote}if (WiiRemote) move with Wii Remote, if (Mouse) move with mouse, if (Joystick) move with joystick, else () bad luck my friend\end{quote}
Every time, support for another input device is required, the code is extended to cover this new input device as well. On the second thought, it may come into ones mind that this is not the perfect solution for the multi-input-devices-support - especially when the demand arises that the user should be able to configure the input controls dynamically... and by this we don't mean to just assign another keyboard key to an action, but to be able to assign a control of a totally different input device to this action, as well!

By using \emph{virtual input controllers} instead of input devices directly, the multi-input-devices-support and dynamic-user-configuration demands can be fulfilled at one and the same time.
