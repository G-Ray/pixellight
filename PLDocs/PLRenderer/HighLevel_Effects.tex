%----- Chapter: High level: Effects --------------
\chapter{High level: Effects}


[TODO] Update!


%----- Section: Overview -------------------------
\section{Overview}
In this section we will explain you how to use the PixelLight material in practice. A material 
is a simple text format which describes the settings of a material - skipped things are set to
default automatically. A text editor is enough to develop materials but theres also a materal
editor provided.



%----- Section: Introduction ---------------------
\section{Introduction}
Advanced material effects which will increase the visual quality of the surfaces is a must have
for modern graphics engines. Therefore PixelLight provides an extensive material system which enables you
to use the capabilities of modern GPU's to create amazing visual effects.\\

This PixelLight material guide will teach you what in detail such a material is, how to use and
build it - further some background information about e.g. what a 3D texture is are provided, but for
detailed information you have to use other documents you can for instance find in the internet. This document
is for both programmers and graphic artists because a material is a complex
and powerful surface description and it's also possible to programm byself advanced shaders to
control how each vertex and fragment is processed by GPU. In this section we will tell you some
backgrounds about the materials to give you an overview over this topic. In section 2 we will talk
about textures because this is the base of the materials. You will learn in section 3 how to use
materials in practice. Section 4 consists of a material format description. In the section
5 we will talk about shaders. Section 6 is a faq.\\

A PixelLight material is a powerfull surface description. It will manage the different render techniques
automatically (also called 'fallbacks') for you so that in fact you don't need to know which technique is
currently used to visualize a surface. Just apply a material to a surface and have fun! ;-)\\
Here's a diagram which shows how the PixelLight material is build up in general:\\

\begin{lstlisting}[caption=PixelLight material system overview]
                  material
                     |
          n techniques per material
                     |
            n passes per technique
                     |
           ----------------------
           |                    |
  n layers per pass      one vertex/fragment shader per pass
\end{lstlisting}

Techniques means that there are different ways how the material can be rendered - in this relation we 
also call this an 'effect'. If you are already familar with e.g. the Direct3D effect file format (fx) it will be
easy for you to understand the PixelLight material. Using advanced effects is an important thing because there
are fallbacks which enables you to use a simpler technique if the hardware or performance is insufficient or for 
material LOD. Such a technique can consist of different render passes which means that the material is rendered 
several times with different settings increasing the visual quality. In such a pass, render states, sampler states 
etc. are set to render the surface in the requested way.
Many render passes per material will reduce the performance because the geometrie the material
is applied to have to be drawn multiple times - so use it wise. Each of such a pass can have that
many texture layers the hardware is supporting. Modern graphic cards normally support 4 direct texture
layers and up to 16 and more layers when using shaders accessing this layers. Using such layers you can
blend together different textures without be forced to redraw the geometry which will blow up the
polycount and reduce problems like z-fighting. Further each layer can have another texture with other properties... 
you can also animate each textur using different changing subtextures to create a cartoon like animation or rotate, 
move etc the texture by changing it's texture matrix or using all this texture animation effects at the SAME TIME
to create amazing materials full of life! :)\\

An advanced material rendering control is supplied through the vertex and fragment shader you can apply
to each pass. A shader is a program running on the GPU. It will 'overwrite' the API (e.g. OpenGL)
fixpass functions which normally process vertices (transformations etc.) and fragments. (get the color
of each fragment etc.) Therefore when using a vertex shader you have to transform the vertices by self
or when using a fragment shader calculating the single fragment colors by hand. In this case the
different pass layers will only feed the active shaders with e.g. the desired textures but not combining
different texture layers automatically. Further you have to consider things like texture matrix animations
in your shaders when you want to use it - in contrast to the fixpass some things don't done automatically!
Note that per pass there's only ONE vertex and ONE fragment active shader! But for instance you can
combine different passes to a final result. It's a lot of work to write such materials with individual
shader effects but therefore you have FULL CONTROL over how it is rendered and you can create impressive
and unique material effects! If it is too much work for you to do all by self you can use the 'standard'
fixpass functions which which will also product good results in less development time - but you haven't
the full freedom shaders offering you.\\

If you don't specify the material format by the filename extension, the framework will search for the material
by adding a filename extension. If a material (*.mat) with the filename isn't found the framework will seach for a 
texture with this filename by adding a texture filename extension like 'bmp'. Therefore the simples
material would be a single texture without any additional settings. Have a look on the texture section
for more details.



%----- Section: Solid and transparent materials --
\section{Solid and transparent materials}
There are two kinds of materials. Solid and blended materials. In order to ensure that blended
materials are rendered in the right way they have to rendered after all other solid materials.
Therfore you have to denote when the material should be rendered. This is done using the GetBlend()
and SetBlend() functions of PLTMaterial. This only a general setting whether the material is blend
in in any pass and doesn't mean that the material is in fact blend - it only denotes when to
render it!




%----- Section: Using materials ------------------
\section{Using materials}
When using PLTMaterial normally you only will use a few functions. Load() to load a material,
Bind() to activate it, SetupPass() so setup the given render pass and Unbind() if rendering using the
material is finished.\\

\begin{lstlisting}[caption=Using materials]
// Somewhere in e.g. your class
PLTMaterial m_cMaterial;

// Load up your material (default extension is 'mat')
m_cMaterial.Load("MyMaterial");

// Rendering
m_cMaterial.Bind(); // Activate material
for (int nPass=0; nPass<m_cMaterial.GetNumOfPasses(); nPass++) {
    m_cMaterial.SetupPass(nPass); // Setup the current render pass
    // Render something
}
m_cMaterial.Unbind(); // Deactivate material
\end{lstlisting}

You don't have to care about how exactly the material is defined internally like if it is using
normal fixpass functions or even advanced shader effects. But you are able to check and change ALL
the settings during runtime!
When loading a material using the Load() function there are different types of possible useable
material data.\\
- A simple texture like a jpg texture, all other material settings will be set to default.\\
- An PixelLight texture animation file. (tani) This is handled in the same way as a simple texture
  but with additonal features like texture rotation.\\
- A PixelLight material which is the default file type with the filename extension 'mat' which is
  added automatically. Within this material file format you are able to create a material with your
  desired properties.\\

As you see the PixelLight material is a extreme flexible and easy useable surface description which will do
all the dirty work for you. The material has an extensive interface which give you more control
over the material during runtime if required.




%----- Section: Material format ------------------
\section{Material format}




%----- Subsection: Overview ----------------------
\subsection{Overview}
A material file is a normal text file were you define the material itself. Unrequired commands can
be skipped - standart values will be used instead. The material format is split up into 3 main sections:\\
- General: General material settings\\
- Parameters: Shader parameter settings\\
- Techniques: There can be different technique blocks. Each of this blocks consitst of different pass blocks
  were you define how the material is rendered in the pass.\\



%----- Subsection: General settings --------------
Boolean settings:\\
For boolean settings you can use 0/false and 1/true.\\

-1 for the default setting means that this setting isn't set by the material and
normally shouldn't set by at material at all because this are basis renderer settings.

Comparison functions:\\
\begin{itemize}
\item{Never        = Never passes}
\item{Less         = Passes if the incoming value is less than the stored value}
\item{Equal        = Passes if the incoming value is equal to the stored value}
\item{LessEqual    = Passes if the incoming value is less than or equal to the stored value}
\item{Greater      = Passes if the incoming value is greater than the stored value}
\item{NotEqual     = Passes if the incoming value is not equal to the stored value}
\item{GreaterEqual = Passes if the incoming value is greater than or equal to the stored value}
\item{Always       = Always passes}
\end{itemize}


Stencil operations:\\
\begin{itemize}
\item{Keep     = Keeps the current value}
\item{Zero     = Sets the stencil buffer value to zero}
\item{Replace  = Sets the stencil buffer value to ref, as specified by StencilRef}
\item{Incr     = Increments the current stencil buffer value. Clamps to the maximum representable unsigned value.}
\item{Decr     = Decrements the current stencil buffer value. Clamps to zero.}
\item{IncrWrap = Increments the current stencil buffer value. Wraps the result.}
\item{DecrWrap = Decrements the current stencil buffer value. Wraps the result.}
\item{Invert   = Bitwise inverts the current stencil buffer value}
\end{itemize}


Texture-addressing modes:\\
\begin{itemize}
\item{Clamp  = Texture coordinates outside the range [0.0, 1.0] are set to the texture color
               at 0.0 or 1.0, respectively.}
\item{Border = Texture coordinates outside the range [0.0, 1.0] are set to the border color.}
\item{Wrap   = Tile the texture at every integer junction. For example, for u values between
               0 and 3, the texture is repeated three times; no mirroring is performed.}
\item{Mirror = Similar to 'Wrap', except that the texture is flipped at every integer junction.
               For u values between 0 and 1, for example, the texture is addressed normally;
               between 1 and 2, the texture is flipped (mirrored); between 2 and 3, the texture is normal again, and so on.}
\end{itemize}
  

Texture filtering modes:\\
\begin{itemize}
\item{None        = Mipmapping disabled. The rasterizer should use the magnification filter instead.}
\item{Point       = Point filtering used as a texture magnification or minification filter. The texel
                    with coordinates nearest to the desired pixel value is used. The texture filter to
                    be used between mipmap levels is nearest-point mipmap filtering. The rasterizer uses
                    the color from the texel of the nearest mipmap texture.}
\item{Linear      = Bilinear interpolation filtering used as a texture magnification or minification filter.
                    A weighted average of a 2x2 area of texels surrounding the desired pixel is used. The
                    texture filter to use between mipmap levels is trilinear mipmap interpolation. The
                    rasterizer linearly interpolates pixel color, using the texels of the two nearest mipmap textures.}
\item{Anisotropic = Anisotropic texture filtering used as a texture magnification or minification filter.
                    Compensates for distortion caused by the difference in angle between the texture polygon
                    and the plane of the screen.}
\end{itemize}


Texture envionment modes:\\
\begin{itemize}
\item{Add                 = Add incoming color with the existing one}
\item{Replace             = Replace existing color with the incoming one}
\item{Modulate            = Modulate colors}
\item{PassThru            = Pass through incoming color}
\item{Dot3                = Dot3}
\item{Interpolate         = Interpolate}
\item{InterpolatePrimary  = Interpolate primary}
\item{InterpolateTexAlpha = Interpolate tex alpha}
\end{itemize}



%----- Subsection: Example -----------------------
\subsection{Example}
Below you will find a full material file example were you can also see the default values:\\
\verbatiminput{examples/example.mat}




%----- Subsection: General -----------------------
\subsection{General}
In this block are general material settings like flags, whether a material should be blend
or not etc.\\

\begin{tabular}{|p{2.5cm}|p{2.5cm}|p{9cm}|}
\hline
\textbf{Command} & \textbf{Default} & \textbf{Description}\\
\hline
Flags="" & 0 & String of material flags (e.g. "1|8|32")\\
\hline
Blend=0  & 0 & Should the material be blend or not?\\
\hline
\end{tabular}




%----- Subsection: Parameters --------------------
\subsection{Parameters}
Shader parameter settings




%----- Subsection: Technique ---------------------
\subsection{Technique}



%----- Subsubsection: RenderStates ---------------
\subsubsection{RenderStates}
Render states\\

Modes:\\
\begin{tabular}{|p{4.5cm}|p{3cm}|p{9cm}|}
\hline
\textbf{Command} & \textbf{Default} & \textbf{Description}\\
\hline
FillMode  & Solid  & Point = Point fill mode\newline
                     Line  = Line fill mode\newline
                     Solid = Solid fill mode\\
\hline
ShadeMode & Smooth & Flat   = No interpolated during rasterizing\newline
                     Smooth = Interpolated during rasterizing\\
\hline
CullMode  & CCW    & None = No culling\newline
                     CW	  = Clockwise culling\newline
                     CCW  = Counterclockwise culling\\
\hline
\end{tabular}


ZBuffer:\\
\begin{tabular}{|p{4.5cm}|p{3cm}|p{9cm}|}
\hline
\textbf{Command} & \textbf{Default} & \textbf{Description}\\
\hline
ZEnable              & true      & false = No Z/depth buffer test\newline
                                   true  = Z/depth buffer test\\
\hline
ZWriteEnable         & true      & false = Do not write into the Z/depth buffer\newline
                                   true  = Write into the Z/depth buffer\\
\hline
ZFunc                & LessEqual & Z/depth buffer comparison function\newline
                                   See 'Comparison functions'\\
\hline
ZBias                & 0.0       & Depth bias/polygon offset units, <0 = towards camera (e.g. -0.001)\newline
                                   Because SlopeScaleDepthBias and DepthBias below are API and 
                                   GPU dependent, their results are NOT the same on each system \& API. Whenever possible, do NOT use 
                                   this 'classic' render states, use ZBias instead. If this state is not null, the renderer 
                                   will automatically manipulate the internal projection matrix to perform an 'z bias' which is more 
                                   predictable as the 'classic' polygon offset.\\
\hline
SlopeScaleDepthBias  & 0.0       & Slope scale bias/polygon offset factor (e.g. -1.0)\\
\hline
DepthBias            & 0.0       & Depth bias/polygon offset units (e.g. -2.0)\\
\hline
\end{tabular}

Note to SlopeScaleDepthBias and DepthBias:\\
Normally there are horrible artefacts when renderning (nearly) co-planar primitives.
To reduce this 'z fighting' you can set 'factor' (default 0) and 'units' (default 0) to
e.g. factor=-1 and units=-2.
Then each fragment's depth value will be offset after it is interpolated from the
depth values of the appropriate vertices. The value of the offset is factor*DZ+r*units,
where DZ is a measurement of the change in depth relative to the screen area of the
polygon, and r is the smallest value that is guaranteed to produce a resolvable offset
for a given implementation. The offset is added before the depth test is performed and
before the value is written into the depth buffer.\\
This is useful for rendering hidden-line images, for applying decals to
surfaces, and for rendering solids with highlighted edges.\\
Notes: - Has no effect on depth coordinates placed in the feedback buffer\\
       - Has no effect on selection\\


AlphaTest:\\
\begin{tabular}{|p{4.5cm}|p{3.5cm}|p{9cm}|}
\hline
\textbf{Command} & \textbf{Default}  & \textbf{Description}\\
\hline
AlphaTestEnable      & false         & false = Do not perform alpha test\newline
                                       true  = Perform alpha test\\
\hline
AlphaFunc            & GreaterEqual  & See 'Comparison functions'\\
\hline
AlphaRef             & 0.5           & Alpha test reference value\\
\hline
\end{tabular}\\

Notes:\\
- Using the alpha test you can create semi transparent textures\\
- An alpha test will normally (when not using for instance shaders :) only work if the first texture
has an alpha channel (RGBA)\\
- For more information about the alpha test have a look at a previous chapter!


Blend:\\
Specifies pixel blend arithmetic\\

In RGB mode, pixels can be drawn using a function that blends the incoming (source)
RGBA values with the RGBA values that are already in the frame buffer (the destination values).

The src parameter specifies which of nine methods is used to scale the source color components.
The dest parameter specifies which of eight methods is used to scale the destination color
components. The eleven possible methods are described in the following table.
Each method defines four scale factors, one each for red, green, blue, and alpha.

In the table and in subsequent equations, source and destination color components are referred
to as (R(s), G(s), B(s), A(s)) and (R(d), G(d), B(d), A(d)). They are understood to
have integer values between zero and (k(R), k(G), k(B), k(A)), where
\(k (c) = 2^m (c) - 1\)
and (m(R), m(G), m(B), m(A)) is the number of red, green, blue, and alpha bitplanes.

Source and destination scale factors are referred to as (s(R), s(G), s(B), s(A)) and
(d(R), d(G), d(B), d(A)). The scale factors described in the table, denoted
(f(R), f(G), f(B), f(A)), represent either source or destination factors.
All scale factors have range [0,1].\\

\begin{tabular}{|p{4.5cm}|p{9cm}|}
\hline
\textbf{Parameter} & \textbf{(f(R), f(G), f(B), f(A))}\\
\hline
Zero        & (0, 0, 0, 0)\\
One         & (1, 1, 1, 1)\\
SrcColor    & (R(s)/k(R), G(s)/k(G), B(s)/k(B), A(s)/k(A))\\
InvSrcColor & (1, 1, 1, 1)\\
SrcAlpha    & (R(d)/k(R), G(d)/k(G), B(d)/k(B), A(d)/k(A))\newline
               A(s)/k(A), A(s)/k(A), A(s)/k(A), A(s)/k(A))\\
InvSrcAlpha & (1, 1, 1, 1)\newline
              (A(s)/k(A), A(s)/k(A), A(s)/k(A), A(s)/k(A))\\
SrcAlphaSat & (i, i, i, 1)\\
DstColor    & (R(d)/k(R), G(d)/k(G), B(d)/k(B), A(d)/k(A))\\
InvDstColor & (1, 1, 1, 1)\\
DstAlpha    & (A(d)/k(A), A(d)/k(A), A(d)/k(A), A(d)/k(A))\\
InvDstAlpha & (1, 1, 1, 1)\newline
              (A(d)/k(A), A(d)/k(A), A(d)/k(A), A(d)/k(A))\\
\hline
\end{tabular}

In the table:\\
\( i = min( A(s), k(A)-A(d) ) / kA \)\\
\pagebreak

To determine the blended RGBA values of a pixel when drawing in RGB mode,
the system uses the following equations:\\
\( R(d) = min(kR, RssR+RddR) \)\\
\( G(d) = min(kG, GssG+GddG) \)\\
\( B(d) = min(kB, BssB+BddB) \)\\
\( A(d) = min(kA, AssA+AddA) \)\\

Despite the apparent precision of the above equations, blending arithmetic is not exactly specified,
because blending operates with imprecise integer color values. However, a blend factor that should
be equal to one is guaranteed not to modify its multiplicand, and a blend factor equal to zero
reduces its multiplicand to zero. Thus, for example, when sfactor is 'src\_alpha', dfactor is
'inv\_src\_alpha', and A(s) is equal to k(A), the equations reduce to simple replacement:\\
\( R (d) = R (s) \)\\
\( G (d) = G (s) \)\\
\( B (d) = B (s) \)\\
\( A (d) = A (s) \)\\

\begin{tabular}{|p{4.5cm}|p{3cm}|p{9cm}|}
\hline
\textbf{Command} & \textbf{Default} & \textbf{Description}\\
\hline
BlendEnable  & false    & false = Disable blending\newline
                          true  = Enable blending\\
\hline
SrcBlendFunc & SrcAlpha & Source blend function\\
\hline
DstBlendFunc & One      & Destination blend function\\
\hline
\end{tabular}
\\

Stencil:\\
\begin{tabular}{|p{4.5cm}|p{3cm}|p{9cm}|}
\hline
\textbf{Command} & \textbf{Default} & \textbf{Description}\\
\hline
StencilEnable       & false      & false = Disable stencil test\newline
                                   true  = Enable stencil test\\
\hline
StencilFunc         & Always     & Stencil test passes if ((ref \& mask) stencilfn (stencil \& mask)) is true\newline
                                   See 'Comparison functions'\\
\hline
StencilRef          & 0          & Reference value used in stencil test\\
\hline
StencilMask         & 0xFFFFFFFF & Mask value used in stencil test\\
\hline
StencilFail         & Keep       & Operation to perform if stencil test fails\newline
                                   See 'Stencil operations'\\
\hline
StencilZFail        & Keep       & Operation to perform if stencil test passes and Z test fails\newline
                                   See 'Stencil operations'\\
\hline
StencilPass         & Keep       & Operation to perform if both stencil and Z tests pass\newline
                                   See 'Stencil operations'\\
                             
\hline
TwoSidedStencilMode & false      & false = Disable two sided stencil test\newline
                                   true  = Enable two sided stencil test\\
\hline
CCWStencilFunc      & Always     & CCW stencil test passes if ((ref \& mask) stencilfn (stencil \& mask)) is true\newline
                                   See 'Comparison functions'\\
\hline
CCWStencilFail      & Keep       & CCW operation to perform if stencil test fails\newline
                                   See 'Stencil operations'\\
\hline
CCWStencilZFail     & Keep       & CCW operation to perform if stencil test passes and Z test fails\newline
                                   See 'Stencil operations'\\
\hline
CCWStencilPass      & Keep       & CCW operation to perform if both stencil and Z tests pass\newline
                                   See 'Stencil operations'\\
\hline
\end{tabular}


Fog: (fix pass relevant)\\
\begin{tabular}{|p{4.5cm}|p{3cm}|p{9cm}|}
\hline
\textbf{Command} & \textbf{Default} & \textbf{Description}\\
\hline
FogEnable  & false & false = Disable fog\newline
                     true  = Enable fog\\
\hline
FogColor   & 0     & Fog color (RGBA)\\
\hline
FogDensity & 1.0   & Fog densitity\\
\hline
FogStart   & 0.0   & Fog start\\
\hline
FogEnd     & 1.0   & Fog end\\
\hline
FogMode    & Exp   & Exp    = Fog effect intensifies exponentially\newline
                     Exp2   = Fog effect intensifies exponentially with the square of the distance\newline
                     Linear = Fog effect intensifies linearly between the start and end points\\
\hline
\end{tabular}


Point sprite:\\
\begin{tabular}{|p{4.5cm}|p{3cm}|p{9cm}|}
\hline
\textbf{Command} & \textbf{Default} & \textbf{Description}\\
\hline
PointSize        & 1.0   & Point size when it is not specified for each vertex. This value is in screen space units 
                           if 'PointScaleEnable' is 'false'; otherwise this value is in world space units.\\
\hline
PointScaleEnable & false & false = Disable point sprite scale\newline
                           true  = Enable point sprite scale\newline
                           Controls computation of size for point primitives. When 'true', the point size is interpreted 
                           as a camera space value and is scaled by the distance function and the frustum to viewport 
                           y-axis scaling to compute the final screen-space point size. When 'false', the point size is 
                           interpreted as screen space and used directly.\\
\hline
PointSizeMin     & 1.0   & Minimum size of point primitives\\
\hline
PointSizeMax     & 64.0  & Maximum size of point primitives, must be greater than or equal to 'PointSizeMin'\\
\hline
PointScaleA      & 1.0   & Controls for distance-based size attenuation for point primitives\\
\hline
PointScaleB      & 0.0   & Controls for distance-based size attenuation for point primitives\\
\hline
PointScaleC      & 0.0   & Controls for distance-based size attenuation for point primitives\\
\hline
\end{tabular}


PN triangles: (TRUFORM)\\
\begin{tabular}{|p{4.5cm}|p{3cm}|p{9cm}|}
\hline
\textbf{Command} & \textbf{Default} & \textbf{Description}\\
\hline
PNTrianglesEnable           & false     & false = Disable PN triangles\newline
                                          true  = Enable PN triangles\\
\hline
PNTrianglesPointMode        & Cubic     & PN triangles point model\newline
                                          Linear = Linear\newline
                                          Cubic  = Cubic\\
\hline
PNTrianglesNormalMode       & Quadratic & PN triangles normal model\newline
                                          Linear    = Linear\newline
                                          Quadratic = Quadratic\\
\hline
PNTrianglesTesselationLevel & 0         & PN triangles tesselation level\\
\hline
\end{tabular}
      

Misc:\\
\begin{tabular}{|p{4.5cm}|p{3cm}|p{9cm}|}
\hline
\textbf{Command} & \textbf{Default} & \textbf{Description}\\
\hline
PointSpriteEnable & false  & false = Disable point sprite\newline
                             true  = Enable point sprite\\
DitherEnable      & false  & false = Disable dithering\newline
                             true  = Enable dithering\\
\hline
ScissorTestEnable & false  & false = Disable scissor test\newline
                             true  = Enable scissor test\\
\hline
Lighting          & true   &  false = Disable lighting\newline
                              true  = Enable lighting\newline
                              (fix pass relevant)\\
Ambient           & 0      &  General RGBA ambient color\newline
                              (fix pass relevant)\\
\hline
NormalizeNormals  & true   & false = Disable normalize normals\newline
                             true  = Enable normalize normals\\
\hline
InvCullMode       & false  & false = Use default cull mode\newline
                             true  = Use invert cull mode\\
\hline
FixedFillMode    & Unknown & If not 'Unknown' this will 'overwrite' 'FillMode'\\
\hline
\end{tabular}

        

%----- Subsubsection: Material -------------------
\subsubsection{Material}
Material settings\\
	
\begin{tabular}{|p{2.5cm}|p{2.5cm}|p{9cm}|}
\hline
\textbf{Command} & \textbf{Default} & \textbf{Description}\\
\hline
Shininess & 0 & Specifies the RGBA specular exponent of the material. Only values in the range [0,128] are accepted.\\
\hline
\end{tabular}


Color block:\\
General material color\\

\begin{tabular}{|p{2.5cm}|p{2.5cm}|p{9cm}|}
\hline
\textbf{Command} & \textbf{Default} & \textbf{Description}\\
\hline
R & 1.0 & Red color component (0-1)\\
G & 1.0 & Green color component (0-1)\\
B & 1.0 & Blue color component (0-1)\\
A & 1.0 & Alpha color component (0-1)\newline
		  Only used if the material is transparent\\
\hline
\end{tabular}


AmbientColor block:\\
Specifies the ambient RGBA reflectance of the material\\

\begin{tabular}{|p{2.5cm}|p{2.5cm}|p{9cm}|}
\hline
\textbf{Command} & \textbf{Default} & \textbf{Description}\\
\hline
R & 0.2 & Red color component (0-1)\\
G & 0.2 & Green color component (0-1)\\
B & 0.2 & Blue color component (0-1)\\
A & 1.0 & Alpha color component (0-1)\\
\hline
\end{tabular}


DiffuseColor block:\\
Specifies the diffuse RGBA reflectance of the material\\

\begin{tabular}{|p{2.5cm}|p{2.5cm}|p{9cm}|}
\hline
\textbf{Command} & \textbf{Default} & \textbf{Description}\\
\hline
R & 0.8 & Red color component (0-1)\\
G & 0.8 & Green color component (0-1)\\
B & 0.8 & Blue color component (0-1)\\
A & 1.0 & Alpha color component (0-1)\\
\hline
\end{tabular}


SpecularColor block:\\
Specifies the specular RGBA reflectance of the material\\

\begin{tabular}{|p{2.5cm}|p{2.5cm}|p{9cm}|}
\hline
\textbf{Command} & \textbf{Default} & \textbf{Description}\\
\hline
R & 0.0 & Red color component (0-1)\\
G & 0.0 & Green color component (0-1)\\
B & 0.0 & Blue color component (0-1)\\
A & 1.0 & Alpha color component (0-1)\\
\hline
\end{tabular}


EmissionColor block:\\
Specifies the RGBA emitted light intensity of the material\\

\begin{tabular}{|p{2.5cm}|p{2.5cm}|p{9cm}|}
\hline
\textbf{Command} & \textbf{Default} & \textbf{Description}\\
\hline
R & 0.0 & Red color component (0-1)\\
G & 0.0 & Green color component (0-1)\\
B & 0.0 & Blue color component (0-1)\\
A & 1.0 & Alpha color component (0-1)\\
\hline
\end{tabular}




%----- Subsubsection: Layer ----------------------
\subsubsection{Layer}
Texture layer\\
The number of texture layers is only limited by hardware, normally there are at least 2 texture
units but modern hardware has at least 4. If more layers are defined as supported, they will be ignored.\\


Texture: (default: "")\\
Texture filename.\\


SamplerStates block:\\
AddressModes:\\
\begin{tabular}{|p{2.5cm}|p{2.5cm}|p{9cm}|}
\hline
\textbf{Command} & \textbf{Default} & \textbf{Description}\\
\hline
AddressU & Wrap & See 'Texture-addressing modes'\\
\hline
AddressV & Wrap & See 'Texture-addressing modes'\\
\hline
AddressW & Wrap & See 'Texture-addressing modes'\\
\hline
\end{tabular}


Filters:\\
\begin{tabular}{|p{2.5cm}|p{2.5cm}|p{9cm}|}
\hline
\textbf{Command} & \textbf{Default} & \textbf{Description}\\
\hline
MagFilter     & Linear & See 'Texture filtering modes'\\
\hline
MinFilter     & Linear & See 'Texture filtering modes'\\
\hline
MipFilter     & Linear & See 'Texture filtering modes'\\
\hline
MipMapLodBias & 0.0    & See 'MipMapLodBias' below this table\\
\hline
MaxMapLevel   & 1000.0 & Maximum map level\\
\hline
MaxAnisotropy & 1      & Maximum anisotropy\\
\hline
\end{tabular}


MipMapLodBias:\\
The renderer API computes a texture level-of-detail parameter, called lambda
in the specification, that determines which mipmap levels and
their relative mipmap weights for use in mipmapped texture filtering.\\

This extension provides a means to bias the lambda computation
by a constant (signed) value.  This bias can provide a way to blur
or pseudo-sharpen standard texture filtering.\\

This blurring or pseudo-sharpening may be useful for special effects
(such as depth-of-field effects) or image processing techniques
(where the mipmap levels act as pre-downsampled image versions).
On some implementations, increasing the texture lod bias may improve
texture filtering performance (at the cost of texture bluriness).\\


TextureStageStates:\\
\begin{tabular}{|p{2.5cm}|p{2.5cm}|p{9cm}|}
\hline
\textbf{Command} & \textbf{Default} & \textbf{Description}\\
\hline
ColorTexEnv & Modulate & See 'Texture envionment modes'\\
\hline
AlphaTexEnv & Modulate & See 'Texture envionment modes'\\
\hline
TexGen      & None     & None          = No texture coordinate generation (passthru)\newline
                         ObjectLinear  = Object linear\newline
                         EyeLinear     = Eye linear\newline
                         ReflectionMap = Reflection map\newline
                         NormalMap     = Normal map\newline
                         SphereMap     = Sphere map\newline\\
\hline
\end{tabular}



%----- Subsubsection: Shader ---------------------
\subsubsection{Shader}
You are able to use vertex and fragment shaders:\\
\begin{tabular}{|p{2.5cm}|p{2.5cm}|p{9cm}|}
\hline
\textbf{Command} & \textbf{Default} & \textbf{Description}\\
\hline
Profile  & 'best available' & Minimum profile requirement (e.g. 'arbvp1')\\
\hline
Filename & ""               & Shader filename (e.g. 'myshader.cg')\\
\hline
\end{tabular}
