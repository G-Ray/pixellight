\chapter{Scene system}




\section{Overview}
The scene system is one of the main components of PixelLight because it brings together multiple components using a scene graph. Within this chapter you get an idea what the \emph{scene system} from PixelLight is and what components it consists of.

The PixelLight scene system is divided into 4 parts: nodes, hierarchies, queries and scene render queries were a renderer can be seen as an advanced scene query.

Each element in the scene is a node with position, scale and rotation. A container is derived from this base scene element and is able to manage sub nodes within it. (composite design pattern)

We decided against an explicit \emph{world space} design because this would restrict the possibilities especially for huge and dynamic worlds. Each node has an axis aligned bounding box in node object coordinate system - we refer to it as \emph{scene node space}. Also, the node will hold a version of this box in \emph{scene container space}. Each time, the position, scale, rotation or bounding box itself is manipulated, this container space bounding box must be recalculated internally - like the transform matrix. But this recalculation is only done \emph{on demand} - meaning, if you request the transform matrix or container space bounding box and this is \emph{out of data}, it's first updated before it's returned.

This is the base of the scene system - create a scene node container and add scene nodes into it - a node can also be a container with more nodes within it. Each node is relative to it's container node, move your container and all your nodes will \emph{automatically move} with it. If a scene node changes position, rotation and so on, events are generated other components can react on. A node can also have modifiers which adds new functionality - see it as a kind of multi inheritance. For instance you can add physical behaviour to a node by adding a physics body modifier - then this physics modifier will take over the control of position and rotation. Normally this modifier will also listening it's owner node by using events, if for instance the position of the node is changed using \emph{SetPosition()}, the physics modifier will update it's rigid body. In reverse, if the rigid body changes translation it informs the node - but this is another story.


\paragraph{Scene hierarchy}
Each container can have a hierarchy assigned to manage the sub-nodes in a more effective way. If such a hierarchy is created it will create sub-hierarchies basing on the volumes of the nodes within the container. If a certain spatial subdivision is reached such a sub hierarchy contains links to nodes within this sub-hierarchy volume. Therefore, if one or more nodes are manipulated in that way their position, scale, rotation or volume was changed the container hierarchy must be updated. If you have a lot of dynamic nodes you have to select a hierarchy type which can deal efficient with frequently changing nodes. This node/container/hierarchy combination can be seen as a database holding your scene description.


\paragraph{Scene query}
To access the \emph{scene database} you can use a query. For instance, if you want to know which nodes of the scene intersecting a line you can create a \emph{line intersection}-query, set the line start and end point and perform the query. Internally the query will now use the hierarchy of the root container and checks which which sub-hierarchies intersecting this line. If a sub-hierarchy with one or more nodes is found the query will inform it's event listeners about this nodes - were each node is only visited one time. We mentioned that because in fact, one node can be within different sub-hierarchies if the node volume is intersecting the volumes of different sub-hierarchies. If a container node is found this container will be traversed using it's own hierarchy, too. (recursive traverse)

The most queries are created and destroyed frequently - but some may exist over multiple frames or for the complete program life time. The general renderer query which offers different visibility determination techniques for instance is such a \emph{long life} query - because for some special visibility algorithms like \emph{coherent occlusion culling} information from the previous frame is used.

In general such a query informs you only about \emph{touched} nodes. For instance, if you want to know which node is the first intersecting this line you have to sort the resulting nodes by self.


\paragraph{Scene renderer}
As mentioned before, a scene renderer itself is also a kind of query. First, we define what we mean with the term \emph{scene renderer}. If something is drawn on the screen you will often read the word renderer - \emph{amazing per pixel lighting renderer}... but WHAT is a renderer, and can I eat it? You known that there are different graphics API's like OpenGL or Direct3D - we call them \emph{renderer API}. Their main task is to deal with render states, textures, shaders, primitives and so on - in short, the most basic rendering stuff. Ignore here that Direct3D is in fact a small framework itself with the possibility to also render whole complex meshes. The PixelLight renderer is a wrapper around this renderer APIs, so the user normally don't have to care about which renderer API is used internally and how for instance rendering into texture is managed internally. So, this renderer will only deal with the basics of rendering - it is NOT able to render large and complex scenes by self, it even does not know that where is something like a \emph{mesh} or a \emph{large scene}. It's world only consists of render states, textures, shaders, primitives and so on. :)

A scene renderer is using the \emph{scene database} (THE scene graph) you provided and the PixelLight renderer tries to visualize your scene in a performant and suitable manner. A very basic scene renderer will traverse your scene and draws each found node using the draw functions of the node interface - maybe before the node is drawn, someone is checking whether or not this node is potential visible by checking whether or not this node is within the camera frustum. But even it such a frustum intersection test is performed, for a large scene with some thousand of nodes this will be quite inefficent to \emph{touch} each node visible or not. The task of a \emph{real} scene renderer can be divided into two parts: visibility determination (optimal: frustum, occlusion \& lod processing) and rendering. If the scene renderer is using complex per pixel lighting, the rendering of the scene is different to the rendering of the basic scene using old fashioned fix pass rendering. Instead only calling the draw functions of the node interface this renderer must set the different render states, textures, shaders etc. by self and therefore the user itself has less control over the rendering itself. But here, such user limitations are required to overcome more complex rendering. Such a complex renderer normally comes with it's internal \emph{shader library} which offers certain shaders for different purposes. For instance shaders for different light types like spot, projective or volumetric light and per light type variations for different surfaces and shading. If the scene renderer should render something, it will \emph{look into} it's shader library to find a shader which fits the requirements. Here, the PixelLight materials are only a kind of \emph{would nice to have} wishes from the artist. If a material doesn't provide a normal map, there's no need to perform normal mapping. Instead a more simple and more performant shader can be chosen. Or if the surface should use nice environment mapping a special shader combination must be chosen by the renderer. As you see, this is a more advanced and not that universal topic - but  you have to make compromises if something should be able to look good and be fast at the same time. Therefore, for PixelLight we decided to implement this scene renderer as \emph{loose} plugins - \footnote{As nearly everything in PixelLight - thank's to the clever RTTI system} and usually, a scene renderer consists of several scene renderer passes, a realtime compositing.

So, there are different scene renderer for different requirements, each with other limitations - it's possible that you have to write/modify a scene renderer for your own project if the provided stuff doesn't match your requirements. Also, it's recommended to write multiple versions of one and the same renderer - like different versions of an advanced per pixel lighting renderer. One for the latest GPU generation which can profit of the latest shader generation and features - and one for older GPU generations with less render details and features so that your product will also run on this hardware. As mentioned, because of the GPU limitations and the different system specifications, it's impossible to create ONE universal scene renderer which supports the whole palette of GPU generations and different systems and can also be used for each scene for any project.\footnote{In Germany we would say to such a thing \emph{Eierlegende Wollmilch Sau}}




\section{Class name conventions}
In order to be able to know what exactly a file/class is for by only looking at the name, there's a name convention for the scene system relevant things:

\begin{tabular}{|p{2cm}|p{5cm}|p{7cm}|}
\hline
\textbf{Prefix} & \textbf{Derived from} & \textbf{Description}\\
\hline
SN  & SceneNode         & Scene node\\
\hline
SC  & SceneContainer    & Scene container\\
\hline
SNM & SceneNodeModifier & Scene node modifier \\
\hline
SH  & SceneHierarchy    & Scene hierarchy\\
\hline
SQ  & SceneQuery        & Scene query\\
\hline
SR  & RenderQuery       & Scene render query\\
\hline
\end{tabular}




\section{Scene context}
Equally to a renderer context, a scene context is an instance that brings together everything that depends on each other. Within scene nodes there may be rendering data like vertex data, so a scene context is using a render context. Besides the renderer context there are other components that are shared amongst all elements within a scene context like meshes. Scene context instances can't share data. For this reasons you usually have only one scene context within your project - else meshes, textures and so on are loaded multiple times.




\section{What's a scene node?}
All objects within your 3D environment are scene nodes. The most obvious example is an actor that moves through the world. But not each scene node has a mesh, lights, triggers etc. are also scene nodes... even load screens can be scene nodes! (but it does not mean that's the best solution) The most time you will spend in programming those elements for different purposes. If you programmed a scene node once you are able to it use for many things... you can even create new scene nodes using another scene node as basis. Or you are able to add new features to a scene node by adding scene node modifier. All scene nodes are managed automatically by a scene node container, using such a container you can e.g. create new scene nodes.




\section{Scene node names}
Because there can be multiple scene nodes in different scene containers with the same names we need a name convention for scene nodes. If you have a pointer to a container and want to get a scene node from it, one can use for instance \emph{pMyContainer->GetByName("MyNode")} without any troubles. If a camera should always look at a target scene node and both, the camera and the target scene nodes are in the same container, you can use \emph{pMyCamera->SetAttribute("TargetNode", "MyTargetNode")}\footnote{Just an example, the camera scene node does not offer this} and you will get the assumed result - your camera will look at the scene node \emph{MyTargetNode}. But what if you have multiple containers and you only know that there's a scene container children node or parent node which has a node with the name \emph{MyTargetNode} inside it, but you don't know the container name at the moment? The scene system overloads the virtual \emph{Get()}-function of the element manager template to deal with such situations - the \emph{Get()}-function of a scene container is not restricted to \emph{it's own nodes} but also to ALL scene nodes available. Therefore you can use \emph{absolute names} like \emph{Root.DefaultSceneRoot.Scene.MyTargetNode} or \emph{relative names} like \emph{Scene.MyTargetNode} if you currently \emph{in} a container which has a scene container named \emph{Scene} which has a scene node called \emph{MyTargetNode}.

Because this, you are NOT allowed to use \emph{.} characters within \emph{concrete} scene node names you can set for instance by using \emph{pMyNode->SetName("MyNode")}. Further the name \emph{Root} is NOT allowed for scene nodes, this name is reserved for the \emph{scene root container} of PixelLight you can request using \emph{SceneContext::GetRoot()} or by \emph{pMyContainer->GetByName("Root")}. The name of the this \emph{root scene container} can NOT be changed. \emph{pMyContainer->GetByName("Parent")} will return the parent scene container. \emph{pMyContainer->GetByName("Parent.MyNode")} returns the scene node \emph{MyNode} of the parent scene container. \emph{pMyContainer->GetByName("This")} will return \emph{pMyContainer}. The sample application \emph{PLDemoSimpleScene} shipped with the SDK contains some more name examples using a concrete scene.




\section{Create, manage and remove scene node instances}
OK, enough scene system theory - now to praxis. First at all, we need to create a scene context using an existing renderer context.

\begin{lstlisting}[caption=Creating a scene context instance]
SceneContext *pSceneContext = new SceneContext(cRendererContext);
\end{lstlisting}

Within a scene context, there's a root scene container holding ALL scenes. Do NEVER use this root scene directly, instead, add for instance a container and use this as your application scene:

\begin{lstlisting}[caption=Creating a new scene container instance]
SceneContainer *pContainer = pSceneContext->GetRoot()->Create("PLScene::SceneContainer");
\end{lstlisting}

As nearly anywhere in PixelLight the scene system is heavily using the RTTI - and the plugin stuff comes for free. If you are not familiar with the PixelLight runtime type information system (RTTI) it's time to read the RTTI documentation\footnote{\emph{PLCore}-documentation} NOW to understand what it is and how to work with it. Your nodes can be implemented within your executable or within in another shared library, there's no different. Just use the node class name, an optional node name and optional node variable parameters when adding a new node into your scene container. Variables that are not set have a default setting.

\begin{lstlisting}[caption=Creating a new scene node instance]
pContainer->Create("PLScene::SNPointLight", "PointLight", "Color=\"1.0 1.0 1.0\"");
\end{lstlisting}

This adds a node of the type \emph{PLScene::SNPointLight} with the (optional) name \emph{PointLight} and sets the light color (Color) to white (1.0, 1.0, 1.0).

Note that each scene node MUST have it's one, unique name within a scene container - but there can be different nodes with the same name within different containers. If the name is already used then a number will be added behind the name automatically. If no name is provided, the node name is generated automatically using the class name and a number behind it. The third parameter is for optional node parameters. This allows to to set up different registered node variables on creation. The function will return a pointer to the new node, or a null pointer if an error occurred. Because the nodes are managed by the container they are in, the returned node pointer is only used in special situations. Note that you should NEVER store a direct pointer to an node by self because it's possible that the node is removed and then the pointer is invalid! (that's the case for all resources!) Instead store the node name and request a pointer to the node if you need one with:

\begin{lstlisting}[caption=Requesting a scene node by name]
SceneNode *pNode = pContainer->GetByName("PointLight");
\end{lstlisting}

If you need this node frequently or if you have multiple containers it's the best to use an scene node handler which keeps track of the node.

\begin{lstlisting}[caption=Scene node handler]
SceneNodeHandler cNodeHandler;
cNodeHandler.SetElement(pNode);     // Set handled node
pNode = cNodeHandler.GetElement();  // Get handled node
\end{lstlisting}

DO NEVER define/embed nodes (like all resources) direct into classes like

\begin{lstlisting}[caption=Invalid scene node instance usage]
class MyClass {
  private:
    int       m_nCounter;
    SceneNode m_cMyNode;
}
\end{lstlisting}

Because this can, and certainly will, cause fatal errors were many will result in a mysterious behaviour or even crash's! Now, we want to have a mesh in our scene:

\begin{lstlisting}[caption=Creating a new scene node mesh instance]
pContainer->Create("PLScene::SNMesh", "Statue", "Position=\"0.0 0.0 5.0\" Rotation=\"0.0 180.0 0.0\" Mesh=\"Statue.3ds\" Flags=\"CalculateNormals\"");
\end{lstlisting}

Nothing new here. Create an instance of \emph{PLScene::SNMesh} which can deal with meshes, setup the position and rotation of the node, define the filename of the used mesh and tell the node that it should (re-)calculate normals for this mesh. Now we need a camera node which can be seen as your window into the scene and backup a pointer to it, we will need it soon:

\begin{lstlisting}[caption=Creating a new camera scene node instance]
SNCamera *pCamera = (SNCamera*)pContainer->Create("PLScene::SNCamera");
\end{lstlisting}

Can you already see this scene within your head? Hopefully not because there's no light within this scene! OK, let's create a light illuminating the scene:

\begin{lstlisting}[caption=Creating a new light scene node instance]
pContainer->Create("PLScene::SNPointLight", "Light");
\end{lstlisting}

Your basic scene now has all it needs - but how to render it? If one of the renderer surfaces like a window is updated/drawn it will inform it's event listeners about this action. Normally you have an event which will draw some nice things into this renderer surface. The framework itself provides a standard renderer surface painter called \emph{SPScene} were things like a scene root node, camera etc. is assigned to.

\begin{lstlisting}[caption=SPScene setup]
pSurfacePainter->SetRootContainer(pContainer);
pSurfacePainter->SetCamera(pCamera);
\end{lstlisting}

Remove a node by use the \emph{SceneNode::Delete()} function. This function will only mark the node as killed, it will be destroyed later by self.

\begin{lstlisting}[caption=Delete a scene node]
pNode->Delete();
\end{lstlisting}




\section{Scene file format}
PixelLight saves it's scenes within the XML (eXtensible Markup Language) scene format with the extension \emph{scene}. Here's a short example scene:

\begin{lstlisting}[caption=Scene file example]
<?xml version="1.0"?>
<Scene Version="1" Name="Scene" Hierarchy="PLScene::SHList">
  <Container Class="PLPhysics::SCPhysicsWorld" Name="PhysicsWorld" PhysicsAPI="PLPhysicsNewton::World">
    <Modifier Class="DoAnything"/>
    <Node Class="SNMyEye" Name="Hero" Position="-3.0 12.0 5.0" Scale="0.5 1.0 0.5">
      <Modifier Class="PLPhysics::SNMPhysicsBodySphere" Mass="1.0" Radius="1.0"/>
      <Modifier Class="PLPhysics::SNMPhysicsCharacterController"/>
    </Node>
    <Node Class="PLScene::SNMesh" Position="1.0 0.0 0.0">
      <Modifier Class="PLPhysics::SNMPhysicsBodyBox" Mass="1.0" />
    </Node>
  </Container>
</Scene>
\end{lstlisting}

Here's another example using a scene node defined within another project:

\begin{lstlisting}[caption=Another scene file example]
<?xml version="1.0"?>
<Scene Version="1">
  <Node Class="PLScene::SNCamera" Name="Camera" Position="10.0 2.0 0.0" Rotation="0.0 -90.0 0.0" />
  <Node Class="PLParticleGroups::PGSmoke" Position="-50 0 -50" />
</Scene>
\end{lstlisting}

Because each scene itself is in fact a scene container you can also add nodes directly into the scene. And here's the general DTD (Document Type Definition) of this format:

\begin{lstlisting}[caption=Scene file format DTD]
<?xml version="1.0"?>
<!DOCTYPE Scene [
  <!ELEMENT Modifier EMPTY>
  <!ATTLIST Modifier Class CDATA #REQUIRED>
  <!ELEMENT Node (Modifier*)>
  <!ATTLIST Node Class CDATA #REQUIRED>
  <!ELEMENT Container (Modifier*, Node*, Container*)>
  <!ATTLIST Container Class CDATA #REQUIRED>
]>
\end{lstlisting}

In words: A scene consists of nodes, containers, modifiers and a hierarchy. Scene itself is a container. A node can have some modifiers changing their behaviour. Because a container is derived from node, it can also have modifiers, further it can contain nodes or other containers. At last you can specify a hierarchy for the container. This hierarchy is used for the spatial management of it's nodes. Each of this elements has an attribute with at least a class name defining what type of element should be used. Further, there can be a lot of further attributes depending on the used class.




\section{Scene node classes}
Each scene node is basing on a node class you create instances from. All this classes are derived from the PixelLight base node class called \emph{PLScene::SceneNode} and therefore all other classes have the full functionality of this base class! There are some framework node base classes like \emph{PLScene::SNLight} defined within the framework itself... but only the core stuff is fixed. There are also many tool-nodes within the different plugins shipped together with PixelLight. With this given tools you are able to create your first scene without any effort in just a few minutes. It's not recommended to manipulate this standard nodes because they will be updated together with the framework. The better way would be to derive this standard tool nodes like all other nodes to expand their functionality... or you take the node class as base of your own complete new node by coping it.




\section{Creating own scene node classes}
In the most cases you will create your own node classes for an unique behaviour by deriving them from other nodes... at least from \emph{PLScene::SceneNode}. Creating an own scene node class is quite simple:

\begin{lstlisting}[caption=Creating a new own scene node class]
#include <PLScene/Scene/SceneNode.h>

class FlickerLight : public PLScene::SceneNode {
};
\end{lstlisting}

It's not recommended to use your own constructors and destructor's in node classes, but this is possible, too. Use instead the offered virtual functions for initialization and de-initialization you will find in \emph{PLScene::SceneNode} for sure. Note that it's not allowed to derive a new node class from more that one base node class!

\begin{lstlisting}[caption=Invalid scene node class creating]
// Not allowed!!
class FlickerLight : public PLScene::SceneNode, public PLScene::SNLight {
};
\end{lstlisting}

This node class definition isn't complete yet. You still have to register your new class in the PixelLight plugin system to be able to create nodes using this class... (see next sections)




\section{Scene system and RTTI}


\subsection{Overview}
The next section will teach you how to use the PixelLight RTTI together with the scene system. In fact, there's no special RTTI stuff within the scene system and if you already read the RTTI documentation here's nothing new for you.


\subsection{Register scene node classes}
All nodes classes must be registered in the RTTI plugin system in order to be able to use them within the framework. If this nodes are defined within shared libraries all other programs using the framework are able to use this node class, too. For instance you can create instances of this scene nodes within the scene editor. The class registration is done through:

\begin{lstlisting}[caption=RTTI and own new scene node class]
class MyFlickerLight : public PLScene::SceneNode { 
  public:
    // Register new scene node class using RTTI macros

    // Define new RTTI class called 'MyFlickerLight' which is derived from 'PLScene::SceneNode'
	pl_class(PLS_RTTI_EXPORT, MyFlickerLight, "", PLScene::SceneNode, "My flicker light")
      // Your new node can be created public using the Create() function of a scene node container
		pl_constructor_0(DefaultConstructor, "Default constructor", "")
	pl_class_end

};
\end{lstlisting}

At last you have to add this into your node source file which completes the registration process:

\begin{lstlisting}[caption=Finishing RTTI class registration process]
pl_implement_class(MyFlickerLight)
\end{lstlisting}

Now you are able to call this new own class by name! To create a node using this class you have to do this:

\begin{lstlisting}[caption=Creating an instance of an own scene node class]
// pContainer is a scene node container
// created somewhere before
pContainer->Create("MyFlickerLight");
\end{lstlisting}

... easy, uh?

This node plugins are automatically managed by the framework and the classes can be used without any extra work. If the application path is set (done at program start automatically) the framework detects all plugins which are in this directory by self and because its possible to change the application path during runtime the plugings registered in the framework can change during runtime!




\section{Scene node meshes}
Normally it's recommended use scene nodes when dealing with meshes. There's a standard class named \emph{PLScene::SNMesh} with comes with a mesh interface by default. If a mesh is loaded the node function \emph{MeshInitFunction()} is called were you can initialize different mesh related stuff like animations. \emph{MeshDeInitFunction()} is called if the mesh get's unloaded.

Because normally you only work with the animations provided by the mesh handlers animation list it's a good idea to store direct pointers to the different animation information in order to increase code readability and performance! This should be done after to mesh was loaded, \emph{MeshInitFunction()} is predicted for such an usage. Example:

\begin{lstlisting}[caption=Derived mesh scene node]
class MyGoblin : public PLScene::SNMesh {

	pl_class(PLS_RTTI_EXPORT, MyGoblin, "", PLScene::SNMesh, "My goblin")
		pl_constructor_0(DefaultConstructor, "Default constructor", "")
	pl_class_end

  private:
    // Animations
    AnimationInfo *m_pStandAnimation,
                  *m_pRunAnimation;

  protected:
    virtual void MeshInitFunction();
    virtual void InitFunction();

}

void MyGoblin::MeshInitFunction()
{
  // Get the desired animations
  m_pStandAnimation = GetMeshHandler()->GetAnimationInfo("Stand");
  m_pRunAnimation = GetMeshHandler()->GetAnimationInfo("Run");
}

void MyGoblin::InitFunction()
{
  // Get/create the animation manager of your mesh handler
  MeshAnimationManager *pAniManager = GetMeshHandler()->GetMeshAnimationManager();
  if (!pAniManager)
    pAniManager = GetMeshHandler()->CreateMeshAnimationManager();

  // Start animation playback - we check whether the animation
  // is available  first... maybe the artist forgot it :)
  if (m_pStandAnimation) {
    // Add a concrete animation to the animation manager...
    Animation *pAni = pAniManager->Create(m_pStandAnimation->GetSourceName());

    // ... and start the playback of your stand animation now
    // if all went fine
    if (pAni)
      pAni->Start(m_pStandAnimation, true);
  }
}
\end{lstlisting}

In this example \emph{MyGoblin} is derived from \emph{PLScene::SNMesh} which loads a mesh if it's initialized and therefore \emph{MeshInitFunction()} is usually called before \emph{InitFunction()}. The animation start function will check automatically whether the given pointer is valid or not, meaning not a null pointer and therefore you don't have to test the pointer by self to ensure that no crash will occur if there wasn't an animation with the requested name! It's a good idea to write a warning message into the log if the desired animation wasn't found. There's a tool function called \emph{GetAnimationInfo()} in the interface of the mesh handler which will do this automatically. This function will return a pointer to the animation information with the given name. If no animation information with the given name was found a warning message can  be written into the log automatically. Note that the second function parameter is the debug mode were the log message will be written, this is set to 1 by default.


\subsection{Debug flags}
It's possible to visualize anchor points, joints, vertices etc. during runtime using the debug flags. This can be quite useful while debugging.
