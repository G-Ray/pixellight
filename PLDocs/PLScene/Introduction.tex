%----- Chapter: Introduction ---------------------
\chapter{Introduction}


\paragraph{Motivation}
The \emph{PLScene} component is the place were the magic happens. While high level libraries like \emph{PLMesh} already made the daily life easier, a single mesh usually is not enough for most applications. Real world applications need to deal with complete scenes consisting of tons of meshes as well as rendering them. This is were \emph{PLScene} comes in. The two main components within \emph{PLScene} are the scene graph and the scene rendering system. While the scene graph is an representation of the scene, the data, the task of the scene rendering system is to take the scene graph and all assigned data and bringing them onto the computer monitor in the best way possible. For legacy hardware, this scene rendering may just mean to render the scene using simple textures - for decent hardware the scene may be rendered using dynamic lighting and shadowing as well as tons of used special effects like normal mapping, SSAO, HDR and so on.




\section{External Dependences}
PLScene depends on the \textbf{PLGeneral}, \textbf{PLCore}, \textbf{PLMath}, \textbf{PLGraphics}, \textbf{PLGui}, \textbf{PLRenderer}, \textbf{PLMesh} and \textbf{PLInput} libraries.
