\chapter{Introduction}


\paragraph{Motivation}
PixelLight itself has NO own sound implementation nor even fixed build in sound support within the scene graph itself. But there are ton's of sound libraries out there - many of them free or even open source. As mentioned before, PixelLight does not come with native sound support - but through the carefully considered design of the PixelLight framework such an 'hacked in'-support is unnecessary, it would even waste the sweet universal design. Because of the extreme plugin nature of PixelLight, it's no problem to add something like sounds to your projects.

The PixelLight SDK itself comes with a few such sound plugins. By using this plugins it's extremely simple to add sound to your scene. In fact, this plugins only have one scene node container for the sound world and a few scene node modifier. Create a scene container using such a sound world scene node container class and add some scene nodes into it. For nodes which should have sound behaviour just add a sound modifier to the node.

You can use multiple sound API's within your project at the same time, but this isn't recommended and doesn't make much sense. So, for your project, you normally have to choose ONE sound API and use it for the hole project. You can extent this PixelLight sound plugins by self (not recommended) or add a plugin for another sound API if required.




\section{External Dependences}
\emph{PLSound} depends on the \textbf{PLScene} library, and therefore an all other libraries \textbf{PLScene} depends on.
