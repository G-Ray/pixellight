\chapter{Introduction}


\paragraph{Motivation}$\;$ \\
Remember the realtime CG days in the previous millennium where a nose of a character consisted of just three triangles? Well, this were the days were rendering a scene was mostly about rastering the triangles, meshes are made up of. Nowadays, rastering mesh triangles is just a part of the complete rendering process. The final image you see on your screen is the result of many compositing steps - just like movies with tons of special effects produce the final image by compositing multiple image layers. The \emph{PLCompositing} component is the place were the compositing steps within the PixelLight framework are implemented. While the scene graph is a representation of the scene - the data, the task of the scene rendering and compositing system is to take the scene graph and all assigned data and bring them onto the computer monitor in the best way possible. For legacy hardware, this scene rendering may just mean to render the scene using simple textures - for decent hardware the scene may be rendered using dynamic lighting and shadowing as well as tons of used special effects like normal mapping, SSAO, HDR and so on.




\section{External Dependences}

PLScene depends on the \textbf{PLGeneral}, \textbf{PLCore}, \textbf{PLMath}, \textbf{PLGraphics}, \textbf{PLGui}, \textbf{PLRenderer}, \textbf{PLMesh}, \textbf{PLInput} and \textbf{PLScene} libraries.
