\chapter{Script backends}
This chapter deals with the different script backends shipping with the PixelLight SDK.




\section{Null}
\begin{itemize}
\item PixelLight component name: PLScriptNull
\item Used dll's: PLScriptNull.dll
\item The null script backend does nothing
\end{itemize}




\section{Lua}
\begin{itemize}
\item Lua 5.1.4 backend\footnote{Lua can be downloaded from \url{http://www.lua.org/}}
\item PixelLight component name: PLScriptLua
\item Used dll's: \emph{PLScriptLua.dll} and \emph{lua51.dll}
\end{itemize}

There's a really useful \emph{Lua reference card for Lua 5.1}\footnote{\url{http://thomaslauer.com/download/luarefv51.pdf} - Lua reference card for Lua 5.1} on Thomas Lauer's website at \url{http://thomaslauer.com/comp/Lua_Short_Reference}.




\section{AngelScript}
\begin{itemize}
\item AngelScript 2.20.2 backend\footnote{AngelScript can be downloaded from \url{http://www.angelcode.com/angelscript/}}
\item PixelLight component name: PLScriptAngelScript
\item Used dll's: \emph{PLScriptAngelScript.dll} and \emph{angelscript.dll}
\end{itemize}

Here's a list of some general notes:
\begin{itemize}
\item It looks like that AngelScript (2.20.2) has currently no support for namespaces... so right now, an ugly hack is used: e.g. \emph{PL.Timing.GetTimeDifference()} is written within scripts as \emph{PL\_Timing\_GetTimeDifference()}
\end{itemize}




\section{V8 (ECMA-262 compliant JavaScript engine)}
\begin{itemize}
\item V8 JavaScript engine 3.3.1 backend\footnote{V8 JavaScript engine can be downloaded from \url{http://code.google.com/p/v8/}}
\item PixelLight component name: PLScriptV8
\item Used dll's: \emph{PLScriptV8.dll} and \emph{v8.dll}
\end{itemize}




\section{Python}
\begin{itemize}
\item Python 2.7.1 backend\footnote{Python can be downloaded from \url{http://www.python.org/}}
\item PixelLight component name: PLScriptPython
\item Used dll's: \emph{PLScriptPython.dll} and \emph{python27.dll}
\end{itemize}

Here's a list of some general notes:
\begin{itemize}
\item Currently when using namespaces, an ugly hack is used: e.g. \emph{PL.Timing.GetTimeDifference()} is written within scripts as \emph{PL['Timing']['GetTimeDifference']()}
\end{itemize}
