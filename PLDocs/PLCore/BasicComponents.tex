\chapter{Basic Components}




\section{Types And Variables}
\paragraph{RTTI Data Type: PLCore::Type}
The \emph{PLCore::Type} template is the foundation of RTTI variables because each variable is from a certain data type. Therefore, you can see the \emph{PLCore::Type} template as a kind of counterpart to \emph{primitive data types in C}. Each RTTI data type must fulfil the \emph{PLCore::Type} interface offering numerous \emph{ConvertTo} and \emph{ConvertFrom} methods so it's possible to convert one data type into another one using known standard data types. This is the key for many RTTI components making the system quite flexible.

The RTTI comes with several known standard data types which can directly be used. Beside primitive data types for boolean, integer, floating point and so on, there are also some advanced data types like \emph{PLGeneral::String} which are usually frequently used. Although C has no real primitive string data type, real strings, meaning not just \emph{char*}, are quite useful and especially heavily used within script languages. As a result, there's a \emph{PLCore::Type::ConvertToString} and \emph{PLCore::Type::ConvertFromString} method to convert a data type into a string representation, and to set a data type value by using a string representation of the value.

Own RTTI data types can be defined as well. Appendix~\ref{Appendix:UserDefinedRTTIDataType} shows an user defined, not totally serious data type - but be warned, if you just started reading about the RTTI and have no experience with it yet, the example within the appendix may make no sense at this moment.


\paragraph{Dynamic Base Class: PLCore::DynVar}
This class is the core of the RTTI flexibility. If you request for example a variable from an object, you will receive a \emph{PLCore::DynVar} pointer to an arbitrary variable which type is not known, yet. By using the \emph{PLCore::DynVar} interface it's possible to
\begin{itemize}
\item{Request the actual type of the variable by using \emph{PLCore::DynVar::GetTypeID} to get an ID or by using \emph{PLCore::DynVar::GetTypeName} to get the type of the variable as string. The type ID is best suited to compare variable types by using for example \begin{quote}bool bEqual = (pMyObject->MyFloat.GetTypeID() == PLCore::Type<float>::TypeID);\end{quote} were \emph{pMyObject} is an instance of a RTTI class with an \emph{float} attribute called \emph{MyFloat}. For the previous example, \emph{pMyObject->MyFloat.GetTypeName} would return ''float'' because the used C type was \emph{float}.}
\item{Access the content of the variable by using known standard types like \emph{PLCore::DynVar::GetBool}, \emph{PLCore::DynVar::SetBool}, \emph{PLCore::DynVar::GetInt}, \emph{PLCore::DynVar::SetInt}, \emph{PLCore::DynVar::GetString}, \emph{PLCore::DynVar::SetString} and so on. The value will be converted automatically into the real type of the variable.}
\item{Figure out whether or not the variable is currently set to it's default value by using the \emph{PLCore::DynVar::IsDefault} method.}
\item{If the variable is actually an attribute, which means it's a class member, it's also possible to use \emph{PLCore::DynVar::GetVarDesc} to access additional information like a variable description.}
\end{itemize}


\paragraph{Typed Variable: PLCore::Var}
This class offers the typed specializations and is realized as a template - as a result, using those variables is type safe. For instance, \emph{PLCore::Var<int>} is a variable of the type \emph{int}, \emph{PLCore::Var<PLGeneral::String>} of type \emph{PLGeneral::String} and so on. Because \emph{PLCore::Var} is derived from \emph{PLCore::DynVar}, it offers the same interface as \emph{PLCore::DynVar}, this is the specialization. In addition to type safe getter and setter methods, there are operators which enable the direct C like usage of the variable like for instance \emph{PLCore::Var<int> cVar = 42}. The mentioned example would not work if you just had a \emph{PLCore::DynVar} pointer.


\paragraph{Variable Within A Class: PLCore::Attribute}
An attribute is a variable which is a member of an object. Like \emph{PLCore::Var}, \emph{PLCore::Attribute} is also typed and just adds a descriptor containing name, description and so on. At first glance, an attribute looks just like a variable, but internally there's more going on. There are different kind of attribute storages, meaning how and where the actual value of the variable is stored - but this is hidden from the daily programmers life through the RTTI macros so that you don't really need to care about the internal storage. The \emph{PLCore::Attribute} template also manages the allowed variable access like read-only or read/write, but this are just internal details, too. More details about attributes and practical examples will be presented within chapter~\ref{Chapter:ClassesModulesAndPlugins}.


\paragraph{RTTI Data Type Information: PLCore::TypeInfo And PLCore::DynTypeInfo}
\emph{PLCore::Type} can only be used directly if the C++ header, the data type is defined in, is available to you. This is not always the case and therefore an alternative approach is required to deal with this situation. Imagine you just received a fancy new \emph{PixelLight plugin}\footnote{PixelLight plugins will be explained within a following chapter in detail}, but you just got the shared library and no C++ headers at all to work with. By using the RTTI, you can access attributes, methods and so on in a generic way without the need for C++ headers - but sometimes, you may need more information about the data type of, for example an RTTI attribute.

By using an \emph{PLCore::DynVar} instance, it's possible to get dynamic type information by using the method \emph{PLCore::DynVar::GetType} which returns an \emph{PLCore::DynTypeInfo} instance. Through the \emph{PLCore::DynTypeInfo} interface, it's possible to get the ID of the data type and the data type as string. This information can for instance be used to present the end user a proper GUI allowing to edit the value of an attribute of this data type in a comfortable way.

The \emph{PLCore::DynTypeInfo} interface also offers methods for dealing with enumeration data types. You may already know enumerations from, for example the C++ \emph{enum} - the RTTI enumeration data type is nearly the same. This data type is quite useful if there are well defined values a variable can be set to. A derivation of an enumeration data type is a flags data type. Enumerations and flags will be discussed in more detail within the RTTI class attribute section~\ref{ClassMembers:EnumerationAndFlagAttributes}. At the moment, this topic would be to advanced for an introduction into the RTTI and would probably be confusing although the daily practical usage of this special data types is not to complex, too.


\paragraph{Automatic Data Type Conversion}
The data type system within PLCore is able to convert some data types into other data types automatically. The best example for this is the C++ data type \emph{enum} which falls out of the series of classic primitive data types. If the system is detecting an \emph{enum}\footnote{Although C++ offers no functionality to figure out whether or not a data type is an \emph{enum}, it's still possible to get this information by using a neat trick}, the primitive data type \emph{int} will be used internally. Therefore, you can use \emph{enum} just as one would use it in C++ as seen within source~code~\ref{Code:AutomaticDataTypeConversion}.
\begin{lstlisting}[float=htb,label=Code:AutomaticDataTypeConversion,caption={Automatic data type conversion}]
// Define an RTTI enumeration variable
enum MyEnum {
	MyEnumValue1 = 42,
	MyEnumValue2 = 24
};
PLCore::Var<MyEnum> cMyVar(MyEnumValue1); 

// Set enumeration value
cMyVar = MyEnumValue2;
MyEnum nMyEnum = cMyVar;	// "nMyEnum" is now "MyEnumValue2"

// Set a none enumeration value
//	cMyVar = 4;			// This will produce an compiler error
cMyVar = (MyEnum)4;	// Will work...
nMyEnum = cMyVar;	// "nMyEnum" is now "4"
\end{lstlisting}




\section{Functions}
One quite important feature, other systems like the event system or the RTTI are build upon, is the \emph{PLCore::Functor} template that represents a form of type-safe function pointer. In general, this principle is known under names like ''functor'', ''delegate'' and ''callback''. The first parameter of the template is the return value, the other parameters are optional and for example not required if a static function doesn't have any parameters.  This template wraps a function and only makes it's signature (<Return> (<Parameters>)) visible to the world. Other components can just use the function without the need to know whether it's a static function or a member method.

The following examples will use a static function and a member method defined within source~code~\ref{Code:StaticFunctionAndMemberMethod}.
\begin{lstlisting}[float=htb,label=Code:StaticFunctionAndMemberMethod,caption={Static function and member method}]
// Static function
double MyFunction(int nFirst, float fSecond) {
	return nFirst*fSecond;
}

// A class with a method
class MyClass {

	public:
		double MyMethod(int nFirst, float fSecond) {
			return nFirst*fSecond;
		}

};
\end{lstlisting}
Source~code~\ref{Code:UsingFunctorsLikeNativeCppFunctions} shows how functors can be used like native C++ functions.
\begin{lstlisting}[float=htb,label=Code:UsingFunctorsLikeNativeCppFunctions,caption={Using functors like native C++ functions}]
// Functor pointing to the static function MyFunction
PLCore::Functor<double, int, float> cMyFunction0(MyFunction);

// Functor pointing to the member method MyClass::MyMethod
MyClass cMyObject;
PLCore::Functor<double, int, float> cMyFunction1(&MyClass::MyMethod, &cMyObject);

// Call the static function MyFunction
double fResult0 = cMyFunction0(42, 3.1415f);

// Call the member method MyClass::MyMethod
double fResult1 = cMyFunction1(42, 3.1415f);
\end{lstlisting}
Both, \emph{cMyFunction0} and \emph{cMyFunction1} are functors with a signature indicating that there's return value of the type \emph{double}, and two parameters with the type \emph{int} and \emph{float}. While \emph{cMyFunction0} points to a static function, \emph{cMyFunction1} points to a member method - as result the last one also needs an object instance. After the definition of the two functors, there's no difference between the usage of the both. It's no longer relevant whether they point to a static function or to a member method.


\paragraph{Dynamic Parameters}
\label{Functions:DynamicParameters}
Using functors like native C++ functions is not the only possible way, it's also possible to use dynamic parameters by using the \emph{PLCore::Params} template.

Source~code~\ref{Code:UsingFunctorsWithDynamicParameters} shows how functors can be used with dynamic parameters.
\begin{lstlisting}[float=htb,label=Code:UsingFunctorsWithDynamicParameters,caption={Using functors with dynamic parameters}]
// Functor pointing to the static function MyFunction
PLCore::Functor<double, int, float> cMyFunction(MyFunction);

// Functor call using dynamic parameters
// [TODO] MyFunction::fSecond is "3" instead of "3.1415"?
cMyFunction.Call("Param0=42 Param1=3.1415");

// Functor call using typed dynamic parameters
PLCore::Params<double, int, float> cParams(42, 3.1415f);
cMyFunction.Call(cParams);
double fResult = cParams.Return;
\end{lstlisting}
As you can see at the \begin{quote}cMyFunction.Call(''Param0=42 Param1=3.1415'');\end{quote} example, when calling a functor by using just a string, there must by any argument name followed by an equal symbol in front of each single argument value. In here, this argument name is totally free to choose (yes, you may also name it \emph{Bob} if you wish), but the order of the arguments must match the one of the called functor. The only reason for this approach is to make the parsing of the given argument string saver. If the given arguments don't match the functor signature, for example because they have the wrong type, the functor call is just ignored. This means this whole process is type safe. Please note, when using just a string to call a functor as shown within \begin{quote}cMyFunction.Call(''Param0=42 Param1=3.1415'');\end{quote}, it's not possible to catch the returned value using this method - use the \emph{CallWithReturn} version instead. If you're using the feature rich \emph{PLCore::Params} template instead of the shown string short cut, it's also possible to catch the return value. Internally, the example \begin{quote}cMyFunction.Call(''Param0=42 Param1=3.1415'')\end{quote} is the same as writing \begin{quote}cMyFunction.Call(PLCore::Params<double, int, float>::FromString(''Param0=42 Param1=3.1415''))\end{quote} - but less painful.

Beside \begin{quote}PLCore::DynFunc::Call(PLCore::DynParams \&)\end{quote} and \begin{quote}PLCore::DynFunc::Call(const PLGeneral::String \&)\end{quote} there's a third version for functor calls with dynamic parameters, one which works with a given XML element. Internally, the method \begin{quote}PLCore::DynFunc::Call(const PLGeneral::XmlElement \&)\end{quote} is using \emph{PLCore::Params::FromXml} to set \emph{PLCore::Params} values by using XML element attributes. Source~code~\ref{Code:DynamicParametersXML} shows dynamic parameters using an XML element as source in action.
\begin{lstlisting}[label=Code:DynamicParametersXML,caption={Dynamic parameters using an XML element as source}]
// Functor pointing to the static function MyFunction
PLCore::Functor<double, int, float> cMyFunction(MyFunction);

// Define an XML element which should be used as source of the dynamic parameters
PLGeneral::XmlElement cXmlElement("MyElement");
cXmlElement.SetAttribute("Param0", "42");
cXmlElement.SetAttribute("Param1", "3.1415");

// Functor call using dynamic parameters
cMyFunction.Call(cXmlElement);

// Functor call using typed dynamic parameters
PLCore::Params<double, int, float> cParams = PLCore::Params<double, int, float>::FromXml(cXmlElement);
cMyFunction.Call(cParams);
double fResult = cParams.Return;
\end{lstlisting}
Only the attributes of the given XML element are used, the name of the element is irrelevant. The names of the attributes are irrelevant, too. When looking at \begin{quote}PLCore::Params<double, int, float> cParams = PLCore::Params<double, int, float>::FromXml(cXmlElement);\end{quote}, it looks long winded and it would come into ones mind that a short version like \begin{quote}PLCore::Params<double, int, float> cParams(cXmlElement);\end{quote} would be nice. But this would also mean that, to be consistent, a version of the \emph{PLCore::Params} constructor that takes a string would be required as \emph{PLCore::Params::FromString} alternative. Such a \emph{PLCore::Params} constructor that takes a string would be problematic when a parameter like \emph{PLCore::Params<PLGeneral::String>} is required, now there would be to versions of a \emph{PLCore::Params} constructor that both would take a string. As a result, such a short cut would bring some nasty side effects which are better avoided. To be honest, the usage of \emph{PLCore::Params::FromString} and \emph{PLCore::Params::FromXml} is in general not used that often, therefore the paperwork is limited to a few places.




\section{Events And Event Handlers}
\label{Chapter:EventsAndEventHandlers}
The event system is an implementation of the \emph{observer design pattern} and consists of events (\emph{Subject}/\emph{Observable}, the source) an event handlers (\emph{Observer}, the destination). In general, this is a quite universal and elegant way to send and receive messages within an application. Within the first years of PixelLight we've used a listener approach quite similar to the Java one. But we noticed that for PixelLight, this wasn't the right thing because it was to complicated and not as universal and flexible as the solution we're now using.

The two main templates of the event system are \emph{PLCore::Event} and \emph{PLCore::EventHandler}. An event is a message you can listening to by connecting one or multiple event handlers to it. One event handler can be connected to one or multiple events and has one functor that is called when a message arrives. Please note that it's important that the signature of event, event handler and functor must match.

If you understand the following source~code~\ref{Code:UsingEventsAndEventHandlers}, you've understand how the event system works and are ready to use it.
\begin{lstlisting}[label=Code:UsingEventsAndEventHandlers,caption={Using events and event handlers}]
// This static function will be called if the cMyEvent event is emitted
void MyFunction(int nValue) {}

// A class with a method
class MyClass {

	public:
		// This member method will be called if the cMyEvent event is emitted
		void OnMyEvent(int nValue) {}

};

// ...

// An event
PLCore::Event<int> cMyEvent;

// Event handler pointing to a static function
PLCore::EventHandler<int> cMyEventHandler0(MyFunction);
cMyEvent.Connect(&cMyEventHandler0);

// Event handler pointing to a member method
MyClass cMyObject;
PLCore::EventHandler<int> cMyEventHandler1(&MyClass::OnMyEvent, &cMyObject);
cMyEvent.Connect(&cMyEventHandler1);

// Emit cMyEvent event, now MyFunction and MyClass are called
cMyEvent(42);
\end{lstlisting}
Please note that you should not rely on that the event handlers are called within a fixed order. For example, it's not guaranteed that \emph{MyFunction} is called before \emph{MyClass::OnMyEvent}.
