\chapter{Documentation}




\section{Overview}
Because the comments in the code are extremely important to work in an efficient way with the code the PixelLight code is well documented via comments etc. and completely in English. The interfaces etc. are documented using a style Doxygen\footnote{Doxygen can be downloaded from \url{http://www.doxygen.org/}} understands and therefore can create code documentations automatically. Further, we are using a comment style convention and order.




\section{Files}
At the top of the files you will find a short comment header:

\begin{lstlisting}[caption=Header comment]
/*********************************************************\
 *  File: <filename>                                     *
 *    <Optional file comment if useful (not within the
 *    source like 'cpp'-files itself!)>
\*********************************************************/
\end{lstlisting}




\section{Functions}
Above each class/function declaration there's a comment block which give some hints about the class/function.

\begin{lstlisting}[caption=Class/function comment block]
/**
*  @brief
*    <Brief function description>
*
*  @param[in] <InputVariable>
*    <Input variable description, the 'original' value
*    from 'outside' is NOT manipulated>
*  @param[out] <OutputVariable>
*    <Output variable description, the 'original' value
*    from 'outside' MAY be manipulated>
*  @param[in, out] <InputOutputVariable>
*    <Input and output variable description, the 'original'
*    value from 'outside' MAY be manipulated>
*
*  @return
*    <Return value description>
*
*  @remarks
*    <Detailed description>
*
*  @note
*    - <Some hints and good to know information>
*
*  @see
*    - <Reference to another documentation for further
*      reading>
*/
\end{lstlisting}

The \emph{@brief} part should consist of one phrase only. If you can't describe a class, function etc. within a single phrase you should consider to refactor your code...




\section{Tags}
Within the code documentation we use multiple tags to \emph{mark} points within the code so we can find all marks simply by searching for the tags. Do also don't forget to write a comment to the tag even if you thing a comment would be useless there... especially then!

\begin{lstlisting}[caption=Comment tags]
// [TODO] Here's still something that must be done

// [HACK] REALLY terrible solution for a problem, it works but
// we are not happy with the solution -> if you have to use
// this tag frequently, something must be totally wrong...

// [DEBUG] Code that is just there for debugging purposes and
// may be removed later within the development process

// [DEPRECATED] Code that exists only for backward
// compatibility and may be removed within a next version
\end{lstlisting}




\section{Semantic documentation}
You can also add documentation with semantic our \emph{PLProject} code post processing tool \emph{knowns}.

The following adds information to \emph{Microsoft Visual Studio} configuration files so we don't jump \emph{into} methods of the string class during debugging:

\begin{lstlisting}[caption=Semantic documentation]
//[-------------------------------------------------------]
//[ Debugging options for Visual Studio                   ]
//[-------------------------------------------------------]
// <<nostepinto>> PLGeneral::String::.*
\end{lstlisting}

For more information look at the \emph{Microsoft Visual Studio} documentation on \emph{nostepinto} and so on.
