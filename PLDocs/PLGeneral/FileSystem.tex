\chapter{File system}
We don't discuss each technical detail of the file system, instead we bring a number of examples and explain them. To find out the rest like creating a file by yourself should be no problem.




\section{File}
The rule for dealing with files is pretty simple: open, read/write, close

Let's have a look how this may look in practice:

\begin{lstlisting}[caption=File usage example]
// Create the file object and set it to the 'MyFile.my' file
File cFile("MyFile.my");

// Open the existing file to read in text
if (cFile.Open(File::FileRead | File::FileText)) {
  // Read a string from the file
  String sLine = cFile.GetS();

  // Close the file
  cFile.Close();
}

// Open the existing file to write in binary data
if (cFile.Open(File::FileWrite)) {
  // Write the byte sequence 'Blub', 1 byte per item, 4 items
  cFile.Write("Blub", 1, 4);

  // Read a string from the file
  String sLine = cFile.GetS();

  // Close the file
  cFile.Close();
}
\end{lstlisting}

After the boring example, let's open a file located in the Internet:

\begin{lstlisting}[caption=http file usage example]
// Create the file object and set it to a file inside a zip
// archive located on a http server - CRAZY... but it works
File cFile("http://www.pixellight.org/test.zip/Test/1.txt");

// Open the file to read in text
if (cFile.Open(File::FileRead | File::FileText)) {
  // Get the OS console instance
  const Console &cConsole = System::GetInstance()->GetConsole();

  // Write some start information into the OS console
  cConsole.Print("Reading content of '" + cFile.GetFilename()
                 + "' (text):\n");

  // Read until the end of the file is reached
  while (!cFile.IsEof()) {
    // Read a line from the file
    String sLine = cFile.GetS();

    // Modify some escape codes
    sLine.Replace("\r", "\\R");
    sLine.Replace("\n", "\\N");

    // Show the read line on the OS console
    cConsole.Print("- " + sLine + '\n');
  }

  // Close the file
  cFile.Close();
}
\end{lstlisting}




\section{Directory and file search}
Let's directly jump into code action, open a directory and list all files within it:

\begin{lstlisting}[caption=Directory and file search usage example]
// Open directory
Directory cDir("MyDirectory");
if (cDir.Exists() && cDir.IsDirectory()) {
  // Initialize search handle
  FileSearch cSearch(cDir);

  // Loop through all found files
  while (cSearch.HasNextFile()) {
    // Get the name of the found file
    String sName = cSearch.GetNextFile();
  }
}
\end{lstlisting}

To search for all \emph{cpp}-files just modify the file search handle a little bit:

\begin{lstlisting}[caption=Wildcard search handle]
// Initialize search handle with a wildcard
FileSearch cSearch(cDir, "*.cpp");
\end{lstlisting}

For more complex file search filters it's recommended to use a regular expressions:

\begin{lstlisting}[caption=Regular expression search handle]
// Initialize search handle with a regular expression
// (PCRE syntax)
SearchFilterRegEx cFilter("(?i)^(win|w)\\w+(32)?.(dll|exe)$");
FileSearch cSearch(cDir, &cFilter);
\end{lstlisting}




\section{URL}
Besides usual strings, strings which contain an URL (Uniform Resource Locator) are often even more painful to deal with because it's not as uniform as the name may indicate. \emph{Microsoft Windows} for instance is using backslashes (\textbackslash\textbackslash) to separate components while Unix systems are using slash's (/), Unix is case-sensitive while \emph{Microsoft Windows} is not, \emph{MyDirectory/Thing} can be an URL pointing to a directory named \emph{Thing} or to a file named \emph{Thing} without a file extension\ldots We think you go the idea. Because this nasty stuff frequently gave us headaches and making writing platform independent applications to a nightmare, we decided to create an \emph{Url}-class that manages this topic automatically for us so we are happy programmers again.

There are no big problems throwing around strings containing an URL, but if they are manipulated or parsed, we strongly recommend to use the \emph{Url}-class - else you have to take care of the situations mentioned above within each code dealing with URL's and this is can be very error prone and frustrating.

Here's an example of how complicated a simple \emph{get me the path, title and extension}-code can get - even with the comfortable \emph{String}-class:

\begin{lstlisting}[caption=File path\, title and extension without using the Url-class]
// A given filename
#ifdef WIN32
String sFilename = "MyDirectory\\Thing.ext";
#elif LINUX
String sFilename = "MyDirectory/Thing.ext";
#endif

// Get the path the file is in
String sPath;
int nPathIndex = sFilename.LastIndexOf('\\');
if (nPathIndex < 0)
  nPathIndex = sFilename.LastIndexOf('/');
if (nPathIndex >= 0)
  sPath = sFilename.GetSubstring(0, nPathIndex+1);

// Get the title and extension of the file
String sTitle, sExtension;
int nTitleIndex = sFilename.IndexOf('.');
if (nTitleIndex >= 0) {
  sTitle = sFilename.GetSubstring(nPathIndex+1, nTitleIndex-nPathIndex-1);
  sExtension = sFilename.GetSubstring(nTitleIndex+1);
}
\end{lstlisting}

Ugly, isn't it? And don't forget that all codes using your parsed filename parts have also to take care of the differences between operation systems! And now the same job done by using the \emph{Url}-class and the convention used within PixelLight that a slash (\emph{/}) is used so separate the components - like Unix system do it:

\begin{lstlisting}[caption=File path\, title and extension using the Url class]
Url cUrl = "MyDirectory/Thing.ext";
String sPath = cUrl.GetPath();
String sTitle = cUrl.GetTitle();
String sExtension = cUrl.GetExtension();
\end{lstlisting}

Can you see the difference?

Sometimes it's necessary to get URL's, the used OS can deal with - for instance if you pass it to the OS or to other projects that need it in that other format. In this case you can request such an URL representation using a function like for instance \emph{Url::GetWindowsPath()} to get a \emph{Microsoft Windows} compatible URL.

At the beginning of this section we wrote: \begin{quote}Unix is case-sensitive while Microsoft Windows is not\end{quote}\ldots that's something the \emph{Url}-class can't compensate for you automatically because this may produce unwanted side effects. You and your team REALLY have to work accurate when working the filenames. PixelLight itself is case-sensitive, but the underlying OS may not and we can't compensate that as mentioned. If for instance your artist saves a texture as \emph{MyTexture.DDS} PixelLight normally can't load it because it only knows \emph{dds} - in this situation your artist can see at once that there's something wrong. But if this artist gives the file the name \emph{MyTexture.dds}, but references to this file in for example a material using \emph{mytexture.dds} he may not notice any problem on \emph{Microsoft Windows} - but on for example Linux this file can't be found! So, no problem at all while developing something for \emph{Microsoft Windows} you may now say - but what, if your product later has to be ported to for instance Linux? Now you have to go over again ALL data and fix it\ldots not really productive! So we recommend it again to do it right in the first place even if your artists may cry loudly!\footnote{... and our experience tells us, they will cry - loudly, frequently, accusingly\ldots !}
