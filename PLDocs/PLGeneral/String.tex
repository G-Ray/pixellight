\chapter{String}




\section{String class}
Sequences of characters, also called \emph{strings}, are often used and frequently come with some nasty features like memory leaks, buffer overruns (security risk!) or even crash's if you are not careful. Also dealing with Unicode is normally not that much fun\ldots so we wrote a comfortable string class making our and especially your life much easier!

Because strings are quite fundamental, there are several optimizations in place to make dealing with strings as fast as possible. The \emph{String}-class uses a \emph{copy on change}-technique - therefore copying one string into another one is quite fast because the internal string buffer is shared as long as a string doesn't change. As result comparing strings can also be very fast and the internal string buffer can be ASCII OR Unicode in a quite flexible way. To enhance the string performance, the internal string buffers are managed by a string buffer manager to avoid to many memory allocations/deallocations. For an additional performance improvement, memory is traded for speed, meaning that additional characters are allocated within the internal string buffer manager for future use. This way, appending new characters to a string is usually quite fast.

As long as you don't save your source codes in an UTF8 format you can also use the ASCII extension Ansi, meaning characters between 128-256. But with an UTF8 format, this may cause serious problems and you should use Unicode instead ASCII for characters above 128\footnote{Using codepage based ASCII is not recommended} to avoid encoding troubles!

Here's a short usage example how to use the \emph{String}-class and how do deal with ASCII/Unicode/UTF8:

\begin{lstlisting}[caption=ASCII/Unicode/UTF8 string example]
// Test string instance
String sS;

// Set string to 'Mini' as ASCII
sS = "Mini";

// Concatenate 'Me' (ASCII) with the string
sS += " Me";

// Get pointer to ASCII string content
const char *pS = sS.GetASCII();

// Get pointer to ASCII string content
// (same as above but not recommended)
const char *pS = sS;

// Set string to 'Mini' as Unicode
sS = L"Mini";

// Set string to 'nihon' (= Japanese)
// as Unicode by using masked characters
sS = L"\u65e5\u672c\u8a9e";

// Set string to 'Mini' as UTF8
sS = String::FromUTF8("Mini");
\end{lstlisting}

As you can see the well known \emph{char*} and \emph{wchar\_t*} are used for ASCII and Unicode strings. We strongly recommend that you don't brother yourself with this stuff as long as not necessary. So, just use \emph{String sS = "Mini";} instead of \emph{String sS = L"Mini";}. It's more readable and ASCII is more performant and requires less memory. The \emph{String}-class takes automatically care of ASCII/Unicode for you, so you can even mix strings with different internal formats!

We strongly recommend to use only this class when working with strings and not using for instance \emph{c}-functions like \emph{strcmp()}. Do also avoid to use the string data you can get from the \emph{String::GetASCII()} directly to search for substrings and so on - use the provided string functions instead. By doing so, the string class will automatically take care of tricky stuff mentioned above.

In general, it's a good idea to use functions like \emph{String::GetASCII()} to get the \emph{internal data} only if really required, for instance if the string data is given to another non PixelLight system.


\paragraph{UTF8}
Originally, the string class had native UTF8 support build in. For each \emph{char} and \emph{wchar\_t} method, there was also an equivalent UTF8 method. Because the compiler can distinguish \emph{char*} and \emph{wchar\_t} by using \emph{L} in front of strings, but not ASCII from UTF8, we introduced the \emph{utf8} data type. If the compiler should interpret a given string as UTF8, we casted it to \emph{utf8} like \emph{String sS = (const utf8*)"Mini";}.

Due to the complexity of the UTF8 implementation and the possible combinations with \emph{char} and \emph{wchar\_t} and a constant lack of time, the UTF8 implementation was not fully implemented and there were a lot of TODO-points. At the end of the year 2010 the decision was made to give up the internal native UTF8 implementation - the alternatives were to keep the unfinished string codes for some more years, or to spend some weeks in completing and testing the native UTF8 implementation... with the possibility that the resulting implementation would have a poor performance and tons of hidden bugs (waiting quietly in the dark to attack) due to it's complexity. Both options were not that attractive.

Now, the string class offers a \emph{ToUTF8}-method to return the current string content as, internally cached, UTF8 formatted string as well as a \emph{FromUTF8}-method to create a PixelLight string instance by using a given UTF8 formatted string. This two methods can be used to exchange strings with for example UNIX systems or with other APIs like Qt.


\paragraph{wchar\_t}
In contrast to UTF8, \emph{wchar\_t} is quite comfortable to use in practice because each character has the same number of bytes. But there's also a dark side... while on Microsoft Windows a \emph{wchar\_t} is two bytes long (UTF-16), it's four bytes long on UNIX systems (UTF-32). When saving a \emph{wchar\_t} string under Microsoft Windows, and loading it on a UNIX system you will get an ugly surprise (\emph{kawoom!}).

To make it even more interesting, there's a compiler flag called "wchar\_t is treated as built-in type" you may set by using \emph{Zc:wchar\_t}, or not by using \emph{-Zc:wchar\_t-}. While the one libraries, like PixelLight, are using "wchar\_t is treated as built-in type", other libraries are using the other way - resulting in incompatible libraries!

So, be carefully when using \emph{wchar\_t} to serialize data or to pass strings to other libraries, or better, don't do it and use UTF8 instead!


\paragraph{Single characters}
If you care about best possible performance (even if nearly not measurable), use character string methods when dealing with characters - for example, write \begin{quote}String sS = 'A';\end{quote} instead of \begin{quote}String sS = "A";\end{quote}. This way, the string method already knows that you provided just a single character and don't need to count characters internally.


\paragraph{Strings and dynamic variable parameters}
Often it's required to give for instance a function a string which is composed of different substrings. For instance \emph{DrawText("My text")} is trivial, \emph{DrawText("\%d: My text '\%s'", 1, "Text")} will only work if the function supports dynamic variable parameters. For PixelLight, we decided to NOT providing such functions with dynamic variable parameters for various reasons. Instead, one can compose own strings by hand or by using the \emph{PLGeneral::String}-class. With \emph{String} the previous example would look like this \emph{DrawText(String::Format("\%d: My text '\%s'", 1, "Text"))}. \emph{String::Format} will compose the string for you and there can't be a buffer overflow if the resulting string gets to long.

Here's an example how to use this string class and how not:

\begin{lstlisting}[caption=Valid and invalid string usage example]
String sOldString = "Moin";
String sNewString;

// NOT correct! (pointer to string object
// instead of string content is used...)
sNewString = String::Format("Old string: %s", sOldString);

// Correct
sNewString = String::Format("Old string: %s", sOldString.GetASCII());

// Even better because for example Unicode safe
sNewString = "Old string: " + sOldString;
\end{lstlisting}

If possible, try to avoid to use this method because it's usually considered to be \emph{slow} due to it's internal complexity. Therefore, write something like
\begin{quote}String sMyString = String("The number ") + 42 + " is fantastic!";\end{quote} instead of \begin{quote}String sMyString = String::Format("The number \%d is fantastic!", 42);\end{quote}


\paragraph{Strings as function parameters}
When using strings as function parameters and/or return values we recommend the following solution:

\begin{lstlisting}[caption=String as function parameter and return value]
String GetOtherString(const String &sString);
\end{lstlisting}

The string parameter is given as reference which is quite performant as you may now. The returned string on the other hand is an object on the runtime stack - please note that this is not performance critical because the internal string data is NOT copied, it's only shared as long as possible! By doing so, you can avoid tricky situations like giving a function a reference to it's own internal string you received from another function of the same class.

Example:

\begin{lstlisting}[caption=Error prone string usage example]
class StringMessUp {
  public:
    const String &GetFilename() const
    {
      return m_sFilename;
    }
    void Cleanup()
    {
      m_sFilename = "";
    }
    void Load(const String &sFilename)
    {
      Cleanup();
      m_sFilename = sFilename;
    }
  private:
    String m_sFilename;
};

// Somewhere in the deep of the codes...
StringMessUp cInstance;

// ...
cInstance.Load("Test.ext");
\end{lstlisting}

Often a clean up or similar is used as shown in the situation above - but in this case the programmer was careless and will get a surprise if we tests the code with \emph{cInstance.Load(cInstance.GetFilename());} for reloading. Think a moment about this\ldots right, because \emph{sFilename} is in fact a reference to the internal \emph{m\_sFilename} that is cleared within this \emph{Load()}-function before loading, \emph{sFilename} is now empty, too - and nothing is loaded.




\section{Regular Expression}
Regular expressions\footnote{At \url{http://www.regular-expressions.info/} are some good regular expression tutorials online available, as well as a basic syntax reference at \url{http://www.regular-expressions.info/reference.html}} are an universal and powerful tool to deal with string operations like \emph{seach for}. Here's a compact example how to use regular expressions to check whether or not a string is a substring of another one:

\begin{lstlisting}[caption=Regular expression example]
  String sBeer = "BeerNumber99";
  RegEx cRegEx("^BeerNumber.*$");
  if (cRegEx.Match(sBeer))
    sBeer = "Drunk";
\end{lstlisting}

Because the string \emph{BeerNumber99} matches the expression \emph{\textasciicircum BeerNumber.*\$} the beer is now gone. Strings like \emph{\_BeerNumber99} or \emph{Number99} don't match the given expression.




\section{Wildcard}
It's practical to only support regular expressions or wildcards, but not both - so you don't need multiple implementations. Usually it's best to support regular expressions because they are more powerful than wildcards. Wildcards on the other side are more compact and users may already be familiar with them through \emph{Microsoft Windows} - therefore we added \emph{PLGeneral::RegEx::WildcardToRegEx()} so one can convert a given wildcard into an regular expression\footnote{Wildcard: \emph{BeerNumber*} = regular expression: \emph{\textasciicircum BeerNumber.*\$}} and then passing this to a function.




\section{Qt string adapter}
Within PLGeneral, there's a static adapter class for mapping Qt\footnote{Qt is a cross-platform application and UI framework, \url{http://qt.nokia.com/}} strings to PixelLight strings and vice versa.

We provide this string adapter class because it's nice to be able to use PixelLight within Qt as comfortable as possible. Please note that the existence of this adapter class doesn't mean that PixelLight depends on Qt - inside PixelLight, neither this class nor Qt are used!\footnote{That's the reason why there's just a header file for this class, and no c++ file}

The following source code example shows how to convert a Qt string to \emph{PLGeneral::String}.
\begin{lstlisting}[caption=Qt string to PLGeneral::String]
	QString sQtString = "Qt says hello to PixelLight";
	String sPLString = QtStringAdapter::QtToPL(sQtString);
\end{lstlisting}

The following source code example shows how to convert \emph{PLGeneral::String} to a Qt string.
\begin{lstlisting}[caption=PLGeneral::String string to Qt]
	String sPLString = "Greetings from PixelLight to Qt";
	QString sQtString = QtStringAdapter::PLToQt(sPLString);
\end{lstlisting}


\paragraph{wchar\_t}
Please note that there's a perfidy when using \emph{wchar\_t} in combination with PixelLight \& Qt...

When using the default settings within e.g. \emph{Microsoft Visual Studio 2010}, \emph{wchar\_t} is treated as built-in type, so third party unicode shared libraries usually use it. So, this is also the default setting for PixelLight projects. On the other hand, if you want to use PixelLight within Qt and write something like \begin{quote}PLGeneral::String sString = L"Test";\end{quote}, the linker will give you an error message like \begin{quote}error:  unresolved external symbol "\_\_declspec(dllimport) public: \_\_thiscall PLGeneral::String::String(unsigned short const *,bool,int)" (\_\_imp\_??0String@PLGeneral@@QAE@PBG\_NH@Z) referenced in function \_main\end{quote} because within Qt, \emph{wchar\_t} is defined as \emph{unsigned short} and not as build in type \emph{wchar\_t}.

One solution is to remove \emph{-Zc:wchar\_t-} from \emph{QMAKE\_CFLAGS} within \emph{qmake.conf} of Qt, and then recompiling Qt. Another solution,  is to recompile PixelLight with a set \emph{-Zc:wchar\_t-}... which means that it might be necessary to modify external dependencies as well! Both solutions are poor because they enforce to much work and potential additional conflicts! Therefore we pass over the string by using UTF8 to avoid the need to recompile PixelLight/Qt with other compiler settings.
