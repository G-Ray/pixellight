\chapter{Introduction}
With the PixelLight scene exporter for \emph{Autodesk 3ds Max} you are able to export scenes into the framework own scene format. The following is exported:
\begin{itemize}
\item{Position, rotation and scale of nodes}
\item{Cameras will be exported as a node of the class \emph{PLEngine::SNCamera}}
\item{Light sources will be exported as \emph{PLEngine::SN<Type>Light} scene nodes}
\item{Objects with a geometry will be exported as \emph{PLEngine::SNMesh} scene node. The mesh of the object will be saved as as PixelLight mesh with the name of the object.}
\item{All unknown \emph{Autodesk 3ds Max} nodes are saved as \emph{PLEngine::SNUnknown} scene nodes, so, they can still be used as target}
\item{\emph{LookAt} and \emph{Path} node controller are supported, they will result in scene node modifiers}
\item{Instancing is supported. If one object is an instance of another one they will share the mesh. Use instances whenever possible to save memory!}
\item{If a \emph{Autodesk 3ds Max} node is casting shadows\footnote{Select object, right mouse button, properties, cast shadows = true} the scene node flag \emph{CastShadow} is set, for receiving shadows\footnote{Select object, right mouse button, properties, receive shadows = true} the scene node flag \emph{ReceiveShadow} is set. If \emph{Hide} and/or \emph{Renderable} is set, the flag \emph{Invisible} is added. If \emph{Freeze} is set, the flag \emph{Frozen} is added. Exclusion from lighting is supported by using the \emph{Object Properties...} -> \emph{Adv. Lighting} -> \emph{Exclude from Adv. Lighting Calculations}, in this case the \emph{NoLighting}-flag is added.}
\item{User properties\footnote{Select a node, right mouse button, properties, user defined} are added to scene nodes as attributes}
\item{\emph{Background color} is exported as a scene key}
\item{\emph{Ambient color} is exported as a scene key}
\item{\emph{File Properties} -> \emph{Summary} information}
\item{Textures can automatically be copied}
\item{Material export}
\item{If a \emph{Autodesk 3ds Max} node has no material, the exporter will create a fallback material}
\item{There's an own PixelLight fx-Shader\footnote{\emph{PixelLight\_SRShaderLighting.fx}} you can use inside \emph{Autodesk 3ds Max}. During export the fx shader parameters are written into the materials.}
\item{Mesh morpher modifier is supported (\emph{morph targets} and \emph{morph target animation})}
\item{Position, rotation and scale keyframe animation of nodes (\emph{keyframe animation})}
\item{Skin and physics modifier support for skeleton animation with skinning}
\item{Point cache support (\emph{PC2} file format version 1) -> adds \emph{PointCacheFrame\_<frame>} named morph targets}
\end{itemize}
