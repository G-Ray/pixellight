\chapter{Nodes}




\section{Cameras}
Within every scene there should be at least ONE camera! Camera's MUST be within cells, too. (for instance \emph{cell\_Room1\_Camera}) One can place multiple cameras in a scene which can for instance be used as observer cameras. But one camera should be marked as \emph{main camera} which may be (= application dependent!) chosen by default if a scene is loaded. If there's a camera assigned with the current active viewport, this camera is used as \emph{start camera}. If the current active viewport has no camera node assigned to it, the exporter will automatically create a camera scene node called \emph{ActiveViewportCamera} to represent this \emph{Autodesk 3ds Max} viewport - to make your life as comfortable as possible, the exporter also adds some modifiers to this automatically created camera:
\begin{itemize}
\item{If currently within \emph{Autodesk 3ds Max} \emph{Walk Through} is active, the exporter adds modifiers resulting in a \emph{free moveable} camera (have a look at your scene scale!)}
\item{If currently within \emph{Autodesk 3ds Max} \emph{Walk Through} is not active, the exporter adds modifiers resulting in an \emph{orbiter} camera orbiting around the automatically created target scene node \emph{ActiveViewportCameraTarget}}
\end{itemize}

Supported properties:
\begin{itemize}
\item{The \textbf{FOV} of the camera is saved within the \emph{FOV} variable of the PixelLight camera.}
\item{If \textbf{Clipping Planes: Clip Manually} is active, near clip is saved within the variable \emph{ZNear} and far clip within \emph{ZFar}. Do NOT make the far clip to small if you don't want to see the scene disappearing in the void. Please take into account, that even if you don't \emph{activating} clip manually there is a near and far clipping using default settings. If your world scale is TO small TO huge this default settings may cause visible artefacts. (z-fighting/tearing)}
\end{itemize}

Sometimes you want to \emph{move around} a camera for debugging purposes. (to inspect the exported scene) PixelLight comes with some default controller scene node modifiers that enable you to move around scene nodes. Add this scene node modifiers to your camera node:

\begin{lstlisting}[caption=Free camera scene node modifiers]
Mod=Class="PLEngine::SNMMouseLookController"
Mod1=Class="PLEngine::SNMRotationFixRoll"
Mod2=Class="PLEngine::SNMMoveController"
\end{lstlisting}

and if you select the camera for instance within \emph{PLViewer}, you can move around this camera. The first modifier modifies the rotation of the camera by using user input. The second scene node modifier fixes undesired \emph{roll} by align the rotation along a given \emph{up-vector} which is \emph{(0 1 0)} by default. If you have a \emph{target camera}, this second modifier is not required. The last modifier is for the movement. If you don't already know HOW to add such PixelLight scene node modifiers within \emph{Autodesk 3ds Max}, you should have a look at the \emph{User Properties}-section below.




\section{Lights}
Without lights, normally one can see absolute nothing. So, put lights into your scene! Try to avoid \emph{overlapping} lights because each batch/object may be rendered per light again.\footnote{This depends on the used scene renderer} If an object is lit by 10 lights, this object may be drawn 11 times, 1 within the ambient lay-down pass and 10 times for the different lights - but this depends, as mentioned before, on the used scene renderer. The \textbf{far attenuation end} variable of light is used as \emph{light range} inside PixelLight.

Supported light types:
\begin{itemize}
\item{Omni: \textbf{PLEngine::SNPointLight} or \textbf{PLEngine::SNProjectivePointLight} (if there's also an active \emph{projector map}) light scene node.}
\item{Spot: \textbf{PLEngine::SNSpotLight} or \textbf{PLEngine::SNProjectiveSpotLight} (if there's also an active \emph{projector map}) light scene node. Falloff/field variable is used for the \emph{OuterAngle} variable of this scene node type. \emph{Circle} and \emph{Rectangle} light shape is supported. \emph{Aspect} is only used for \emph{Rectangle} spot lights. \emph{Hotspot}/\emph{beam} is used for the \emph{InnerAngle} variable of this scene node type - only supported for \emph{Circle} light shape. The \textbf{far attenuation start} variable of light is used as \emph{ZNear} of the spot light inside PixelLight. (use responsible settings!)}
\item{Directional: \textbf{PLEngine::SNDirectionalLight} light scene node.}
\end{itemize}

If a light is not \emph{on} within \emph{Autodesk 3ds Max}, the PixelLight light scene node receives an \emph{Invisible}-flag.

Shadow map for shadow casting lights is supported. For performance reasons, it's NOT recommended to set this flag for each light! If a light with another shadow technique than shadow mapping is found, a hint is written into the log and shadow casting is ignored for that light.

Light color multiplier is supported.




\section{Paths}
If a node is moving on a path, a \emph{PLEngine::SNMPositionPath} scene node modifier is added to the exported node automatically. If \emph{Follow} is checked within a path, a \emph{PLEngine::SNMRotationMoveDirection} scene node modifier is added, too. So the node will always look into the movement direction.





\section{User properties}
By using the \emph{User Properties} of a \emph{Autodesk 3ds Max} node it's possible to give the exporter some more information what to do with the node during the export.

Here's an example how to add \emph{scene node modifiers}:

\begin{lstlisting}[caption=Multiple scene node modifiers]
Mod=Class="PLEngine::SNMMouseLookController"
Mod1=Class="PLEngine::SNMRotationFixRoll"
Mod2=Class="PLEngine::SNMMoveController"
\end{lstlisting}

Here's an example how to change the PixelLight node class of the exported node:

\begin{lstlisting}[caption=Setting the scene node class]
Class="PLPostProcessEffects::PGFire"
\end{lstlisting}

Here's an example how to setup some variables of the exported PixelLight node:

\begin{lstlisting}[caption=Setting scene node variables]
Vars=Size="0.2" Particles="30"
\end{lstlisting}

Here's an example how to setup some flags of the exported PixelLight node:

\begin{lstlisting}[caption=Setting scene node flags]
Vars=Flags="Corona"
\end{lstlisting}
