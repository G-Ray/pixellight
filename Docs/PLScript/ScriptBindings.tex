\chapter{Script bindings plugin}
\label{ScriptBindingsPlugin}
This chapter is about script bindings in general and in particular about the loose plugin \emph{PLScriptBindings} which exposes certain parts of PixelLight to script languages.




\section{PL}
Exposes general PixelLight features to scripts.

\paragraph{<RTTI object> PL.GetApplication()}
Write \emph{PL.GetApplication()} in order to get an instance of the currently used RTTI application class. This can be a null pointer, but usually it isn't.




\section{PL.System}
Exposes \emph{PLGeneral::System}-features to scripts.

\paragraph{<bool> PL.System.IsWindows()}
Write \emph{PL.System.IsWindows()} in order to figure out whether or not the application is currently running on MS Windows. Returns \emph{true} if we're currently running on a Windows platform, else \emph{false}.

\paragraph{<bool> PL.System.IsLinux()}
Write \emph{PL.System.IsLinux()} in order to figure out whether or not the application is currently running on Linux. Returns \emph{true} if we're currently running on a Linux platform, else \emph{false}.




\section{PL.Log}
Exposes \emph{PLGeneral::Log}-features to scripts.

\paragraph{PL.Log.OutputAlways(<string>)}
Write \emph{PL.System.OutputAlways(<string>)} in order to write the given string into the log ('always' log level).

\paragraph{PL.Log.OutputCritical(<string>)}
Write \emph{PL.System.OutputCritical(<string>)} in order to write the given string into the log ('critical' log level).

\paragraph{PL.Log.OutputError(<string>)}
Write \emph{PL.System.OutputError(<string>)} in order to write the given string into the log ('error' log level).

\paragraph{PL.Log.OutputWarning(<string>)}
Write \emph{PL.System.OutputWarning(<string>)} in order to write the given string into the log ('warning' log level).

\paragraph{PL.Log.OutputInfo(<string>)}
Write \emph{PL.System.OutputInfo(<string>)} in order to write the given string into the log ('info' log level).

\paragraph{PL.Log.OutputDebug(<string>)}
Write \emph{PL.System.OutputDebug(<string>)} in order to write the given string into the log ('debug' log level).




\section{PL.System.Console}
Exposes \emph{PLGeneral::Console}-features to scripts.

\paragraph{PL.System.Console.Print(<string>)}
Write \emph{PL.System.Console.Print(<string>)} in order to write the given string into the system console.




\section{PL.Timing}
Exposes \emph{PLGeneral::Timing}-features to scripts.

\paragraph{<float> PL.Timing.GetTimeDifference()}
Write \emph{PL.Timing.GetTimeDifference()} in order to get the past time since last frame (seconds).

\paragraph{<float> PL.Timing.GetFramesPerSecond()}
Write \emph{PL.Timing.GetFramesPerSecond()} in order to get the current frames per second (FPS).
