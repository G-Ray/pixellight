\chapter{Base}




\section{PLCore}
\begin{itemize}
\item{Generic functor class\footnote{Delegate, a form of type-safe function pointer, 'callback'}}
\item{Event system\footnote{The principle is also known as signals \& slots}}
\item{\ac{RTTI} with attributes, methods, signals, slots and properties}
\item{Plugin system with support for delayed shared library loading to speed up the program start and to reduce memory consumption}
\item{Loadable system. Everything that can be loaded and saved is not limited to just one 'hacked in' file format. Hence, it is possible to add, for instance, a loader plugin for your own mesh or image file formats in a quite universal way!}
\item{Localization system}
\item{Configuration system}
\end{itemize}


\subsection{Text Processing}
\begin{itemize}
\item{An advanced dynamic string class with \ac{ASCII}, Unicode and UTF-8 support + extensive UTF-8 tool class}
\item{An advanced tokenizer which allows you to parse texts without any effort}
\item{\ac{XML} classes (\ac{DOM}) making it comfortable to load, edit and save \ac{XML} files}
\item{Regular expression class\footnote{Internally PCRE (\url{http://www.pcre.org/}) is used}}
\item{Extensive command line parser}
\end{itemize}


\subsection{System}
\begin{itemize}
\item{Multi threading functionality (thread, semaphore, mutex)}
\item{Dynamic/shared libraries}
\item{Pipes}
\item{Class for comfortable processes interaction}
\item{Easy access to system information like used \ac{OS}, user name, available memory, etc.}
\item{Access to environment variables}
\item{Basic \ac{OS} console functions}
\item{Class for working with the registry (if the \ac{OS} has one)}
\end{itemize}


\subsection{Network}
\begin{itemize}
\item{Basic universal standard network functionality like sockets}
\item{Additional basic network functionality like client/server classes}
\end{itemize}


\subsection{File System}
\begin{itemize}
\item{File class for access to standard \ac{OS} files, archives like zip\footnote{Internally zlib (\url{http://www.zlib.org/}) is used} and even http with password support and other special access settings}
\item{Quite universal design. For instance, it is possible to open a file within an archive from an http server.}
\item{Comfortable \ac{URL} class taking the pain when dealing with filenames, especially when you want to work platform independent}
\item{Access to the \ac{OS} standard streams using the file class}
\item{File search functionality with filters (wildcard, regular expression)}
\end{itemize}


\subsection{Script}
\begin{itemize}
\item{Generic script language independent script interface\footnote{The PixelLight \ac{SDK} comes with Null and Lua (\url{http://www.lua.org/}) backends, within the repository are also experimental JavaScript (using V8, a ECMA-262 compliant JavaScript engine, see \url{http://code.google.com/p/v8/})), Python (\url{http://www.python.org/}) and AngelScript (\url{http://www.angelcode.com/angelscript/}) backends.}}
\item{Supported script features are global variables, global functions\footnote{C++ calls script and script calls C++}, \ac{RTTI} objects\footnote{Properties, attributes, methods, signals, slots} and namespaces}
\item{Scripting is heavily using PLCore features like the \ac{RTTI}, therefore adding script bindings or using \ac{RTTI} objects within scripts is fairly straightforward and don't require the writing of thousands of proxy/wrapper classes exposing C++ functionality to script languages}
\item{Certain parts of PixelLight are exposed to script languages through the loose plugin \emph{PLScriptBindings}}
\item{Scripting is completely optional, not mandatory}
\end{itemize}


\subsection{A huge set of Additional Tools}
\begin{itemize}
\item{Log with flexible implementation so we can output into a file (unformatted text, \ac{XML}, \ac{HTML}) or into the console}
\item{Unified abstract checksum interface (the \ac{SDK} comes with a MD5, CRC32 and SHA-1 plugin). Supports checksum from string, memory or files.}
\item{General list interface with concrete implementations for linked list, array, bitset and so on}
\item{Many basic tool classes like stack, queue, hash map, heap, quick sort, singleton, iterator etc.}
\item{PixelLight comes with an efficient resource manager template system which is used by several managers, e.g. textures, meshes, sounds, paths, shaders etc. }
\item{PixelLight provides an advanced timing class which offers a lot of timing relevant functions. Time difference since the last frame with clamp functionality, past time since start, slow motion, pause, freeze, \ac{FPS} limiter, stopwatch}
\item{Basic memory manager so you can mix release and debug builds as good as possible}
\end{itemize}




\section{PLDatabase}
\begin{itemize}
\item{Abstract unified database interfaces\footnote{The PixelLight \ac{SDK} comes with Null and SQLite (\url{http://sqlitebrowser.sourceforge.net/}) backends, within the repository are also experimental MySQL (\url{http://www.mysql.com/}) and PostgreSQL (\url{http://www.postgresql.org/}) backends.}}
\end{itemize}




\section{PLGraphics}
\begin{itemize}
\item{Extensive image class which is able to load and save the image formats dds, png\footnote{Using libpng \url{http://www.libpng.org/}}, tga, jpg\footnote{Using libjpeg \url{http://www.ijg.org/}}, ppm and bmp by default. Further formats can be added without any effort.}
\item{Comfortable rgb and rgba color classes}
\end{itemize}




\section{PLMath}
\begin{itemize}
\item{Various comfortable and feature rich vector classes for 2D, 3D and 4D vectors}
\item{Easily usable 3x3, 3x4 and 4x4 matrices with also offers important functions, like converting a direction vector into a rotation matrix}
\item{Quaternions}
\item{Planes}
\item{Bounding box (AABB and OBB)}
\item{Advanced Euler angles conversion tool class (from/to rotation matrix, from/to quaternion -> and all for multiple axis orders)}
\item{Helper functions like transforming a 2D coordinate to an 3D and backward to, for example, find out were in the 3D world the 2D mouse cursor is in}
\end{itemize}




\section{PLGui}
\begin{itemize}
\item{PLGui is an universal library that offers many functions for creating editors, etc. The framework itself also uses this \ac{GUI} for its own 'ingame' dialogs.}
\item{The PLGui library provides simple classes for creating standard \ac{UI} applications (using windows, dialogs etc.) and is designed to easily integrate PixelLight rendering into those programs. But you can also use other \ac{UI} libraries and integrate PixelLight rendering into them if you wish (e.g. \ac{MFC} or Qt)}
\item{By using the PLGui for creating editors, they will also be compilable under every platform the framework is ported to (e.g. Linux)}
\item{Tons of controls like bitmap, button, check box, combo box, edit box, group box, group box group, label, list box, list view, menu, menu bar, pull down button, progress bar, slider, spinner, splitter, status bar, tab bar, tool bar, tree view)}
\item{Various window types like normal window, content window, frame, logo window, progress window, scroll window, tooltip, tool window}
\item{Different dialogs like class view, color dialog, config view, file/directory dialog, file system view, input box, message box, system view}
\end{itemize}




\section{PLInput}
\begin{itemize}
\item{Aimed at modern input devices (6DOF tracking devices)}
\item{The input component gives you a feature rich access to standard devices like keyboard, mouse and joystick/joypad}
\item{Wii Remote support}
\item{SpaceMouse (SpaceNavigator, SpacePilot and so on) support}
\item{Device output controls such as rumble and force-feedback effects as well as control over device LEDs}
\item{Internally HID, Bluetooth and special \ac{OS} functionality is used}
\end{itemize}




\section{PLRenderer}
\begin{itemize}
\item{Dynamic \ac{API} design\footnote{The PixelLight \ac{SDK} comes with Null and \ac{OpenGL} (using FreeType (\url{http://www.freetype.org/}) for font support) backends, within the repository there's also a OpenGL ES 2.0 backend and an experimental Direct3D 9 backend}}
\item{Multiple output windows}
\item{\ac{RTT} (also rendering to floating-point for \ac{HDR} buffer is possible)}
\item{\ac{MRT} for rendering into different textures at the same time}
\item{Primitives are drawn through \ac{VBO} and \ac{IBO} for maximum performance}
\item{Vertex streaming for combining the data of different \ac{VBO}s}
\item{1D, 2D, 2D array\footnote{Kind of 3D texture, but without filtering between the depth layers}, rectangle\footnote{Predecessor of the modern and universal \ac{NPOT} textures \ac{GPU} feature}, 3D and cube textures}
\item{Mipmaps and texture compression support\footnote{DXT1 (= BC1), DXT3 (= BC2), DXT5 (= BC3), LATC1 (= 3DC+/ATI1N/BC4) and LATC2 (= 3DC/ATI2N/BC5)}. These can be created automatically on the fly or used from given dds data, for instance, for maximum control and best loading times.}
\item{Abstract \ac{GPU} program interface with support for vertex, tessellation control\footnote{Named \emph{hull shader} in DirectX 11}, tessellation evaluation\footnote{Named \emph{domain shader} in DirectX 11}, geometry and fragment shaders}
\item{The \ac{OpenGL} renderer backend has a build in \ac{GPU} program interface implementation for \ac{GLSL}}
\item{The \ac{OpenGL} renderer backend has an optional \ac{GPU} program interface implementation for Cg (\url{http://developer.nvidia.com/object/cg_toolkit.html}) via plugin}
\item{\ac{UBO} support}
\item{Fonts and draw helpers so it's easy to draw lines, images and so on}
\item{Interface for fixed functions to support legacy graphics cards without, or just limited shader support}
\item{A lot of standard functions like stencil, blend, fog, point sprites, anisotropic texture filtering and much more not worth to be mentioned in here because it's just standard must have stuff}
\item{\ac{GPU} program generator class for �ber-shaders}
\end{itemize}


\subsection{Texturing}
\begin{itemize}
\item{Different texture creator plugins (for instance blank texture, normalization cube map, angle cube map...)}
\item{The image class of PLGraphics is used to load and save textures which makes things quite comfortable}
\item{Alpha blended textures which can be loaded through formats like tga automatically. It is also possible to define a color key by providing a RGB and tolerance value to describe transparent areas}
\item{Maximum texture size is only limited by hardware, today nearly every card has at least textures with a size of 2048x2048, a GeForce4 for instance has maximum texture size of 4096x4096... and ATI Radeon HD 5870 up to 16384x16384... }
\item{Textures are automatically resized by the framework if their size is too large for the given hardware or if its dimensions are invalid\footnote{When the texture is for instance \ac{NPOT} but the modern and universal \ac{NPOT} textures \ac{GPU} feature is not supported}. Moreover, there is a texture quality option in the configuration where the user can change the texture quality in order to gain more performance}
\item{Procedural texturing}
\item{Different texture animation techniques are provided, e.g. changing textures, texture matrix and color animations for a maximum freedom of creativity}
\end{itemize}


\subsection{Materials \& Effects}
\begin{itemize}
\item{PixelLight comes with a powerful material \& effect system which enables you to create amazing effects like Normal-Mapping etc.}
\item{Each material consists of different flexible properties like color, shininess etc.}
\item{In addition to the main effect properties, it is possible to setup a lot of options like blending, culling, polygon offset etc. for each material}
\item{You can use as many texture layers as supported by hardware. Today there are at least 2 texture layers available, but a GeForce4 for instance has 4 and the latest hardware even has up to 16! Each texture layer can be animated which opens a huge animation freeness.}
\item{The effect system supports different techniques (fallbacks), passes were you can assigned shaders to each pass etc.}
\end{itemize}




\section{PLMesh}


\subsection{Meshes}
\begin{itemize}
\item{Own flexible binary chunk based mesh format. The mesh library comes with different mesh import plugins for 3ds, lwo, ase, obj, smd, x (any many more) - through the nature of PixelLight it's no problem to implement more importer and exporter by self.}
\item{Meshes can consist of different geometries and therefore it's possible to have a mesh with different materials per geometry.}
\item{Mesh animations can be mixed together, therefore different animation channels for vertex and skeleton animations are provided per mesh}
\item{Different mesh creator plugins (for instance sphere, cube, cylinder, disk, torus...)}
\end{itemize}


\subsection{Animation System}
\begin{itemize}
\item{All animation types are handled equally, meaning that the same animation interface is used to control vertex, skeleton, texture etc. animation. So every animation type can be accessed in the same way and with the same features!}
\item{Animations can cause events at certain frames. Those events will be sent to their owner entity which can react on it}
\item{Vertex (morph targets, can for example be used for facial animations) and skeleton animation system with multiple weights per vertex}
\item{Blending of different animations}
\end{itemize}




\section{PLScene}


\subsection{Scene System}
The dynamic and universal hierarchical scene system consists of scene nodes and scene containers (composite design pattern). A scene node is the most basic scene element, and a scene container manages such scene nodes. A scene container can also manage child scene containers for a hierarchical scene. Within a scene container, a scene hierarchy like a KD-tree can be used to manage/access the scene nodes in a more effective way. Scene queries operate on this scene description. For instance, there is a query which returns all scene nodes intersecting a line or plane set. To make the system more efficient, scene nodes can have different modifiers. For instance, you can use a physics scene node modifier to give your scene node a physical behaviour. Physics are not fixed build in the scene graph itself - it comes through such plugins to make the framework more universal.
\begin{itemize}
\item{Powerfull 'scene node modifier' concept allowing to add as many modifiers to scene nodes you want taking for instance control over the position of the node, controlling morph targets of meshes and so on - the possibilities are nearly unlimited!}
\item{Because the \ac{RTTI} is used nearly everywhere in PixelLight, the scene system can be extended without any effort and components can be reused in other projects}
\item{KD-tree and list scene hierarchy plugins provided}
\item{Line/plane set intersection and culling queries provided as well as multiple other queries}
\item{Scenes can be rendered using no culling at all, frustum culling or coherent hierarchical occlusion culling for more complex scenes}
\item{A comfortable post process manager class and a lot of prepared post processing effects are provided. Adding bloom, grey scale or even visualizing the scene using \ac{ASCII} characters is no problem at all. You can combine existing effects to create a new one.}
\item{Various useful scene nodes like camera, light, sky, terrain, ingame \ac{GUI}, mirror and so on are built in}
\item{The sky scene node comes with multiple sky layers and creates impressive animated and atmospheric backgrounds like hills, where slowly moving clouds appear behind them}
\item{The terrain scene node is using GeoMipMapping to create easily usable, impressive and large landscapes}
\item{The scene format is a simple \ac{XML} text format and therefore it's nearly version change save}
\item{Lights can have coronas, lens flares or can even blind the screen looking into them! Lights can also project images over the scene, like a projector, which produces impressive effects! Because nearly all content in the framework is managed in the same way, you can also project a video texture over the scene where you will see a movie!}
\end{itemize}


\subsection{Compositing System}
Scenes are rendered using a realtime compositing system. For example a first layer may clear the screen color to black, another may write down depth information and ambient color, a next may add lighting, another one fog and so on. The system is using the PixelLight \ac{RTTI}, as such, it's expandable and can be heavily configured\footnote{Adding new layers, changing layer order, changing layer attributes, everything directly while a program is running}.
\begin{itemize}
\item{There are fixed functions and shaders based compositing layers to support a broad range of graphics cards}
\item{Fixed function: For legacy hardware without shader support and just fixed build in graphics features}
\item{Forward: A classic forward renderer using shaders. Each object is drawn per light again.}
\item{Deferred: A modern deferred renderer approach performing for example lighting in image space}
\item{Scene rendering is usually using �ber-shaders to enable many shader features, while using just the currently required features}
\item{Several types of light sources: directional, omnidirectional, spot, omnidirectional projective and projective spot}
\item{Shadow mapping}
\item{Post process system with 'build in' effects like \ac{DOF} and many effects as plugins}
\item{\ac{HDRR}}
\item{Reinhard tone mapping, light adaptation, \ac{HDR} bloom}
\item{Glow}
\item{Depth fog and volumetric fog}
\item{God rays\footnote{This effect is also known as crepuscular rays, sunbeams, sunbursts, star flare, sun shafts, or light shafts}}
\item{\ac{SSAO} with implementations for \ac{HBAO} and \ac{HDAO}}
\item{\ac{FXAA}}
\item{Gamma correction}
\item{Layers with debug information like wireframe, scene node icons, scene node names and so on}
\item{Many material features like diffuse and opacity mapping, normal mapping, detail normal mapping, parallax mapping, two sided lighting, specular and gloss mapping, \ac{AO} mapping, light mapping, emissive mapping, glow mapping, fresnel reflection, spherical environment mapping, cubic environment mapping, reflectivity mapping}
\item{Normal map compression using swizzled DXT5 (xGxR), LATC2\footnote{Formally \emph{3DC}/\emph{ATI2N} on ATI \ac{GPU}s only} and alternate XY swizzle LATC2 is supported}
\end{itemize}




\section{PLSound}
\begin{itemize}
\item{Abstract and universal sound \ac{API}\footnote{The PixelLight \ac{SDK} comes with Null and OpenAL (\url{http://www.openal.org/}) backends, within the repository are also experimental FMOD (\url{http://www.fmod.org/}) and FMOD Ex (\url{http://www.fmod.org/}) backends. \emph{FMOD Sound System, copyright � Firelight Technologies Pty, Ltd., 1994-2011.}}}
\item{2D and 3D sound}
\item{Streaming}
\item{Global volume and pitch control (for instance to slow down the sound playback)}
\end{itemize}




\section{PLPhysics}
\begin{itemize}
\item{Abstract and universal physics \ac{API}\footnote{The PixelLight \ac{SDK} comes with Null and Newton Game Dynamics (\url{http://www.newtondynamics.com/}) backends, within the repository are also experimental Open Dynamics Engine (\url{http://www.ode.org/}) and PhysX (\url{http://developer.nvidia.com/object/physx.html}, backends.}}
\item{Ragdoll scene node (also called 'online animation', 'articulated character')}
\item{Physics tool scene nodes}
\item{Physics tool scene node modifiers: Normally you add physics to a scene node by just adding a physics scene node modifier to it}
\end{itemize}




\section{PLEngine}


\begin{itemize}
\item{High-level application classes}
\item{Comfortable picking features}
\item{Screenshot tool class}
\end{itemize}
