%----- Chapter: Introduction ---------------------
\chapter{Introduction}


\paragraph{Motivation}
\emph{PLCore} is the most basic library of PixelLight. In here, you'll find container implementations like linked lists, arrays up to hash maps. Strings, files, checksum, XML, log and interfaces for the general interaction with the used operation system are also available. Usually, every PixelLight basing project is using \emph{PLCore} features because they are just essential and by using it, one can avoid to implement linked lists, strings etc. multiple times within multiple projects which would just be inefficient and error prone.\footnote{When using external third party libraries, it's usually fact that they implement their own linked lists, strings and so on - this shouldn't be the case for PixelLight based projects.}

The PixelLight core component also provides more advanced features that are usually quite useful for all types of projects. As the project name \emph{PLCore} implies, it's the foundation of the PixelLight framework which is used for the development of complete applications. RTTI, plugins, the application framework.


\paragraph{Overview}
The heart of PLCore is the RTTI which is completely basing on C++ templates and additionally offers some macros to enhance the readability and usability. By using C++ templates, the system is type safe. The RTTI is using several fundamental components like variables, functions, events and event handler. This components, the RTTI is composed of, can be subdivided into three levels as summarized within table~\ref{Table:FundamentalRTTIComponents}.
\begin{table}[htb]
	\centering
	\begin{tabular}{|l||l|l|l|}
		\hline
		Components & 1. Virtual Base Class & 2. Typed Variant & 3. Inside Object\\
		\hline
		\hline
		Variable & DynVar & Var\textless type\textgreater & Attribute\textless type\textgreater\\
		\hline
		Function & DynFunc & Func\textless type\textgreater & Method\textless type\textgreater\\
		\hline
		Event & DynEvent & Event\textless type\textgreater & Signal\textless type\textgreater\\
		\hline
		EventHandler & DynEventHandler & EventHandler\textless type\textgreater & Slot\textless type\textgreater\\
		\hline
	\end{tabular} 
	\caption{Fundamental RTTI components and their levels}
	\label{Table:FundamentalRTTIComponents}
\end{table}


\paragraph{'this' Used In Constructor Initializer List}
When you try to compile examples shown within this document, you will probably get a compiler warning like \begin{quote}warning C4355: 'this' : used in base member initializer list\end{quote}. This warning occurs because within the initializer-list, the current object denoted by \emph{this} may not be fully initialized, yet. In here, you can just ignore this warning because internally the given pointer is just stored but not yet accessed - so, there is no immediate danger in using \emph{this}. Source~code~\ref{Code:DisableThisInitializerListWarning} shows how this warning can be disabled within \emph{Microsoft Visual Studio 2010}.
\begin{lstlisting}[float=htb,label=Code:DisableThisInitializerListWarning,caption={Disable warning when using \emph{this} within the initializer list}]
// Disable warning
PL_WARNING_PUSH
PL_WARNING_DISABLE(4355) // "'this' : used in base member initializer list"

	// Default constructor
	MyClass::MyClass() :
		MyInt(this),
		MyFloat(this)
	{
	}

// Reset to previous warning settings
PL_WARNING_POP
\end{lstlisting}
While doing this for each case in a controlled way would be the clean way - it would also mean that, for each class, there would be more typing work and the code wouldn't look that compact any more. Therefore, we decided to disable this one warning by default within \emph{PLCoreWindows.h} - usually we don't do that because warnings are in general there for a good and helpful reason, but in this case the alternative wouldn't be better.




\section{External Dependences}
\emph{PLCore} is using some open source third party libraries which are static linked. As a result, the resulting \emph{PLCore} library has no additional external shared library dependencies.


\paragraph{zlib}
\begin{itemize}
\item ZIP library
\item \emph{zlib} license
\item Version \emph{1.2.3}
\item Used by the zip file classes
\item Downloaded from \url{http://www.zlib.org/}
\end{itemize}


\paragraph{PCRE}
\begin{itemize}
\item Regular expression library
\item \emph{BSD} license
\item Version \emph{8.10}
\item Used by the \emph{RegEx} class
\item Downloaded from \emph{http://www.pcre.org/}
\end{itemize}
