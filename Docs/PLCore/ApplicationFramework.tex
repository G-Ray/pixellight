\chapter{Application Framework}
The main purpose of the application framework is to make your life as an application programmer as comfortable as possible. There are may tasks like initialization, de-initialization, input handling and so on that are equal for multiple applications - so it wouldn't be efficient to do the same work over and over again within multiple applications. Instead, derive your application from a given application base class that are closest to the requirements of our new application and just overwrite required virtual methods if there's something special you need to do.

Source~code~\ref{Code:SimpleApplication} shows an extreme usage case of the application framework: An existing application class is just instanced and you have a running application at once!
\begin{lstlisting}[label=Code:SimpleApplication,caption={Simple application using the application framework}]
// Includes
#include <PLCore/Main.h>
#include <PLCore/ModuleMain.h>
#include <PLEngine/Application/RenderApplication.h>

// Namespace
using namespace PLCore;
using namespace PLEngine;

// Module definition
pl_module("MyApplication")
	pl_module_vendor("Copyright (C) 2002-2011 by Me")
	pl_module_license("LGPL")
	pl_module_description("Description of my application")
	pl_module_version("1.0")
pl_module_end

// Program entry point
int PLMain(const String &sFilename, const Array<String> &lstArguments)
{
	// Create an instance (on the stack) of the default
	// render application class
	RenderApplication cApplication;

	// Let's rock'n'roll!
	return cApplication.Run(sFilename, lstArguments);
}
\end{lstlisting}
The example application results in a window in which you can see the PixelLight logo and a text.
