\chapter{User Defined RTTI Data Type}
\label{Appendix:UserDefinedRTTIDataType}
Although the RTTI comes with many standard RTTI data types, additional data types can be defined as well. In here, we create a user defined data type to show how this is done. Please note that this data type is not totally serious and we think it's possible to write applications without the use of this data type. We implement a \emph{troll} data type which is able to remember a complete number, as long as it's 0, 1 or 2 - in short: a completely useless example data type. The data type is defined within the source~codes~\ref{Code:UserDefinedRTTIDataTypeHeader}, \ref{Code:UserDefinedRTTIDataTypeInline} and \ref{Code:UserDefinedRTTIDataTypeSource}.

The new RTTI data type can then be used like all the other data types as shown within source~code~\ref{Code:UserDefinedRTTIDataTypeClassDefinition}.
\begin{lstlisting}[label=Code:UserDefinedRTTIDataTypeClassDefinition,caption={RTTI class using a user defined RTTI data type}]
// Class definition of MyClass
#include <PLCore/Base/Object.h>
class MyClass : public PLCore::Object {

	// RTTI interface
	pl_class(MY_RTTI_EXPORT, MyClass, "", "PLCore::Object", "Description")
		pl_constructor_0(MyConstructor, "Default constructor", "")
		pl_attribute(MyTroll, TrollType, TrollType(1), ReadWrite, DirectValue, "A troll, don't feed him!", "")
	pl_class_end

	// Default constructor
	public:
		MyClass() : MyTroll(this) {}

};

// MyClass RTTI implementation (not done within headers)
pl_implement_class(MyClass)
\end{lstlisting}
The RTTI class containing a user defined RTTI data type can than be used as seen within source~code~\ref{Code:UserDefinedRTTIDataTypeClassDefinition}.
\begin{lstlisting}[label=Code:UserDefinedRTTIDataTypeClassUsage,caption={Using a RTTI class containing a user defined RTTI data type}]
// Get RTTI class description
const PLCore::Class *pMyClass = PLCore::ClassManager::GetInstance()->GetClass("MyClass");
if (pMyClass != nullptr) {
	// Create an instance of the RTTI class
	MyClass *pMyObject = (MyClass*)pMyClass->Create();
	if (pMyObject != nullptr) {
		// Ask the troll about the number he should currently
		// remember -> sNumber is now "one"
		PLGeneral::String sNumber = pMyObject->MyTroll.GetString();

		// Give the smart troll a new number to remember
		pMyObject->MyTroll.SetInt(42);

		// Ask the troll about the number he should currently
		// remember -> sNumber is now "many"... because the troll
		// is only able to remember a number as long as it's 0, 1
		// or 2, he has better things do to than remembering huge
		// numbers...
		sNumber = pMyObject->MyTroll.GetString();

		// Cleanup
		delete pMyObject;
	}
}
\end{lstlisting}


\paragraph{Header Of The User Defined RTTI Data Type}
\begin{lstlisting}[label=Code:UserDefinedRTTIDataTypeHeader,caption={Header of the user defined RTTI data type}]
#pragma once

// Includes
#include <PLCore/Base/Type/Type.h>

// Classes
class TrollType {

	// Public functions
	public:
		TrollType();
		TrollType(int nValue);
		TrollType(const TrollType &cOther);
		~TrollType();
		TrollType &operator =(const TrollType &cOther);
		bool operator ==(const TrollType &cOther) const;
		int GetValue() const;
		void SetValue(int nValue);
		PLGeneral::String GetComment() const;

	// Private data
	private:
		int m_nValue;

};

// Include type implementation
#include "TrollType.inl"
\end{lstlisting}


\paragraph{Inline Of The User Defined RTTI Data Type}
\begin{lstlisting}[label=Code:UserDefinedRTTIDataTypeInline,caption={Inline of the user defined RTTI data type}]
#pragma once

// Namespace
namespace PLCore {

// Classes
template <>
class Type<TrollType> {

	// Public static type information
	public:
		typedef TrollType _Type;
		typedef TrollType _StorageType;
		static const int TypeID = 100101;
		static PLGeneral::String GetTypeName() {
			return "troll";
		}
		static TrollType ConvertFromVar(const DynVar *pVar) {
			TrollType cTroll;
			cTroll.SetValue(pVar->GetInt());
			return cTroll;
		}
		static bool ConvertToBool(const TrollType &cTroll) {
			return (bool)(cTroll.GetValue() != 0);
		}
		static TrollType ConvertFromBool(bool bValue) {
			TrollType cTroll;
			cTroll.SetValue((int)bValue);
			return cTroll;
		}
		static int ConvertToInt(const TrollType &cTroll) {
			return cTroll.GetValue();
		}
		static TrollType ConvertFromInt(int nValue) {
			TrollType cTroll;
			cTroll.SetValue(nValue);
			return cTroll;
		}
		static PLGeneral::int8 ConvertToInt8(const TrollType &cTroll) {
			return (PLGeneral::int8)ConvertToInt(cTroll);
		}
		static TrollType ConvertFromInt8(PLGeneral::int8 nValue) {
			return ConvertFromInt((int)nValue);
		}
		static PLGeneral::int16 ConvertToInt16(const TrollType &cTroll) {
			return (PLGeneral::int16)ConvertToInt(cTroll);
		}
		static TrollType ConvertFromInt16(PLGeneral::int16 nValue) {
			return ConvertFromInt((int)nValue);
		}
		static PLGeneral::int32 ConvertToInt32(const TrollType &cTroll) {
			return (PLGeneral::int32)ConvertToInt(cTroll);
		}
		static TrollType ConvertFromInt32(PLGeneral::int32 nValue) {
			return ConvertFromInt((int)nValue);
		}
		static PLGeneral::int64 ConvertToInt64(const TrollType &cTroll)
		{
			return (PLGeneral::int64)ConvertToInt(cTroll);
		}
		static TrollType ConvertFromInt64(PLGeneral::int64 nValue)
		{
			return ConvertFromInt((int)nValue);
		}
		static PLGeneral::uint8 ConvertToUInt8(const TrollType &cTroll) {
			return (PLGeneral::uint8)ConvertToInt(cTroll);
		}
		static TrollType ConvertFromUInt8(PLGeneral::uint8 nValue) {
			return ConvertFromInt((int)nValue);
		}
		static PLGeneral::uint16 ConvertToUInt16(const TrollType &cTroll) {
			return (PLGeneral::uint16)ConvertToInt(cTroll);
		}
		static TrollType ConvertFromUInt16(PLGeneral::uint16 nValue) {
			return ConvertFromInt((int)nValue);
		}
		static PLGeneral::uint32 ConvertToUInt32(const TrollType &cTroll) {
			return (PLGeneral::uint32)ConvertToInt(cTroll);
		}
		static TrollType ConvertFromUInt32(PLGeneral::uint32 nValue) {
			return ConvertFromInt((int)nValue);
		}
		static PLGeneral::uint64 ConvertToUInt64(const TrollType &cTroll)
		{
			return (PLGeneral::uint64)ConvertToInt(cTroll);
		}
		static TrollType ConvertFromUInt64(PLGeneral::uint64 nValue)
		{
			return ConvertFromInt((int)nValue);
		}
		static PLGeneral::uint_ptr ConvertToUIntPtr(const TrollType &cTroll)
		{
			return (PLGeneral::uint_ptr)ConvertToInt(cTroll);
		}
		static TrollType ConvertFromUIntPtr(PLGeneral::uint_ptr nValue)
		{
			return ConvertFromInt((int)nValue);
		}
		static float ConvertToFloat(const TrollType &cTroll) {
			return (float)cTroll.GetValue();
		}
		static TrollType ConvertFromFloat(float fValue) {
			TrollType cTroll;
			cTroll.SetValue((int)fValue);
			return cTroll;
		}
		static double ConvertToDouble(const TrollType &cTroll) {
			return (double)cTroll.GetValue();
		}
		static TrollType ConvertFromDouble(double dValue) {
			TrollType cTroll;
			cTroll.SetValue((int)dValue);
			return cTroll;
		}
		static PLGeneral::String ConvertToString(const TrollType &cTroll) {
			return cTroll.GetComment();
		}
		static TrollType ConvertFromString(const PLGeneral::String &sString) {
			TrollType cTroll;
				 if (sString.CompareNoCase("nuthin'"))	cTroll.SetValue(0);
			else if (sString.CompareNoCase("one"))		cTroll.SetValue(1);
			else if (sString.CompareNoCase("two"))		cTroll.SetValue(2);
			else if (sString.CompareNoCase("many"))		cTroll.SetValue(3);
			else										cTroll.SetValue(-1);
			return cTroll;
		}
		static TrollType &ConvertRealToStorage(TrollType &cValue) {
			return cValue;
		}
		static TrollType &ConvertStorageToReal(TrollType &cValue) {
			return cValue;
		}

};

// Namespace
} // PLCore
\end{lstlisting}
Please note that the declaration of new RTTI data types, in this case \emph{Type<TrollType>} must happen within the \emph{PLCore} namespace. If you don't do it this way, it may work on the \emph{Microsoft Visual Studio Compiler} but may fail on the \emph{gcc compiler for Linux}.


\paragraph{Source Of The User Defined RTTI Data Type}
\begin{lstlisting}[label=Code:UserDefinedRTTIDataTypeSource,caption={Source of the user defined RTTI data type}]
// Includes
#include "TrollType.h"

// Namespace
using namespace PLGeneral;

// Public functions
TrollType::TrollType() :
	m_nValue(0) {
}

TrollType::TrollType(int nValue) :
	m_nValue(0) {
	SetValue(nValue);
}

TrollType::TrollType(const TrollType &cOther) :
	m_nValue(cOther.m_nValue) {
}

TrollType::~TrollType() {
}

TrollType &TrollType::operator =(const TrollType &cOther) {
	m_nValue = cOther.m_nValue;
	return *this;
}

bool TrollType::operator ==(const TrollType &cOther) const {
	return (m_nValue == cOther.m_nValue);
}

int TrollType::GetValue() const {
	// Return value
	return m_nValue;
}

void TrollType::SetValue(int nValue) {
	// Try to remember value ...
	if (nValue >= 0 && nValue <=2)	m_nValue = nValue;
	else if (nValue > 2)			m_nValue =  3;
	else							m_nValue = -1;
}

String TrollType::GetComment() const {
	// Return comment about current value
		 if (m_nValue == 0)	return "nuthin'";
	else if (m_nValue == 1)	return "one";
	else if (m_nValue == 2)	return "two";
	else if (m_nValue  > 0)	return "many";
	else					return "wuz?";
}
\end{lstlisting}
