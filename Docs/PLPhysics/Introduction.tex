\chapter{Introduction}


\paragraph{Motivation}
PixelLight itself has NO own physics implementation nor even fixed build in physics support within the scene graph itself. Such a physics engine is an own complex task - and at the moment the PixelLight team has no possibility (lack of time and man-power) to write a complete own physics solution.

But there are ton's of physics libraries out there - many of them free or even open source and the most modern even with hardware support.

As mentioned above, PixelLight does not come with native physics support within the scene graph component itself - but through the carefully considered design of the PixelLight framework such an 'hacked in'-support is unnecessary, it would even waste the sweet universal design. Because of the extreme plugin nature of PixelLight, it's no problem to add something like physics to your projects.

The PixelLight SDK itself comes with a few such physics plugins. By using this plugins it's extremely simple to add physics to your scene. In fact, this plugins only have one scene node container for the physics world/simulation and a few scene node modifier. Create a scene container using such a physics world scene node container class and add some scene nodes into it. For nodes which should have physical behaviour just add a physics body modifier to the node. If a body has no mass it's considered to be static... et voil�.

To connect two physics bodies you can add a joint modifier to one of the bodies and 'hang in' the second one into the joint. For more advanced physics control you have to use the functions of the physics API of your choice directly - we decided against wrapping all these functions. This would be to much work and to less advanced because the physics API differ at many points a lot of. It's impossible to create ONE universal interface for all physics API's wrapping all available features.

You can use multiple physics API's within your project at the same time, but this isn't recommended and doesn't make much sense - this different simulations also can't interact with each other. So, for your project, you normally have to choose ONE physics API and use it for the hole project. Changing during development to another physics API wouldn't be that easy. You can extent this PixelLight physics plugins by self (not recommended) or add a plugin for another physics API if required.




\section{External Dependences}
\emph{PLPhysics} depends on the \textbf{PLScene} library, and therefore an all other libraries \textbf{PLScene} depends on.
