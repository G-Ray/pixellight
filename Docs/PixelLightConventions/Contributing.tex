\chapter{Contributing}
Sorry in advance for the following quite long introduction. Don't take everything to serious, it should just point out what can go wrong within a free open-source project and hints how this can be avoided. PixelLight is in development since the year 2002 and experience tells that it's required to mention a few things, first. The intent of this chapter is to make the collaboration with and within the PixelLight developer team more comfortable, relaxed and to reduce time and energy consuming pointless detail discussions to a minimum. We should spend most of our time for developing, not arguing. It should be fun to work on PixelLight, not frustrating due to fighting inside the team. Please note that we are no lawyers nor do we have money to spend on lawyers in order to check this text. In case you love to sue other people, please stay away from this free open-source project. There's no money to get out of it.


\paragraph{Tiny Contributions}
We hope that a lot of people will provide patches and small contributions, but we hope you understand that, for organisation purpose, we can't accept every tiny contribution. As in every open-source project you will be mentioned in the repository as contributor but it's not possible to list everybody within the \emph{AUTHORS}-file. Nevertheless we will thank you very much for your efforts to make Pixellight a better and more stable software. For sending bug reports or providing a prebuild external package we are really thankfull and everyone who is using Pixellight will profite from this helpfull contribution.


\paragraph{Warning about Convention}
Before you contribute something please note that we take coding conventions and proper behaviour within the development team seriously. In order to mange a big project like PixelLight, rules, discipline, respecting each other and especially structure and order are important. That's why this document exists in the first place\footnote{... or maybe because we German guys are over exact by nature...}. If the laws of the game are known, one can focus on the development itself without, or at least reduced constant and time consuming discussions about for example code style or whether or not to use a certain C++ feature. Additionally, the team members are frequently making code reviews. This means that code you contribute may be constantly changed or even completely removed if it's no longer required or there are other serious reasons for doing so. We will also change pointless and stupid stuff like spaces and tabs, moving around code, adding nonsense comments for code were it should be clear to everyone what it's doing and why. In case you're hating it when other developers are touching your code, or you demand that before someone in the team changes anything should first have to contact you - maybe you should think again about contributing to this project. The development of a technology like PixelLight is an iterative and evolutionary process in which we already refactored or even dropped code we spend a lot of time in writing it for the greater good. So, even if this sounds extreme, we don't accept contributions blocking the further development of the technology. Sorry. In case you're open minded, willing to share and learn - please read on.


\paragraph{Warning about Copyright and Patents}
We take copyright and patents seriously in order to avoid conflicts. In case we become aware that you contributed something which is not from you and you haven't mentioned the source which has to be compatible with PixelLights license, the contribution will be removed. If you follow the rules applying to scientific work, there shouldn't be any problems. Any contribution will become part of PixelLight, meaning that it also will be released under the same license. If the PixelLight licence is changed for example to a less restrictive licence at some point of time, you agree that this is ok for you.


\paragraph{Errors and Bugs}
We expect that the contribution is free of any errors... just kidding. As you might now, it's impossible to prove that, at least a little bit longer code, is free of any errors or undesired side effects. But we expect that you're taking responsible measurements and best practices in order to at least reduce the number of errors which can be avoided. If possible, do also provide unit tests so we can ensure that the quality stays high over the time by performing automatic unit tests. If we notice any errors, breaking stuff and so on no one will yell at you because something like this happens even to the most experienced developer. In case we're noticing the same error over and over again, we might discus how we can avoid, or at least reduce it in the future to keep the productivity at a decent level. But to be fair, you also have to accept that the other team members are not perfect and will make errors on a regular basis. We have to work together to keep up the quality as high as possible.


\paragraph{Behaviour}
Be as polite as possible to each other. Critics is always good in order go get improvement, but do never get personal or take something personal. Be careful with something like jokes, because a person from another culture may totally misunderstand it. Even persons from one and the same culture may think totally different, take this into account. Experience shows that a smooth teamwork is really hard to archive, but please, show your best side.


\paragraph{Language}
Do only write in English. English isn't the native language of the project founders, it's German and that's the only reason why the previous diary entries are in German. To be able to work on a global scale with an international team, everything must be in English so that everyone has at least a real chance to be able to understand it without using an automatic translator.


\paragraph{Documentation and Diaries}
Source code has to be documented. Thrown in undocumented source code is worthless. Additionally we're writing diaries right from the beginning of the project in the year 2002. This makes it possible to figure out who has written certain parts, background information and thoughts can be followed as well. The diaries are also a security measurement in order to detect frequent change forth and change back of certain parts over the years. Although it appears that this diary approach is not standard in software development, when working on PixelLight it has several times proven to be quite useful. As a result it's required to add diary entries describing what you've done, why, helpful insight into your thoughts and everything else that helps to understand and maintain your work. You don't need to write a complete book each time, depending on the made change keywords might also be fine. In general, writing more detailed diary entries has also proven to be a quality enhancement. While explaining what you've done, from time to time one's noticing that there's something totally wrong with the made change. So, we motivate all team members to try to write at least a little bit more detailed diary entries.


\paragraph{Trying to take over PixelLight}
Sorry to mention this point at all. The team will not accept over dominant persons trying to take over the control of PixelLight and constantly telling us what's best and how things should be handled the right way. Even if it sounds stupid and totally selfishness, a project like PixelLight requires a certain hierarchy within the team in order to get progress and quality. Everything else would end in anarchy, poor results and endles discussions without any sense. In extreme it would lead to the dead of the project due to constant fighting inside the team. We respect the freedom of speech and are trying to give everybody the chance to get more control. But this has to be done by constant contributing and showing the rest of the team that you really know what you're talking about. If there are open questions what's the best for the project in general, the order of the listed persons should give a hint who has the last word when there's a total disagreement inside the team. Important stuff, like changing the licence PixelLight is released under, is suspect to the project founders working on the project since the beginning. No one else. Right now, on the PixelLight homepage at \url{http:\\www.pixellight.org} only the founders are listed to know which are the contact persons. They started the project and worked for it on their own in private for nearly a decade.




\section{Welcome}
In case you're still here reading this and have not closed the document either shocked or in total disagreement about what you've read in the introduction of this chapter... we're always totally happy to get new contributors. The PixelLight technology covers a wide area of topics and it's always wonderful to have developers willing to focus to improve certain parts of the technology. In general, it will make the organisation much easier if you concentrate on a certain part. Managing persons working on everything at one and the same time is hard to manage and also requires this persons to have a deep inside and understanding of the complete project.

TODO: CLA

TODO: Commit checklist *review own commit before sending it*