\chapter{General}




\section{Compiler and Project Settings}
Whenever you create a project using the PixelLight framework you have to set the used runtime library to \emph{Multithreaded \ac{DLL}}\footnote{Multithreaded or multithreaded debug, for the release and debug versions, respectively}, else there may occur undefined errors! In \emph{Microsoft Visual Studio} you can find this properties in the project settings, C++/Code generation menu. Further you have to add the folders were the \ac{API} includes and libraries can be found. In \emph{Microsoft Visual Studio} you will find this options under: \emph{Menu -> Tools -> Options -> Projects and Solutions -> VC++ Directories}, if you want to set global paths. In general we recommend to setup this within the project dependent settings instead within the global ones. Note that sometimes the order is important, therefore if problems appear place the PixelLight directories over other directories. You also have to add the libraries into your project. (e.g. \emph{PLCore.lib}) Note that within the PixelLight \ac{SDK} only relative paths are used. So you can for instance compile the sample applications at once without any additional manual changes.


\paragraph{Character Set}
Set the used character set to \emph{Unicode} and not to \emph{Multi-Byte} which is normally the default setting.


\paragraph{'this' used in Constructor Initializer List}
When you try to compile examples shown within this document, you will probably get a compiler warning like \begin{quote}warning C4355: 'this' : used in base member initializer list\end{quote}. This warning occurs because within the initializer-list, the current object denoted by \emph{this} may not be fully initialized, yet. In here, you can just ignore this warning because internally the given pointer is just stored but not yet accessed - so, there is no immediate danger in using \emph{this}. Source~code~\ref{Code:DisableThisInitializerListWarning} shows how this warning can be disabled within \emph{Microsoft Visual Studio 2010}.
\begin{lstlisting}[float=htb,label=Code:DisableThisInitializerListWarning,caption={Disable warning when using \emph{this} within the initializer list}]
// Disable warning
PL_WARNING_PUSH
PL_WARNING_DISABLE(4355) // "'this' : used in base member initializer list"

	// Default constructor
	MyClass::MyClass() :
		MyInt(this),
		MyFloat(this)
	{
	}

// Reset to previous warning settings
PL_WARNING_POP
\end{lstlisting}
While doing this for each case in a controlled way would be the clean way - it would also mean that, for each class, there would be more typing work and the code wouldn't look that compact any more. Therefore, we decided to disable this one warning by default within \emph{PLCoreWindows.h} - usually we don't do that because warnings are in general there for a good and helpful reason, but in this case the alternative wouldn't be better.




\section{Flags}
Often flags are used within the PixelLight framework to set up different properties. This is quite efficient because all those properties are stored in one single variable. Normally a \SI{32}{\bit} variable is used where you can manage up to 32 flags. To manipulate the flags you have to use the binary operations:
\begin{tabular}{ll}
Set flags:&
  \emph{nFlags = 2 | 8;}\\
Remove a flag:&
  \emph{Flags \&= ~2;}\\
if a flag is set:&
  \emph{if (nFlags \& 8)}\\
\end{tabular}
It's also possible to store such flags in a string like \emph{1|8|64}. For this purpose the function \emph{PLCore::ParseTools::GetStringFromFlags()} is supplied which converts flags into a string and \emph{PLCore::ParseTools::GetFlagsFromString()} which will convert a string into normal flags. When working with flags we highly recommend to use definitions like \emph{MyAmazingFlag} instead of \emph{8} - your brain will thank you for that.




\section{Program Entry Point}
To write applications which are totally platform independent one have to write different program entry points (\emph{WinMain}/\emph{main}) for the different platforms. PixelLight offers you a macro called \emph{PLMain} so you don't have to brother your self every time with this when writing a new program.

\begin{lstlisting}[caption=Program entry point example]
//[-------------------------------------------------------]
//[ Includes                                              ]
//[-------------------------------------------------------]
#include <PLCore/Main.h>


//[-------------------------------------------------------]
//[ Namespace                                             ]
//[-------------------------------------------------------]
using namespace PLCore;


//[-------------------------------------------------------]
//[ Program entry point                                   ]
//[-------------------------------------------------------]
int PLMain(const String &sExecutableFilename, const Array<String> &lstArguments)
{
  // ... add some astounding code here...
  return 0;
}
\end{lstlisting}

\begin{itemize}
\item{This function returns \emph{0} (normally the standard error code of operation systems) if all went fine, else another number.}
\item{\emph{sExecutableFilename} is the absolute application filename.}
\item{\emph{lstArguments} is a list of arguments given to the application - because the string class is used, you don't brother yourself with Unicode, ASCII and so on. There's a class called \emph{CommandLine} you can use to parse the given arguments.}

If an application is started, it's current directory may differ from the one the executable is in - for instance if it's started using a link. In the most cases this is undesired because the data the application is using is relative to the current directory - in this cases, the directory the executable is in. You can add the following lines of code into you \emph{PLMain()} function to change the current directory to the one the application executable is in:

\begin{lstlisting}[caption=Set current directory example]
// Use the executable directory as the current directory
PLCore::System::GetInstance()->SetCurrentDir(PLCore::Url(sExecutableFilename).CutFilename());
\end{lstlisting}
\end{itemize}

Where \emph{sExecutableFilename} is the absolute application filename provided by \emph{PLMain()}.




\section{Version and Memory Management}
Because your project is normally split up into several sub-projects like scene node plugins you have to ensure that all this projects are always up to date to avoid mysterious behaviour or even illogical crash's!

In general you should never give out debug versions of your projects because they are larger, slower and sometimes there may be undesired differences in runtime behaviour between release and debug mode.\footnote{Like everything runs quite fine within the debug version, but crash's within the release version}

For \emph{Microsoft Windows}, there's a useful free tool called \emph{Dependency Walker}\footnote{\emph{Dependency Walker} can be downloaded from \url{http://www.dependencywalker.com/}} which will help you when \ac{DLL} problems and so on occur. This tool will show you the dependencies of applications and \ac{DLL}s - so have a look on it! For Linux, \emph{objdump} and \emph{ldd} will do this job quite fine.

If you don't use the PixelLight memory manager, all your executables and dynamic linked libraries should be compiled in the same mode to avoid troubles during e.g. debugging. If you don't do any special and using \emph{PLMain} as program entry point, the PixelLight memory manager is used automatically by default and you can mix debug/release builds without any problems because any memory operation is done within \emph{PLCore}. This is the strongly recommended way to go. It's also recommended that you never use \emph{old C style}-functions like \emph{malloc}/\emph{free} to allocate/deallocate memory because the PixelLight memory manager has no chance to get informed about the allocations.

You can use for instance \emph{memcpy}, but using for example \emph{PLCore::MemoryManager::Copy()} is the more elegant way to go because it's more readable and all memory handling is going through the memory manager and you don't have to care about whether the implementation may have a special behaviour on different systems.
