\chapter{PLInput}


\paragraph{Motivation}
The PixelLight input component, or short \emph{PLInput}, provides access and control over several input devices. Although the purpose of this project is to have an input library that perfectly integrates into the PixelLight framework, \emph{PLInput} can be used without other PixelLight components like the rendering system as well.




\section{External Dependences}
The core of PLInput depends on the \textbf{PLCore} and \textbf{PLMath} libraries. PLInput is using some platform dependent third party libraries, but usually, the resulting binary of PLInput library is stand alone and does not force you to deliver additional shared libraries, too.


\paragraph{Microsoft Windows}
When compiling for \emph{Microsoft Windows}, there are no additional external dependencies - the few required \emph{hid.dll} system shared library functions are loaded dynamically.




\section{Important Terminology}
Within the PixelLight input system, there's some some important terminology we first need to define before we can go into details.


\paragraph{Controller, Controls, Buttons, Axis}
A \emph{controller} represents an input device, which can either be a real device like e.g. a mouse or joystick, or a virtual device that is used to map real input devices to virtual axes and keys. A controller consists of a list of \emph{controls}, e.g. \emph{buttons} or \emph{axes} and provides methods to obtain the status. While a button can be pressed, not not pressed, an axis has quantized values within a given interval.


\paragraph{Absolute And Relative Axis}
Depending on the input device and axis type, the value of an axis can be \emph{absolute} or \emph{relative}. A good example for an absolute axis is the x axis of a joystick. As long as you keep the joystick pulled to the left, you get a certain value. On the other hand, a mouse is a good example for a relative axis. If you move our mouse to the left, the position difference is recognized once, and send to the system. This means that the relative axis contains time information\footnote{The position difference of the mouse since the last position test some time ago} while a absolute axis contains no time information\footnote{Ignoring the fact that the axis value can change over time, but that's not interesting in here}. It's important to keep the time information in mind when using the device controls in order to, for instance, moving around a character.




% Include the other sections of the chapter
\section{Physical Input Devices}
\label{Chapter:PhysicalInputDevices}
This section deals with the actual, physical devices an, for instance human being, is using in the physical world in order to communicate with the computer. While starting with traditional input devices like the keyboard or the mouse, we also cover classic gaming devices like the joystick/gamepad and modern gaming devices like Wii Remote\footnote{Sometimes unofficially nicknamed \emph{Wiimote}}. Especially for developers, there are input devices like a space mouse making navigation within virtual worlds quite comfortable - therefore, they are covered as well.




\subsection{Keyboard}
As seen within source~code~\ref{Code:KeyboardUsageExample}, accessing a keyboard is quite simple.
\begin{lstlisting}[float=htb,label=Code:KeyboardUsageExample,caption={Keyboard usage example}]
// Get keyboard input device
Keyboard *pKeyboard = InputManager::GetInstance()->GetKeyboard();
if (pKeyboard) {
  // Update movement vector
  if (pKeyboard->KeyUp.IsPressed())
    vMovement += vDirVector;
  if (pKeyboard->KeyDown.IsPressed())
    vMovement -= vDirVector;
}
\end{lstlisting}




\subsection{Mouse}
As seen within source~code~\ref{Code:MouseUsageExample}, accessing a mouse is quite simple.
\begin{lstlisting}[float=htb,label=Code:MouseUsageExample,caption={Mouse usage example}]
// Get mouse input device
Mouse *pMouse = InputManager::GetInstance()->GetMouse();
if (pMouse && pMouse->Left.IsPressed()) {
  // Get the current time difference
  const float fTimeDiff = Timing::GetInstance()->GetTimeDifference();

  // Get mouse movement
  const float fX = pMouse->X.GetValue() * fTimeDiff;
  const float fY = pMouse->Y.GetValue() * fTimeDiff;

  // Get a quaternion representation of the rotation delta
  Quaternion qRotInc;
  EulerAngles::ToQuaternion(float(fX * Math::DegToRad),
                            float(fY * Math::DegToRad),
                            0.0f,
                            qRotInc);

  // Set new rotation
  qNewRotation = qOldRotation * qRotInc;
}
\end{lstlisting}




\subsection{Joystick}
When talking about \emph{joysticks}, we also cover \emph{gamepads} by this term because technically, their're practically the same. Therefore, when looking over the PixelLight input device classes, one may suspect that there's no support for gamepads, but that's not true because joysticks and gamepads are both handled by the joystick class.

As seen within source~code~\ref{Code:JoystickUsageExample}, accessing a joystick is quite simple.
\begin{lstlisting}[float=htb,label=Code:JoystickUsageExample,caption={Joystick usage example}]
Joystick *pJoystick = (Joystick*)InputManager::GetInstance()->GetDevice("Joystick0");
if (pJoystick && pJoystick->GetButtons()[0]->IsPressed()) {
  // Get the current time difference
  const float fTimeDiff = Timing::GetInstance()->GetTimeDifference();

  // Get joystick axis
  const float fX = pJoystick->X.GetValue() * fTimeDiff;
  const float fY = pJoystick->Y.GetValue() * fTimeDiff;

  // Get a quaternion representation of the rotation delta
  Quaternion qRotInc;
  EulerAngles::ToQuaternion(float(fX * Math::DegToRad),
                            float(fY * Math::DegToRad),
                            0.0f,
                            qRotInc);

  // Set new rotation
  qNewRotation = qOldRotation * qRotInc;
}
\end{lstlisting}




\subsection{Wii Remote}
As seen within source~code~\ref{Code:WiiRemoteUsageExample}, accessing a Wii Remote is quite simple.
\begin{lstlisting}[float=htb,label=Code:WiiRemoteUsageExample,caption={Wii Remote usage example}]
// Get Wii Remote input device
WiiMote *pWiiMote = (WiiMote*)InputManager::GetInstance()->GetDevice("WiiMote0");
if (pWiiMote && (pWiiMote->ButtonA.IsPressed())) {
  // Get the current time difference
  const float fTimeDiff = Timing::GetInstance()->GetTimeDifference();

  // Get orientation
  const float fX = float(pWiiMote->OrientX.GetValue() * fTimeDiff * Math::DegToRad);
  const float fY = float(pWiiMote->OrientY.GetValue() * fTimeDiff * Math::DegToRad);
  const float fZ = float(pWiiMote->OrientZ.GetValue() * fTimeDiff * Math::DegToRad);

  // Get a quaternion representation of the rotation delta
  Quaternion qRotInc;
  EulerAngles::ToQuaternion(fX, fY, fZ, qRotInc);

  // Set new rotation
  qNewRotation = qOldRotation * qRotInc;
}
\end{lstlisting}




\subsection{Space mouse}
Several space mouses from \emph{3DConnexion} are supported: \emph{SpaceMousePlus}, \emph{SpaceTraveler}, \emph{SpaceBall}, \emph{SpacePilot} and \emph{SpaceExplorer}.

Controlling an orbiting or free camera by using a space mouse is quite comfortable and is replacing mouse AND keyboard for this purpose.

As seen within source~code~\ref{Code:SpaceMouseUsageExample}, accessing a space mouse is quite simple.
\begin{lstlisting}[float=htb,label=Code:SpaceMouseUsageExample,caption={Space mouse usage example}]
// Get SpaceMouse device
InputManager *pInputManager = InputManager::GetInstance();
SpaceMouse *pSpaceMouse = (SpaceMouse*)pInputManager->GetDevice("SpaceMouse0");
if (pSpaceMouse) {
  // Get the current time difference
  const float fTimeDiff = Timing::GetInstance()->GetTimeDifference();

  // Get orientation
  const float fX = float(pSpaceMouse->RotX.GetValue() * fTimeDiff * Math::DegToRad);
  const float fY = float(pSpaceMouse->RotY.GetValue() * fTimeDiff * Math::DegToRad);
  const float fZ = float(pSpaceMouse->RotZ.GetValue() * fTimeDiff * Math::DegToRad);

  // Get a quaternion representation of the rotation delta
  Quaternion qRotInc;
  EulerAngles::ToQuaternion(fX, fY, fZ, qRotInc);

  // Set new rotation
  qNewRotation = qOldRotation * qRotInc;

  // Get translation
  const Vector3 vTrans(pSpaceMouse->TransX.GetValue() * fTimeDiff, pSpaceMouse->TransY.GetValue() * fTimeDiff, pSpaceMouse->TransZ.GetValue() * fTimeDiff);

  // Set new position
  vNewPos = vOldPos + vTrans;
}
\end{lstlisting}

\cleardoublepage
\section{Virtual Input Controllers}
Using physical input devices directly, as seen within chapter~\ref{Chapter:PhysicalInputDevices}, is quite comfortable and also an obvious way to deal with input devices - maybe that's the reason why most input systems out there just support this approach. While someone may be happy with this \emph{direct} solution for some time, there will probably come the point were the need arises to support multiple input devices to, for example, moving around a camera by the keyboard, the mouse, the joystick, the space mouse or by using all devices at once in a combination. A trivial solution for multi-input-devices-support would be, to just build in the support right at the place were it's required like
\begin{quote}if (WiiRemote) move with Wii Remote, if (Mouse) move with mouse, if (Joystick) move with joystick, else () bad luck my friend\end{quote}
Every time, support for another input device is required, the code is extended to cover this new input device as well. On the second thought, it may come into ones mind that this is not the perfect solution for the multi-input-devices-support - especially when the demand arises that the user should be able to configure the input controls dynamically... and by this we don't mean to just assign another keyboard key to an action, but to be able to assign a control of a totally different input device to this action, as well!

By using \emph{virtual input controllers} instead of input devices directly, the multi-input-devices-support and dynamic-user-configuration demands can be fulfilled at one and the same time.

\cleardoublepage
