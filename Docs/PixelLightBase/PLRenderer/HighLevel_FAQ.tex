%----- Chapter: High level: FAQ ------------------
\chapter{High level: \ac{FAQ}}


[TODO] Update!



%----- Section: Overview -------------------------
\section{Overview}
This \ac{FAQ} answers general questions on the fly without mentioning each detail - for more information
about several topics have a look at the detailed material documentation above.\\




%----- Section: FAQ ------------------------------
\section{\ac{FAQ}}


\emph{How can I setup additional per texture settings like if this texture can be compressed or not?}\\
This can be set in the plt-file for each texture. Example: If your texture is 'MyTexture.bmp' the
corresponding plt file is 'MyTexture.plt' which is in the same directory. A plt file is a normal
text file were you can write in several special per texture settings.\\


\emph{Which texture formats PixelLight can load in?}\\
bmp, cut, ico, jpg, pcx, tif, png, tga, dds, psd, hdr, exr\\


\emph{Which texture formats PixelLight can save?}\\
bmp, jpg, pcx, png, pnm, raw, sgi, tga, tif, pal, hdr, exr\\


\emph{How can I foce a texture to never use texture compression?}\\
This can be set in the plt-file in the general block using Compression=0 to define that this
texture shoult never use texture compression. Example:\\

\begin{lstlisting}[caption=plt-file disable texture compression]
<?xml version="1.0"?>
<Texture>
	<General Compression="0" />
</Texture>
\end{lstlisting}

... to disable the texture compression for this texture. Note that texture compression will
increase the performance and save memory!\\


\emph{How to use textures which are non power of two for e.g. ingame menus?}\\
Non power of two textures are a special type of texture which is also called 'rectangle texture'
because the graphic cards are optimized on power of two textures you should only use such
textures in special situations like for menus. In the plt file you have to setup:\\

\begin{lstlisting}[caption=plt-file rectangle texture]
<?xml version="1.0"?>
<Texture>
	<General Rectangle="1" />
</Texture>
\end{lstlisting}

... to mark this texture as rectangle texture.\\


\emph{What is the maximum possible texture size?}\\
The maximum possible texture size depends und the available hardware and the texture type.
(2D, 3D, rectangle, cube map etc.)
A standart maximum texture size of 2D textures today is 2048x2048, but the most cards can use
textures with a size up to 4096x4096. If a texture is to large for the available hardware the
texture is scaled down automatically. Example: A GeForce4 Ti4200 has a maximum texture size of
4096x4096 and 4 texture layers. An Radion 9600 Mobile a maximum texture size of 2048x2048 and
8 texture layers.\\


\emph{How to use cube maps?}\\
Cube maps are a special type of textures. In fact a cube map is build up of 6 different textures -
one for each cube side. The cub map ifself is used like the other normal textures. Whether a
texture is a cube map or not is defined in the texture plt-file:\\

\begin{lstlisting}[caption=plt-file cube map]
<?xml version="1.0"?>
<Texture>
	<General CubeMap="1" />
</Texture>
\end{lstlisting}

The other 5 textures will be loaded automatically using texture filename + ID. For example if
'Cube.jpg' has a plt-file with says that this is a cube map the other texture files 'Cube1.jpg',
'Cube2.jpg'... 'Cube5.jpg' will be loaded automatically. The order is x-positive (0), x-negative (1),
y-positive (2), y-negative (3), z-positive (4), z-negative(5).\\


\emph{How to create semi-transparent textures like a grate texture you can look through certain parts?}\\
First at all you need a alpha mask which says whats transparent and how much.
There are two different ways to do this. For instance you can put an alpha channel into your
texture (rgba) - for instance using a tga texture. Or you can define a color key for the texture
using the plt format by writing:\\

\begin{lstlisting}[caption=plt-file color key]
<?xml version="1.0"?>
<Texture>
	<ColorKey R="0" G="0" B="0" Tolerance="0" />
</Texture>
\end{lstlisting}

... were in this example all texel which are black (0/0/0) should be transparent. Internally a
alpha channel for the texture is created using the color key. We recommend to e.g. use a tga
texture to supply a alpha channel because there you have more control over it - there you can
also control how transparent a texel is.\\
    
When using the texture in a material you have to activate the alpha test for the material pass
using the semi-transparent texture. Example:\\

\begin{lstlisting}[caption=Semi-transparent material]
<?xml version="1.0"?>
<Material>
	<Technique Name="Semi transparent">
		<Pass Name="Pass0">
			<RenderStates AlphaTestEnable="1" />
			<Layer>
				<Texture>MyTexture.tga</Texture>
			</Layer>
		</Pass>
	</Technique>
</Material>
\end{lstlisting}

In the AlphaTest block you can also setup the used alpha test function which is by default 'greaterequal' 
which means that pass if the incomming alpha value is greater than or equal to the reference value.
(default 0.5 but can also be set within this block)\\


\emph{How to create texture animations like rotating textures or textures with several animation frames?}\\
A texture animation is defined in a tani file which is a normal text file. You define a texture
animation in a tani file and then you have to use it like all other textures (bmp etc.). PixelLight
will playback etc. your animation automatically when the texture is used. Note that using a material
you can create amazing effects when using different animated texture layers. :)\\


\emph{What are materials and how to use them?}\\
A material is a powerfull surface description which controls how surfaces will look. You can define
materials in mat files which simple text files. A material combines different texture layers, shaders,
render state settings etc. Because this is a more extensive topic we recommend you to read the material
documentation. Note that using material and shaders you can create nearly each effect you want to have
and you have a huge degree of freedom when implementing your material effects!\\


\emph{I HATE writing pure text, I will see at once how different settings will affect the visual material
  apperance...}\\
You are not forced to create your materials only by using a text editor. Theres a material editor
within the model editor which allows you to create and tweak your materials in a quite comfortable
way.\\


\emph{Can I use more than one texture per material?}\\
Yes. The number of texture layers is only limited by the available hardware - normally at least 2
texture layers are supported - but 4 is the standard today. Note that by using shaders you are
able to use on several graphic cards up to 16 or more different texture layers - 'fix pass' - meaning
without using a shader you can normally using less texture layers as when using shaders!\\


\emph{How to create a material with 'silhouettes'?}\\
A easy way to do this is to draw the material twice were the second time the culling is flipped and
the fill mode is set to wireframe. Example:\\

\begin{lstlisting}[caption=Silhouette material]
<?xml version="1.0"?>
<Material>
	<Technique Name="Simple silhouette">
		<Pass Name="Mesh">
			<Layer>
				<Texture>MyTexture.bmp</Texture>
			</Layer>
		</Pass>
		<Pass Name="Silhouette">
			<RenderStates FillMode="line" CullMode="cw" Lighting="0" />
			<Material Color="0.0 0.0 0 0.0" />
		</Pass>
	</Technique>
</Material>
\end{lstlisting}

The color controled by the set material color.\\


\emph{How to fix terrible z-fighing (ugly artefacts) problems?}\\
z-fighing happens when two polygons are (nearly) coplanar. To fix this precision problem you have
to 'shif' the z values of your material - this is also called setting the polygon offset which is
also known as depth bias. You can set the polygon offset in the material using:\\

\begin{lstlisting}[caption=Polygon offset]
<?xml version="1.0"?>
<Material>
	<Technique Name="Polygon offset">
		<Pass Name="Pass0">
			<RenderStates SlopeScaleDepthBias="-1" 
							DepthBias="-2" />
			<Layer>
				<Texture>MyTexture.bmp</Texture>
			</Layer>
		</Pass>
	</Technique>
</Material>
\end{lstlisting}

... were you have to play around with the factor und unit settings to find good settings for your
situation.\\


\emph{How to use vertex and/or fragment shaders?}\\
PixelLight supports cg shaders which are comfortable and easy to deal with. You can apply vertex
and/or fragment shaders to each material per pass.\\

\begin{lstlisting}[caption=Material and shaders]
<?xml version="1.0"?>
<Material>
	<Technique Name="Shaders">
		<Pass Name="Pass0">
			<VertexShader Filename="MyShader.cg" />
			<FragmentShader Filename="MyShader.cg" />
		</Pass>
	</Technique>
</Material>
\end{lstlisting}

... if the shaders will not work on the current hardware this this technique will be 'invalid'
- when using shaders you should always supply additional techniques with simpler shaders and or
techniques without any shaders to catch the worst case (fallbacks issue).\\


\emph{Can I use more than one vertex and/or fragment shader per material pass?}\\
No, thats not supported by current hardware so you have to do all within one shader at once.
But note that you can also use different passes per material were each pass is using another
shader... ;-)\\


\emph{How to create environment mapping?}\\
To create environment mapping you just have to change the texture generation mode in a material 
layer. Example:\\

\begin{lstlisting}[caption=Environment mapping]
<?xml version="1.0"?>
<Material>
	<Technique Name="Environment mapping">
		<Pass Name="Pass0">
			<Layer>
				<Texture>EnvironmentMap.jpg</Texture>
				<TextureStageStates TexGen="sphere_map" />
			</Layer>
		</Pass>
	</Technique>
</Material>
\end{lstlisting}


\emph{How to create specular highlights?}\\
The simples way would be to use environment mapping with another texture environment mode to
receive a 'highlight' effect. Example:\\

\begin{lstlisting}[caption=Specular highlights]
<?xml version="1.0"?>
<Material>
	<Technique Name="Specular highlight">
		<Pass Name="Pass0">
			<Layer>
				<Texture>Highlight.jpg</Texture>
				<TextureStageStates ColorTexEnv="add" TexGen="sphere_map" />
			</Layer>
		</Pass>
	</Technique>
</Material>
\end{lstlisting}

But there are also several other more complicated and situation dependent ways to create such
an effect like using the material provied color setting, shaders etc.\\


\emph{How to use detail maps?}\\
Normally detail maps are applied using multitexturing. The detail map is blended over another
texture to add more detail - normally this detail map is 'wrapped' meaning that the texture will
repeat if the texture coordinates pointing 'outside' the texture. Example:\\

\begin{lstlisting}[caption=Detail texture]
<?xml version="1.0"?>
<Material>
	<Technique Name="Detail texture">
		<Pass Name="Pass0">
			<Layer>
				<Texture>MyTexture.bmp</Texture>
			</Layer>
			<Layer>
				<Texture>MyDetailMap.bmp</Texture>
			</Layer>
		</Pass>
	</Technique>
</Material>
\end{lstlisting}

\emph{How to use light maps?}\\
Normally light maps are applied using multitexturing. The light map is blended over another
texture to add static lighting. Example:\\

\begin{lstlisting}[caption=Light mapping]
<?xml version="1.0"?>
<Material>
	<Technique Name="Light mapping">
		<Pass Name="Pass0">
			<Layer>
				<Texture>MyTexture.bmp</Texture>
			</Layer>
			<Layer>
				<Texture>MyLightMap.bmp</Texture>
			</Layer>
		</Pass>
	</Technique>
</Material>
\end{lstlisting}


\emph{How to use BumpMapping?}\\
BumpMapping is done through shaders and the underlying model normally has to provide several
additional data like binormals etc. There are tons of different BumpMapping techniques so its
up to you how to create this effect. Here's an example how such a BumpMapping material can look
like:\\

\begin{lstlisting}[caption=BumpMapping]
<?xml version="1.0"?>
<Material>
	<Parameter Name="mvp" Type="float4x4" Semantic="WorldViewProjection" />
	<Parameter Name="camPos" Type="float4" Semantic="EyePos" />
	<Parameter Name="lightPos" Type="float3" />
	<Parameter Name="invRad" Type="float" />
	<Parameter Name="lightColor" Type="float4" />
	<Parameter Name="Base" Type="texture" Semantic="Texture0" />
	<Parameter Name="Normal" Type="texture" Semantic="Texture1" />
	<Technique Name="Lighting">
		<Pass Name="Pass0">
			<Layer>
				<Texture>Fieldstone.jpg</Texture>
			</Layer>
			<Layer>
				<Texture>FieldstoneNormal.tga</Texture>
			</Layer>
			<VertexShader Filename="BumpMapping.cg" />
			<FragmentShader Filename="BumpMapping.cg" />
		</Pass>
	</Technique>
</Material>
\end{lstlisting}


\emph{Can I use "Cel Shading"?}\\
Yes, this is done through shaders processing a 1D texture. It's up to you to write a "cel shading"
shader - there are tons of references and cg shader examples. :)\\


\emph{How to create transparent/blended materials?}\\
Because a material can have several passes were some may be transparent but other not you have
to 'mark' whether the material is transparent or not to let the renderer not how to deal with
this material. Then in each pass you can activate blending and setup several things like the blend
mode. Normally when blending you will not write into the z buffer because the material will not
hide things behind it. How much transparent a material is is controlled through the material color.
Example:\\

\begin{lstlisting}[caption=Transparent/blended material]
<?xml version="1.0"?>
<Material>
	<General Blend="1" />
	<Technique Name="Transparent">
		<Pass Name="Pass0">
			<RenderStates ZWriteEnable="0" BlendEnable="1" 
							ScrBlendFunc="scr_alpha" DstBlendFunc="one" />
			<Layer>
				<Texture>MyTexture.bmp</Texture>
			</Layer>
		</Pass>
	</Technique>
</Material>
\end{lstlisting}
