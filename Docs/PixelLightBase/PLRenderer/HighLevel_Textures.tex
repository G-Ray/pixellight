----- Chapter: High level: Textures --------------
\chapter{High level: Textures}


[TODO] Update!


%----- Section: Overview -------------------------
\section{Overview}
Because the materials normally using textures it's good to know some hinds about it - read this
section carefully!
There are different types of textures like 1D, 2D, \hyperlink{Rectangle textures}{rectangle textures},
\hyperlink{3D textures}{3D}), \hyperlink{Cube maps}{cube map textures}) etc.




%----- Section: Texture formats ------------------
\section{Texture formats}
PixelLight can load the following texture formats:\\

bmp, cut, ico, jpg, pcx, tif, png, tga, dds, psd, hdr, exr\\

and save this formats:\\

bmp, jpg, pcx, png, pnm, raw, sgi, tga, tif, pal, hdr, exr\\

If you don't specify the texture format by the filename extension the framework will search for the
texture by adding a filename extension in the order you could see above. PixelLight has no own texture format
because this would be to complex to the graphical artists to convert the textures to a special format.
Normally textures etc. are packed in a zip file and the user will not see them.

Beside the texture files like bmp, tga, jpg etc. which store the texture data itself there are
several additional PixelLight texture description file formats. This PixelLight texture file formats are described below in
detail, here were we only want to mention them in order to let you know that they exist and how to
use them.\\
Each texture file (bmp, tga etc.) can have additional PixelLight relevant information which are defined
in a plt-file with the same texture filename but with the filename ending 'plt' e.g. 'MyTexture.bmp'
would have a 'MyTexture.plt' plt-file which is in the same directory as the owner texture. In this
additional texture file you are able to setup different fixed properties for this texture like if
the texture can be scaled, if \ac{GPU} texture compression is allowed, the color key for alpha transparency
etc. (such a color key can also be defined in a tga RGBA texture).\\
In the \hyperlink{tani}{tani-format} format you can define texture animations like sliding/rolling
texturs, texture animations through different texture frames etc. Such tani-files can be loaded by
texture handlers and materials as all other texturs, too.

It's recommended to use dds textures with precalculated mipmaps and compression whenever possible
for better loading performance.




%----- Section: PixelLight texture format (plt) --
\section{PixelLight texture format (plt)}
In the plt-format files which have the same name as the 'real' texture (bmp, jpg etc) with the texture
data itself but with the ending 'plt' you can setup some different fixed texture properties like its
color key for alpha masking, whether texture compression is allowed or not etc. This texture
configuration file is a normal text file you can edit with every text editor. Unrequired commands
can be skipped - in this case standard values will be used instead.

First here's a full plt-format example with the standard values:\\
\verbatiminput{examples/example.plt}


General texture configurations\\

\begin{tabular}{|p{2.5cm}|p{2.5cm}|p{9cm}|}
\hline
\textbf{Command} & \textbf{Default} & \textbf{Description}\\
\hline
CubeMap & 0 &
\begin{tabular}{|p{9.5cm}|}
\hline
Is this a cube map?
If yes the other 5 textures will be loaded automatically using the texture name + ID\\

\begin{tabular}{|p{3cm}|p{2.5cm}|}
\hline
\textbf{Side} & \textbf{Index}\\
\hline
positive x & -\\
negative x & 1\\
positive y & 2\\
negative y & 3\\
positive z & 4\\
negative z & 5\\
\hline
\end{tabular}

Example:\\
Sky.tga (this texture you load)\\
Sky1.tga Sky2.tga Sky3.tga Sky4.tga Sky5.tga (loaded automatically)\\
\end{tabular}\\

\hline
Mipmaps		& config & Are mipmaps allowed?\\
\hline
Compression	& config & Is texture compression allowed? (less quality but better performance) Do not use standard texture 
                       compression for normal maps, because this textures stores vector data instead of visible colors 
                       the result would be wrong. If for instance the loaded dds texture is already compressed by
                       default, this compression option is ignored because it would be pretty useless to uncompress
                       this texture before uploading to the \ac{GPU}... there's already a loss of quality.\\
\hline
Gamma		& 1.0	 & Gamma factor\\
\hline
FitLower	& ?		 & Specifies the automatic texture scale behaviour.
					   If not defined, the texture manager settings will be used. If 1
					   the next valid lower texture size is used, if 0 the next higher one.\\
\hline
Rectangle	& 0		 & If 1 there's no power of 2 limitation for the texture size. If 0, the texture will normally be
                       resized automatically to a valid dimension. If rectangle textures are not supported by the \ac{GPU},
                       this settings is ignored and the texture is resized automatically.
 \\
\hline
\end{tabular}


Texture resize settings\\

\begin{tabular}{|p{2.5cm}|p{2.5cm}|p{9cm}|}
\hline
\textbf{Command} & \textbf{Default} & \textbf{Description}\\
\hline
Active	  & 1 & Can the texture size be resized to reduce the texture quality? If the texture dimension
                is not correct and the \ac{GPU} can't use the texture without resizing it this is internally
                always set to true. Use the 'force' settings below to force a given size even if the \ac{GPU}
                can't handle it.\\
\hline
MinWidth  & 4 & The minimum allow texture width (should normally be a power of 2!)\\
\hline
MinHeight & 4 & The minimum allow texture height (should normally be a power of 2!)\\
\hline
MinDepth  & 1 & The minimum allow texture depth (should normally be a power of 2!)\\
\hline
\end{tabular}


Forces a special static texture size if not -1 (texture resizing will be disabled!).\\
Use this only in special situations because the \ac{GPU} may not be able to handle certain
sizes...

\begin{tabular}{|p{2.5cm}|p{2.5cm}|p{9cm}|}
\hline
\textbf{Command} & \textbf{Default} & \textbf{Description}\\
\hline
Width  & -1 & Forced texture width (should normally be a power of 2!)\\
\hline
Height & -1 & Forced texture height (should normally be a power of 2!)\\
\hline
Depth  & -1 & Forced texture depth (should normally be a power of 2!)\\
\hline
\end{tabular}


The color key defines the color which should be transparent.
You can also use a RGBA texture like a tga to create semi transparent textures, in that
case this RGB settings are not used! But note that in this case the material pass using this
RGBA texture in the first layer must have an active alpha test to use the alpha mask - else
you won't see any difference!\\

\begin{tabular}{|p{2.5cm}|p{2.5cm}|p{9cm}|}
\hline
\textbf{Command} & \textbf{Default} & \textbf{Description}\\
\hline
R		  & 0 & Red component (0-255)\\
\hline
G		  & 0 & Green component (0-255)\\
\hline
B		  & 0 & Blue component (0-255)\\
\hline
Tolerance & 0 & Alpha color tolerance value (0-128)\\
\hline
\end{tabular}



%----- Section: Texture handler ------------------
\section{Texture handler}
When working with textures you will use texture handler in your code (PLTTextureHandler).\\
They will manage the basic texture stuff for you like loading them or set the correct texture states
like the wrapping functions. The texture handler will offer you an interface to work with the used texture.
The texture handlers are also responsible for texture animations like changing textures or rotation, move etc.
ones. Such texture animations you can create within your application by programming it directly
or through the PixelLight \hyperlink{tani}{tani-format} which are load by and dealed the texture
handlers like 'normal' texture, too.
The different material layers are implemented using such texture handlers.




%----- Section: Texture animation format (tani) --
\section{Texture animation format (tani)}
\hypertarget{tani}{}
With the texture animation configuration (normal ini text file) you can create animated textures.
There are three texture animation types:
\begin{description}
\item[Texture animations:] Animation through texture changing
\item[Matrix animations:]  Animation through texture transformation matrix manipulation
\item[Color animations:]   Animation through color changing
\end{description}
Unrequired commands can be skipped - standard values will then be used.\\

Heres a full texture animation example:\\
\verbatiminput{examples/example.tani}


Texture frame definitions\\
\\
\textbf{Frame:}\\
\\
Texture resize settings\\
\\
\begin{tabular}{|p{2.5cm}|p{2.5cm}|p{9cm}|}
\hline
\textbf{Command} & \textbf{Default} & \textbf{Description}\\
\hline
Texture & - & Texture filename of for this frame\\
\hline
\end{tabular}


Matrix frame definitions\\

\textbf{Frame:}\\

\begin{tabular}{|p{2.5cm}|p{2.5cm}|p{9cm}|}
\hline
\textbf{Command} & \textbf{Default} & \textbf{Description}\\
\hline
Pos   & 0.0 0.0 0.0 & Texture position\\
\hline
Scale & 1.0 1.0 1.0 & Texture scale\\
\hline
Rot   & 0.0 0.0 0.0 & Texture rot\\
\hline
\end{tabular}


Color frame definitions\\

\textbf{Frame:}\\

\begin{tabular}{|p{2.5cm}|p{2.5cm}|p{9cm}|}
\hline
\textbf{Command} & \textbf{Default} & \textbf{Description}\\
\hline
Color & 1.0 1.0 1.0 1.0 & Texture color (RGBA)\\
\hline
\end{tabular}


Animation definitions\\

There's always a texture, matrix and color animation with all frames by default.
The first animation of each type is played automatically after loading!\\

\begin{tabular}{|p{2.5cm}|p{2.5cm}|p{9cm}|}
\hline
\textbf{Command} & \textbf{Default} & \textbf{Description}\\
\hline
Type      & TEXTURE & Animation type
\begin{tabular}{p{3cm}p{6cm}}
Texture & Texture animation\\
\hline
Matrix  & Matrix animation\\
\hline
Color   & Color animation\\
\end{tabular}\\
\hline
Name      & -       & Animation name\\
\hline
Start     & 0       & Start frame\\
\hline
End       & 0       & End frame\\
\hline
Speed     & 1.0     & Animation speed (higher = faster)\\
\hline
Loop      & 1       & Loop animation?\\
\hline
Ping pong & 0       & Ping pong animation?\\
\hline
\end{tabular}


Frame settings\\

\begin{tabular}{|p{2.5cm}|p{2.5cm}|p{9cm}|}
\hline
\textbf{Command} & \textbf{Default} & \textbf{Description}\\
\hline
ID    & 0   & Frame ID\\
\hline
Speed & 0.0 & Speed of this frame (higher value = faster animation)\\
              &&  0.0 = Set value and skip this frame\\
              && < 0.0 = No interpolation\\
\hline
\end{tabular}


Event settings\\

\begin{tabular}{|p{2.5cm}|p{2.5cm}|p{9cm}|}
\hline
\textbf{Command} & \textbf{Default} & \textbf{Description}\\
\hline
FrameID & 0 &
If this frame is reached in the animation the event ID message is send to the entity connected
with the animation. You could use it e.g. to mark frames were a sound should be played.\\
\hline
ID      & 0 & Event ID\\
\hline
\end{tabular}




%----- Section: Procedural textures --------------
\section{Procedural textures}
Prodecural textures are textures which will be created and possibly updated during runtime.
Normally it's usefull to implement them as scene node which controls the procedural texture. Here's
an example how such an procedural texture entity might look:\\

\begin{lstlisting}[caption=Procedural texture class]
class SNProceduralTexture : public PLTSceneNode {
	...
	private:
		float m_fTimer;             // Timer

		// Exported variables
		char  m_szTextureName[64];  // Name of the texture
		int   m_nWidth;             // Texture width
		int   m_nHeight;            // Texture height
		float m_fSpeed;             // Animation speed

	private:
		virtual void InitFunction();
		virtual void DeInitFunction();
		virtual void UpdateFunction();
	...
};

void SNProceduralTexture::InitFunction()
{
	// Init data
	m_fTimer = 0.f;

	// Create texture
	PLTTextureManager::GetInstance()->CreateTexture(m_szTextureName, m_nWidth, m_nHeight);
}

void SNProceduralTexture::DeInitFunction()
{
	PLTTextureManager::GetInstance()->Unload(PLTTextureManager::GetInstance()->Get(m_szTextureName));
}

void SNProceduralTexture::UpdateFunction()
{
	// Animation speed
	m_fTimer += PLTTimer::GetInstance()->GetTimeDifference()*m_fSpeed;
	if (m_fTimer < 1.f) return;
	m_fTimer = 0.f;

	// Create texture
	PLTTexture *pTex = PLTTextureManager::GetInstance()->Get(m_szTextureName);
	if (!pTex) return;
	unsigned char *pData = pTex->GetData();
	if (!pData) return;
	for (int i=0; i<pTex->GetWidth()*pTex->GetHeight(); i++) {
		*pData = *(pData+1) = *(pData+2) = PL::Math.GetRand() % 255;
		pData += 3;
	}

	// Upload new texture
	pTex->Upload();
}
\end{lstlisting}

This example creates a new texture using the texture manager function CreateTexture(). Then from
time to time the texture data is filled up with random color values and is finally uploaded to the \ac{GPU}.
This would result in a dynamic random noise texture. Dynamic created textures are used like all other normal
textures. E.g. you can load them in a texture handler using Load("Texturename"). But note that
this texture must already exist before you are able load them in a texture handler or material else
this texture will be unknown!\\
In this way you can also create special textures you e.g. need for your shaders. For instance
if you need a normalization cube map you implement an scene node which will create this texture for you
so that you can use it in the materials like all other textures, too! But note that this textures
must be created before a material is loading them! This is an intuitive way to deal with self calculated
textures without to be forced to handle such textures in a special manner.



%----- Section: Alpha test -----------------------
\section{Alpha test}
Alpha test is one operation in the per-fragment operations stage of the renderer pipeline that
allows any further processing of the fragment to be aborted based on the value of its alpha component. The
fragment's alpha component is compared with an application-specified reference value
using an application-specified comparison function. If the fragment passes the test, it will
be processed by the subsequent fragment operation, otherwise it will be discarded. Alpha
test does not incur any extra overhead even if all the fragments pass the test. Unlike
stencil testing, depth testing, texturing, and blending, alpha testing do not involve
fetching data from memories external to the \ac{GPU}. Alpha test is only available in RGBA
mode for instance when using tga textures with an alpha channel. You can setup the alpha
test withing the render state section, either directly by hand in the renderer (for programmers)
or in the render state section of a material technique pass (see the material format in another chapter).

Alpha test provides a great way to save fill rate. When used to draw alpha blended
textures, or to do additive blending in multi-pass, it provides a means to reject the
fragment as early as possible in order to reduce the memory traffic due to stencil, depth,
and color buffer reads and writes.
Consider the case when an application draws big triangles on the screen, like lens
flares or explosions, and those triangles are transparent or blended with additive blending.
Additionally, consider the case of multi-pass rendering, which is common practice to
"shade" models and significantly increases the depth complexity. Typically, one can
render one pass to apply a textured diffuse lighting to a model, render another pass for the
specular highlight. In this second pass, an enormous number of fragments are just black,
due to the nature of the specular computation. We will demonstrate a technique, which
can optimize the application in such situation by mixing the use of alpha test and shaders,
in order to discard the black fragments.
Graphic programmers should reduce the overhead, caused by such rendering
architectures or effects, by utilizing fill rate saving techniques such as alpha test and
other tricks derived from specific hardware functionalities that can take advantage of
alpha test.

Here's a more advanced example when the alpha test can be quite useful to discard the black fragments
when dealing with range-limited lights and shaders:\\
Whenever a fragment is outside the range of the light, you can skip all the lighting math and just return
zero immediately. With pixel shader 3.0 you'd simply implement this with an if-statement. Without pixel
shader 3.0, you just move this check to another shader. This shader returns a truth value to alpha. The
alpha test then kills all unlit fragments. Surviving fragments will write 1 to stencil. In the next pass
you simply draw lighting as usual. The stencil test will remove any fragments that aren't tagged as lit.
If the hardware performs the stencil test prior to shading you will save a lot of shading power. In fact,
the shading workload is reduced to the same level of pixel shader 3.0, or in some cases even slightly below
since the alpha test can do a compare that otherwise would have to be performed in the shader. 
The result is that early-out speeds things up considerably. In theory you are able to to this completely within
the PixelLight material! Imagine a material with two techniques the more advanced one is using pixel shader 3.0 to ignore
unrequired fragments and if this fragment profile isn't available an alternative technique is used with two
passes: The first which will use the alpha test, stencil buffer and a special 'if' shader and the second one drawing
the stuff were it passes the stencil buffer which was set by the pass before.

By tuning appropriately the reference value of alpha test to reject the transparent or almost transparent
fragments, one can improve the application performance significantly. Basically, varying the reference value
acts like a threshold setting up how many fragments are going to be evicted. The more fragments
are being discarded, the faster the application will run. On the other hand, the more
fragment are being discarded, the more texture subtleties are not being drawn, potentially
leading to some undesirable visual side effects. In this particular case, it is crucial to take
into consideration the visual artifacts introduced by setting a reference value threshold
too aggressively. Therefore, setting up alpha test's reference value shouldn't be done
without taking the final visual quality into account.
Because using alpha test can speed up your application, one should consider adding an
explicit alpha channel to textures that are used in a way that burns a lot of fill rate and
that have areas that are not used.
