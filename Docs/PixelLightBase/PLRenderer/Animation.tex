\section{Animations}
This is a separate chapter because PixelLight deals all animations using the same interface \emph{Animation}, there's no difference whether it's a mesh, texture etc. animation! Therefore all types of animation have the same features like causing an event at a given frame. An animation consists of different frames which are played in a given order with defined settings like loop etc. The animation class ONLY deals with frame ID which makes it usable for different kinds of animations.




\subsection{Animation Playback}
The most primitive way of playing an animation is to call the start function like with e.g. \emph{Start(0, 10)} which will cause a playback from frame 0 until 10 without any special. Each frame you have to call the animation \emph{Update()} function which is using the time difference since last frame to increase the current frame. Using \emph{GetCurrentFrame()} you will receive the current animation frame while the function \emph{GetFrame()} will return an interpolated frame like 5.25. The animation playback can be paused or continued or even stopped on desire. Further you are able to setup whether the animation should start from beginning when it's end is reached used \emph{SetLoop()}, there are also some other functions for controlling the animation playback by hand.




\subsection{Advanced Animation Playback}
The hole animation process could be automated using the animation information class \emph{AnimationInfo}. There all information about an animation like start, end frame, playback speed, loop etc. are stored and you only have to pass a pointer to such a class when starting an animation like \emph{cAnimation.Start(cMyAnimationInfo)}. Further this information class offers new animation features which enables you e.g. to control the playback speed of each individual frame through extra animation frame information \emph{AnimationFrameInfo}. Using this you can even create frames which don't interpolate to create e.g. clonus animations.
