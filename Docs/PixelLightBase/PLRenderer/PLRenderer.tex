\chapter{PLRenderer}


\paragraph{Motivation}
The PixelLight renderer component provides \ac{API}-independence from the underlying 3D graphics \ac{API}s using renderer libraries that are dynamically loaded at runtime. The public interface of the PixelLight renderer can itself be seen as an object-oriented 3D graphics \ac{API}. The render component acts as an abstraction layer between the application and the underlying rendering \ac{API}, such as OpenGL, DirectX, or possibly even theoretically a software renderer. It maps the functionality offered by the underlying rendering \ac{API} to a well-defined, object-oriented interface that the application uses to render scenes. The application only works with the interfaces offered by the render component and usually does not have to concern itself much with the peculiarities of the underlying 3D graphics \ac{API}.

If you have the headers and and libraries of a renderer backend, you can also use the \ac{API} functions by hand, for instance to use special/new OpenGL extensions. In fact, even if you don't have the headers and you know that you are using for instance OpenGL, no one will hinder you using \ac{API} dependent calls like \emph{glVertex}. But this is not recommended because you limit yourself to one \ac{API} and porting your application is more work. If you are missing features - ask the PixelLight team for it.

The main functionality of the render component comprises routines to create render contexts and render targets, to begin and end a scene, to swap back with front buffers, to render primitives, to create vertex and \ac{IBO}s, to manage render states and texture sampler stages and to offer support for high-level shading languages.

More than one renderer can be used by an application at the same time, which allows using different 3D graphics \ac{API}s to render into multiple windows - but this isn't recommended in practice and the PixelLight application framework itself can only use one renderer at the same time to keep things simple. Else you need to store your textures, shaders, vertex data etc. several times - one copy for each renderer \ac{API}. One texture manager, one shader manager etc. are quite handy - aren't they?

Things like meshes, scene graphs etc. are not implemented within the renderer itself because the renderer is developed as an universal and basic component. On top of the abstract renderer interfaces there are texture, shader etc. managers within the renderer to avoid storing the same resource multiple times and to be able to use them in a more comfortable way.\footnote{But the backend implementations don't know anything about this and don't need to!} More advanced things are implemented within projects using the renderer. For instance the mesh library will add meshes and more complex animated meshes. An advanced scene system is added by the framework itself.


\paragraph{External Dependences}
PLRenderer depends on the \textbf{PLCore}, \textbf{PLMath} and \textbf{PLGraphics} libraries.




% Include the other sections of the chapter
\chapter{General}




\section{Keep it Simple!}
The world is complicated enough - if there are multiple solutions, prefer the simplest over the most complicated one! This way, the chances are high that other will understand the solution as well as you when looking at the code some years later.




\section{Encoding}
We work with multiple operation systems so we have to take into account \emph{how} text files are saved. All \emph{Diary}, \emph{Readme}, \emph{Todo} and \emph{Plan} text files saved as "Unicode (UTF-8 with signature) - Codepage 65001". All other text files like code or make files saved in classic \emph{ANSI}.




\section{File Extension}
First of all, we use the old fashion \emph{.h}-file extension to mark header files - \emph{hpp} would be the \emph{correct} extension for C++, but it's not widely used. Usually, we outsource inline implementations into files with an \emph{inl}-extension to keep the header files good readable. For source codes, we use the file extension \emph{cpp}.




\section{C++11 Language Features}
Using C++11 (previously known as C++0x) language features is fine as long as
\begin{itemize}
\item{It's possible to emulate, or at least deactivate the feature within compilers don't supporting it (yet)}
\item{There's no comfortable/acceptable way to solve a task without using the feature, but those situations have to be discussed within the team and community}
\end{itemize}
\textsl{}
Currently the following C++11 language features are used:
\begin{itemize}
\item{\emph{nullptr} - a null pointer literal}
\item{\emph{override} - gives the compiler a chance to detect and blame errors related to overwriting methods}
\end{itemize}


\paragraph{Extern Templates}
Don't use \emph{extern templates}\footnote{\url{http://www2.research.att.com/~bs/C++0xFAQ.html\#extern-templates}} in order to avoid template instantiation in other modules. This is a feature we can't emulate and there are still legacy compilers like \ac{GCC} 4.2.1 used on Mac OS X 10.6 actively used around the world. So, at least for now, don't use this useful feature.




\section{For-Scope}
Within the PixelLight projects, the \emph{for-scope} is active by default. The following for instance will produce a compiler error:
\begin{lstlisting}[caption=for-scope]
for (int i=0; i<1; i++) {
	// Do anything
}
i = 20;	// i has already gone out of scope
\end{lstlisting}




\section{Null Pointer}
For null pointers, use \emph{nullptr}\footnote{For example \emph{Microsoft Visual Studio 2010} and \emph{\ac{GCC} 4.6} have native support for \emph{nullptr}} from \emph{C++11} and not for example the legacy, but traditional \emph{NULL} definition or even directly a integer $0$.

\begin{lstlisting}[caption=Null pointer]
char *pszMyString = nullptr;
\end{lstlisting}




\section{Overwriting Methods}
Put the methods within the base class, which are allowed to be overwritten, into a separate code block so everyone is able to find them at once.
\begin{lstlisting}[caption=Virtual methods within a base class]
//[-------------------------------------------------------]
//[ Public virtual MyClass functions                      ]
//[-------------------------------------------------------]
virtual void MyMethod() = 0;
\end{lstlisting}

When overwriting virtual methods within a derived class, put the overwritten methods into a code block telling were those methods originally came from.
\begin{lstlisting}[caption=Overwriting virtual methods within a derived class]
//[-------------------------------------------------------]
//[ Public virtual MyClass functions                      ]
//[-------------------------------------------------------]
virtual void MyMethod() override;
\end{lstlisting}
Although technically not required, do also add a \emph{virtual} to make it absolutely clear that this is a virtual method. To give the compiler a chance to find and blame possible errors like a signature change within the base class, use the C++11 language keyword \emph{override}.




\section{Casting}
PixelLight is using \emph{C++ style casts} (\emph{int i = static\_cast<int>(42.21f)}), not \emph{C style casts} (\emph{int i = (int)42.21f}). It's much easier to search for \emph{C++ style casts}\footnote{\emph{static\_cast}, \emph{reinterpret\_cast}, \emph{const\_cast}, and \emph{dynamic\_cast}} and they are less vulnerable to unintended effects as well - and because they are not that compact as the \emph{C style casts}, one may think about it a second time why there's a need for a cast.

\paragraph{\ac{GCC}}
\ac{GCC}, offers an option called \emph{-Wold-style-cast} to let the compiler warn if an old-style (C-style) cast to a non-void type is used within a C++ program.




\section{Const Correctness}
Define functions, variables etc. whenever possible to be constant. By giving the compiler this hint, it may be possible to use special optimizations or uncover bugs within the implementation.


\paragraph{Const Example}
Have a look at the following example function:
\begin{lstlisting}[caption=Non-constant function parameter]
void MyFunction(PLMath::Vector3 &vPosition)
{
	// ...
}
\end{lstlisting}
In case the function is considered to manipulate \emph{vPosition}, all's fine. Let's continue with another usage example:
\begin{lstlisting}[caption=Using a temporary variable instance as non-constant function parameter]
MyFunction(PLMath::Vector3(0.0f, 1.0f, 2.0f));
\end{lstlisting}
The compiler creates a temporary \emph{PLMath::Vector3} instance on the runtime stack and is passing a reference into the function. Looks fine as well? In case the function manipulates the variable passed by reference, the temporary instance is changed. In principle that's no problem because it's thrown away after the function call anyway. In practice, this situation is considered to be evil. \ac{MSVC} 2010 will shy tell you via a warning
\begin{quote}warning C4239: nonstandard extension used : 'argument' : conversion from 'PLMath::Vector3' to 'PLMath::Vector3 \&' A non-const reference may only be bound to an lvalue\end{quote}
So, it's possible to just ignore or even deactivate this warning... isn't it? In fact, it isn't because other compilers might not be that tollerant and will throw an error message at you. To sum this up: In the example above, you really want to write
\begin{lstlisting}[caption=Constant function parameter]
void MyFunction(const PLMath::Vector3 &vPosition)
{
	// ...
}
\end{lstlisting}
in order to be cross-platform safe.


\paragraph{Const Exception}
There's one situation were we do not use \emph{const} - when dealing with function parameters because

\begin{lstlisting}[caption=Function parameters]
void MyFunction(int nVariable1, int nVariable2);
\end{lstlisting}

is inside headers better readable than

\begin{lstlisting}[caption=Constant function parameters]
void MyFunction(const int nVariable1, const int nVariable2);
\end{lstlisting}

In this situation, the readability is more important for us. This rule does not apply for pointer or reference parameters like

\begin{lstlisting}[caption=Constant function pointer/reference parameter]
void MyFunction(const String &sVariable);
\end{lstlisting}

because the user should be able to see whether or not a function is going to manipulate the parameter variable!




\section{static const Vs. const static}
Use \emph{static const} instead of \emph{const static}. Have a look at e.g. the \ac{GCC} option \emph{Wold-style-declaration} resulting in the warning \begin{quote}`static' is not at beginning of declaration\end{quote} or into chapter 6.11 of ISO C99 (\emph{Future Language Directions} -> \emph{Storage-class specifiers}).




\section{Namespaces}
PixelLight is using multiple namespaces, one for each sub-project. If you want to use for instance the string class which is defined in \emph{PLCore} you need to do this:

\begin{lstlisting}[caption=Explicit namespace]
PLCore::String sMyString;
\end{lstlisting}

Or this:

\begin{lstlisting}[caption=Using namespace]
using namespace PLCore;
...
String sMyString;
\end{lstlisting}

Try to avoid using \emph{using namespace} too often or this will result in name conflicts which you then have to resolve by hand by adding for instance \emph{PLCore::}. We recommend to never use \emph{using namespace} within header files!




\section{Dynamic Parameters}
When dynamic parameters are used and the name of the parameters inside a string is irrelevant, as this is the case for \emph{PLCore::Params::FromString}, the parameters are named using \emph{Param<x>} were x starts with $0$ (example: \emph{Param0=1 Param1=''Hello''}).  




\section{Names}
In general, names of classes, functions, variables and so on must have human readable names. The name has to tell as much as possible about the usage - if the user can guess correctly the usage of for example a variable by just looking at it's name, the name is perfect.

General rules:

\begin{itemize}
\item A single character as name for local (only local!) control variables like \emph{i} within a for-loop is acceptable as long as there are not to much of those at once (else use reasonable names to avoid confusion!)
\item Short cuts should be avoided whenever possible because they may leads to confusion\footnote{True story: When using \emph{Rot} as short cut for \emph{Rotation}, we once had the situation that a German speaking programmer asked confused what the color \emph{Rot} should do inside the scene node... in German, \emph{Rot} is the word for \emph{red}...} (NO stuff like \emph{stricmp()}!)
\item If there's a \emph{commonly used} name for something, just this name instead of creating a totally new one
\item Avoid long names, if there's an expressive shorter name it's the preferred one... but keep the short cut rule in mind!
\end{itemize}

Classes, structures and so on have a upper case letter at the beginning. Example:

\begin{lstlisting}[caption=Name convention]
class Player {
};
struct Info {
};
\end{lstlisting}




\section{Prefix}
Because the readability of code is extremely important when working in a team and/or using code from others, one of our goals was to make the PixelLight code as readable and well structured as possible. We are using a name style convention\footnote{We know that there are a lot of discussions around the internet whether or not prefixes should be used. In the year 2002 we decided to use them and we don't change it - due to the dimension of the PixelLight project, it would be a huge effort to change it anyway.}.

Variable prefixes for standard types:

\begin{lstlisting}[caption=Variable prefixes for standard types]
Type               Prefix    Example
bool               b         bool bActive

(n for all none standard floating point types)
int                n         int  nNumber
char               n         char nCharacter
long               n         long nHuge

float              f
double             d

(Character arrays -> strings)
char[]             sz        char szName[64]
char*              psz       char *pszName

(Pointers)
*                  p         Player *pPlayer

(References)
&                  -         char &nTest = nTest2;

struct instance    s         Info sPlayer (struct Info)

class instance     c         Player cPlayer (class Player)
\end{lstlisting}

General variable prefix for class variables:
m\_ (m for member)\\
Example: char *m\_pszName\\

Variable prefixes for PixelLight types:

\begin{lstlisting}[caption=Variable prefixes for PixelLight types]
Type               Prefix    Example
String             s         String sName

Container          lst       List lstNames

Map                map       HashMap mapNames

VectorX            v         Vector3 vPosition
(X for dimension: 2, 3 or 4)

MatrixXxX          m         Matrix4x4 mRotation
(X for dimension: 3 or 4)

Quaternion         q         Quaternion qRotation

ColorX             c         Color3 cColor (same as class)
(X for dimension: 3 or 4)
\end{lstlisting}




\section{Postfix}
We recommend you to use the PixelLight name convention and marking debug versions with a \emph{D} at the end of the filename. Example: \emph{MyPlugins.dll} = release version, \emph{MyPluginsD.dll} = debug version.




\section{Events and Signals}
As soon as an event is inside a class, we refer to it as \emph{signal}. As such, the prefix \emph{Event} like within \emph{EventKeyDown} is used outside classes while prefix \emph{Signal} like within \emph{SignalKeyDown} is used inside classes.




\section{Event Handlers and Slots}
Within our name convention for event handlers and \ac{RTTI} slot names, there's a \emph{On} within for example \emph{OnMyEvent} indicating that this is a handler/slot method. The other part of the name consists of the name of the event/signal - for \emph{OnMyEvent} this would be an event/signal with the name \emph{MyEvent}.




\section{\ac{RTTI} Interface}
Within PixelLight, the \ac{RTTI} class properties and members are always defined in the following order:
\begin{itemize}
\item Properties
\item Attributes
\item Constructors
\item Methods
\item Signals
\item Slots
\end{itemize}
This way one knows exactly were to look for something. Further, within the \ac{RTTI} class properties and members definitions, only tabs and no spaces are used to make it easier to write the definitions like a table. This makes it more comfortable for the eyes and brain to navigate to certain definition parts without to much searching around.

Here's an example source code showing the common \ac{RTTI} interface layout (without the word wrap):
\begin{lstlisting}[caption=\ac{RTTI} interface (without the word wrap)]
//[-------------------------------------------------------]
//[ RTTI interface                                        ]
//[-------------------------------------------------------]
pl_class(pl_rtti_export, MyRTTIClass, "", PLCore::Object, "Sample RTTI class, don't take it to serious")
	// Properties
	pl_properties
		pl_property("MyProperty",	"This is a property value")
	pl_properties_end
	// Attributes
	pl_attribute(Name,	PLCore::String,	"Bob",	ReadWrite,	GetSet,			"A name, emits MySignal after the name was changed",			"")
	pl_attribute(Level,	int,			1,		ReadWrite,	DirectValue,	"Level, automatically increased on get/set name and OnMyEvent",	"")
	// Constructors
	pl_constructor_0(DefaultConstructor,	"Default constructor",	"")
	// Methods
	pl_method_0(Return42,			int,					"Returns 42",							"")
	pl_method_1(IgnoreTheParameter,	void,			float,	"Ignores the provided parameter",		"")
	pl_method_0(SaySomethingWise,	void,					"Says something wise",					"")
	pl_method_0(GetSelf,			MyRTTIClass*,			"Returns a pointer to this instance",	"")
	// Signals
	pl_signal_1(MySignal,	PLCore::String,	"My signal, automatically emitted after the name was changed",	"")
	// Slots
	pl_slot_0(OnMyEvent,	"My slot",	"")
pl_class_end
\end{lstlisting}



\section{Reuseability and adding new Stuff}
Before you add new classes, functions an so on - check first whether there's already something similar within PixelLight. If there's something you can already use directly, use it instead of writing new stuff. If there's something quite similar, have a more detailed look at it and contact your team colleagues to discuss whether a refactoring is possible and reasonable to update and/or to enhance existing stuff.

Reuseability is one of the most important concepts when creating frameworks like PixelLight... and reuseability does not mean that it's possible to copy'n'past it and then hacking around for a certain project! Reuseability means that it's possible to directly reuse, to share, something between multiple projects in a quite universal way without the need to enhance and hack around constantly!

\cleardoublepage
\section{Animations}
This is a separate chapter because PixelLight deals all animations using the same interface \emph{Animation}, there's no difference whether it's a mesh, texture etc. animation! Therefore all types of animation have the same features like causing an event at a given frame. An animation consists of different frames which are played in a given order with defined settings like loop etc. The animation class ONLY deals with frame ID which makes it usable for different kinds of animations.




\subsection{Animation Playback}
The most primitive way of playing an animation is to call the start function like with e.g. \emph{Start(0, 10)} which will cause a playback from frame 0 until 10 without any special. Each frame you have to call the animation \emph{Update()} function which is using the time difference since last frame to increase the current frame. Using \emph{GetCurrentFrame()} you will receive the current animation frame while the function \emph{GetFrame()} will return an interpolated frame like 5.25. The animation playback can be paused or continued or even stopped on desire. Further you are able to setup whether the animation should start from beginning when it's end is reached used \emph{SetLoop()}, there are also some other functions for controlling the animation playback by hand.




\subsection{Advanced Animation Playback}
The hole animation process could be automated using the animation information class \emph{AnimationInfo}. There all information about an animation like start, end frame, playback speed, loop etc. are stored and you only have to pass a pointer to such a class when starting an animation like \emph{cAnimation.Start(cMyAnimationInfo)}. Further this information class offers new animation features which enables you e.g. to control the playback speed of each individual frame through extra animation frame information \emph{AnimationFrameInfo}. Using this you can even create frames which don't interpolate to create e.g. clonus animations.

\cleardoublepage
\chapter{Renderer backends}
This chapter deals with the different renderer backends shipping with the PixelLight SDK.




\section{Null}
\begin{itemize}
\item The null renderer backend is for situations were you don't need \emph{real} rendering output. It consists of dummy functions which will try to \emph{emulate} the set \& get functions that good as possible.
\item PixelLight component name: PLRendererNull
\item Used dll's: \emph{PLRendererNull.dll}
\end{itemize}




\section{OpenGL}
\begin{itemize}
\item This is the preferred renderer backend of the PixelLight team because OpenGL\footnote{More information about OpenGL can be found at \url{http://www.opengl.org/}} provides some nice features like line width and it's also supported by other operation systems like Linux.
\item PixelLight component name: PLRendererOpenGL
\item Used dll's: \emph{PLRendererOpenGL.dll}
\item FreeType\footnote{FreeType can be downloaded from \url{http://www.freetype.org/}} is used for font support.
\end{itemize}

Here's a list of some general notes:
\begin{itemize}
\item Normally when using rectangle textures (none power of 2) under OpenGL, you have to work with none normalized texture coordinates - but because we don't want to deal with texture coordinates for rectangle textures in a special way, within the backend the texture matrix is scaled. So, you always work with normalized texture coordinates. But if you use rectangle textures within a shader, you should take the texture matrix within your vertex shader into account!
\end{itemize}

There's a plugin for PLRendererOpenGL adding support for the Cg\footnote{(\url{http://developer.nvidia.com/object/cg_toolkit.html})} shader language:
\begin{itemize}
\item PixelLight component name: PLRendererOpenGLCg
\item Used dll's: \emph{PLRendererOpenGLCg.dll}, \emph{cg.dll} and \emph{cgGL.dll}
\end{itemize}




\section{OpenGL ES 2.0}
\begin{itemize}
\item Currently not within the official PixelLight SDK
\item OpenGL ES 2.0 renderer backend for mobile devices
\item PixelLight component name: PLRendererOpenGLES
\item FreeType\footnote{FreeType can be downloaded from \url{http://www.freetype.org/}} used for font support.
\end{itemize}




\section{Direct3D 9}
\begin{itemize}
\item Currently not within the official PixelLight SDK
\item This Direct3D 9\footnote{For more Direct3D information, please have a look at \url{http://www.microsoft.com/downloads/Browse.aspx?displaylang=de&categoryid=2}} renderer backend is for \emph{Microsoft Windows} systems only, and has some missing features.
\item PixelLight component name: PLRendererD3D9
\item Used dll's: \emph{PLRendererD3D9.dll}, \emph{cg.dll} and \emph{cgD3D9.dll} and DirectX 9 must be installed
\end{itemize}

Here's a list of some general notes:
\begin{itemize}
\item No line width support. \emph{SetLineWidth()} has no effect and \emph{GetLineWidth()} will always return 1
\item Because D3D doesn't have any constant color, \emph{SetColor()} must be emulated within this backend using for instance a vertex shader
\end{itemize}
Further, within this backend there are still some bugs to fix.




\section{Direct3D 11}
\begin{itemize}
\item Currently not within the official PixelLight SDK, project is quite new and heavily under construction
\item This Direct3D 11\footnote{For more Direct3D information, please have a look at \url{http://www.microsoft.com/downloads/Browse.aspx?displaylang=de&categoryid=2}} renderer backend is for \emph{Microsoft Windows} systems only, and has some missing features.
\item Windows Vista and Windows 7 only
\item PixelLight component name: PLRendererD3D11
\item Used dll's: \emph{PLRendererD3D11.dll} and DirectX 11 must be installed
\end{itemize}

\cleardoublepage
%\chapter{GPU Programs}
There's a broad range of shader languages and APIs like GLSL, HLSL and Cg. Therefore, when designing the GPU program interfaces for PLRenderer, one design goal was to be able to implement as many GPU program backends as possible - and this without producing to much internal overhead.

In PixelLight, we're using OpenGL and GLSL terminology if not otherwise mentioned. The reason for this is simple: OpenGL and GLSL are popular open, multi-platform standard.




\section{Shader}


\paragraph{Uniforms}
Uniforms are per-program variables that are constant during program execution.




\section{Vertex Shader}
A vertex shader can reference a number of variables as it executes.


\paragraph{Vertex Attributes}
Vertex attributes are the per-vertex values.


\paragraph{Varying Vertex Shader Input}
In the first years of programmable shaders, there was a need to connect the legacy fixed functions with the new programmable shaders. Therefore, within certain shader languages, binding semantic names for varying vertex shader input were introduces. Table~\ref{Table:VaryingVertexShaderInput} gives an overview over the first 16 vertex shader input arguments within GLSL and Cg and their corresponding binding semantic names.
\begin{table}[htb]
	\centering
	\begin{ThreePartTable}
		\begin{tabular}{|l|l|p{0.2\textwidth}|l|}
			\toprule
				\textbf{Index} & \textbf{GLSL\tnote{1}} & \textbf{Cg\tnote{2}} & \textbf{Description}\\
			\midrule
				\hline
				0	&	gl\_Vertex			&	POSITION, ATTR0				&	Position\\
				\hline
				1	&	-					&	BLENDWEIGHT, ATTR1			&	Weight\\
				\hline
				2	&	gl\_Normal			&	NORMAL, ATTR2				&	Normal\\
				\hline
				3	&	gl\_Color			&	COLOR0, DIFFUSE, ATTR3		&	Primary color\\
				\hline
				4	&	gl\_SecondaryColor	&	COLOR1, SPECULAR, ATTR4		&	Secondary color\\
				\hline
				5	&	gl\_FogCoord		&	TESSFACTOR, FOGCOORD, ATTR5	&	Fog coordinate\\
				\hline
				6	&	-					&	PSIZE, ATTR6				&	Point size\\
				\hline
				7	&	-					&	BLENDINDICES, ATTR7			&	Blend indices\\
				\hline
				8	&	gl\_MultiTexCoord0	&	TEXCOORD0, ATTR8			&	Texture coordinate 0\\
				\hline
				9	&	gl\_MultiTexCoord1	&	TEXCOORD1, ATTR9			&	Texture coordinate 1\\
				\hline
				10	&	gl\_MultiTexCoord2	&	TEXCOORD2, ATTR10			&	Texture coordinate 2\\
				\hline
				11	&	gl\_MultiTexCoord3	&	TEXCOORD3, ATTR11			&	Texture coordinate 3\\
				\hline
				12	&	gl\_MultiTexCoord4	&	TEXCOORD4, ATTR12			&	Texture coordinate 4\\
				\hline
				13	&	gl\_MultiTexCoord4	&	TEXCOORD5, ATTR13			&	Texture coordinate 5\\
				\hline
				14	&	gl\_MultiTexCoord6	&	TEXCOORD6, TANGENT, ATTR14	&	Texture coordinate 6, tangent\\
				\hline
				15	&	gl\_MultiTexCoord7	&	TEXCOORD7, BINORMAL, ATTR15	&	Texture coordinate 7, binormal\\
				\hline
			\bottomrule
		\end{tabular}
		\begin{tablenotes}
			\item[1] Cg Users Manual (2.2, Release 1.4 September 2005), \url{http://developer.download.nvidia.com/cg/Cg\_2.2/CgUsersManual.pdf}, Page 299, Table 30. vp20 Varying Input Binding Semantics
			\item[2] OpenGL Shading Language 3.30.6 Specification (updated March 11, 2010), \url{http://www.opengl.org/registry/doc/GLSLangSpec.3.30.6.clean.pdf} Page 73, 7.3 Compatibility Profile Vertex Shader Built-In Inputs
		\end{tablenotes}
		\caption{Binding semantic names for varying vertex shader input in GLSL and Cg}
		\label{Table:VaryingVertexShaderInput}
	\end{ThreePartTable}
\end{table}
Nowadays, one should avoid using this binding semantic names because they are often depriciated and in general, restricts the usage of programmable shaders. Modern graphic API's and shader languages no longer demand that for example, the first vertex shader input argument is the vertex position. One should carefully plan the vertex shader input design. If not otherwise mentioned we're using within PixelLight the vertex shader input layout as seen within the table~\ref{Table:VaryingVertexShaderInput} above - this way we stay compatible to fixed function stuff and in general, it's less confusing to use a standard layout if possible.




\section{Fragment Shader}
A vertex shader can reference a number of variables as it executes.
			%[TODO] Update!
\cleardoublepage
%----- Chapter: High level: Textures --------------
\chapter{High level: Textures}


[TODO] Update!


%----- Section: Overview -------------------------
\section{Overview}
Because the materials normally using textures it's good to know some hinds about it - read this
section carefully!
There are different types of textures like 1D, 2D, \hyperlink{Rectangle textures}{rectangle textures},
\hyperlink{3D textures}{3D}), \hyperlink{Cube maps}{cube map textures}) etc.




%----- Section: Texture formats ------------------
\section{Texture formats}
PixelLight can load the following texture formats:\\

bmp, cut, ico, jpg, pcx, tif, png, tga, dds, psd, hdr, exr\\

and save this formats:\\

bmp, jpg, pcx, png, pnm, raw, sgi, tga, tif, pal, hdr, exr\\

If you don't specify the texture format by the filename extension the framework will search for the
texture by adding a filename extension in the order you could see above. PixelLight has no own texture format
because this would be to complex to the graphical artists to convert the textures to a special format.
Normally textures etc. are packed in a zip file and the user will not see them.

Beside the texture files like bmp, tga, jpg etc. which store the texture data itself there are
several additional PixelLight texture description file formats. This PixelLight texture file formats are described below in
detail, here were we only want to mention them in order to let you know that they exist and how to
use them.\\
Each texture file (bmp, tga etc.) can have additional PixelLight relevant information which are defined
in a plt-file with the same texture filename but with the filename ending 'plt' e.g. 'MyTexture.bmp'
would have a 'MyTexture.plt' plt-file which is in the same directory as the owner texture. In this
additional texture file you are able to setup different fixed properties for this texture like if
the texture can be scaled, if GPU texture compression is allowed, the color key for alpha transparency
etc. (such a color key can also be defined in a tga RGBA texture)\\
In the \hyperlink{tani}{tani-format} format you can define texture animations like sliding/rolling
texturs, texture animations through different texture frames etc. Such tani-files can be loaded by
texture handlers and materials as all other texturs, too.

It's recommended to use dds textures with precalculated mipmaps and compression whenever possible
for better loading performance.




%----- Section: PixelLight texture format (plt) --
\section{PixelLight texture format (plt)}
In the plt-format files which have the same name as the 'real' texture (bmp, jpg etc) with the texture
data itself but with the ending 'plt' you can setup some different fixed texture properties like its
color key for alpha masking, whether texture compression is allowed or not etc. This texture
configuration file is a normal text file you can edit with every text editor. Unrequired commands
can be skipped - in this case standard values will be used instead.

First here's a full plt-format example with the standard values:\\
\verbatiminput{examples/example.plt}


General texture configurations\\

\begin{tabular}{|p{2.5cm}|p{2.5cm}|p{9cm}|}
\hline
\textbf{Command} & \textbf{Default} & \textbf{Description}\\
\hline
CubeMap & 0 &
\begin{tabular}{|p{9.5cm}|}
\hline
Is this a cube map?
If yes the other 5 textures will be loaded automatically using the texture name + ID\\

\begin{tabular}{|p{3cm}|p{2.5cm}|}
\hline
\textbf{Side} & \textbf{Index}\\
\hline
positive x & -\\
negative x & 1\\
positive y & 2\\
negative y & 3\\
positive z & 4\\
negative z & 5\\
\hline
\end{tabular}

Example:\\
Sky.tga (this texture you load)\\
Sky1.tga Sky2.tga Sky3.tga Sky4.tga Sky5.tga (loaded automatically)\\
\end{tabular}\\

\hline
Mipmaps		& config & Are mipmaps allowed?\\
\hline
Compression	& config & Is texture compression allowed? (less quality but better performance) Do not use standard texture 
                       compression for normal maps, because this textures stores vector data instead of visible colors 
                       the result would be wrong. If for instance the loaded dds texture is already compressed by
                       default, this compression option is ignored because it would be pretty useless to uncompress
                       this texture before uploading to the GPU... there's already a loss of quality.\\
\hline
Gamma		& 1.0	 & Gamma factor\\
\hline
FitLower	& ?		 & Specifies the automatic texture scale behaviour.
					   If not defined, the texture manager settings will be used. If 1
					   the next valid lower texture size is used, if 0 the next higher one.\\
\hline
Rectangle	& 0		 & If 1 there's no power of 2 limitation for the texture size. If 0, the texture will normally be
                       resized automatically to a valid dimension. If rectangle textures are not supported by the GPU,
                       this settings is ignored and the texture is resized automatically.
 \\
\hline
\end{tabular}


Texture resize settings\\

\begin{tabular}{|p{2.5cm}|p{2.5cm}|p{9cm}|}
\hline
\textbf{Command} & \textbf{Default} & \textbf{Description}\\
\hline
Active	  & 1 & Can the texture size be resized to reduce the texture quality? If the texture dimension
                is not correct and the GPU can't use the texture without resizing it this is internally
                always set to true. Use the 'force' settings below to force a given size even if the GPU
                can't handle it.\\
\hline
MinWidth  & 4 & The minimum allow texture width (should normally be a power of 2!)\\
\hline
MinHeight & 4 & The minimum allow texture height (should normally be a power of 2!)\\
\hline
MinDepth  & 1 & The minimum allow texture depth (should normally be a power of 2!)\\
\hline
\end{tabular}


Forces a special static texture size if not -1. (texture resizing will be disabled!)\\
Use this only in special situations because the GPU may not be able to handle certain
sizes...

\begin{tabular}{|p{2.5cm}|p{2.5cm}|p{9cm}|}
\hline
\textbf{Command} & \textbf{Default} & \textbf{Description}\\
\hline
Width  & -1 & Forced texture width (should normally be a power of 2!)\\
\hline
Height & -1 & Forced texture height (should normally be a power of 2!)\\
\hline
Depth  & -1 & Forced texture depth (should normally be a power of 2!)\\
\hline
\end{tabular}


The color key defines the color which should be transparent.
You can also use a RGBA texture like a tga to create semi transparent textures, in that
case this RGB settings are not used! But note that in this case the material pass using this
RGBA texture in the first layer must have an active alpha test to use the alpha mask - else
you won't see any difference!\\

\begin{tabular}{|p{2.5cm}|p{2.5cm}|p{9cm}|}
\hline
\textbf{Command} & \textbf{Default} & \textbf{Description}\\
\hline
R		  & 0 & Red component (0-255)\\
\hline
G		  & 0 & Green component (0-255)\\
\hline
B		  & 0 & Blue component (0-255)\\
\hline
Tolerance & 0 & Alpha color tolerance value (0-128)\\
\hline
\end{tabular}



%----- Section: Texture handler ------------------
\section{Texture handler}
When working with textures you will use texture handler in your code. (PLTTextureHandler)\\
They will manage the basic texture stuff for you like loading them or set the correct texture states
like the wrapping functions. The texture handler will offer you an interface to work with the used texture.
The texture handlers are also responsible for texture animations like changing textures or rotation, move etc.
ones. Such texture animations you can create within your application by programming it directly
or through the PixelLight \hyperlink{tani}{tani-format} which are load by and dealed the texture
handlers like 'normal' texture, too.
The different material layers are implemented using such texture handlers.




%----- Section: Texture animation format (tani) --
\section{Texture animation format (tani)}
\hypertarget{tani}{}
With the texture animation configuration (normal ini text file) you can create animated textures.
There are three texture animation types:
\begin{description}
\item[Texture animations:] Animation through texture changing
\item[Matrix animations:]  Animation through texture transformation matrix manipulation
\item[Color animations:]   Animation through color changing
\end{description}
Unrequired commands can be skipped - standard values will then be used.\\

Heres a full texture animation example:\\
\verbatiminput{examples/example.tani}


Texture frame definitions\\
\\
\textbf{Frame:}\\
\\
Texture resize settings\\
\\
\begin{tabular}{|p{2.5cm}|p{2.5cm}|p{9cm}|}
\hline
\textbf{Command} & \textbf{Default} & \textbf{Description}\\
\hline
Texture & - & Texture filename of for this frame\\
\hline
\end{tabular}


Matrix frame definitions\\

\textbf{Frame:}\\

\begin{tabular}{|p{2.5cm}|p{2.5cm}|p{9cm}|}
\hline
\textbf{Command} & \textbf{Default} & \textbf{Description}\\
\hline
Pos   & 0.0 0.0 0.0 & Texture position\\
\hline
Scale & 1.0 1.0 1.0 & Texture scale\\
\hline
Rot   & 0.0 0.0 0.0 & Texture rot\\
\hline
\end{tabular}


Color frame definitions\\

\textbf{Frame:}\\

\begin{tabular}{|p{2.5cm}|p{2.5cm}|p{9cm}|}
\hline
\textbf{Command} & \textbf{Default} & \textbf{Description}\\
\hline
Color & 1.0 1.0 1.0 1.0 & Texture color (RGBA)\\
\hline
\end{tabular}


Animation definitions\\

There's always a texture, matrix and color animation with all frames by default.
The first animation of each type is played automatically after loading!\\

\begin{tabular}{|p{2.5cm}|p{2.5cm}|p{9cm}|}
\hline
\textbf{Command} & \textbf{Default} & \textbf{Description}\\
\hline
Type      & TEXTURE & Animation type
\begin{tabular}{p{3cm}p{6cm}}
Texture & Texture animation\\
\hline
Matrix  & Matrix animation\\
\hline
Color   & Color animation\\
\end{tabular}\\
\hline
Name      & -       & Animation name\\
\hline
Start     & 0       & Start frame\\
\hline
End       & 0       & End frame\\
\hline
Speed     & 1.0     & Animation speed (higher = faster)\\
\hline
Loop      & 1       & Loop animation?\\
\hline
Ping pong & 0       & Ping pong animation?\\
\hline
\end{tabular}


Frame settings\\

\begin{tabular}{|p{2.5cm}|p{2.5cm}|p{9cm}|}
\hline
\textbf{Command} & \textbf{Default} & \textbf{Description}\\
\hline
ID    & 0   & Frame ID\\
\hline
Speed & 0.0 & Speed of this frame (higher value = faster animation)\\
              &&  0.0 = Set value and skip this frame\\
              && < 0.0 = No interpolation\\
\hline
\end{tabular}


Event settings\\

\begin{tabular}{|p{2.5cm}|p{2.5cm}|p{9cm}|}
\hline
\textbf{Command} & \textbf{Default} & \textbf{Description}\\
\hline
FrameID & 0 &
If this frame is reached in the animation the event ID message is send to the entity connected
with the animation. You could use it e.g. to mark frames were a sound should be played.\\
\hline
ID      & 0 & Event ID\\
\hline
\end{tabular}




%----- Section: Procedural textures --------------
\section{Procedural textures}
Prodecural textures are textures which will be created and possibly updated during runtime.
Normally it's usefull to implement them as scene node which controls the procedural texture. Here's
an example how such an procedural texture entity might look:\\

\begin{lstlisting}[caption=Procedural texture class]
class SNProceduralTexture : public PLTSceneNode {
	...
	private:
		float m_fTimer;             // Timer

		// Exported variables
		char  m_szTextureName[64];  // Name of the texture
		int   m_nWidth;             // Texture width
		int   m_nHeight;            // Texture height
		float m_fSpeed;             // Animation speed

	private:
		virtual void InitFunction();
		virtual void DeInitFunction();
		virtual void UpdateFunction();
	...
};

void SNProceduralTexture::InitFunction()
{
	// Init data
	m_fTimer = 0.f;

	// Create texture
	PLTTextureManager::GetInstance()->CreateTexture(m_szTextureName, m_nWidth, m_nHeight);
}

void SNProceduralTexture::DeInitFunction()
{
	PLTTextureManager::GetInstance()->Unload(PLTTextureManager::GetInstance()->Get(m_szTextureName));
}

void SNProceduralTexture::UpdateFunction()
{
	// Animation speed
	m_fTimer += PLTTimer::GetInstance()->GetTimeDifference()*m_fSpeed;
	if (m_fTimer < 1.f) return;
	m_fTimer = 0.f;

	// Create texture
	PLTTexture *pTex = PLTTextureManager::GetInstance()->Get(m_szTextureName);
	if (!pTex) return;
	unsigned char *pData = pTex->GetData();
	if (!pData) return;
	for (int i=0; i<pTex->GetWidth()*pTex->GetHeight(); i++) {
		*pData = *(pData+1) = *(pData+2) = PL::Math.GetRand() % 255;
		pData += 3;
	}

	// Upload new texture
	pTex->Upload();
}
\end{lstlisting}

This example creates a new texture using the texture manager function CreateTexture(). Then from
time to time the texture data is filled up with random color values and is finally uploaded to the GPU.
This would result in a dynamic random noise texture. Dynamic created textures are used like all other normal
textures. E.g. you can load them in a texture handler using Load("Texturename"). But note that
this texture must already exist before you are able load them in a texture handler or material else
this texture will be unknown!\\
In this way you can also create special textures you e.g. need for your shaders. For instance
if you need a normalization cube map you implement an scene node which will create this texture for you
so that you can use it in the materials like all other textures, too! But note that this textures
must be created BEFORE a material is loading them! This is an intuitive way to deal with self calculated
textures without to be forced to handle such textures in a special manner.



%----- Section: Alpha test -----------------------
\section{Alpha test}
Alpha test is one operation in the per-fragment operations stage of the renderer pipeline that
allows any further processing of the fragment to be aborted based on the value of its alpha component. The
fragment's alpha component is compared with an application-specified reference value
using an application-specified comparison function. If the fragment passes the test, it will
be processed by the subsequent fragment operation, otherwise it will be discarded. Alpha
test does not incur any extra overhead even if all the fragments pass the test. Unlike
stencil testing, depth testing, texturing, and blending, alpha testing do not involve
fetching data from memories external to the GPU. Alpha test is only available in RGBA
mode for instance when using tga textures with an alpha channel. You can setup the alpha
test withing the render state section, either directly by hand in the renderer (for programmers)
or in the render state section of a material technique pass. (see the material format in another chapter)

Alpha test provides a great way to save fill rate. When used to draw alpha blended
textures, or to do additive blending in multi-pass, it provides a means to reject the
fragment as early as possible in order to reduce the memory traffic due to stencil, depth,
and color buffer reads and writes.
Consider the case when an application draws big triangles on the screen, like lens
flares or explosions, and those triangles are transparent or blended with additive blending.
Additionally, consider the case of multi-pass rendering, which is common practice to
"shade" models and significantly increases the depth complexity. Typically, one can
render one pass to apply a textured diffuse lighting to a model, render another pass for the
specular highlight. In this second pass, an enormous number of fragments are just black,
due to the nature of the specular computation. We will demonstrate a technique, which
can optimize the application in such situation by mixing the use of alpha test and shaders,
in order to discard the black fragments.
Graphic programmers should reduce the overhead, caused by such rendering
architectures or effects, by utilizing fill rate saving techniques such as alpha test and
other tricks derived from specific hardware functionalities that can take advantage of
alpha test.

Here's a more advanced example when the alpha test can be quite useful to discard the black fragments
when dealing with range-limited lights and shaders:\\
Whenever a fragment is outside the range of the light, you can skip all the lighting math and just return
zero immediately. With pixel shader 3.0 you'd simply implement this with an if-statement. Without pixel
shader 3.0, you just move this check to another shader. This shader returns a truth value to alpha. The
alpha test then kills all unlit fragments. Surviving fragments will write 1 to stencil. In the next pass
you simply draw lighting as usual. The stencil test will remove any fragments that aren't tagged as lit.
If the hardware performs the stencil test prior to shading you will save a lot of shading power. In fact,
the shading workload is reduced to the same level of pixel shader 3.0, or in some cases even slightly below
since the alpha test can do a compare that otherwise would have to be performed in the shader. 
The result is that early-out speeds things up considerably. In theory you are able to to this completely within
the PixelLight material! Imagine a material with two techniques the more advanced one is using pixel shader 3.0 to ignore
unrequired fragments and if this fragment profile isn't available an alternative technique is used with two
passes: The first which will use the alpha test, stencil buffer and a special 'if' shader and the second one drawing
the stuff were it passes the stencil buffer which was set by the pass before.

By tuning appropriately the reference value of alpha test to reject the transparent or almost transparent
fragments, one can improve the application performance significantly. Basically, varying the reference value
acts like a threshold setting up how many fragments are going to be evicted. The more fragments
are being discarded, the faster the application will run. On the other hand, the more
fragment are being discarded, the more texture subtleties are not being drawn, potentially
leading to some undesirable visual side effects. In this particular case, it is crucial to take
into consideration the visual artifacts introduced by setting a reference value threshold
too aggressively. Therefore, setting up alpha test's reference value shouldn't be done
without taking the final visual quality into account.
Because using alpha test can speed up your application, one should consider adding an
explicit alpha channel to textures that are used in a way that burns a lot of fill rate and
that have areas that are not used.
	%[TODO] Update!
\cleardoublepage
%%----- Chapter: High level: Shaders --------------
\chapter{High level: Shaders}


[TODO] Update!



%----- Section: Overview -------------------------
\section{Overview}
In general PixelLight is using the Cg shader language for vertex and fragment shader effects. Have 
a look at e.g. \url{http://www.shadertech.com/} or developer.nvidia.com/cg to learn the basics
about this technology. Because Cg is API and platform independent you can work with Cg functions
directly - therefore have a look at the Cg documentation to see how to use the Cg functions.\\
The PixelLight material is encapsulating this shaders and you haven't to work with them directly. But
because you can do alot with shaders you have also low level access to the shaders to be able
to create e.g. grass effects etc.\\

Vertex/fragment programs will process every vertices/fragments on the GPU. When using such programs
you normally will only work with the provided program parameters. The PixelLight shader system will give the
programmer the ability to use this shaders in the application.\\
The PixelLight shader handler PLTShaderHandler provides an interface with all required functions to work with
such programs. Each individual program needs it's own derived shader handler because each program has
different parameters and functionality! So, if you want to use a certain vertex/fragment program you
have to use it through its own shader handler. The shader handler is the obverse of the program on the GPU.




%----- Section: Shader handler -------------------
\section{Shader handler}
The main shader handler functions are Load(), Bind() and Update(). First you have to load up such
a vertex/fragment program using the Load() function of the shader handler. If the program filename
ends with a '+' it's a fragment program, else a vertex program. The filename ending of the programs
is '.cg'.\\
Then you should update the shader hander each frame to ensure that the internal parameters are
updated correctly. Before rendering anything the shader is activated using the Bind() function and
after the render process is finished it should be deactivated using Unbind().\\
Note that only one vertex and one fragment program can be active at the same time!\\
In the example below two derived shader handlers are used:\\

\begin{lstlisting}[caption=Using shader handler]
// Somewere in e.g. your class
TVertexShaderHandler   m_cVertexShader;
TFragmentShaderHandler m_cFragmentShader;

// Load somewere the vertex and fragment programs
  m_cVertexShader.Load("myVertexShader");
m_cFragmentShader.Load("myFragmentShader+");

// Render process
// Bind shaders
m_cVertexShader.Bind();
m_cFragmentShader.Bind();

// Draw anything

// Unbind shaders
m_cFragmentShader.Unbind();
m_cVertexShader.Unbind();

// Somewere in your code...
// Update shader handlers
m_cFragmentShader.Update();
m_cVertexShader.Update();
\end{lstlisting}




%----- Section: Derive shader handler ------------
\section{Derive shader handler}
To be able to use your own program you have to derive PLTShaderHandler to implement your shader
handler which will manage this the program.\\
After a program was loaded the virtual function CustomLoad() is called. There you should initialize
your handler and getting the program parameters. If the shader handler is bind the function CustomBind()
is called were the program parameters should be set. Finally use CustomUpdate() to update your shader
handler. Example:\\

\begin{lstlisting}[caption=Creating own shader handler]
class TVertexShaderHandler : public PLTShaderHandler {

    // Private data
    private:
        float m_fTime;

    // Programm parameters
    private:
        CGparameter m_ConstantsParam;
        CGparameter m_ModelViewProjParam;
        CGparameter m_ModelViewITParam;
        CGparameter m_ModelViewParam;

    // Virtual PLTShaderHandler functions
    private:
        virtual void CustomLoad();
        virtual void CustomBind();
        virtual void CustomUpdate();


};

void TVertexShaderHandler::CustomLoad()
{
    // Init data
    m_fTime = 0.f;

    // Get program variables
    m_ConstantsParam     = GetCGNamedParameter("Constants");
    m_ModelViewProjParam = GetCGNamedParameter("ModelViewProj");
    m_ModelViewITParam   = GetCGNamedParameter("ModelViewIT");
    m_ModelViewParam     = GetCGNamedParameter("ModelView");
}

void TShaderGrass::CustomBind()
{
    // Set program parameters
             cgGLSetParameter4f(m_ConstantsParam,     m_fTime, 0, 0, 1);
    cgGLSetStateMatrixParameter(m_ModelViewProjParam, CG_GL_MODELVIEW_PROJECTION_MATRIX, CG_GL_MATRIX_IDENTITY);
    cgGLSetStateMatrixParameter(m_ModelViewITParam,   CG_GL_MODELVIEW_MATRIX,            CG_GL_MATRIX_INVERSE_TRANSPOSE);
    cgGLSetStateMatrixParameter(m_ModelViewParam,     CG_GL_MODELVIEW_MATRIX,            CG_GL_MATRIX_IDENTITY);
}

void TShaderGrass::CustomUpdate()
{
    // Update timer
    m_fTime += PL::Timer.GetTimeDifference();
}
\end{lstlisting}




%----- Section: Material shader effects ----------
\section{Material shader effects}
In the shader section above you saw how to use shader effects as programmer directly in your program
by programming their usage by hand for special situations like grass effects were you also have to
create and setup the vertices in the correct way.
Because the vertex/fragment programs are extreme implementation dependent and the programer has to
do the most work graphic artists haven't that many control over them and new shader effects can't be
used without write some new code in your application - except the new effect is using the same parameters
etc. as another effect.\\
Therefore PixelLight has an advanced effect system which is implemented in the material itself to give the
graphic artists more control over the appearance of  surfaces etc - but in reverse the programmer
itself has less control because he isn't able to manipulate this cg programs directly in his code.\\




%----- Section: CgFX -----------------------------
\section{CgFX}
During development graphic artists can use CgFX which also provides the environment the vertex/fragment
programs will work in and is setting its own render states, used textures etc. Further the fx files
provide in contrast to cg files also a description of the program parameters with default values, gui
definitions in annotations to manipulare this values in editors using gui elements etc.
Also different techniques/fallbacks can be defined. The 'fx' file format is nealy the same as the
DirectX effect format and if you write your own programs in there it is identical to Cg or assembly
programs. There are plugins for \emph{Autodesk 3ds Max} etc which enables the graphic artists to develop the
advanced shader effects directly in their well kown development environment! The CgFX 'fx' file format
can be compared with the PixelLight own material file format 'mat', therefore is isn't that difficult to 
convert such CgFX effects artists had developed in their own development environment into the
PixelLight own material format.\\

PixelLight itself doesn't supply direct CgFX support because it doesn't provide a nice programmer
interface and is setting render states etc. automatically without enabling the programmer to find
out what in fact was set. Further CgFX isn't THAT compatible to other graphic cards (e.g. ATI) like
Cg itself which would result in many troubles. Therefore we decided to don't implementet CgFX directly
- but as mentioned above, the PixelLight own material format has nearly the same capability as CgFX
without the compatibility troubles CgFX has!\\

The material itself which can contain shaders is nearly used in the same way as the 'by hand' Cg shaders
are used. But unlike the direct shader interface the material effects provides a more comfortable
parameter interface. Further there are different known parameter semantics which will setup the
shader parameter automatically. (e.g. set current projection matrix) Therefore you mustn't derive
the PLTMaterial to deal with the different shader parameters. Some parameters with kown semantic
will be managed complete automatically and the other you can manipulate in a quite comfortable way
over the interface of the material.\\




%----- Section: Material shader parameters -------
\section{Material shader parameters}
The material shader parameters are split up into two parameter types:\\
- Parameter: Parameters were the semantic is unknown.
             They can be manipulated by hand.\\
- Semantic:  Parameters were the semantic is known.
             They will be updated automatically and shouldn't be manipulated
             by hand.\\




%----- Section: Parameter semantic ---------------
\section{Parameter semantic}
Here's a list of all known parameter semantic:\\

 \begin{tabular}{|p{7cm}|p{7cm}|}
\hline
\textbf{Parameter semantic} & \textbf{Description}\\
\hline
World               & World matrix (float4x4)\\
\hline
WorldI	            & World inverse matrix (float4x4)\\
\hline
WorldIT	            & World inverse transpose matrix (float4x4)\\
\hline
WorldView           & World view matrix (float4x4)\\
\hline
WorldViewI          & World view inverse matrix (float4x4)\\
\hline
WorldViewIT         & World view inverse transpose matrix (float4x4)\\
\hline
WorldViewProjection & World view projection matrix (float4x4)\\
\hline
Projection	        & Projection matrix (float4x4)\\
\hline
View	            & View matrix (float4x4)\\
\hline
ViewIT	            & View inverse transpose matrix (float4x4)\\
\hline
EyePos	            & Eye position (object space, float4)\\
\hline
EyeVector           & Eye direction vector (object space, float3)\\
\hline
EyePosWorld	        & Eye position (world space, float4)\\
\hline
EyeVectorWorld      & Eye direction vector (world space, float3)\\
\hline
Time	            & Timer (float)\\
\hline
Shininess	        & Shininess (float)\\
\hline
Color	            & Color (float4)\\
\hline
AmbientColor        & Ambient color (float4)\\
\hline
DiffuseColor        & Diffuse color (float4)\\
\hline
SpecularColor       & Specular color (float4)\\
\hline
EmissionColor       & Emission color (float4)\\
\hline
Texture<n>          & Texture 0-7 (texture)\\
\hline
TextureMatrix<n>    & Texture matrix 0-7\\
\hline
\end{tabular}
	%[TODO] Update!
\cleardoublepage
%%----- Chapter: High level: Effects --------------
\chapter{High level: Effects}


[TODO] Update!


%----- Section: Overview -------------------------
\section{Overview}
In this section we will explain you how to use the PixelLight material in practice. A material 
is a simple text format which describes the settings of a material - skipped things are set to
default automatically. A text editor is enough to develop materials but theres also a materal
editor provided.



%----- Section: Introduction ---------------------
\section{Introduction}
Advanced material effects which will increase the visual quality of the surfaces is a must have
for modern graphics engines. Therefore PixelLight provides an extensive material system which enables you
to use the capabilities of modern GPU's to create amazing visual effects.\\

This PixelLight material guide will teach you what in detail such a material is, how to use and
build it - further some background information about e.g. what a 3D texture is are provided, but for
detailed information you have to use other documents you can for instance find in the internet. This document
is for both programmers and graphic artists because a material is a complex
and powerful surface description and it's also possible to programm byself advanced shaders to
control how each vertex and fragment is processed by GPU. In this section we will tell you some
backgrounds about the materials to give you an overview over this topic. In section 2 we will talk
about textures because this is the base of the materials. You will learn in section 3 how to use
materials in practice. Section 4 consists of a material format description. In the section
5 we will talk about shaders. Section 6 is a faq.\\

A PixelLight material is a powerfull surface description. It will manage the different render techniques
automatically (also called 'fallbacks') for you so that in fact you don't need to know which technique is
currently used to visualize a surface. Just apply a material to a surface and have fun! ;-)\\
Here's a diagram which shows how the PixelLight material is build up in general:\\

\begin{lstlisting}[caption=PixelLight material system overview]
                  material
                     |
          n techniques per material
                     |
            n passes per technique
                     |
           ----------------------
           |                    |
  n layers per pass      one vertex/fragment shader per pass
\end{lstlisting}

Techniques means that there are different ways how the material can be rendered - in this relation we 
also call this an 'effect'. If you are already familar with e.g. the Direct3D effect file format (fx) it will be
easy for you to understand the PixelLight material. Using advanced effects is an important thing because there
are fallbacks which enables you to use a simpler technique if the hardware or performance is insufficient or for 
material LOD. Such a technique can consist of different render passes which means that the material is rendered 
several times with different settings increasing the visual quality. In such a pass, render states, sampler states 
etc. are set to render the surface in the requested way.
Many render passes per material will reduce the performance because the geometrie the material
is applied to have to be drawn multiple times - so use it wise. Each of such a pass can have that
many texture layers the hardware is supporting. Modern graphic cards normally support 4 direct texture
layers and up to 16 and more layers when using shaders accessing this layers. Using such layers you can
blend together different textures without be forced to redraw the geometry which will blow up the
polycount and reduce problems like z-fighting. Further each layer can have another texture with other properties... 
you can also animate each textur using different changing subtextures to create a cartoon like animation or rotate, 
move etc the texture by changing it's texture matrix or using all this texture animation effects at the SAME TIME
to create amazing materials full of life! :)\\

An advanced material rendering control is supplied through the vertex and fragment shader you can apply
to each pass. A shader is a program running on the GPU. It will 'overwrite' the API (e.g. OpenGL)
fixpass functions which normally process vertices (transformations etc.) and fragments. (get the color
of each fragment etc.) Therefore when using a vertex shader you have to transform the vertices by self
or when using a fragment shader calculating the single fragment colors by hand. In this case the
different pass layers will only feed the active shaders with e.g. the desired textures but not combining
different texture layers automatically. Further you have to consider things like texture matrix animations
in your shaders when you want to use it - in contrast to the fixpass some things don't done automatically!
Note that per pass there's only ONE vertex and ONE fragment active shader! But for instance you can
combine different passes to a final result. It's a lot of work to write such materials with individual
shader effects but therefore you have FULL CONTROL over how it is rendered and you can create impressive
and unique material effects! If it is too much work for you to do all by self you can use the 'standard'
fixpass functions which which will also product good results in less development time - but you haven't
the full freedom shaders offering you.\\

If you don't specify the material format by the filename extension, the framework will search for the material
by adding a filename extension. If a material (*.mat) with the filename isn't found the framework will seach for a 
texture with this filename by adding a texture filename extension like 'bmp'. Therefore the simples
material would be a single texture without any additional settings. Have a look on the texture section
for more details.



%----- Section: Solid and transparent materials --
\section{Solid and transparent materials}
There are two kinds of materials. Solid and blended materials. In order to ensure that blended
materials are rendered in the right way they have to rendered after all other solid materials.
Therfore you have to denote when the material should be rendered. This is done using the GetBlend()
and SetBlend() functions of PLTMaterial. This only a general setting whether the material is blend
in in any pass and doesn't mean that the material is in fact blend - it only denotes when to
render it!




%----- Section: Using materials ------------------
\section{Using materials}
When using PLTMaterial normally you only will use a few functions. Load() to load a material,
Bind() to activate it, SetupPass() so setup the given render pass and Unbind() if rendering using the
material is finished.\\

\begin{lstlisting}[caption=Using materials]
// Somewhere in e.g. your class
PLTMaterial m_cMaterial;

// Load up your material (default extension is 'mat')
m_cMaterial.Load("MyMaterial");

// Rendering
m_cMaterial.Bind(); // Activate material
for (int nPass=0; nPass<m_cMaterial.GetNumOfPasses(); nPass++) {
    m_cMaterial.SetupPass(nPass); // Setup the current render pass
    // Render something
}
m_cMaterial.Unbind(); // Deactivate material
\end{lstlisting}

You don't have to care about how exactly the material is defined internally like if it is using
normal fixpass functions or even advanced shader effects. But you are able to check and change ALL
the settings during runtime!
When loading a material using the Load() function there are different types of possible useable
material data.\\
- A simple texture like a jpg texture, all other material settings will be set to default.\\
- An PixelLight texture animation file. (tani) This is handled in the same way as a simple texture
  but with additonal features like texture rotation.\\
- A PixelLight material which is the default file type with the filename extension 'mat' which is
  added automatically. Within this material file format you are able to create a material with your
  desired properties.\\

As you see the PixelLight material is a extreme flexible and easy useable surface description which will do
all the dirty work for you. The material has an extensive interface which give you more control
over the material during runtime if required.




%----- Section: Material format ------------------
\section{Material format}




%----- Subsection: Overview ----------------------
\subsection{Overview}
A material file is a normal text file were you define the material itself. Unrequired commands can
be skipped - standart values will be used instead. The material format is split up into 3 main sections:\\
- General: General material settings\\
- Parameters: Shader parameter settings\\
- Techniques: There can be different technique blocks. Each of this blocks consitst of different pass blocks
  were you define how the material is rendered in the pass.\\



%----- Subsection: General settings --------------
Boolean settings:\\
For boolean settings you can use 0/false and 1/true.\\

-1 for the default setting means that this setting isn't set by the material and
normally shouldn't set by at material at all because this are basis renderer settings.

Comparison functions:\\
\begin{itemize}
\item{Never        = Never passes}
\item{Less         = Passes if the incoming value is less than the stored value}
\item{Equal        = Passes if the incoming value is equal to the stored value}
\item{LessEqual    = Passes if the incoming value is less than or equal to the stored value}
\item{Greater      = Passes if the incoming value is greater than the stored value}
\item{NotEqual     = Passes if the incoming value is not equal to the stored value}
\item{GreaterEqual = Passes if the incoming value is greater than or equal to the stored value}
\item{Always       = Always passes}
\end{itemize}


Stencil operations:\\
\begin{itemize}
\item{Keep     = Keeps the current value}
\item{Zero     = Sets the stencil buffer value to zero}
\item{Replace  = Sets the stencil buffer value to ref, as specified by StencilRef}
\item{Incr     = Increments the current stencil buffer value. Clamps to the maximum representable unsigned value.}
\item{Decr     = Decrements the current stencil buffer value. Clamps to zero.}
\item{IncrWrap = Increments the current stencil buffer value. Wraps the result.}
\item{DecrWrap = Decrements the current stencil buffer value. Wraps the result.}
\item{Invert   = Bitwise inverts the current stencil buffer value}
\end{itemize}


Texture-addressing modes:\\
\begin{itemize}
\item{Clamp  = Texture coordinates outside the range [0.0, 1.0] are set to the texture color
               at 0.0 or 1.0, respectively.}
\item{Border = Texture coordinates outside the range [0.0, 1.0] are set to the border color.}
\item{Wrap   = Tile the texture at every integer junction. For example, for u values between
               0 and 3, the texture is repeated three times; no mirroring is performed.}
\item{Mirror = Similar to 'Wrap', except that the texture is flipped at every integer junction.
               For u values between 0 and 1, for example, the texture is addressed normally;
               between 1 and 2, the texture is flipped (mirrored); between 2 and 3, the texture is normal again, and so on.}
\end{itemize}
  

Texture filtering modes:\\
\begin{itemize}
\item{None        = Mipmapping disabled. The rasterizer should use the magnification filter instead.}
\item{Point       = Point filtering used as a texture magnification or minification filter. The texel
                    with coordinates nearest to the desired pixel value is used. The texture filter to
                    be used between mipmap levels is nearest-point mipmap filtering. The rasterizer uses
                    the color from the texel of the nearest mipmap texture.}
\item{Linear      = Bilinear interpolation filtering used as a texture magnification or minification filter.
                    A weighted average of a 2x2 area of texels surrounding the desired pixel is used. The
                    texture filter to use between mipmap levels is trilinear mipmap interpolation. The
                    rasterizer linearly interpolates pixel color, using the texels of the two nearest mipmap textures.}
\item{Anisotropic = Anisotropic texture filtering used as a texture magnification or minification filter.
                    Compensates for distortion caused by the difference in angle between the texture polygon
                    and the plane of the screen.}
\end{itemize}


Texture envionment modes:\\
\begin{itemize}
\item{Add                 = Add incoming color with the existing one}
\item{Replace             = Replace existing color with the incoming one}
\item{Modulate            = Modulate colors}
\item{PassThru            = Pass through incoming color}
\item{Dot3                = Dot3}
\item{Interpolate         = Interpolate}
\item{InterpolatePrimary  = Interpolate primary}
\item{InterpolateTexAlpha = Interpolate tex alpha}
\end{itemize}



%----- Subsection: Example -----------------------
\subsection{Example}
Below you will find a full material file example were you can also see the default values:\\
\verbatiminput{examples/example.mat}




%----- Subsection: General -----------------------
\subsection{General}
In this block are general material settings like flags, whether a material should be blend
or not etc.\\

\begin{tabular}{|p{2.5cm}|p{2.5cm}|p{9cm}|}
\hline
\textbf{Command} & \textbf{Default} & \textbf{Description}\\
\hline
Flags="" & 0 & String of material flags (e.g. "1|8|32")\\
\hline
Blend=0  & 0 & Should the material be blend or not?\\
\hline
\end{tabular}




%----- Subsection: Parameters --------------------
\subsection{Parameters}
Shader parameter settings




%----- Subsection: Technique ---------------------
\subsection{Technique}



%----- Subsubsection: RenderStates ---------------
\subsubsection{RenderStates}
Render states\\

Modes:\\
\begin{tabular}{|p{4.5cm}|p{3cm}|p{9cm}|}
\hline
\textbf{Command} & \textbf{Default} & \textbf{Description}\\
\hline
FillMode  & Solid  & Point = Point fill mode\newline
                     Line  = Line fill mode\newline
                     Solid = Solid fill mode\\
\hline
ShadeMode & Smooth & Flat   = No interpolated during rasterizing\newline
                     Smooth = Interpolated during rasterizing\\
\hline
CullMode  & CCW    & None = No culling\newline
                     CW	  = Clockwise culling\newline
                     CCW  = Counterclockwise culling\\
\hline
\end{tabular}


ZBuffer:\\
\begin{tabular}{|p{4.5cm}|p{3cm}|p{9cm}|}
\hline
\textbf{Command} & \textbf{Default} & \textbf{Description}\\
\hline
ZEnable              & true      & false = No Z/depth buffer test\newline
                                   true  = Z/depth buffer test\\
\hline
ZWriteEnable         & true      & false = Do not write into the Z/depth buffer\newline
                                   true  = Write into the Z/depth buffer\\
\hline
ZFunc                & LessEqual & Z/depth buffer comparison function\newline
                                   See 'Comparison functions'\\
\hline
ZBias                & 0.0       & Depth bias/polygon offset units, <0 = towards camera (e.g. -0.001)\newline
                                   Because SlopeScaleDepthBias and DepthBias below are API and 
                                   GPU dependent, their results are NOT the same on each system \& API. Whenever possible, do NOT use 
                                   this 'classic' render states, use ZBias instead. If this state is not null, the renderer 
                                   will automatically manipulate the internal projection matrix to perform an 'z bias' which is more 
                                   predictable as the 'classic' polygon offset.\\
\hline
SlopeScaleDepthBias  & 0.0       & Slope scale bias/polygon offset factor (e.g. -1.0)\\
\hline
DepthBias            & 0.0       & Depth bias/polygon offset units (e.g. -2.0)\\
\hline
\end{tabular}

Note to SlopeScaleDepthBias and DepthBias:\\
Normally there are horrible artefacts when renderning (nearly) co-planar primitives.
To reduce this 'z fighting' you can set 'factor' (default 0) and 'units' (default 0) to
e.g. factor=-1 and units=-2.
Then each fragment's depth value will be offset after it is interpolated from the
depth values of the appropriate vertices. The value of the offset is factor*DZ+r*units,
where DZ is a measurement of the change in depth relative to the screen area of the
polygon, and r is the smallest value that is guaranteed to produce a resolvable offset
for a given implementation. The offset is added before the depth test is performed and
before the value is written into the depth buffer.\\
This is useful for rendering hidden-line images, for applying decals to
surfaces, and for rendering solids with highlighted edges.\\
Notes: - Has no effect on depth coordinates placed in the feedback buffer\\
       - Has no effect on selection\\


AlphaTest:\\
\begin{tabular}{|p{4.5cm}|p{3.5cm}|p{9cm}|}
\hline
\textbf{Command} & \textbf{Default}  & \textbf{Description}\\
\hline
AlphaTestEnable      & false         & false = Do not perform alpha test\newline
                                       true  = Perform alpha test\\
\hline
AlphaFunc            & GreaterEqual  & See 'Comparison functions'\\
\hline
AlphaRef             & 0.5           & Alpha test reference value\\
\hline
\end{tabular}\\

Notes:\\
- Using the alpha test you can create semi transparent textures\\
- An alpha test will normally (when not using for instance shaders :) only work if the first texture
has an alpha channel (RGBA)\\
- For more information about the alpha test have a look at a previous chapter!


Blend:\\
Specifies pixel blend arithmetic\\

In RGB mode, pixels can be drawn using a function that blends the incoming (source)
RGBA values with the RGBA values that are already in the frame buffer (the destination values).

The src parameter specifies which of nine methods is used to scale the source color components.
The dest parameter specifies which of eight methods is used to scale the destination color
components. The eleven possible methods are described in the following table.
Each method defines four scale factors, one each for red, green, blue, and alpha.

In the table and in subsequent equations, source and destination color components are referred
to as (R(s), G(s), B(s), A(s)) and (R(d), G(d), B(d), A(d)). They are understood to
have integer values between zero and (k(R), k(G), k(B), k(A)), where
\(k (c) = 2^m (c) - 1\)
and (m(R), m(G), m(B), m(A)) is the number of red, green, blue, and alpha bitplanes.

Source and destination scale factors are referred to as (s(R), s(G), s(B), s(A)) and
(d(R), d(G), d(B), d(A)). The scale factors described in the table, denoted
(f(R), f(G), f(B), f(A)), represent either source or destination factors.
All scale factors have range [0,1].\\

\begin{tabular}{|p{4.5cm}|p{9cm}|}
\hline
\textbf{Parameter} & \textbf{(f(R), f(G), f(B), f(A))}\\
\hline
Zero        & (0, 0, 0, 0)\\
One         & (1, 1, 1, 1)\\
SrcColor    & (R(s)/k(R), G(s)/k(G), B(s)/k(B), A(s)/k(A))\\
InvSrcColor & (1, 1, 1, 1)\\
SrcAlpha    & (R(d)/k(R), G(d)/k(G), B(d)/k(B), A(d)/k(A))\newline
               A(s)/k(A), A(s)/k(A), A(s)/k(A), A(s)/k(A))\\
InvSrcAlpha & (1, 1, 1, 1)\newline
              (A(s)/k(A), A(s)/k(A), A(s)/k(A), A(s)/k(A))\\
SrcAlphaSat & (i, i, i, 1)\\
DstColor    & (R(d)/k(R), G(d)/k(G), B(d)/k(B), A(d)/k(A))\\
InvDstColor & (1, 1, 1, 1)\\
DstAlpha    & (A(d)/k(A), A(d)/k(A), A(d)/k(A), A(d)/k(A))\\
InvDstAlpha & (1, 1, 1, 1)\newline
              (A(d)/k(A), A(d)/k(A), A(d)/k(A), A(d)/k(A))\\
\hline
\end{tabular}

In the table:\\
\( i = min( A(s), k(A)-A(d) ) / kA \)\\
\pagebreak

To determine the blended RGBA values of a pixel when drawing in RGB mode,
the system uses the following equations:\\
\( R(d) = min(kR, RssR+RddR) \)\\
\( G(d) = min(kG, GssG+GddG) \)\\
\( B(d) = min(kB, BssB+BddB) \)\\
\( A(d) = min(kA, AssA+AddA) \)\\

Despite the apparent precision of the above equations, blending arithmetic is not exactly specified,
because blending operates with imprecise integer color values. However, a blend factor that should
be equal to one is guaranteed not to modify its multiplicand, and a blend factor equal to zero
reduces its multiplicand to zero. Thus, for example, when sfactor is 'src\_alpha', dfactor is
'inv\_src\_alpha', and A(s) is equal to k(A), the equations reduce to simple replacement:\\
\( R (d) = R (s) \)\\
\( G (d) = G (s) \)\\
\( B (d) = B (s) \)\\
\( A (d) = A (s) \)\\

\begin{tabular}{|p{4.5cm}|p{3cm}|p{9cm}|}
\hline
\textbf{Command} & \textbf{Default} & \textbf{Description}\\
\hline
BlendEnable  & false    & false = Disable blending\newline
                          true  = Enable blending\\
\hline
SrcBlendFunc & SrcAlpha & Source blend function\\
\hline
DstBlendFunc & One      & Destination blend function\\
\hline
\end{tabular}
\\

Stencil:\\
\begin{tabular}{|p{4.5cm}|p{3cm}|p{9cm}|}
\hline
\textbf{Command} & \textbf{Default} & \textbf{Description}\\
\hline
StencilEnable       & false      & false = Disable stencil test\newline
                                   true  = Enable stencil test\\
\hline
StencilFunc         & Always     & Stencil test passes if ((ref \& mask) stencilfn (stencil \& mask)) is true\newline
                                   See 'Comparison functions'\\
\hline
StencilRef          & 0          & Reference value used in stencil test\\
\hline
StencilMask         & 0xFFFFFFFF & Mask value used in stencil test\\
\hline
StencilFail         & Keep       & Operation to perform if stencil test fails\newline
                                   See 'Stencil operations'\\
\hline
StencilZFail        & Keep       & Operation to perform if stencil test passes and Z test fails\newline
                                   See 'Stencil operations'\\
\hline
StencilPass         & Keep       & Operation to perform if both stencil and Z tests pass\newline
                                   See 'Stencil operations'\\
                             
\hline
TwoSidedStencilMode & false      & false = Disable two sided stencil test\newline
                                   true  = Enable two sided stencil test\\
\hline
CCWStencilFunc      & Always     & CCW stencil test passes if ((ref \& mask) stencilfn (stencil \& mask)) is true\newline
                                   See 'Comparison functions'\\
\hline
CCWStencilFail      & Keep       & CCW operation to perform if stencil test fails\newline
                                   See 'Stencil operations'\\
\hline
CCWStencilZFail     & Keep       & CCW operation to perform if stencil test passes and Z test fails\newline
                                   See 'Stencil operations'\\
\hline
CCWStencilPass      & Keep       & CCW operation to perform if both stencil and Z tests pass\newline
                                   See 'Stencil operations'\\
\hline
\end{tabular}


Fog: (fix pass relevant)\\
\begin{tabular}{|p{4.5cm}|p{3cm}|p{9cm}|}
\hline
\textbf{Command} & \textbf{Default} & \textbf{Description}\\
\hline
FogEnable  & false & false = Disable fog\newline
                     true  = Enable fog\\
\hline
FogColor   & 0     & Fog color (RGBA)\\
\hline
FogDensity & 1.0   & Fog densitity\\
\hline
FogStart   & 0.0   & Fog start\\
\hline
FogEnd     & 1.0   & Fog end\\
\hline
FogMode    & Exp   & Exp    = Fog effect intensifies exponentially\newline
                     Exp2   = Fog effect intensifies exponentially with the square of the distance\newline
                     Linear = Fog effect intensifies linearly between the start and end points\\
\hline
\end{tabular}


Point sprite:\\
\begin{tabular}{|p{4.5cm}|p{3cm}|p{9cm}|}
\hline
\textbf{Command} & \textbf{Default} & \textbf{Description}\\
\hline
PointSize        & 1.0   & Point size when it is not specified for each vertex. This value is in screen space units 
                           if 'PointScaleEnable' is 'false'; otherwise this value is in world space units.\\
\hline
PointScaleEnable & false & false = Disable point sprite scale\newline
                           true  = Enable point sprite scale\newline
                           Controls computation of size for point primitives. When 'true', the point size is interpreted 
                           as a camera space value and is scaled by the distance function and the frustum to viewport 
                           y-axis scaling to compute the final screen-space point size. When 'false', the point size is 
                           interpreted as screen space and used directly.\\
\hline
PointSizeMin     & 1.0   & Minimum size of point primitives\\
\hline
PointSizeMax     & 64.0  & Maximum size of point primitives, must be greater than or equal to 'PointSizeMin'\\
\hline
PointScaleA      & 1.0   & Controls for distance-based size attenuation for point primitives\\
\hline
PointScaleB      & 0.0   & Controls for distance-based size attenuation for point primitives\\
\hline
PointScaleC      & 0.0   & Controls for distance-based size attenuation for point primitives\\
\hline
\end{tabular}


PN triangles: (TRUFORM)\\
\begin{tabular}{|p{4.5cm}|p{3cm}|p{9cm}|}
\hline
\textbf{Command} & \textbf{Default} & \textbf{Description}\\
\hline
PNTrianglesEnable           & false     & false = Disable PN triangles\newline
                                          true  = Enable PN triangles\\
\hline
PNTrianglesPointMode        & Cubic     & PN triangles point model\newline
                                          Linear = Linear\newline
                                          Cubic  = Cubic\\
\hline
PNTrianglesNormalMode       & Quadratic & PN triangles normal model\newline
                                          Linear    = Linear\newline
                                          Quadratic = Quadratic\\
\hline
PNTrianglesTesselationLevel & 0         & PN triangles tesselation level\\
\hline
\end{tabular}
      

Misc:\\
\begin{tabular}{|p{4.5cm}|p{3cm}|p{9cm}|}
\hline
\textbf{Command} & \textbf{Default} & \textbf{Description}\\
\hline
PointSpriteEnable & false  & false = Disable point sprite\newline
                             true  = Enable point sprite\\
DitherEnable      & false  & false = Disable dithering\newline
                             true  = Enable dithering\\
\hline
ScissorTestEnable & false  & false = Disable scissor test\newline
                             true  = Enable scissor test\\
\hline
Lighting          & true   &  false = Disable lighting\newline
                              true  = Enable lighting\newline
                              (fix pass relevant)\\
Ambient           & 0      &  General RGBA ambient color\newline
                              (fix pass relevant)\\
\hline
NormalizeNormals  & true   & false = Disable normalize normals\newline
                             true  = Enable normalize normals\\
\hline
InvCullMode       & false  & false = Use default cull mode\newline
                             true  = Use invert cull mode\\
\hline
FixedFillMode    & Unknown & If not 'Unknown' this will 'overwrite' 'FillMode'\\
\hline
\end{tabular}

        

%----- Subsubsection: Material -------------------
\subsubsection{Material}
Material settings\\
	
\begin{tabular}{|p{2.5cm}|p{2.5cm}|p{9cm}|}
\hline
\textbf{Command} & \textbf{Default} & \textbf{Description}\\
\hline
Shininess & 0 & Specifies the RGBA specular exponent of the material. Only values in the range [0,128] are accepted.\\
\hline
\end{tabular}


Color block:\\
General material color\\

\begin{tabular}{|p{2.5cm}|p{2.5cm}|p{9cm}|}
\hline
\textbf{Command} & \textbf{Default} & \textbf{Description}\\
\hline
R & 1.0 & Red color component (0-1)\\
G & 1.0 & Green color component (0-1)\\
B & 1.0 & Blue color component (0-1)\\
A & 1.0 & Alpha color component (0-1)\newline
		  Only used if the material is transparent\\
\hline
\end{tabular}


AmbientColor block:\\
Specifies the ambient RGBA reflectance of the material\\

\begin{tabular}{|p{2.5cm}|p{2.5cm}|p{9cm}|}
\hline
\textbf{Command} & \textbf{Default} & \textbf{Description}\\
\hline
R & 0.2 & Red color component (0-1)\\
G & 0.2 & Green color component (0-1)\\
B & 0.2 & Blue color component (0-1)\\
A & 1.0 & Alpha color component (0-1)\\
\hline
\end{tabular}


DiffuseColor block:\\
Specifies the diffuse RGBA reflectance of the material\\

\begin{tabular}{|p{2.5cm}|p{2.5cm}|p{9cm}|}
\hline
\textbf{Command} & \textbf{Default} & \textbf{Description}\\
\hline
R & 0.8 & Red color component (0-1)\\
G & 0.8 & Green color component (0-1)\\
B & 0.8 & Blue color component (0-1)\\
A & 1.0 & Alpha color component (0-1)\\
\hline
\end{tabular}


SpecularColor block:\\
Specifies the specular RGBA reflectance of the material\\

\begin{tabular}{|p{2.5cm}|p{2.5cm}|p{9cm}|}
\hline
\textbf{Command} & \textbf{Default} & \textbf{Description}\\
\hline
R & 0.0 & Red color component (0-1)\\
G & 0.0 & Green color component (0-1)\\
B & 0.0 & Blue color component (0-1)\\
A & 1.0 & Alpha color component (0-1)\\
\hline
\end{tabular}


EmissionColor block:\\
Specifies the RGBA emitted light intensity of the material\\

\begin{tabular}{|p{2.5cm}|p{2.5cm}|p{9cm}|}
\hline
\textbf{Command} & \textbf{Default} & \textbf{Description}\\
\hline
R & 0.0 & Red color component (0-1)\\
G & 0.0 & Green color component (0-1)\\
B & 0.0 & Blue color component (0-1)\\
A & 1.0 & Alpha color component (0-1)\\
\hline
\end{tabular}




%----- Subsubsection: Layer ----------------------
\subsubsection{Layer}
Texture layer\\
The number of texture layers is only limited by hardware, normally there are at least 2 texture
units but modern hardware has at least 4. If more layers are defined as supported, they will be ignored.\\


Texture: (default: "")\\
Texture filename.\\


SamplerStates block:\\
AddressModes:\\
\begin{tabular}{|p{2.5cm}|p{2.5cm}|p{9cm}|}
\hline
\textbf{Command} & \textbf{Default} & \textbf{Description}\\
\hline
AddressU & Wrap & See 'Texture-addressing modes'\\
\hline
AddressV & Wrap & See 'Texture-addressing modes'\\
\hline
AddressW & Wrap & See 'Texture-addressing modes'\\
\hline
\end{tabular}


Filters:\\
\begin{tabular}{|p{2.5cm}|p{2.5cm}|p{9cm}|}
\hline
\textbf{Command} & \textbf{Default} & \textbf{Description}\\
\hline
MagFilter     & Linear & See 'Texture filtering modes'\\
\hline
MinFilter     & Linear & See 'Texture filtering modes'\\
\hline
MipFilter     & Linear & See 'Texture filtering modes'\\
\hline
MipMapLodBias & 0.0    & See 'MipMapLodBias' below this table\\
\hline
MaxMapLevel   & 1000.0 & Maximum map level\\
\hline
MaxAnisotropy & 1      & Maximum anisotropy\\
\hline
\end{tabular}


MipMapLodBias:\\
The renderer API computes a texture level-of-detail parameter, called lambda
in the specification, that determines which mipmap levels and
their relative mipmap weights for use in mipmapped texture filtering.\\

This extension provides a means to bias the lambda computation
by a constant (signed) value.  This bias can provide a way to blur
or pseudo-sharpen standard texture filtering.\\

This blurring or pseudo-sharpening may be useful for special effects
(such as depth-of-field effects) or image processing techniques
(where the mipmap levels act as pre-downsampled image versions).
On some implementations, increasing the texture lod bias may improve
texture filtering performance (at the cost of texture bluriness).\\


TextureStageStates:\\
\begin{tabular}{|p{2.5cm}|p{2.5cm}|p{9cm}|}
\hline
\textbf{Command} & \textbf{Default} & \textbf{Description}\\
\hline
ColorTexEnv & Modulate & See 'Texture envionment modes'\\
\hline
AlphaTexEnv & Modulate & See 'Texture envionment modes'\\
\hline
TexGen      & None     & None          = No texture coordinate generation (passthru)\newline
                         ObjectLinear  = Object linear\newline
                         EyeLinear     = Eye linear\newline
                         ReflectionMap = Reflection map\newline
                         NormalMap     = Normal map\newline
                         SphereMap     = Sphere map\newline\\
\hline
\end{tabular}



%----- Subsubsection: Shader ---------------------
\subsubsection{Shader}
You are able to use vertex and fragment shaders:\\
\begin{tabular}{|p{2.5cm}|p{2.5cm}|p{9cm}|}
\hline
\textbf{Command} & \textbf{Default} & \textbf{Description}\\
\hline
Profile  & 'best available' & Minimum profile requirement (e.g. 'arbvp1')\\
\hline
Filename & ""               & Shader filename (e.g. 'myshader.cg')\\
\hline
\end{tabular}
	%[TODO] Update!
\cleardoublepage
%%----- Chapter: High level: FAQ ------------------
\chapter{High level: FAQ}


[TODO] Update!



%----- Section: Overview -------------------------
\section{Overview}
This FAQ answers general questions on the fly without mentioning EACH detail - for more information
about several topics have a look at the detailed material documentation above.\\




%----- Section: FAQ ------------------------------
\section{FAQ}


\emph{How can I setup additional per texture settings like if this texture can be compressed or not?}\\
This can be set in the plt-file for each texture. Example: If your texture is 'MyTexture.bmp' the
corresponding plt file is 'MyTexture.plt' which is in the same directory. A plt file is a normal
text file were you can write in several special per texture settings.\\


\emph{Which texture formats PixelLight can load in?}\\
bmp, cut, ico, jpg, pcx, tif, png, tga, dds, psd, hdr, exr\\


\emph{Which texture formats PixelLight can save?}\\
bmp, jpg, pcx, png, pnm, raw, sgi, tga, tif, pal, hdr, exr\\


\emph{How can I foce a texture to never use texture compression?}\\
This can be set in the plt-file in the general block using Compression=0 to define that this
texture shoult NEVER use texture compression. Example:\\

\begin{lstlisting}[caption=plt-file disable texture compression]
<?xml version="1.0"?>
<Texture>
	<General Compression="0" />
</Texture>
\end{lstlisting}

... to disable the texture compression for this texture. Note that texture compression will
increase the performance and save memory!\\


\emph{How to use textures which are non power of two for e.g. ingame menus?}\\
Non power of two textures are a special type of texture which is also called 'rectangle texture'
because the graphic cards are optimized on power of two textures you should only use such
textures in special situations like for menus. In the plt file you have to setup:\\

\begin{lstlisting}[caption=plt-file rectangle texture]
<?xml version="1.0"?>
<Texture>
	<General Rectangle="1" />
</Texture>
\end{lstlisting}

... to mark this texture as rectangle texture.\\


\emph{What is the maximum possible texture size?}\\
The maximum possible texture size depends und the available hardware and the texture type.
(2D, 3D, rectangle, cube map etc.)
A standart maximum texture size of 2D textures today is 2048x2048, but the most cards can use
textures with a size up to 4096x4096. If a texture is to large for the available hardware the
texture is scaled down automatically. Example: A GeForce4 Ti4200 has a maximum texture size of
4096x4096 and 4 texture layers. An Radion 9600 Mobile a maximum texture size of 2048x2048 and
8 texture layers.\\


\emph{How to use cube maps?}\\
Cube maps are a special type of textures. In fact a cube map is build up of 6 different textures -
one for each cube side. The cub map ifself is used like the other normal textures. Whether a
texture is a cube map or not is defined in the texture plt-file:\\

\begin{lstlisting}[caption=plt-file cube map]
<?xml version="1.0"?>
<Texture>
	<General CubeMap="1" />
</Texture>
\end{lstlisting}

The other 5 textures will be loaded automatically using texture filename + ID. For example if
'Cube.jpg' has a plt-file with says that this is a cube map the other texture files 'Cube1.jpg',
'Cube2.jpg'... 'Cube5.jpg' will be loaded automatically. The order is x-positive (0), x-negative (1),
y-positive (2), y-negative (3), z-positive (4), z-negative(5).\\


\emph{How to create semi-transparent textures like a grate texture you can look through certain parts?}\\
First at all you need a alpha mask which says whats transparent and how much.
There are two different ways to do this. For instance you can put an alpha channel into your
texture (rgba) - for instance using a tga texture. Or you can define a color key for the texture
using the plt format by writing:\\

\begin{lstlisting}[caption=plt-file color key]
<?xml version="1.0"?>
<Texture>
	<ColorKey R="0" G="0" B="0" Tolerance="0" />
</Texture>
\end{lstlisting}

... were in this example all texel which are black (0/0/0) should be transparent. Internally a
alpha channel for the texture is created using the color key. We recommend to e.g. use a tga
texture to supply a alpha channel because there you have more control over it - there you can
also control HOW transparent a texel is.\\
    
When using the texture in a material you have to activate the alpha test for the material pass
using the semi-transparent texture. Example:\\

\begin{lstlisting}[caption=Semi-transparent material]
<?xml version="1.0"?>
<Material>
	<Technique Name="Semi transparent">
		<Pass Name="Pass0">
			<RenderStates AlphaTestEnable="1" />
			<Layer>
				<Texture>MyTexture.tga</Texture>
			</Layer>
		</Pass>
	</Technique>
</Material>
\end{lstlisting}

In the AlphaTest block you can also setup the used alpha test function which is by default 'greaterequal' 
which means that pass if the incomming alpha value is greater than or equal to the reference value.
(default 0.5 but can also be set within this block)\\


\emph{How to create texture animations like rotating textures or textures with several animation frames?}\\
A texture animation is defined in a tani file which is a normal text file. You define a texture
animation in a tani file and then you have to use it like all other textures. (bmp etc.) PixelLight
will playback etc. your animation automatically when the texture is used. Note that using a material
you can create amazing effects when using different animated texture layers. :)\\


\emph{What are materials and how to use them?}\\
A material is a powerfull surface description which controls how surfaces will look. You can define
materials in mat files which simple text files. A material combines different texture layers, shaders,
render state settings etc. Because this is a more extensive topic we recommend you to read the material
documentation. Note that using material and shaders you can create nearly each effect you want to have
and you have a huge degree of freedom when implementing your material effects!\\


\emph{I HATE writing pure text, I will see at once how different settings will affect the visual material
  apperance...}\\
You are not forced to create your materials only by using a text editor. Theres a material editor
within the model editor which allows you to create and tweak your materials in a quite comfortable
way.\\


\emph{Can I use more than one texture per material?}\\
Yes. The number of texture layers is only limited by the available hardware - normally at least 2
texture layers are supported - but 4 is the standard today. Note that by using shaders you are
able to use on several graphic cards up to 16 or more different texture layers - 'fix pass' - meaning
without using a shader you can normally using less texture layers as when using shaders!\\


\emph{How to create a material with 'silhouettes'?}\\
A easy way to do this is to draw the material twice were the second time the culling is flipped and
the fill mode is set to wireframe. Example:\\

\begin{lstlisting}[caption=Silhouette material]
<?xml version="1.0"?>
<Material>
	<Technique Name="Simple silhouette">
		<Pass Name="Mesh">
			<Layer>
				<Texture>MyTexture.bmp</Texture>
			</Layer>
		</Pass>
		<Pass Name="Silhouette">
			<RenderStates FillMode="line" CullMode="cw" Lighting="0" />
			<Material Color="0.0 0.0 0 0.0" />
		</Pass>
	</Technique>
</Material>
\end{lstlisting}

The color controled by the set material color.\\


\emph{How to fix terrible z-fighing (ugly artefacts) problems?}\\
z-fighing happens when two polygons are (nearly) coplanar. To fix this precision problem you have
to 'shif' the z values of your material - this is also called setting the polygon offset which is
also known as depth bias. You can set the polygon offset in the material using:\\

\begin{lstlisting}[caption=Polygon offset]
<?xml version="1.0"?>
<Material>
	<Technique Name="Polygon offset">
		<Pass Name="Pass0">
			<RenderStates SlopeScaleDepthBias="-1" 
							DepthBias="-2" />
			<Layer>
				<Texture>MyTexture.bmp</Texture>
			</Layer>
		</Pass>
	</Technique>
</Material>
\end{lstlisting}

... were you have to play around with the factor und unit settings to find good settings for your
situation.\\


\emph{How to use vertex and/or fragment shaders?}\\
PixelLight supports cg shaders which are comfortable and easy to deal with. You can apply vertex
and/or fragment shaders to each material per pass.\\

\begin{lstlisting}[caption=Material and shaders]
<?xml version="1.0"?>
<Material>
	<Technique Name="Shaders">
		<Pass Name="Pass0">
			<VertexShader Filename="MyShader.cg" />
			<FragmentShader Filename="MyShader.cg" />
		</Pass>
	</Technique>
</Material>
\end{lstlisting}

... if the shaders will not work on the current hardware this this technique will be 'invalid'
- when using shaders you should ALWAYS supply additional techniques with simpler shaders and or
techniques without any shaders to catch the worst case. (fallbacks issue)\\


\emph{Can I use more than one vertex and/or fragment shader per material pass?}\\
No, thats not supported by current hardware so you have to do all within one shader at once.
But note that you can also use different passes per material were each pass is using another
shader... ;-)\\


\emph{How to create environment mapping?}\\
To create environment mapping you just have to change the texture generation mode in a material 
layer. Example:\\

\begin{lstlisting}[caption=Environment mapping]
<?xml version="1.0"?>
<Material>
	<Technique Name="Environment mapping">
		<Pass Name="Pass0">
			<Layer>
				<Texture>EnvironmentMap.jpg</Texture>
				<TextureStageStates TexGen="sphere_map" />
			</Layer>
		</Pass>
	</Technique>
</Material>
\end{lstlisting}


\emph{How to create specular highlights?}\\
The simples way would be to use environment mapping with another texture environment mode to
receive a 'highlight' effect. Example:\\

\begin{lstlisting}[caption=Specular highlights]
<?xml version="1.0"?>
<Material>
	<Technique Name="Specular highlight">
		<Pass Name="Pass0">
			<Layer>
				<Texture>Highlight.jpg</Texture>
				<TextureStageStates ColorTexEnv="add" TexGen="sphere_map" />
			</Layer>
		</Pass>
	</Technique>
</Material>
\end{lstlisting}

But there are also several other more complicated and situation dependent ways to create such
an effect like using the material provied color setting, shaders etc.\\


\emph{How to use detail maps?}\\
Normally detail maps are applied using multitexturing. The detail map is blended over another
texture to add more detail - normally this detail map is 'wrapped' meaning that the texture will
repeat if the texture coordinates pointing 'outside' the texture. Example:\\

\begin{lstlisting}[caption=Detail texture]
<?xml version="1.0"?>
<Material>
	<Technique Name="Detail texture">
		<Pass Name="Pass0">
			<Layer>
				<Texture>MyTexture.bmp</Texture>
			</Layer>
			<Layer>
				<Texture>MyDetailMap.bmp</Texture>
			</Layer>
		</Pass>
	</Technique>
</Material>
\end{lstlisting}

\emph{How to use light maps?}\\
Normally light maps are applied using multitexturing. The light map is blended over another
texture to add static lighting. Example:\\

\begin{lstlisting}[caption=Light mapping]
<?xml version="1.0"?>
<Material>
	<Technique Name="Light mapping">
		<Pass Name="Pass0">
			<Layer>
				<Texture>MyTexture.bmp</Texture>
			</Layer>
			<Layer>
				<Texture>MyLightMap.bmp</Texture>
			</Layer>
		</Pass>
	</Technique>
</Material>
\end{lstlisting}


\emph{How to use BumpMapping?}\\
BumpMapping is done through shaders and the underlying model normally has to provide several
additional data like binormals etc. There are tons of different BumpMapping techniques so its
up to you how to create this effect. Here's an example how such a BumpMapping material can look
like:\\

\begin{lstlisting}[caption=BumpMapping]
<?xml version="1.0"?>
<Material>
	<Parameter Name="mvp" Type="float4x4" Semantic="WorldViewProjection" />
	<Parameter Name="camPos" Type="float4" Semantic="EyePos" />
	<Parameter Name="lightPos" Type="float3" />
	<Parameter Name="invRad" Type="float" />
	<Parameter Name="lightColor" Type="float4" />
	<Parameter Name="Base" Type="texture" Semantic="Texture0" />
	<Parameter Name="Normal" Type="texture" Semantic="Texture1" />
	<Technique Name="Lighting">
		<Pass Name="Pass0">
			<Layer>
				<Texture>Fieldstone.jpg</Texture>
			</Layer>
			<Layer>
				<Texture>FieldstoneNormal.tga</Texture>
			</Layer>
			<VertexShader Filename="BumpMapping.cg" />
			<FragmentShader Filename="BumpMapping.cg" />
		</Pass>
	</Technique>
</Material>
\end{lstlisting}


\emph{Can I use "Cel Shading"?}\\
Yes, this is done through shaders processing a 1D texture. It's up to you to write a "cel shading"
shader - there are tons of references and cg shader examples. :)\\


\emph{How to create transparent/blended materials?}\\
Because a material can have several passes were some may be transparent but other not you have
to 'mark' whether the material is transparent or not to let the renderer not how to deal with
this material. Then in each pass you can activate blending and setup several things like the blend
mode. Normally when blending you will not write into the z buffer because the material will not
hide things behind it. How much transparent a material is is controlled through the material color.
Example:\\

\begin{lstlisting}[caption=Transparent/blended material]
<?xml version="1.0"?>
<Material>
	<General Blend="1" />
	<Technique Name="Transparent">
		<Pass Name="Pass0">
			<RenderStates ZWriteEnable="0" BlendEnable="1" 
							ScrBlendFunc="scr_alpha" DstBlendFunc="one" />
			<Layer>
				<Texture>MyTexture.bmp</Texture>
			</Layer>
		</Pass>
	</Technique>
</Material>
\end{lstlisting}
		%[TODO] Update!
\cleardoublepage
