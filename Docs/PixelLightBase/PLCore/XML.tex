\section{XML}
All text based file formats\footnote{Have a look at \emph{SDKBrowser.chm} for a list of all this formats} of PixelLight are XML (eXtensible Markup Language) based. This way they follow a well known syntax. The XML classes you will find in \emph{PLCore} are only wrapper classes. Further we added some additional functions to make the usage of XML more efficient. After you load in such a XML document you can browse and edit it using the DOM (Document Object Model) interface which is quite comfortable. \footnote{But not as performant as for example a SAX/StAX API} After you created a new XML document or edited an existing one, you can also save the XML or even print it into the console. Here's an example how it would look like if you want to load in a PixelLight configuration file. (\emph{cfg}-extension) \emph{Config} will do this for you but this is only an example to see how to use the XML classes in practice:

\begin{lstlisting}[caption=XML DOM usage example]
// Create XML document
XmlDocument cDocument;
if (!cDocument.Load(cFile)) {
  PL_LOG(Error, cDocument.GetValue() + ": " + cDocument.GetErrorDesc())

  // Error!
  return false;
}

// Get config element
const XmlElement *pConfigElement =
  cDocument.GetFirstChildElement("Config");
if (!pConfigElement) {
  PL_LOG(Error, "Can't find 'Config' element")

  // Error!
  return false;
}

// Iterate through all groups
const XmlElement *pGroupElement =
  pConfigElement->GetFirstChildElement("Group");
while (pGroupElement) {
  // Get group class name
  String sClass = pGroupElement->GetAttribute("Class");
  if (sClass.GetLength()) {
    // Get config class instance
    ConfigGroup *pClass = cConfig.GetClass(sClass);
    if (pClass) {
      // Set variables
      pClass->SetVarsFromXMLElement(*pGroupElement, 0);
    }
  }

  // Next element, please
  pGroupElement = pConfigElement->GetNextSiblingElement("Group");
}

// Done
return true;
\end{lstlisting}

To print a XML document into the console you can write for example the following:

\begin{lstlisting}[caption=Print XML document into the console]
XmlDocument cDocument;
cDocument.Load(cFile);
cDocument.Save(File::StandardOutput);
\end{lstlisting}
