\section{\ac{XML}}
All text based file formats\footnote{Have a look at \emph{SDKBrowser.chm} for a list of all this formats} of PixelLight are \ac{XML} based. This way they follow a well known syntax. The \ac{XML} classes you will find in \emph{PLCore} are only wrapper classes. Further we added some additional functions to make the usage of \ac{XML} more efficient. After you load in such a \ac{XML} document you can browse and edit it using the \ac{DOM} interface which is quite comfortable\footnote{But not as performant as for example a \ac{SAX}/\ac{StAX} \ac{API}}. After you created a new \ac{XML} document or edited an existing one, you can also save the \ac{XML} or even print it into the console. Here's an example how it would look like if you want to load in a PixelLight configuration file (\emph{cfg}-extension). \emph{Config} will do this for you but this is only an example to see how to use the \ac{XML} classes in practice:

\begin{lstlisting}[caption=\ac{XML} \ac{DOM} usage example]
// Create XML document
PLCore::XmlDocument cDocument;
if (!cDocument.Load(cFile)) {
	PL_LOG(Error, cDocument.GetValue() + ": " + cDocument.GetErrorDesc())

	// Error!
	return false;
}

// Get config element
PLCore::XmlElement *pConfigElement =
	cDocument.GetFirstChildElement("Config");
if (!pConfigElement) {
	PL_LOG(Error, "Can't find 'Config' element")

	// Error!
	return false;
}

// Iterate through all groups
PLCore::XmlElement *pGroupElement =
	pConfigElement->GetFirstChildElement("Group");
while (pGroupElement) {
	// Get group class name
	PLCore::String sClass = pGroupElement->GetAttribute("Class");
	if (sClass.GetLength()) {
		// Get config class instance
		PLCore::ConfigGroup *pClass = cConfig.GetClass(sClass);
		if (pClass) {
			// Set variables
			pClass->SetVarsFromXMLElement(*pGroupElement, 0);
		}
	}

	// Next element, please
	pGroupElement = pConfigElement->GetNextSiblingElement("Group");
}

// Done
return true;
\end{lstlisting}

To print a \ac{XML} document into the console you can write for example the following:

\begin{lstlisting}[caption=Print \ac{XML} document into the console]
PLCore::XmlDocument cDocument;
cDocument.Load(cFile);
cDocument.Save(PLCore::File::StandardOutput);
\end{lstlisting}
