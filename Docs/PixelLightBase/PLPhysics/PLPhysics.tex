\chapter{PLPhysics}


\paragraph{Motivation}
PixelLight itself has NO own physics implementation nor even fixed build in physics support within the scene graph itself. Such a physics engine is an own complex task - and at the moment the PixelLight team has no possibility (lack of time and man-power) to write a complete own physics solution.

But there are ton's of physics libraries out there - many of them free or even open source and the most modern even with hardware support.

As mentioned above, PixelLight does not come with native physics support within the scene graph component itself - but through the carefully considered design of the PixelLight framework such an 'hacked in'-support is unnecessary, it would even waste the sweet universal design. Because of the extreme plugin nature of PixelLight, it's no problem to add something like physics to your projects.

The PixelLight SDK itself comes with a few such physics plugins. By using this plugins it's extremely simple to add physics to your scene. In fact, this plugins only have one scene node container for the physics world/simulation and a few scene node modifier. Create a scene container using such a physics world scene node container class and add some scene nodes into it. For nodes which should have physical behaviour just add a physics body modifier to the node. If a body has no mass it's considered to be static... et voil�.

To connect two physics bodies you can add a joint modifier to one of the bodies and 'hang in' the second one into the joint. For more advanced physics control you have to use the functions of the physics API of your choice directly - we decided against wrapping all these functions. This would be to much work and to less advanced because the physics API differ at many points a lot of. It's impossible to create ONE universal interface for all physics API's wrapping all available features.

You can use multiple physics API's within your project at the same time, but this isn't recommended and doesn't make much sense - this different simulations also can't interact with each other. So, for your project, you normally have to choose ONE physics API and use it for the hole project. Changing during development to another physics API wouldn't be that easy. You can extent this PixelLight physics plugins by self (not recommended) or add a plugin for another physics API if required.


\paragraph{External Dependences}
\emph{PLPhysics} depends on the \textbf{PLScene} library, and therefore an all other libraries \textbf{PLScene} depends on.




% Include the other sections of the chapter
\section{Physics Backends}
This section deals with the different physics backends shipping with the PixelLight SDK.


\paragraph{Null}
\begin{itemize}
\item The null physics backend does nothing - can be useful if you e.g. are in debug mode and don't want have any physics.
\item PixelLight component name: PLPhysicsNull
\item Used dll's: \emph{PLPhysicsNull.dll}
\end{itemize}


\paragraph{Newton Game Dynamics}
\begin{itemize}
\item This is the preferred physics plugin of the PixelLight team because Newton Game Dynamics\footnote{Newton Game Dynamics can be downloaded from \url{http://newtondynamics.com/}} is really easy to use, offers a lot of geometry and joint types by default, has a nice interface (in fact JUST one header) and some usefull tool functions. Also the community is very active.
\item PixelLight component name: PLPhysicsNewton
\item Used dll's: \emph{PLPhysicsNewton.dll} and \emph{newton.dll}
\end{itemize}


\paragraph{Open Dynamics Engine (ODE)}
\begin{itemize}
\item Currently not within the official PixelLight SDK
\item The Open Dynamics Engine\footnote{Open Dynamics Engine can be downloaded from \url{http://www.ode.org/}} or short ODE is a well-known open source physics engine.
\item PixelLight component name: PLPhysicsODE
\item Used dll's: \emph{PLPhysicsODE.dll} and \emph{ode.dll}
\end{itemize}


\paragraph{PhysX}
\begin{itemize}
\item Currently not within the official PixelLight SDK
\item PhysX\footnote{PhysX can be downloaded from \url{http://developer.nvidia.com/object/physx.html}} is a commercial multithreaded physics API with support for the physics GPU. Please keep in mind that the \emph{PhysX SDK System Software} must be installed to be able to use PhysX - this may not be acceptable for each project.
\item PixelLight component name: PLPhysicsPhysX
\item Used dll's: \emph{PLPhysicsPhysX.dll} and \emph{PhysXLoader.dll}
\end{itemize}

\cleardoublepage
