\section{Debugging and Tweaking}




\subsection{Console}
PixelLight provides and comfortable and expandable console. This feature is especially during the development and debugging extreme useful. You will find a list of the standard PixelLight commands within the section below. Note that the framework handles some commands in a special way. This special commands are called debug commands and they are only available if the framework runs in debug mode. This avoids that cheater can use this special and most extreme powerful commands to solve puzzles in \emph{the other way} like removing the locked door via console. :) The console is connected with the log. The log receives all performed commands and you will see the log in the console itself... even if the log is manipulated from outside you will see this changes at once! The console must be activated in the PixelLight configuration in order to be able use it during runtime:

\begin{lstlisting}[caption=Activate the console within the PixelLight configuration]
<Group Class="PLScene::EngineGeneralConfig" ConsoleActivated="true" />
\end{lstlisting}

Then you can pull down the console using the key \(\land\). There are different features which makes the console usage more comfortable. E.g. there's an auto complete function which shows you in grey the first available command which begins with this characters. If the suggested command is the desired one you could hit the tab-key to perform the auto complete and you only have to confirm the command in order to perform it. You can scroll through the log output with the page-up/page-down keys. The current cursor in the command line can be moved by pressing the left/right keys. You can remove single characters or removing the complete line. All entered commands are stored in a command history were you are able to select old commands by pressing the up/down key. Further it's also possible to copy'n'past text.



\subsubsection{Standard Commands}
\begin{tabular}{|p{4cm}|p{2.4cm}|p{7.2cm}|}
\hline
\textbf{Command} & \textbf{Parameters} & \textbf{Description}\\
\hline
help                 &          & Show help text\\
\hline
list                 &          & Show a list of all available commands\\
\hline
bulkylist            &          & Show a list of all available commands with detailed information\\
\hline
clear                &          & Clear command history\\
\hline
quit/exit/bye/logout &          & Shut down the framework\\
\hline
about                &          & Show about information\\
\hline
version              &          & Show version information\\
\hline
fps                  &          & Toggle \ac{FPS} display\\
\hline
fpslimit             & [limit]  & Sets the \ac{FPS} limit, 0 if there's no \ac{FPS} limitation\\
\hline
profiling            &          & Activates / deactivates the profiling system\\
\hline
nextprofile          &          & Selects the next profile group\\
\hline
preprofile           &          & Selects the previous profile group\\
\hline
coordinateaxis       &          & Toggle coordinate axis visualisation\\
\hline
xzplane              &          & Toggle xz-plane visualisation\\
\hline
xyplane              &          & Toggle xy-plane visualisation\\
\hline
yzplane              &          & Toggle yz-plane visualisation\\
\hline
setloglevel          & [mode]   & Sets the debug mode. Normally 0 means that the debug mode is deactivated and no additional debug information will be written into the log.\\
\hline
pause                &          & Enable/disable pause\\
\hline
timescale            & [factor] & Sets the time scale factor\\
\hline
\end{tabular}



\subsubsection{Register and Un-Register Commands}
It's possible to add new custom commands or remove old one with these functions:

\begin{lstlisting}[caption=Register and un-register commands]
PLEngine::SNConsoleBase::RegisterCommand()
PLEngine::SNConsoleBase::UnRegisterCommand()
\end{lstlisting}

It's even possible to customize the complete layout of the console! Example:

\begin{lstlisting}[caption=Register command]
pMyConsole->RegisterCommand(1, "setlevel", "I", "<level>", PLCore::Functor<void, PLEngine::ConsoleCommand &>(ConsoleCommandSetLevel));
\end{lstlisting}

This would register the new debug command \emph{setlevel} which takes one integer parameter. It will call the function \emph{ConsoleCommandSetLevel} if the command is performed which will receive more information about it's \emph{ConsoleCommand} parameter. Fore more information have a look at the console code comments.
