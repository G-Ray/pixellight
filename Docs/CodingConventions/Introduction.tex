\chapter{Introduction}
C++ is a mighty programming language, and has evolved over the time - as a result, there are many ways how certain goals can be archived and a lot of pitfalls. This huge freedom often leads to confusion which way is the "best" - and often, the answer is that there's no best way, just widely used practices or literature from C++ gurus and accepted design patterns.

If a project has a certain size, it's usually a good idea to write a document about the used \emph{coding conventions}. So, here's the \emph{PixelLight coding conventions} document.

This documentation can't describe everything\footnote{Although it has grown over the years...} - because then you would spend more time in reading and rereading this document instead of actual programming. In here, we especially mention things that are important to us. The rest should be pretty self explanatory and it should be possible to \emph{find out} other style conventions by self. We strongly recommend you to spend time in writing clean code yourself to support the \emph{work flow}! It's not important HOW FAST you wrote the code, it's important that this code is well designed, well commented and cleanly written because you may use this code for a loonng time.\footnote{Belief us, at the time of writing we're working already 8 years on PixelLight and are always happy when there's no need to touch code again because it's just fine the way it is}
