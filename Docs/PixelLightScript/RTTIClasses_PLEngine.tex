\chapter{PLEngine \ac{RTTI} Classes}




\section{PLEngine::EngineApplication Class}


\subsection{Methods}

\paragraph{<RTTI object> GetScene()}
Write \emph{<RTTI object>:GetScene()} in order to get the scene container (the \emph{concrete scene}), can be a null pointer.

\paragraph{SetScene(<RTTI object>)}
Write \emph{<RTTI object>:SetScene(<RTTI object>)} in order to set the scene container (the \emph{concrete scene}). New scene container as first parameter (can be a null pointer).

\paragraph{ClearScene()}
Write \emph{<RTTI object>:ClearScene()} in order to clear the scene, after calling this method the scene is empty.

\paragraph{<bool> LoadScene(<string>)}
Write \emph{<RTTI object>:LoadScene(<string>)} in order to load a scene. Filename of the scene to load as first argument. Returns \emph{true} if all went fine, else \emph{false}. This method will completly replace the current scene.

\paragraph{<RTTI object> GetCamera()}
Write \emph{<RTTI object>:GetCamera()} in order to get the scene camera, can be a null pointer.

\paragraph{SetCamera(<RTTI object>)}
Write \emph{<RTTI object>:SetCamera(<RTTI object>)} in order to set the scene camera. New scene camera as first parameter (can be a null pointer).

\paragraph{<RTTI object> GetInputController()}
Write \emph{<RTTI object>:GetInputController()} in order to get the virtual input controller (can be a null pointer). See appendix~\ref{Appendix:VirtualStandardController} for information about the virtual standard controller and appendix~\ref{Appendix:InputExamples} for input examples.

\paragraph{SetInputController(<RTTI object>)}
Write \emph{<RTTI object>:SetInputController(<RTTI object>)} in order to set the virtual input controller. Virtual input controller (can be a null pointer) as first parameter.

\paragraph{<RTTI object> GetSceneRendererTool()}
Write \emph{<RTTI object>:GetSceneRendererTool()} in order to get the scene renderer tool.

\paragraph{<RTTI object> GetRootScene()}
Write \emph{<RTTI object>:GetRootScene()} in order to 

\paragraph{<RTTI object> GetScreenshotTool()}
Write \emph{<RTTI object>:GetScreenshotTool()} in order to get the screenshot tool.


\subsection{Signals}

\paragraph{SignalCameraSet}
A new camera has been set.

\paragraph{SignalSceneLoadingFinished}
Scene loading has been finished successfully.




\section{PLEngine::ScriptApplication Class}


\subsection{Attributes}

\paragraph{OnInitFunction}
Name of the optional script function called by C++ when the application should initialize itself. Default is \emph{OnInit}.

\paragraph{OnUpdateFunction}
Name of the optional script function called by C++ when the application should update itself. Default is \emph{OnUpdate}.

\paragraph{OnDeInitFunction}
Name of the optional script function called by C++ when the application should de-initialize itself. Default is \emph{OnDeInit}.


\subsection{Methods}

\paragraph{<string> GetBaseDirectory()}
Write \emph{<RTTI object>:GetBaseDirectory()} in order to get the base directory of the application (native path style, e.g. on Windows: 'C:\textbackslash MyApplication\textbackslash ').

\paragraph{SetBaseDirectory(<string>)}
Write \emph{<RTTI object>:SetBaseDirectory(<string>)} in order to set the base directory of the application (e.g. on Windows: 'C:\textbackslash MyApplication\textbackslash '). Base directory as the first parameter.

\paragraph{<string> GetScript()}
Write \emph{<RTTI object>:GetScript()} in order to get the used script instance.

\paragraph{<string> GetScriptFilename()}
Write \emph{<RTTI object>:GetScriptFilename()} in order to get the absolute filename of the used script (native path style, e.g. on Windows: 'C:\textbackslash MyApplication\textbackslash Main.lua').

\paragraph{<string> GetScriptDirectory()}
Write \emph{<RTTI object>:GetScriptDirectory()} in order to get the absolute directory the used script is in (native path style, e.g. on Windows: 'C:\textbackslash MyApplication\textbackslash ' if currently the script 'C:\textbackslash MyApplication\textbackslash Main.lua' is used).
