\chapter{Script Bindings Plugin}
\label{ScriptBindingsPlugin}
This chapter is about script bindings in general and in particular about the loose plugin \emph{PLScriptBindings} which exposes certain parts of PixelLight to script languages.




\section{PL}
Exposes general PixelLight features to scripts.

\paragraph{<RTTI object> PL.GetApplication()}
Write \emph{PL.GetApplication()} in order to get an instance of the currently used \ac{RTTI} application class. This can be a null pointer, but usually it isn't.




\section{PL.System}
Exposes \emph{PLCore::System}-features to scripts.

\paragraph{<bool> PL.System.IsWindows()}
Write \emph{PL.System.IsWindows()} in order to figure out whether or not the application is currently running on \ac{MS} Windows. Returns \emph{true} if we're currently running on a Windows platform, else \emph{false}.

\paragraph{<bool> PL.System.IsLinux()}
Write \emph{PL.System.IsLinux()} in order to figure out whether or not the application is currently running on Linux. Returns \emph{true} if we're currently running on a Linux platform, else \emph{false}.

\paragraph{<string> PL.System.GetPlatformArchitecture()}
Write \emph{PL.System.GetPlatformArchitecture()} in order to figure out the platform architecture (for instance \emph{x86}, \emph{x64}, \emph{armeabi}, \emph{armeabi-v7a} and so on).




\section{PL.Log}
Exposes \emph{PLCore::Log}-features to scripts.

\paragraph{PL.Log.OutputAlways(<string>)}
Write \emph{PL.System.OutputAlways(<string>)} in order to write the given string into the log ('always' log level).

\paragraph{PL.Log.OutputCritical(<string>)}
Write \emph{PL.System.OutputCritical(<string>)} in order to write the given string into the log ('critical' log level).

\paragraph{PL.Log.OutputError(<string>)}
Write \emph{PL.System.OutputError(<string>)} in order to write the given string into the log ('error' log level).

\paragraph{PL.Log.OutputWarning(<string>)}
Write \emph{PL.System.OutputWarning(<string>)} in order to write the given string into the log ('warning' log level).

\paragraph{PL.Log.OutputInfo(<string>)}
Write \emph{PL.System.OutputInfo(<string>)} in order to write the given string into the log ('info' log level).

\paragraph{PL.Log.OutputDebug(<string>)}
Write \emph{PL.System.OutputDebug(<string>)} in order to write the given string into the log ('debug' log level).




\section{PL.System.Console}
Exposes \emph{PLCore::Console}-features to scripts.

\paragraph{PL.System.Console.Print(<string>)}
Write \emph{PL.System.Console.Print(<string>)} in order to write the given string into the system console.




\section{PL.Timing}
Exposes \emph{PLCore::Timing}-features to scripts.

\paragraph{<float> PL.Timing.GetTimeDifference()}
Write \emph{PL.Timing.GetTimeDifference()} in order to get the past time since last frame (seconds).

\paragraph{<float> PL.Timing.GetFramesPerSecond()}
Write \emph{PL.Timing.GetFramesPerSecond()} in order to get the current \ac{FPS}.

\paragraph{<bool> PL.Timing.IsPaused()}
Write \emph{PL.Timing.IsPaused()} in order to receive \emph{true} when the timing is paused, else \emph{false}. If the timing is paused scene nodes, particles etc. are not updated. The timing will still be updated.

\paragraph{<float> PL.Timing.Pause(<boolean>)}
Write \emph{PL.Timing.Pause(<boolean>)} in order to set pause mode. \emph{true} as first parameter when timing should be pause, else \emph{false}.

\paragraph{<float> PL.Timing.GetTimeScaleFactor()}
Write \emph{PL.Timing.GetTimeScaleFactor()} in order to get the time scale factor. The global time scale factor should only be manipulated for debugging. A factor of <= 0 is NOT allowed because this may cause problems in certain situations, pause the timer instead by hand! Do NOT make the factor \emph{too} (for example > 4) extreme, this may cause problems in certain situations!

\paragraph{<float> PL.Timing.SetTimeScaleFactor(<float>)}
Write \emph{PL.Timing.SetTimeScaleFactor(<float>)} in order to set the time scale factor. Time scale as first parameter (a factor of <= 0 is NOT allowed!). Returns \emph{true} if all went fine, else \emph{false} (maybe the given factor is <= 0?).




\section{PL.ClassManager}
Exposes \emph{PLCore::ClassManager}-features to scripts.

\paragraph{PL.ClassManager.ScanPlugins(<string>, <boolean>, <boolean>)}
Write \emph{PL.ClassManager.ScanPlugins(<string>, <boolean>, <boolean>)} in order to scan a directory for compatible plugins and load them in. Directory to search in as first parameter, boolean value deciding whether or not to take sub-directories into account as second parameter, boolean value deciding whether or not its allowed to perform delayed shared library loading to speed up the program start as third parameter. Returns \emph{true} if all went fine, else \emph{false}.

\paragraph{PL.ClassManager.Create(<string>, <string>)}
Write \emph{PL.ClassManager.Create(<string>, <string>)} in order to create a new RTTI class instance by using the default constructor. Name of the RTTI class to create an instance from as first parameter and optional parameter string for the created instance as second parameter. Returns a pointer to the new RTTI class instance or a null pointer if something went wrong (maybe unknown class). Created instance has initially no references, meaning that a script usually automatically destroys the instance when no longer used.

\paragraph{PL.ClassManager.CreateByConstructor(<string>, <string>, <string>, <string>)}
Write \emph{PL.ClassManager.CreateByConstructor(<string>, <string>, <string>, <string>)} in order to create a new RTTI class instance by using a specified constructor. Name of the RTTI class to create an instance from as first parameter, constructor name as second parameter, constructor parameters as third parameter and optional parameter string for the created instance as fourth parameter. Returns a pointer to the new RTTI class instance or a null pointer if something went wrong (maybe unknown class). Created instance has initially no references, meaning that a script usually automatically destroys the instance when no longer used.
