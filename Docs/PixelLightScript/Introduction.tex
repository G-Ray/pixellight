\chapter{Introduction}


\paragraph{Motivation}
This script documentation is intended for script programmers. While this document also talks about some quite basic stuff, the targeted audience has already some fundamental programming experience. If you're totally new to the field of programming in general, it's highly recommended to read some beginner literature first \footnote{For example \emph{Programming in Lua (first edition)} which is online available at \url{http://www.lua.org/pil/} if you're just interested in scripting using Lua}.

The reason for creating the script interface was to have a minimalistic and universal interface to work with script languages. Instead of inventing an own script language, we use a backend design pattern to use already available common script languages like \emph{Lua}, \emph{JavaScript}, \emph{Python} or \emph{AngelScript}.

Why should you care about scripting in the first place when PixelLight is written in C++ and you have the whole C++ API and even the complete source code of the project at our hands? Well, let me say it this way: Not every programmer is a programmer. What's meant by this statement is, that within development teams not every team member is able to or want's to work in C++. When it comes to implement application logic like \begin{quote}''When the user clicks on this door then let an anvil fall down on him''\end{quote}, there's no real need to let everything be written by experienced (and therefore usually expensive) software developers within C++. Let's say a graphics artist has finished his work and has enough of colourful things for a few hours - why shouldn't he be allowed to implement some simple application logic? By using scripting, it's usually no big deal to let other persons than experienced software developers do some coding.

Please note that the purpose of this document is not to describe each and every possible global function, RTTI classes, methods, attributes and so on. Due to the highly plugin driven architecture of PixelLight, this would be impossible anyway. There are several chapters about fundamental RTTI classes, but that's just a tiny portion of what's available. Script language support was added to PixelLight in the release 0.9.7 - so it's a young component. At the moment, you'll have to look into the C++ source codes\footnote{Shouldn't be hard too find the stuff accessible by scripts, just look for \emph{pl\_class}} so see what can be accessed by the script. This is definitely not optimal and in the future it's probably possible to access a dynamic realtime script API documentation which is using RTTI information. An editor should also include an script editor with build in automatic script API documentation and auto-complete. This way, such a documentation would automatically be always up-to-date and would also contain your own, custom extensions. If you have any questions or problems, feel free to use the official PixelLight forum at \url{http://dev.pixellight.org/forum/}.




\section{External Dependences}
There's no individual \emph{PLScript} project, it's part of \emph{PLCore}. \emph{PLCore} only depends on the \textbf{PLCore} library and doesn't use any external third party libraries itself.
