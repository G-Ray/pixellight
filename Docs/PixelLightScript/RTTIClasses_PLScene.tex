\chapter{PLScene \ac{RTTI} Classes}
See appendix~\ref{Appendix:SceneExamples} for scene examples.




\section{PLScene::SceneApplication Class}


\subsection{Methods}

\paragraph{<RTTI object> GetRootScene()}
Write \emph{<RTTI object>:GetRootScene()} in order to get the root scene container, can be a null pointer.




\section{PLScene::SceneNode Class}


\subsection{Attributes}

\paragraph{Flags}
Flags.

\paragraph{DebugFlags}
Debug flags.

\paragraph{Position}
Position.

\paragraph{Rotation}
Rotation as Euler angles in degree, [0, 360].

\paragraph{Scale}
Scale.

\paragraph{MaxDrawDistance}
Maximum draw distance of the scene node to the camera, if 0 do always draw, if negative, do always draw this node before other.

\paragraph{AABBMin}
Minimum position of the 'scene node space' axis aligned bounding box.

\paragraph{AABBMax}
Maximum position of the 'scene node space' axis aligned bounding box.

\paragraph{Name}
Optional scene node name. If not defined, a name is chosen automatically.


\subsection{Methods}

\paragraph{<RTTI object> GetContainer()}
Write \emph{<RTTI object>:GetContainer()} in order to get the scene container the scene node is in or a null pointer if this is the root node.

\paragraph{SetContainer(<RTTI object>)}
Write \emph{<RTTI object>:SetContainer(<RTTI object>)} in order to set the scene container the scene node is in. Scene container this node is in as first parameter. Returns \emph{true} if all went fine, else \emph{false} (Position, rotation, scale etc. are not manipulated, only the container is changed!).

\paragraph{<RTTI object> GetRootContainer()}
Write \emph{<RTTI object>:GetRootContainer()} in order to get the scene root container (this scene container can be the root scene container), a null pointer on error.

\paragraph{<RTTI object> GetCommonContainer(<RTTI object>)}
Write \emph{<RTTI object>:GetCommonContainer(<RTTI object>)} in order to get the common container of this and another scene node. The other scene node as first parameter. Returns the common container, or a null pointer.

\paragraph{<RTTI object> GetHierarchy()}
Write \emph{<RTTI object>:GetHierarchy()} in order to get the scene hierarchy this scene node is linked into. Returns the scene hierarchy this scene node is linked into, a null pointer on error.

\paragraph{<string> GetAbsoluteName()}
Write \emph{<RTTI object>:GetAbsoluteName()} in order to get the unique absolute name of the scene node (for instance ''Root.MyScene.MyNode'').

\paragraph{<bool> IsActive()}
Write \emph{<RTTI object>:IsActive()} in order to check whether the scene node is active or not. Returns \emph{true} if the scene node is active, else \emph{false}.

\paragraph{SetActive(<bool>)}
Write \emph{<RTTI object>:SetActive(<bool>)} in order to set whether the scene node is active or not. \emph{true} as first parameter if the scene node should be active, else \emph{false} (sets/unsets the \emph{Inactive}-flag).

\paragraph{<bool> IsVisible()}
Write \emph{<RTTI object>:IsVisible()} in order to check whether the scene node is visible or not. Returns \emph{true} if the scene node is visible, else \emph{false} (invisible/inactive). If the scene node is not active it's automatically invisible but the \emph{Invisible}-flag is not touched. \emph{Visible} doesn't mean \emph{currently on screen}, it just means \emph{can be seen in general}.

\paragraph{SetVisible(<bool>)}
Write \emph{<RTTI object>:SetVisible(<bool>} in order set whether the scene node is visible or not. \emph{true} as first parameter if the scene node should be visible, else \emph{false} (sets/unsets the \emph{Invisible}-flag). See \emph{IsVisible()}-method for more information.

\paragraph{<bool> IsFrozen()}
Write \emph{<RTTI object>:IsFrozen()} in order to check whether the scene node is frozen or not. Returns \emph{true} if the scene node is frozen, else \emph{false}.

\paragraph{SetFrozen(<bool>)}
Write \emph{<RTTI object>:SetFrozen(<bool>)} in order to set whether the scene node is frozen or not. \emph{true} as first parameter if the scene node should be frozen, else \emph{false} (sets/unsets the \emph{Frozen}-flag).

\paragraph{<bool> IsContainer()}
Write \emph{<RTTI object>:IsContainer()} in order to check whether this scene node is a scene container (\emph{PLScene::SceneContainer}) or not. Returns \emph{true} if this scene node is a scene container, else \emph{false}.

\paragraph{<bool> IsCell()}
Write \emph{<RTTI object>:IsCell()} in order to check whether this scene node is a cell (\emph{PLScene::SCCell}) or not. Returns \emph{true} if this scene node is a cell, else \emph{false}.

\paragraph{<bool> IsPortal()}
Write \emph{<RTTI object>:IsPortal()} in order to check whether this scene node is a portal (\emph{PLScene::SNPortal}) or not. Returns \emph{true} if this scene node is a portal, else \emph{false}.

\paragraph{<bool> IsCamera()}
Write \emph{<RTTI object>:IsCamera()} in order to check whether this scene node is a camera (\emph{PLScene::SNCamera}) or not. Returns \emph{true} if this scene node is a camera, else \emph{false}.

\paragraph{<bool> IsLight()}
Write \emph{<RTTI object>:IsLight()} in order to check whether this scene node is a light (\emph{PLScene::SNLight}) or not. Returns \emph{true} if this scene node is a light, else \emph{false}.

\paragraph{<bool> IsFog()}
Write \emph{<RTTI object>:IsFog()} in order to check whether this scene node is a fog (\emph{PLScene::SNFog}) or not. Returns \emph{true} if this scene node is a fog, else \emph{false}.

\paragraph{<integer> GetNumOfModifiers(<string>)}
Write \emph{<RTTI object>:GetNumOfModifiers(<string>)} in order to get the number of modifiers. Optional modifier class name to return the number of instances from as first parameter (if empty return the total number of modifiers).

\paragraph{<RTTI object> AddModifier(<string>, <string>)}
Write \emph{<RTTI object>:AddModifier(<string>, <string>))} in order to add a modifier. Modifier class name of the modifier to add as first parameter and optional parameter string as second parameter. Returns a pointer to the modifier instance if all went fine, else a null pointer (maybe unknown/incompatible modifier).

\paragraph{<RTTI object> AddModifierAtIndex(<string>, <string>, <integer>))}
Write \emph{<RTTI object>:AddModifierAtIndex(<string>, <string>, <integer>))} in order to add a modifier at a certain index inside the child list. Modifier class name of the modifier to add as first parameter and optional parameter string as second parameter, optional index position specifying the location within the child list where the scene node modifier should be added as third parameter (<0 for at the end). Returns a pointer to the modifier instance if all went fine, else a null pointer (maybe unknown/incompatible modifier).

\paragraph{<RTTI object> GetModifier(<string>, <integer>)}
Write \emph{<RTTI object>:GetModifier(<string>, <integer>)} in order to get a modifier. Modifier class name of the modifier to return as first parameter, optional modifier index as second parameter (used if class name is empty or if there are multiple instances of this modifier class). Returns the requested modifier, a null pointer on error.

\paragraph{<bool> RemoveModifierByReference(<RTTI object>)}
Write \emph{<RTTI object>:RemoveModifierByReference(<RTTI object>)} in order to remove a modifier by using a given reference to the modifier to remove. Modifier to remove as first parameter. Returns \emph{true} if all went fine, else \emph{false} (maybe invalid modifier). After this method succeeded, the given reference is no longer valid.

\paragraph{<bool> RemoveModifier(<string>, <integer>)}
Write \emph{<RTTI object>:RemoveModifier(<string>, <integer>)} in order to remove a modifier. Modifier class name of the modifier to remove as first parameter, modifier index as second parameter (used if class name is empty or if there are multiple instances of this modifier class). Returns \emph{true} if all went fine, else \emph{false} (maybe invalid modifier).

\paragraph{ClearModifiers()}
Write \emph{<RTTI object>:ClearModifiers()} in order to clear all modifiers.

\paragraph{<bool> Delete(<bool>)}
Write \emph{<RTTI object>:Delete(<bool>)} in order to delete this scene node. If the first parameter is \emph{true} the scene node will also be deleted when it's protected. Returns \emph{true} when all went fine, else \emph{false}.

\paragraph{<RTTI object> GetInputController()}
Write \emph{<RTTI object>:GetInputController()} in order to get the input controller. Returns the input controller (can be a null pointer).


\subsection{Signals}

\paragraph{SignalDestroy}
Scene node destruction signal. Unlike \emph{PLCore::Object::SignalDestroyed}, the scene node is still intact at the point the signal is emitted.

\paragraph{SignalActive}
Scene node active state change signal.

\paragraph{SignalVisible}
Scene node visible state change signal.

\paragraph{SignalContainer}
Scene node parent container change signal.

\paragraph{SignalAABoundingBox}
Scene node axis aligned bounding box change signal.

\paragraph{SignalInit}
Scene node initialization signal.

\paragraph{SignalDeInit}
Scene node de-initialization change signal.

\paragraph{SignalAddedToVisibilityTree}
Scene node was added to a visibility tree signal. Visibility node representing this scene node within the visibility tree as parameter.




\section{PLScene::SceneNodeModifier Class}


\subsection{Attributes}

\paragraph{Flags}
Flags.


\subsection{Methods}

\paragraph{<RTTI object> GetSceneNode()}
Write \emph{<RTTI object>:GetSceneNode()} in order to get the owner scene node.

\paragraph{<RTTI object> GetSceneNodeIndex()}
Write \emph{<RTTI object>:GetSceneNodeIndex()} in order to get the index of this scene node modifier within the scene node modifier list of the owner scene node, <0 on failure (e.g. the scene node modifier is no member of this scene node).			

\paragraph{<string> GetAbsoluteName()}
Write \emph{<RTTI object>:GetAbsoluteName()} in order to get an unique absolute name for the scene node modifier. The name is constructed by using \emph{<absolute owner scene node name>:<scene node modifier class name>.<zero based index>} (for instance \emph{Root.MyScene.MyNode:SNMRotationLinearAnimation.0}). Do not use this method on a regular basis.

\paragraph{<bool> IsActive()}
Write \emph{<RTTI object>:IsActive()} in order to check whether the scene node modifier is active or not. Returns \emph{true} if the scene node modifier is active, else \emph{false}.

\paragraph{SetActive(<bool>)}
Write \emph{<RTTI object>:SetActive(<bool>)} in order to set whether the scene node modifier is active or not. \emph{true} as first parameter if the scene node modifier should be active, else \emph{false} (sets/unsets the \emph{Inactive}-flag).

\paragraph{<RTTI object> GetInputController()}
Write \emph{<RTTI object>:GetInputController()} in order to get the input controller. Returns the input controller (can be a null pointer).




\section{PLScene::SceneContainer Class}


\subsection{Attributes}

\paragraph{Hierarchy}
Class name of the scene container hierarchy.

\paragraph{Filename}
Filename of the file to load the container from.


\subsection{Methods}

\paragraph{<bool> Clear(<bool>)}
Write \emph{<RTTI object>:Clear(<bool>)} in order to destroy all scene nodes within this scene container. If the first parameter is \emph{true} protected scene nodes are destroyed as well. Returns \emph{true} if all went fine, else \emph{false}.

\paragraph{<RTTI object> GetByIndex(<integer>)}
Write \emph{<RTTI object>:GetByIndex(<integer>)} in order to get a scene node by using the given index, result can be a null pointer.

\paragraph{<RTTI object> GetByName(<string>)}
Write \emph{<RTTI object>:GetByName(<string>)} in order to get a scene node by using the given name, result can be a null pointer.

\paragraph{<RTTI object> Create(<string>, <string>, <string>)}
Write \emph{<RTTI object>:Create(<string>, <string>, <string>)} in order to create a new scene node. Name of the scene node class to create an instance from as first parameter, scene node name as second parameter and optional parameter string as third parameter. Returns a pointer to the new scene node or a null pointer if something went wrong (maybe unknown class or the class is not derived from \emph{PLScene::SceneNode}).

\paragraph{<RTTI object> CreateAtIndex(<string>, <string>, <string>), <integer>}
Write \emph{<RTTI object>:CreateAtIndex(<string>, <string>, <string>, <integer>)} in order to create a new scene node at a certain index inside the child list. Name of the scene node class to create an instance from as first parameter, scene node name as second parameter and optional parameter string as third parameter, optional index position specifying the location within the child list where the scene node should be added as fouth parameter (<0 for at the end). Returns a pointer to the new scene node or a null pointer if something went wrong (maybe unknown class or the class is not derived from \emph{PLScene::SceneNode}).

\paragraph{CalculateAABoundingBox()}
Write \emph{<RTTI object>:CalculateAABoundingBox()} in order to calculate and sets the axis align bounding box in \emph{scene node space}. Because the \emph{scene node space} axis aligned bounding box should always cover all scene nodes of this container, you can use this function to calculate and set this a bounding box automatically.

\paragraph{LoadByFilename(<string>, <string>, <string>))}
Write \emph{<RTTI object>:LoadByFilename()} in order to load a scene from a file given by filename. Scene filename as first parameter, optional load method parameters as second parameter, optional name of the load method to use as third parameter. Returns \emph{true} if all went fine, else \emph{false}.


\subsection{Signals}

\paragraph{SignalLoadProgress}
Scene load progress signal. Current load progress as parameter - if not within 0-1 loading is done.
