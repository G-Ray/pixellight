\chapter{PLCore \ac{RTTI} Classes}




\section{PLCore::Object Class}


\subsection{Methods}

\paragraph{<bool> IsInstanceOf(<string>)}
Write \emph{<RTTI object>:IsInstanceOf(<string>)} in order to check if object is instance of a given class. Class name (with namespace) as first parameter. Returns \emph{true} if the object is an instance of the class or one of it's derived classes, else \emph{false}.

\paragraph{SetAttribute(<string>, <string>)}
Write \emph{<RTTI object>:SetAttribute(<string>, <string>)} in order to set an attribute value. Attribute name as first parameter, attribute value as second parameter.

\paragraph{SetAttributeDefault(<string>)}
Write \emph{<RTTI object>:SetAttributeDefault(<string>)} in order to set an attribute to it's default value. Attribute name as first parameter.

\paragraph{CallMethod(<string>, <string>)}
Write \emph{<RTTI object>:CallMethod(<string>, <string>} in order to call a method. Method name as first parameter, parameters as string (e.g. ''Param0='x' Param1='y' '') as second parameter.

\paragraph{SetValues(<string>)}
Write \emph{<RTTI object>:SetValues(<string>)} in order to set multiple attribute values as a string at once. String containing attributes and values as first parameter (e.g. ''Name='Bob' Position='1 2 3' '').

\paragraph{SetDefaultValues()}
Write \emph{<RTTI object>:SetDefaultValues()} in order to set all attributes to default.

\paragraph{<string> ToString()}
Write \emph{<RTTI object>:ToString()} in order to get the object as string. Returns string representation of object.

\paragraph{FromString(<string>)}
Write \emph{<RTTI object>:FromString(<string>)} in order to set the object from string. String representation of object as first parameter.


\subsection{Signals}

\paragraph{SignalDestroyed}
Object destroyed signal. When this signal is emitted the object is already in the destruction phase and parts may already be invalid. Best to e.g. only update our object pointer.




\section{PLCore::CoreApplication Class}


\subsection{Methods}

\paragraph{<RTTI object> GetApplicationContext()}
Write \emph{<RTTI object>:GetApplicationContext()} in order to receive the application context.

\paragraph{Exit(<integer>)}
Write \emph{<RTTI object>:Exit(<integer>)} in order to exit the application. Return code for application as first parameter (usually 0 means no error).




\section{PLCore::FrontendApplication Class}


\subsection{Methods}

\paragraph{<RTTI object> GetFrontend()}
Write \emph{<RTTI object>:GetFrontend()} in order to get the frontend this application is running in.




\section{PLCore::Frontend Class}


\subsection{Methods}

\paragraph{Redraw()}
Write \emph{<RTTI object>:Redraw()} in order redraw the frontend.

\paragraph{Ping()}
Write \emph{<RTTI object>:Ping()} in order to give the frontend a chance to process \ac{OS} messages.

\paragraph{RedrawAndPing()}
Write \emph{<RTTI object>:RedrawAndPing()} in order to redraw the frontend and give the frontend a chance to process \ac{OS} messages.

\paragraph{<string> GetTitle()}
Write \emph{<RTTI object>:GetTitle()} in order receive the frontend title.

\paragraph{SetTitle(<string>)}
Write \emph{<RTTI object>:SetTitle(<string>)} in order to set the frontend title.

\paragraph{<integer> GetX()}
Write \emph{<RTTI object>:GetX()} in order receive the x position of the frontend (in screen coordinates).

\paragraph{<integer> GetY()}
Write \emph{<RTTI object>:GetY()} in order receive the y position of the frontend (in screen coordinates).

\paragraph{<integer> GetWidth()}
Write \emph{<RTTI object>:GetWidth()} in order receive the width of the frontend.

\paragraph{<integer> GetHeight()}
Write \emph{<RTTI object>:GetHeight()} in order receive the height of the frontend.

\paragraph{<bool> GetToggleFullscreenMode()}
Write \emph{<RTTI object>:GetToggleFullscreenMode()} in order to request whether it's allowed to toggle the fullscreen mode using hotkeys. \emph{true} if it's possible to toggle the fullscreen mode using hotkeys, else \emph{false}.

\paragraph{SetToggleFullscreenMode(<bool>)}
Write \emph{<RTTI object>:SetToggleFullscreenMode(<bool>)} in order set whether it's allowed to toggle the fullscreen mode using hotkeys. \emph{true} as first parameter to allow it, else \emph{false}.

\paragraph{<bool> GetFullscreenAltTab()}
Write \emph{<RTTI object>:GetFullscreenAltTab()} in order to request whether it's allowed to use Alt-Tab if fullscreen mode is used. \emph{true} if it's possible to use Alt-Tab if fullscreen mode is used, else \emph{false}.

\paragraph{SetFullscreenAltTab(<bool>)}
Write \emph{<RTTI object>:SetFullscreenAltTab(<bool>)} in order to set whether it's allowed to use Alt-Tab if fullscreen mode is used. \emph{true} as first parameter to allow it, else \emph{false}.

\paragraph{<bool> IsFullscreen()}
Write \emph{<RTTI object>:IsFullscreen()} in order to request whether or not the frontend is currently fullscreen or not. Returns \emph{true} if the frontend is currently fullscreen, else \emph{false}.

\paragraph{SetFullscreen(<bool>)}
Write \emph{<RTTI object>:SetFullscreen(<bool>)} in order to set whether or not the frontend is currently fullscreen or not. \emph{true} as first parameter if the frontend is currently fullscreen, else \emph{false}.

\paragraph{<bool> IsMouseOver()}
Write \emph{<RTTI object>:IsMouseOver()} in order to request whether or not the mouse cursor is currently over the frontend. Returns \emph{true} if the mouse cursor is currently over the frontend, else \emph{false}.

\paragraph{<integer> GetMousePositionX()}
Write \emph{<RTTI object>:GetMousePositionX()} in order to request the current mouse cursor X position inside the frontend, negative value if the mouse cursor isn't currently over the frontend.

\paragraph{<integer> GetMousePositionY()}
Write \emph{<RTTI object>:GetMousePositionY()} in order to request the current mouse cursor Y position inside the frontend, negative value if the mouse cursor isn't currently over the frontend.

\paragraph{<bool> IsMouseVisible()}
Write \emph{<RTTI object>:IsMouseVisible()} in order to request whether or not the mouse cursor is currently visible. Returns \emph{true} if the mouse cursor is currently visible, else \emph{false}.

\paragraph{SetMouseVisible(<bool>)}
Write \emph{<RTTI object>:SetMouseVisible(<bool>)} in order to set the mouse cursor visibility. \emph{true} as first parameter if the mouse cursor shall be visible.

\paragraph{SetTrapMouse(<bool>)}
Write \emph{<RTTI object>:SetTrapMouse(<bool>)} in order to trap the mouse inside the frontend. \emph{true} as first parameter if the mouse should be trapped inside the frontend, else \emph{false}.




\section{PLCore::ApplicationContext Class}


\subsection{Methods}

\paragraph{<string> GetExecutableFilename()}
Write \emph{<RTTI object>:GetExecutableFilename()} in order to receive the absolute path of application executable (native path style, e.g. on Windows: "C:\MyApplication\Test.exe").

\paragraph{<string> GetAppDirectory()}
Write \emph{<RTTI object>:GetAppDirectory()} in order to receive the directory of application executable (native path style, e.g. on Windows: "C:\MyApplication").

\paragraph{<string> GetStartupDirectory()}
Write \emph{<RTTI object>:GetStartupDirectory()} in order to receive the current directory when the application constructor was called (native path style, e.g. on Windows: "C:\MyApplication").

\paragraph{<string> GetLogFilename()}
Write \emph{<RTTI object>:GetLogFilename()} in order to receive the absolute path to log file, empty if log has not been opened (native path style).

\paragraph{<string> GetConfigFilename()}
Write \emph{<RTTI object>:GetConfigFilename()} in order to receive the absolute path to config file, empty if no config is used (native path style).
