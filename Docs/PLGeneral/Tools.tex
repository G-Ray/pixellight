\chapter{Tools}




\section{Log}
Normally PixelLight will write down different information in the text file \emph{PLLog.txt} in the current application root directory when using \emph{PLEngine} and above. This log will receive e.g. system information and warning/error messages. After an error or even crash occurred it's useful to have a look at the log which can maybe offer some information about the error. You can write into the log by self using \emph{Log::Output()}.

Example:

\begin{lstlisting}[caption=Using the log singleton directly]
Log::GetInstance()->Output(String::Format("State of my variable is %d", nMyVariable));
\end{lstlisting}

There's a special log write function for debugging proposes taking as first parameter the log mode level at which this message should be written into the log. This should be used if you want to write special messages into the log only if the framework currently runs in a special log mode. The current log level of the log can be set with \emph{Log::SetLogLevel()}. There's a log macro that is more comfortable to use than using the log singleton directly.

Example:

\begin{lstlisting}[caption=Using the log singleton though the log macros]
PL_LOG(Error, "Can't find 'Effect' element")
\end{lstlisting}





\section{Timing}


\paragraph{Overview}
The PixelLight framework offers a class with a huge set of timing relevant functions which can be found in \emph{PLGeneral::System} and \emph{PLGeneral::Timing}. The function \emph{System::Sleep()} will hold on the program for a given number of milliseconds. Use this function only in special situations! \emph{System::GetMilliseconds()} retrieves the time that have elapsed since the system was started and \emph{Timing::GetPastTime()} the past time since the application was started. The returned information could be used to e.g. cause an event after a given time since the last check. Measure the past time between two points can be done with the \emph{Stopwatch}-class which is especially useful for debugging.


\paragraph{Getting the time}
Within the \emph{Timing}-class, there are different functions dealing with timing relevant data. The most important of them is the \emph{GetTimeDifference()}-function which returns the time difference in seconds since the last frame. This information is that important because it is used to create FPS independent movement, meaning that your actors will always walk with the same velocity at 100 FPS or 10 FPS. If the time difference is less your actors will move smooth\ldots if your FPS is low the time difference between two frames is high and your actors will stammer or even jump around. If your system is blocked through something and the program holds for a while, this unexpected extreme high time difference could cause many problems. Therefore you are able to set a maximum allowed time difference with \emph{SetMaxTimeDifference()}. Normally 0.15 is a good setting which is set by default. Now you could ensure that the time difference is NEVER higher than this value! If the FPS is extreme low and the time difference is clamped always to the maximum allow your actors would apparent slow down but that's better then jumping into something terrible which would brother the user. Go sure that you ALWAYS multiply the time difference if you move, animate etc. something. If you don't do this it might look OK on our system in the tested situation, but on another system or situation the animation would be too fast or slow which might end in unexpected errors. By holding this rule you will automatically profit of different features coming with the timer e.g. you get slow motion for free!


\paragraph{Timing effects}
The PixelLight timing system offers different functions to manipulate the timing.

Scaling the time is one of the functions which could be used to create impressing special effects like slow motion or freezing the scene while moving with the camera around an object. There are 3 time scale functions. The first \emph{SetTimeScaleFactor()} is a general time scale factor which is always applied. Use this only in special situations like for debugging proposes. For example its useful to artificially slow down the time to be able to see whether there are objects with a wrong timing. (maybe they don't use the time difference!) For slow motion effects take \emph{SetSlowMotionFactor()} instead because you are able to activate/deactivate it using \emph{SetSlowMotion()}. \emph{SetCustomSlowMotionFactor()} is the third provided time scale function which could be used if different slow motion factors should be combined without any effort. Use \emph{SetActive()} to activate/deactivate. If the timer is inactive the time difference isn't updated. Using \emph{Freeze()} you freeze the timer and the time difference is set to 0 and all other timing data aren't updated. Also, usually scene node relevant things like the input aren't updated it the timer is frozen\ldots all will hold until you unfreeze it. A normal pause is done with the function \emph{Pause()}. The pause mode has no influence on the timer itself, but e.g. the scene nodes are not updated if their \emph{NoPause} flag is not set. By set this scene node variable to 0 the scene node is not influenced through the pause which could e.g. be used to create special effects like a frozen scene and a free moveable camera.

Another timing feature is a FPS limit which could be set to avoid that the FPS is never higher that the limit you set with \emph{SetFPSLimit()}. The FPS limitation could be used to avoid that the application uses to much system resources and therefore to save energy. (normally 30 FPS are enough) Or you are able to simulate a low FPS for debugging proposes which could be useful sometimes. But in general you should use a FPS limitation only in special situations - users love FPS over 120 even if its more or less senseless!





\section{Profiling}


\paragraph{Overview}
The PixelLight profiling tool is extreme useful to acquire information about certain code parts and their performance. If the profiling is activated with \emph{Profiling::GetInstance()->SetActive(true)} it may display you information like current FPS, triangle count, number of scene nodes, node update time etc. With this information you are able to find out were your performance is burned. The profiling information are subdivided in groups like general, scenes, textures etc. containing elements were you can browse through using offered functions like \emph{Profiling::SelectNextGroup()}.

By default the PixelLight console offers all required commands to handle the profiling tool in a comfortable way therefore there's no real need to use the profiling browse functions offered by the profiling interface.


\paragraph{Adding own groups and elements}
The profiling information can be expand with own information you maybe want to add to inspect the runtime behaviour of certain codes. In fact that's much easier as it may sound first\ldots in real it's only ONE function called \emph{Profiling::Set()} you need to use to customize the PixelLight profiling tool!!

Example:

\begin{lstlisting}[caption=Profiling usage example]
// Add profiling information
Profiling *pProfiling = Profiling::GetInstance()
pProfiling->Set("MyProfilingGroup", "Name: " + sName);
pProfiling->Set("MyProfilingGroup", "Level: " + nLevel);
\end{lstlisting}

This will add the defined elements to the own profiling group \emph{MyProfilingGroup}. If this group doesn't exist yet it's created automatically, therefore you done have to be interested in the stuff behind all! If you want to do more than only the described above it's still possible because you have FULL access to all required to remove groups and so on.\footnote{Because the profiling class is derived from \emph{ElementManager}}
