\chapter{General}




\section{Keep it Simple}
In case you're one of the persons really hating this phrase and laughing about people using it, stop reading this document because you're probably don't like the rest of it either. For the rest, go on reading.

The world is complicated enough - if there are multiple solutions, prefer the simplest over the most complicated one. This way, the chances are high that other will understand the solution as well as you when looking at the code some years later.




\section{Encoding}
We work with multiple operation systems so we have to take into account \emph{how} text files are saved. All \emph{Diary}, \emph{Readme}, \emph{Todo} and \emph{Plan} text files saved as "Unicode (UTF-8 with signature) - Codepage 65001". All other text files like code or make files saved in classic \emph{ANSI}.




\section{File Extension}
\label{General:FileExtension}
First of all, we use the old fashion \emph{.h}-file extension to mark header files - \emph{hpp} would be the \emph{correct} extension for C++, but it's not widely used. Usually, we outsource inline implementations into files with an \emph{inl}-extension to keep the header files good readable. For source codes, we use the file extension \emph{cpp}.




\section{C++11 Language Features}
Using C++11 (previously known as C++0x) language features is fine as long as
\begin{itemize}
\item{It's possible to emulate, or at least deactivate the feature within compilers don't supporting it (yet)}
\item{There's no comfortable/acceptable way to solve a task without using the feature, but those situations have to be discussed within the team and community}
\end{itemize}
\textsl{}
Currently the following C++11 language features are used:
\begin{itemize}
\item{\emph{nullptr} - a null pointer literal}
\item{\emph{override} - gives the compiler a chance to detect and blame errors related to overwriting methods}
\end{itemize}


\paragraph{Extern Templates}
Don't use \emph{extern templates}\footnote{\url{http://www2.research.att.com/~bs/C++0xFAQ.html\#extern-templates}} in order to avoid template instantiation in other modules. This is a feature we can't emulate and there are still legacy compilers like \ac{GCC} 4.2.1 used on Mac OS X 10.6 actively used around the world. So, at least for now, don't use this useful feature.




\section{For-Scope}
\label{General:For-Scope}
Within the PixelLight projects, the \emph{for-scope} is active by default. The following for instance will produce a compiler error:
\begin{lstlisting}[caption=for-scope]
for (int i=0; i<1; i++) {
	// Do anything
}
i = 20;	// i has already gone out of scope
\end{lstlisting}




\section{Locality and Scope}
Locality means that one should keep things together whenever possible. A good example is the \emph{for-scope} from section~\ref{General:For-Scope}. Do not define all variables you're using within a function right at the top of the function. Define and initialize local variables when you first need them. If reasonable, it's also fine to add a new scope by using brackets in order to restrict the life of a local variable which is only required within a tiny section of your implementation. Consider moving this inner bracket into an own function if it makes sense in some way, but this is no requirement if you can's see any point in adding another tiny function.

Having only a hand full of local variables with a well defined scope makes it easier to understand and debug the code. It also avoids trashing the valuable C runtime stack. If there's only a limited number of local variables, a good compiler might also have the chance to produce more effective code. Another benefit of this approach is that this local variable may never be created and touched at all in case the part of the implementation is never executed.




\section{C runtime stack Vs. C heap}
Prefer allocating variables on the C runtime stack over the C heap whenever possible. Dynamic memory allocation and deallocation via the C heap is considered to be slow compared to using the C runtime stack. The C heap introduces the risk of memory leaks while the C runtime stack is cleaned up automatically when leaving a scope. Also the C heap suffers from something called fragmentation and caching may be an issue. In concurrent programming the C runtime stack is less painful because every thread has it's own C runtime stack, while all threads share the same C heap which is a source of really nasty bugs in case one is sloppy when dealing with dynamically allocated memory.

Be aware of the difference between C runtime stack and C heap. If you're not familiar or unsafe about this terminology and topic, please refer to the widely available literature about C/C++. For example, adding a class member variable instance directly to a class, meaning without using a pointer, doesn't necessarily mean that this will be super fast because this member variable will be automatically allocated on the C runtime stack. If this class is allocated on the C heap, the member variable will be allocated on the C heap as well.




\section{Global Variables and Global Functions}
Whenever possible, try to avoid using global variables and global functions. If you really have to, because e.g. everything else would be overcomplex for a tiny application, document at least why you had to make it global. In general, especially global variables lead to hard to understand, debuggable and maintainable code.




\section{Null Pointer}
For null pointers, use \emph{nullptr}\footnote{For example \emph{Microsoft Visual Studio 2010} and \emph{\ac{GCC} 4.6} have native support for \emph{nullptr}} from \emph{C++11} and not for example the legacy, but traditional \emph{NULL} definition or even directly a integer $0$.

\begin{lstlisting}[caption=Null pointer]
char *pszMyString = nullptr;
\end{lstlisting}




\section{Overwriting Methods}
Put the methods within the base class, which are allowed to be overwritten, into a separate code block so everyone is able to find them at once.
\begin{lstlisting}[caption=Virtual methods within a base class]
//[-------------------------------------------------------]
//[ Public virtual MyClass functions                      ]
//[-------------------------------------------------------]
virtual void MyMethod() = 0;
\end{lstlisting}

When overwriting virtual methods within a derived class, put the overwritten methods into a code block telling were those methods originally came from.
\begin{lstlisting}[caption=Overwriting virtual methods within a derived class]
//[-------------------------------------------------------]
//[ Public virtual MyClass functions                      ]
//[-------------------------------------------------------]
virtual void MyMethod() override;
\end{lstlisting}
Although technically not required, do also add a \emph{virtual} to make it absolutely clear that this is a virtual method. To give the compiler a chance to find and blame possible errors like a signature change within the base class, use the C++11 language keyword \emph{override}.




\section{Casting}
PixelLight is using \emph{C++ style casts} (\emph{int i = static\_cast<int>(42.21f)}), not \emph{C style casts} (\emph{int i = (int)42.21f}). It's much easier to search for \emph{C++ style casts}\footnote{\emph{static\_cast}, \emph{reinterpret\_cast}, \emph{const\_cast}, and \emph{dynamic\_cast}} and they are less vulnerable to unintended effects as well - and because they are not that compact as the \emph{C style casts}, one may think about it a second time why there's a need for a cast.

\paragraph{\ac{GCC}}
\ac{GCC}, offers an option called \emph{-Wold-style-cast} to let the compiler warn if an old-style (C-style) cast to a non-void type is used within a C++ program.




\section{Const Correctness}
Define functions, variables etc. whenever possible to be constant. By giving the compiler this hint, it may be possible to use special optimizations or uncover bugs within the implementation.


\paragraph{Const Example}
Have a look at the following example function:
\begin{lstlisting}[caption=Non-constant function parameter]
void MyFunction(PLMath::Vector3 &vPosition)
{
	// ...
}
\end{lstlisting}
In case the function is considered to manipulate \emph{vPosition}, all's fine. Let's continue with another usage example:
\begin{lstlisting}[caption=Using a temporary variable instance as non-constant function parameter]
MyFunction(PLMath::Vector3(0.0f, 1.0f, 2.0f));
\end{lstlisting}
The compiler creates a temporary \emph{PLMath::Vector3} instance on the runtime stack and is passing a reference into the function. Looks fine as well? In case the function manipulates the variable passed by reference, the temporary instance is changed. In principle that's no problem because it's thrown away after the function call anyway. In practice, this situation is considered to be evil. \ac{MSVC} 2010 will shy tell you via a warning
\begin{quote}warning C4239: nonstandard extension used : 'argument' : conversion from 'PLMath::Vector3' to 'PLMath::Vector3 \&' A non-const reference may only be bound to an lvalue\end{quote}
So, it's possible to just ignore or even deactivate this warning... isn't it? In fact, it isn't because other compilers might not be that tollerant and will throw an error message at you. To sum this up: In the example above, you really want to write
\begin{lstlisting}[caption=Constant function parameter]
void MyFunction(const PLMath::Vector3 &vPosition)
{
	// ...
}
\end{lstlisting}
in order to be cross-platform safe.


\paragraph{Const Exception}
There's one situation were we do not use \emph{const} - when dealing with function parameters because

\begin{lstlisting}[caption=Function parameters]
void MyFunction(int nVariable1, int nVariable2);
\end{lstlisting}

is inside headers better readable than

\begin{lstlisting}[caption=Constant function parameters]
void MyFunction(const int nVariable1, const int nVariable2);
\end{lstlisting}

In this situation, the readability is more important for us. This rule does not apply for pointer or reference parameters like

\begin{lstlisting}[caption=Constant function pointer/reference parameter]
void MyFunction(const String &sVariable);
\end{lstlisting}

because the user should be able to see whether or not a function is going to manipulate the parameter variable.




\section{static const Vs. const static}
Use \emph{static const} instead of \emph{const static}. Have a look at e.g. the \ac{GCC} option \emph{Wold-style-declaration} resulting in the warning \begin{quote}`static' is not at beginning of declaration\end{quote} or into chapter 6.11 of ISO C99 (\emph{Future Language Directions} -> \emph{Storage-class specifiers}).




\section{Namespaces}
PixelLight is using multiple namespaces, one for each sub-project. If you want to use for instance the string class which is defined in \emph{PLCore} you need to do this:

\begin{lstlisting}[caption=Explicit namespace]
PLCore::String sMyString;
\end{lstlisting}

Or this:

\begin{lstlisting}[caption=Using namespace]
using namespace PLCore;
...
String sMyString;
\end{lstlisting}

Try to avoid using \emph{using namespace} too often or this will result in name conflicts which you then have to resolve by hand by adding for instance \emph{PLCore::}. We recommend to never use \emph{using namespace} within header files.




\section{Inline}
Whenever reasonable, mark methods as inline in order to give the compiler a chance to optimize out function calls. This may not always result in a huge performance improvement, but we have to use features like inline to get a decent overall performance and restricting the performance impact introduced by heavily using \ac{OOP}. This is especially true when considering devices like Smartphones which have more restricted resources as a powerful workstation. For example, by marking a getter-method as inline, requesting the value of a member variable may not introduce any performance drawback compared to making the attribute public and accessing it directly.

Potential candidates for inline are one-line-methods which don't require to include more headers within the class headers, meaning that the header complexity is not significantly increasing by using inline. Other candidates are methods which require only to include really lightweight headers or headers which have usually already been included. Beside the mentioned one-line-methods, methods with multiple instructions and lines can also be marked as inline as long as you consider it as reasonable after thinking it thru again. Complex methods should not be marked as inline to avoid increasing the resulting binary size and therefore making the life of the cache harder. Do only mark complex methods as inline if there are extremely good, and especially documented reasons for doing so. Do not inline methods if this requires to include system or so called \emph{standard}-headers, except there's a really good and heavily documented reason for doing so.

Do not directly write the inline implementation into the class header. The class headers have to stay human readable and shouldn't disclose to many implementation details - not because they are top secret, but because the class has to be usable by only looking at it's interface. So, move the inline implementations into a separate \emph{inl} file as described within section~\ref{General:FileExtension}. Of course, the inline implementation is still visible to the user, but not when looking at the class header directly.




\section{Dynamic Parameters}
When dynamic parameters are used and the name of the parameters inside a string is irrelevant, as this is the case for \emph{PLCore::Params::FromString}, the parameters are named using \emph{Param<x>} were x starts with $0$ (example: \emph{Param0=1 Param1=''Hello''}).  




\section{Names}
In general, names of classes, functions, variables and so on must have human readable names. The name has to tell as much as possible about the usage - if the user can guess correctly the usage of for example a variable by just looking at it's name, the name is perfect.

General rules:

\begin{itemize}
\item A single character as name for local (really, only local) control variables like \emph{i} within a for-loop is acceptable as long as there are not to much of those at once (else use reasonable names to avoid confusion)
\item Short cuts should be avoided whenever possible because they may leads to confusion\footnote{True story: When using \emph{Rot} as short cut for \emph{Rotation}, we once had the situation that a German speaking programmer asked confused what the color \emph{Rot} should do inside the scene node... in German, \emph{Rot} is the word for \emph{red}...} (no stuff like \emph{stricmp()})
\item If there's a \emph{commonly used} name for something, use this name instead of creating a totally new one. Example: Don't call an array class \emph{Bob} just because this name is cool - use the name \emph{Array} for an array class instead and the users might be able to find it.
\item Avoid long names, if there's an expressive shorter name it's the preferred one... but keep the short cut rule in mind
\end{itemize}

Classes, structures and so on have a upper case letter at the beginning. Example:

\begin{lstlisting}[caption=Name convention]
class Player {
};
struct Info {
};
\end{lstlisting}




\section{Prefix}
Because the readability of code is extremely important when working in a team and/or using code from others, one of our goals was to make the PixelLight code as readable and well structured as possible. We are using a name style convention\footnote{We know that there are a lot of discussions around the internet whether or not prefixes should be used. In the year 2002 we decided to use them and we don't change it - due to the dimension of the PixelLight project, it would be a huge effort to change it anyway.}.

Variable prefixes for standard types:

\begin{lstlisting}[caption=Variable prefixes for standard types]
Type               Prefix    Example
bool               b         bool bActive

(n for all none standard floating point types)
int                n         int  nNumber
char               n         char nCharacter
long               n         long nHuge

float              f
double             d

(Character arrays -> strings)
char[]             sz        char szName[64]
char*              psz       char *pszName

(Pointers)
*                  p         Player *pPlayer

(References)
&                  -         char &nTest = nTest2;

struct instance    s         Info sPlayer (struct Info)

class instance     c         Player cPlayer (class Player)
\end{lstlisting}

General variable prefix for class variables:
m\_ (m for member)\\
Example: char *m\_pszName\\

Variable prefixes for PixelLight types:

\begin{lstlisting}[caption=Variable prefixes for PixelLight types]
Type               Prefix    Example
String             s         String sName

Container          lst       List lstNames

Map                map       HashMap mapNames

VectorX            v         Vector3 vPosition
(X for dimension: 2, 3 or 4)

MatrixXxX          m         Matrix4x4 mRotation
(X for dimension: 3 or 4)

Quaternion         q         Quaternion qRotation

ColorX             c         Color3 cColor (same as class)
(X for dimension: 3 or 4)
\end{lstlisting}




\section{Postfix}
We recommend you to use the PixelLight name convention and marking debug versions with a \emph{D} at the end of the filename. Example: \emph{MyPlugins.dll} = release version, \emph{MyPluginsD.dll} = debug version.




\section{Events and Signals}
As soon as an event is inside a class, we refer to it as \emph{signal}. As such, the prefix \emph{Event} like within \emph{EventKeyDown} is used outside classes while prefix \emph{Signal} like within \emph{SignalKeyDown} is used inside classes.




\section{Event Handlers and Slots}
Within our name convention for event handlers and \ac{RTTI} slot names, there's a \emph{On} within for example \emph{OnMyEvent} indicating that this is a handler/slot method. The other part of the name consists of the name of the event/signal - for \emph{OnMyEvent} this would be an event/signal with the name \emph{MyEvent}.




\section{\ac{RTTI} Interface}
Within PixelLight, the \ac{RTTI} class properties and members are always defined in the following order:
\begin{itemize}
\item Properties
\item Attributes
\item Constructors
\item Methods
\item Signals
\item Slots
\end{itemize}
This way one knows exactly were to look for something. Further, within the \ac{RTTI} class properties and members definitions, only tabs and no spaces are used to make it easier to write the definitions like a table. This makes it more comfortable for the eyes and brain to navigate to certain definition parts without to much searching around.

Here's an example source code showing the common \ac{RTTI} interface layout (without the word wrap):
\begin{lstlisting}[caption=\ac{RTTI} interface (without the word wrap)]
//[-------------------------------------------------------]
//[ RTTI interface                                        ]
//[-------------------------------------------------------]
pl_class(pl_rtti_export, MyRTTIClass, "", PLCore::Object, "Sample RTTI class, don't take it to serious")
	// Properties
	pl_properties
		pl_property("MyProperty",	"This is a property value")
	pl_properties_end
	// Attributes
	pl_attribute(Name,	PLCore::String,	"Bob",	ReadWrite,	GetSet,			"A name, emits MySignal after the name was changed",			"")
	pl_attribute(Level,	int,			1,		ReadWrite,	DirectValue,	"Level, automatically increased on get/set name and OnMyEvent",	"")
	// Constructors
	pl_constructor_0(DefaultConstructor,	"Default constructor",	"")
	// Methods
	pl_method_0(Return42,			int,					"Returns 42",							"")
	pl_method_1(IgnoreTheParameter,	void,			float,	"Ignores the provided parameter",		"")
	pl_method_0(SaySomethingWise,	void,					"Says something wise",					"")
	pl_method_0(GetSelf,			MyRTTIClass*,			"Returns a pointer to this instance",	"")
	// Signals
	pl_signal_1(MySignal,	PLCore::String,	"My signal, automatically emitted after the name was changed",	"")
	// Slots
	pl_slot_0(OnMyEvent,	"My slot",	"")
pl_class_end
\end{lstlisting}




\section{Reuseability and adding new Stuff}
Before you add new classes, functions and so on - check first whether there's already something similar within PixelLight. If there's something you can already use directly, use it instead of writing new stuff. If there's something quite similar, have a more detailed look at it and contact your team colleagues to discuss whether a refactoring is possible and reasonable to update and/or to enhance existing stuff.

This is also true for adding new precompiler definitions. Look deeply whether or not there's already a definition (or better method) existing which can be used. If new precompiler definitions are really required we should dicuss them first because adding new of thoses may make it harder to compile PixelLight in the correct way.

Reuseability is one of the most important concepts when creating frameworks like PixelLight... and reuseability does not mean that it's possible to copy'n'past it and then hacking around for a certain project. Reuseability means that it's possible to directly reuse, to share something between multiple projects in a quite universal way without the need to enhance and hack around constantly.
