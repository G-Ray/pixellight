\chapter{Introduction}
This is the PixelLight feature list which is split into the different project packages. This introduction will inform you about the general features. Please note that due the size of PixelLight, it's not possible to mention each and every single feature within this document.





\section{API}
\begin{itemize}
\item{By design the framework is split up into an 'immediate' and a 'composed' part}
\item{The 'immediate' functions make it possible to immediately display all kinds of resources (e.g. textures, meshes or sounds) on the screen. These low-level functions can be used for customization of the framework or for simplicity where no internal structures (like objects or scenes) are needed}
\item{The 'composed' part actually contains the main framework features by defining structures for whole virtual scenes and objects reacting to each other. See 'Scene System' for detailed description}
\end{itemize}





\section{Documentation and examples}
\begin{itemize}
\item{In general, the PixelLight SDK comes with a huge set of documents and diagrams which will make your work with this technology much easier}
\item{Quite comfortable SDK browser}
\item{Separate documentation for each main component}
\item{Detailed API documentation for programmers - automatically created with Doxygen\footnote{\url{http://www.doxygen.org/}}}
\item{A huge set of different example programs and scenes which will demonstrate different things}
\item{More documentation, tutorials, FAQ's and so on can be found within the official PixelLight Wiki\footnote{\url{https://developer.pixellight.org/wiki/index.php/Main_Page}}}
\end{itemize}





\section{PixelLight is NOT an...}
\begin{itemize}
\item{Pure Click'n'Play toolkit - there are several useful tools, but you still need programmers to add own special functionality!}
\item{Pure game engine. PixelLight is an universal framework which can also be used for 'serious' applications like product visualization, simulation or E-learning.}
\item{ONE package - PixelLight consists of different components working together}
\end{itemize}





\section{General}
\begin{itemize}
\item{Completely programmed in C++ using modern C++0x language features like the null pointer literal \emph{nullptr}, \emph{extern templates} or \emph{override} enabling the compiler to detect and blame errors related to overwriting methods}
\item{Whenever possible, well known design patterns are used}
\item{Well structured and understandable code, due to strictly object oriented design}
\item{Detailed documented code with explanations of parameters and return values, examples and notes}
\item{Easily portable to other operation systems - currently Microsoft Windows, Linux and the mobile platform Maemo\footnote{Used for example by \emph{Nokia N900}} are supported}
\item{Build system using CMake to make multi-platform development more comfortable\footnote{\emph{Microsoft Visual Studio 2010} project files are provided as well for comfortable development under \emph{Microsoft Windows}}}
\item{Clever RTTI (runtime type information) system which handles different classes with attributes, methods, signals, slots and properties. Classes are defined in modules where a module can be an executable or dynamic library. Therefore, the plugin-system' is naturally provided through the RTTI in an efficient manner.}
\item{The framework comes with several development tools, plugins and libraries, as well as a lot of documentation and example programs}
\item{The SDK itself is split up into several sub-projects to increase productivity. For instance there is a project with general classes like lists, hashtables etc. a mathematics library with vector, matrix etc.}
\item{Most PixelLight file formats are XML based\footnote{The primary chunk and mesh formats are binary based for efficiency, but they have XML counterparts as well for better data exchange}}
\item{Many components are implemented with abstract interfaces and multiple default implementations. Own implementations can be added without any effort, making PixelLight extremely flexible and extensible.}
\item{The headers are compact by using forward declarations where possible instead of hard includes. They are well documented and contain everything needed, and as such don't require to include many resources just to use one.}
\end{itemize}





\section{Debugging and tweaking tools}
\begin{itemize}
\item{PixelLight comes with a set of tools which shorten the development time. Using them makes it easier to find bugs and performance consumers in order to eliminate them!}
\item{A console with a user friendly interface including auto-complete, history etc. It is also possible to register own new commands!}
\item{Log system which is connected with the console and is able to print messages which could also be split up into debug mode dependent information}
\item{Useful and customizable profiling tool which allows you to control different code parts during runtime in order to find out where your performance is burned! There is a lot of standard framework information like current fps, triangle count, rendering time etc.}
\item{For debugging purposes, PixelLight provides an extensive debug dialog were you are able to, for example, inspect and manipulate the different managers like the textures, meshes etc. There you can see whether your resources were loaded in the correct way and how often they are used by the certain resource managers. It is also possible to load new or replace/update existing resources as well as deleting some. For example, it is possible to open a texture, edit it and then reload it in order to see the changes at once in your project... it is not required to restart your program! This is an extremely useful feature of the technology for the whole development team, as it enables real-time editing and saves a lot of time.}
\end{itemize}
\newpage
