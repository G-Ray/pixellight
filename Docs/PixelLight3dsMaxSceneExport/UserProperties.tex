\chapter{User Properties}
Each \emph{Autodesk 3ds Max} node can have different user properties. To edit this properties select a node, open the popup menu using the right mouse button, select \emph{Object Properties...} and change into the \emph{User Defined} tab. By using this properties, you can append scene node modifiers or set custom \ac{RTTI} variables. For instance, if there's a \emph{monster class} provided by your programmers which has the variables \emph{Health}, \emph{Speed} etc. One can directly set this variables within \emph{Autodesk 3ds Max} by adding this user properties to an object:

\begin{lstlisting}[caption=Setting scene node variables]
Vars=Health="100" Speed="0.2"
\end{lstlisting}

\textbf{Put the variable value into "" (\emph{"<value>"}) for sure. If the variable name has blanks (not recommended!) within it or special characters, set the variable name into "", too!}

\begin{lstlisting}[caption=Setting scene node variables safely]
Vars="Health"="100" "Speed"="0.2"
\end{lstlisting}

As told above, this variables are not implemented within the exporter itself. If you don't know which variables can be set, look into a documentation (if there's one) or ask your programmer.

One PixelLight scene node can have multiple modifiers assigned to it. This modifiers are flexible/universal, too. Look into a documentation (if there's one) or ask your programmer again which modifiers are available.

For instance it's possible to let a node \emph{walk} on a path by using:

\begin{lstlisting}[caption=Path scene node modifier]
Mod=Class=PLScene::SNMPositionPath Filename="SampleSound\_Soldier.path" Speed="0.2"
\end{lstlisting}

If the node should also \emph{look} into the direction it moves, add a second modifier controlling the rotation:

\begin{lstlisting}[caption=Look into movement direction scene node modifier]
Mod1=Class="PLScene::SNMRotationMoveDirection"
\end{lstlisting}

As you see, the first modifier is called \emph{Mod}, the second \emph{Mod1} and so on. Because the exporter doesn't know and also don't need to know anything about the stuff \emph{behind} for instance \emph{Mod=} or \emph{Vars=} the \emph{value} of the \emph{key} is written directly into the exported PixelLight scene file. So, do not put this string into "" like

\begin{lstlisting}[caption=Invalid scene node modifier definition]
Mod1="Class=PLScene::SNMRotationMoveDirection"
\end{lstlisting}

because this may cause problems.

Physics attributes are also set by using modifiers:

\begin{lstlisting}[caption=Physics mesh scene node modifier]
Mod=Class="PLPhysics::SNMPhysicsBodyMesh"
\end{lstlisting}

This modifier is adding a static physics collision object using the mesh of the node. If you have a ball which should \emph{roll away} you can for instance add this modifier:

\begin{lstlisting}[caption=Physics sphere scene node modifier]
Mod=Class="PLPhysics::SNMPhysicsBodySphere" Mass="1" Radius="1" AutoFreeze="0" InitFrozen="0"
\end{lstlisting}

Note, normally physics objects are \emph{frozen} automatically if they don't move a while. If physics objects are created they are normally within a frozen state. In the sample above we don't allow auto freeze and the physics object is active after creation - so your ball is rolling away by self. Note that it's not recommended to activate all physics objects after creation or deactivate auto freeze - this would slow down the physics simulation!


\textbf{Additional node flags:}
You can give the PixelLight scene node additional flags by writing for instance

\begin{lstlisting}[caption=Setting scene node flags]
Vars=Flags="NoLighting|NoCulling"
\end{lstlisting}

which will disable the lighting and culling in this sample.
