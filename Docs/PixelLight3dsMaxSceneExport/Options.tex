\chapter{Options}
The PixelLight scene exporter has multiple options you can load, save and reset to their default values. Further, the current settings are automatically saved and restored - if you restart \emph{Autodesk 3ds Max}, your settings will be still there. You can find the PixelLight scene exporter configuration file under \emph{*3dsMaxPath*/plugcfg/PixelLight\_SceneExporterSettings.ini}.

\textbf{General export settings:}
\begin{itemize}
\item{\textbf{Remove spaces/tabs within names:} If a name with spaces/tabs is found, a log warning is written and the name is fixed automatically if this option is active.}
\item{\textbf{Correct portals:} If a cell-portal is found which doesn't look into the cell it is in, a log warning is written and the vertex order is flipped automatically if this option is active.}
\item{\textbf{Overwrite ambient color:} By using this option you can overwrite the environment ambient color of \emph{Autodesk 3ds Max} used as default ambient color within the exported scene.}
\item{\textbf{Animation playback:} By default animations within the scene are played automatically. By using this option you can disable this.}
\item{\textbf{Scene renderer:} Within this combobox you can select the desired scene renderer that should be used for rendering the scene. Note that this is just a \emph{hint}, programs are not forced to use this information. You can add your own scene renderers by editing \emph{PixelLight\_SceneExporterSettings.ini}.}
\item{\textbf{Show exported scene:} If the checkbox \emph{Show} is checked, the exported scene is loaded automatically by using the given program. The absolute filename of the exported scene is given to this program as parameter, so the program must accept this parameter. If the PixelLight \ac{SDK} is installed correctly, this viewer filename is by default the correct absolute path to \emph{PLViewer.exe} which can be used as \emph{scene viewer}.}
\item{\textbf{Publish:} If the checkbox \emph{Publish} is checked, the exporter will also put everything into the exported directory you need to run \emph{PLViewer.exe} with the \emph{standard features} without the PixelLight \ac{SDK}. The PixelLight \ac{SDK} must be installed, else this feature will fail.}
\end{itemize}

\textbf{User properties export settings}
\begin{itemize}
\item{\textbf{Variables:} Use the \emph{Vars} key within the user properties?}
\item{\textbf{Modifiers:} Use the \emph{Mod<n>} key within the user properties?}
\end{itemize}

\textbf{Material export settings:}
\begin{itemize}
\item{\textbf{Create materials:} Create PixelLight material files? (\emph{mat}-extension)}
\item{\textbf{Smart material parameters:} If this is active, known parameters with default values are not saved, parameters like textures without an influence are ignored and so on - so the resulting materials are quite compact.}
\item{\textbf{Copy textures:} Copy the textures (bitmaps) used by the scene into the same directory the scene is saved? If textures have valid absolute filenames those will be used. Else the absolute bitmap source filename is constructed by using the filename without the given original path and the path of the current \emph{Autodesk 3ds Max} scene will be used. If the scene was loaded using drag'n'drop and the scene was not dropped within a render viewport this may not work correct. If the texture can't be found within this directory, the \emph{Autodesk 3ds Max} map directories will be used instead. If a given texture can't be found, a log message is written into the log and the texture can't be copied. If there's a \emph{plt}-file for a texture, this file is copied, too.}
\item{\textbf{PL directories:} Put the data (scene, meshes, textures and materials) into different PixelLight style directories? If yes, the scene is put into the \emph{Scenes} subdirectory, meshes are put into the \emph{Meshes} subdirectory, all textures into the \emph{Textures} subdirectory and all materials into the \emph{Materials} subdirectory.}
\item{\textbf{Sub directories:} If this option is active and \emph{PixelLight directories} is also active, the exporter data is placed in sub directories to avoid name conflicts with other resources. For instance all textures of the mesh named \emph{MyMesh} will be put into \emph{Data/Textures/MyMesh/} and all materials will be put into \emph{Data/Materials/MyMesh/}. The scene itself is saved within \emph{Data/Scenes/MyScene.scene}. Now you can pack the scene together with its resources without any problems into a zip file and put it into the data sub-directory of your project. Then when loading the scene \emph{MyScene.scene} all should went fine.}
\end{itemize}

\textbf{Mesh export settings:}
\begin{itemize}
\item{\textbf{UV channels:} Number of texture coordinates per vertex mapping coordinates. u, uv or uvw is allowed were uv is the default value for all channels.}
\item{\textbf{Normals:} Export vertex normals?}
\item{\textbf{Tangents:} Export vertex tangents?}
\item{\textbf{Binormals:} Export vertex binormals?}
\end{itemize}

You can deactivate \textbf{Tangents} and \textbf{Binormals} if you for instance use no per pixel lighting - this way you safe resources.

\textbf{Log export settings:}
If the export log is enabled a log is written within the application data directory of the used viewer - but with the filename ending \emph{log}. (simple ASCII file) The log may help you to find errors or solutions when something went wrong during the export. The
log is opened automatically after the export is done.
\begin{itemize}
\item{\textbf{Inactive:} Do not create a log}
\item{\textbf{Errors:} Write errors into the log}
\item{\textbf{Warnings:} Write errors and warnings into the log}
\item{\textbf{Hints:} Write errors, warnings and some additional information that may be interesting/useful into the log}
\item{\textbf{Scene data:} All above + write detailed scene export information into the log}
\end{itemize}

\textbf{Other:}
\begin{itemize}
\item{\textbf{Scene container:} If this is not empty, the scene data is \emph{within} another container. If for instance the scene using physics, the scene must be within a physics world.}
\item{\textbf{Scene renderer:} The desired scene renderer - normally it's not required to change anything in here.}
\end{itemize}
