\chapter{Cells, Portals and Instancing}
\textbf{For the scene exporter, the \emph{Autodesk 3ds Max} node names are quite important because they are used to identify cells, portals, instancing and so on.} Within \emph{Autodesk 3ds Max} this names are not cases sensitive, (\emph{cell\_Room1\_Char = cell\_ROOm1\_ChAr}) but it's recommended to take account of how you write something because PixelLight itself is case-sensitive.

Your scene should be divided into cells which are connected through portals. Each node of your scene must be within a cell, else the exporter will write a hint into the log and append this node at the root of the exported PixelLight scene. The spatial subdivision of your scene is defined by using the \emph{Autodesk 3ds Max} node names. \emph{cell\_Room1\_Table1}, \emph{cell\_Room1\_Wall2}, \emph{cell\_Room2\_Monster}, \emph{cell\_Room2\_Floor} -> This will result in two cells called \emph{Room1} and \emph{Room2}. Within \emph{Room1} are the nodes \emph{Table1} and \emph{Wall2}, within \emph{Room2} are the nodes \emph{Monster} and \emph{Floor}.

The scene exporter will detect instanced meshes automatically. In this case, this mesh is only saved once and then used for multiple nodes within your scene. This way the scene is exported faster, the exported scene will need less space on the hard-drive - but most important, the time to load this scene will be shorter and the scene will eat up less system \& \ac{GPU} memory during runtime.

\textbf{Use instancing whenever possible.}\footnote{There are \emph{Autodesk 3ds Max} scripts like \url{http://www.scriptspot.com/3ds-max/scripts/instance-replace} helping with instancing}

There are two ways to do instancing:

\begin{itemize}
\item{By providing a mesh name within the \emph{Autodesk 3ds Max} node name}
\item{By using \emph{Autodesk 3ds Max} instancing, the recommended way to go}
\end{itemize}

As result, there are two ways to create scene nodes:

\textbf{1.}
\textbf{cell\_<cell name>\_<node name>}
\begin{itemize}
\item{\emph{<cell name>} = Name of the cell the node is in}
\item{\emph{<node name>} = Name of the resulting scene node within this cell. Each node within a cell should have an unique name. The exporter will write a warning into the log if nodes within a cell have the same name and resolves the name conflicts automatically by adding a number. If the \emph{Autodesk 3ds Max} node has a mesh, this mesh will have the same name as the node itself. If there's already a mesh with the same name within the scene, the name conflict is resolved automatically and a warning is written into the log. If the mesh is instanced by \emph{Autodesk 3ds Max}, the mesh is only saved once.}
\end{itemize}

\textbf{2.}
\textbf{cell\_<cell name>\_<mesh name>\_<instance name>}
\begin{itemize}
\item{\emph{<cell name>} = Name of the cell the node is in}
\item{\emph{<mesh name>} = Name of the mesh the node is using. Automatic mesh instancing is using this name during the export. The mesh of the first found \emph{Autodesk 3ds Max} node will be used for all other nodes with this mesh name - even if they have in fact another mesh within \emph{Autodesk 3ds Max}!}
\item{\emph{<instance name>} = Name of the instance. The resulting name of the scene node within this cell is given by \emph{<mesh name>\_<instance name>}. Each node within a cell should have an unique name. The exporter will write a warning into the log if nodes within a cell have same names and resolves the name conflicts automatically by adding a number.}
\end{itemize}

\textbf{Each node within a cell should have an unique name!}

\textbf{Do not use \emph{\_} within a name, this is a reserved character! Try to avoid using spaces/tabs within names, too. The exporter will write a warning into the log if spaces are found within a name.}

\textbf{Do not use \emph{.} within a name, within PixelLight this is a reserved character used for name hierarchies! If  a \emph{.} is found within a node name, a warning is written into the log and it is replaced by \emph{-}.}

\textbf{Do not use the names \emph{Root}, \emph{Parent} or \emph{This}, within PixelLight this are reserved names for the name hierarchy!} If nodes with such names are found, a warning is written into the log and the names are changed automatically.

Examples:
\begin{itemize}
\item{\emph{cell\_Room1\_Chair\_Chair2} -> Object with the name \emph{Chair2} within the cell \emph{Room1} using the mesh \emph{Chair}}
\item{\emph{cell\_Room2\_Chair3} -> Object with the name \emph{Chair3} within the cell \emph{Room2}, no mesh instancing by names, if the \emph{Autodesk 3ds Max} node itself doesn't use instancing, no automatic instancing during export can be performed and the same mesh is exported multiple times!}
\item{\emph{cell\_Room1\_Light1} -> Light with the name \emph{Light1} within the cell \emph{Room1}.}
\end{itemize}

Cells are only a collection of objects, they do not need a convex room mesh enclosing this area. But it's recommended to ensure that it's only possible to look \emph{out of} the cell through portals (doors, windows etc.).

\textbf{You can only look from one cell to another through portals!}

Portals will only go into one direction. To place a portal from \emph{Room1} to \emph{Room2} add for instance a flat rectangle within the door and call this portal object \emph{portal\_Room1\_Room2}. A second portal \emph{portal\_Room2\_Room1} is required to be able to go back from cell \emph{Room2} to \emph{Room1}.

\textbf{portal\_<from cell>\_<to cell>}

\textbf{The number of incoming portals should be the same as the number of outgoing portals - besides you will do something crazy}

The face normal of a portal must point into the cell the portal is in. For instance, if you have a portal \emph{portal\_Room1\_Room2}, this portal must point into \emph{Room1} because you look through this portal from \emph{Room1} into \emph{Room2}! Within \emph{Autodesk 3ds Max}, if you select the portal or it's faces, the selection (for instance the white edges or red faces) must be visible from inside the cell (for example \emph{Room1}) the portal is in. The exporter will write a warning into the log if a portal is looking into the wrong direction and if set within the exporter options, the polygon winding is fixed automatically. During export, the portals are converted into an polygon object, so it must be possible to convert your portal into such an object!\footnote{Simple quad geometries are pretty good for portals}

\textbf{Be carefully when placing portals. If portals are placed wrong, for instance two portals facing each other this may result in an infinite recursion which isn't a good thing during runtime, or at all.}

Take care of the size of your portals: If there's a portal \emph{within a door}, this portal should have the same size as this door and not overlap some miles out of the door. If a portal is smaller, normally more \emph{invisible} geometry and fragments can be throw away - which will increase the performance of your scene.
But do not create to much, to small portals. If there are for example multiple small \emph{windows} pointing all into the same target cell, it's recommended to combine this small portals into a huge one.

Beside the \emph{cell-portals} there are also \emph{anti-portals}. While \emph{cell-portals} allowing the viewer to \emph{look into} other cells, \emph{anti-portals} cover everything behind them (\emph{occlusion culling}) - this is for example usable within an outdoor scene.

\textbf{antiportal\_<name>}
\emph{antiportal\_MyPortal} is an example for a plane name.
