\chapter{Introduction}


\paragraph{Motivation}
The PixelLight input component, or short \emph{PLInput}, provides access and control over several input devices. Although the purpose of this project is to have an input library that perfectly integrates into the PixelLight framework, \emph{PLInput} can be used without other PixelLight components like the rendering system as well.




\section{External Dependences}
The core of PLInput depends on the \textbf{PLCore} and \textbf{PLMath} libraries. PLInput is using some platform dependent third party libraries, but usually, the resulting binary of PLInput library is stand alone and does not force you to deliver additional shared libraries, too.


\paragraph{Microsoft Windows}
When compiling for \emph{Microsoft Windows}, there are no additional external dependencies - the few required \emph{hid.dll} system shared library functions are loaded dynamically.




\section{Important Terminology}
Within the PixelLight input system, there's some some important terminology we first need to define before we can go into details.

\paragraph{Controller, Controls, Buttons, Axis}
A \emph{controller} represents an input device, which can either be a real device like e.g. a mouse or joystick, or a virtual device that is used to map real input devices to virtual axes and keys. A controller consists of a list of \emph{controls}, e.g. \emph{buttons} or \emph{axes} and provides methods to obtain the status. While a button can be pressed, not not pressed, an axis has quantized values within a given interval.

\paragraph{Absolute And Relative Axis}
Depending on the input device and axis type, the value of an axis can be \emph{absolute} or \emph{relative}. A good example for an absolute axis is the x axis of a joystick. As long as you keep the joystick pulled to the left, you get a certain value. On the other hand, a mouse is a good example for a relative axis. If you move our mouse to the left, the position difference is recognized once, and send to the system. This means that the relative axis contains time information\footnote{The position difference of the mouse since the last position test some time ago} while a absolute axis contains no time information\footnote{Ignoring the fact that the axis value can change over time, but that's not interesting in here}. It's important to keep the time information in mind when using the device controls in order to, for instance, moving around a character.
