\chapter{Android}
In principle, there's no difference in building PixelLight for Android on a \ac{MS} Windows or Linux host. So, the following description is e.g. not \ac{MS} Windows only.

Warning: We noticed several issues when using the \ac{NDK} under \ac{MS} Windows (tested with \emph{ndk r6b})
\begin{itemize}
\item{The linker eat characters resulting in errors like "CMakeFiles/PLCore.dr/src/File/FileSearchLinux.cpp not found" while the filename was "CMakeFiles/PLCore.dir/src/File/FileSearchLinux.cpp"}
\item{Performance issues, the build process is slow, really slow}
\end{itemize}

In case you never ever did native Android development before, you may have a look at the appendix~\ref{Appendix:AndroidNativeBuildTutorial}, first. Under Linux, in case you're using environment variables as described within the appendix, don't forget that you have to start CMake-GUI out of a terminal. If you don't, the CMake-GUI doesn't know anything about your environment variables.




\section{Prerequisites}
\begin{itemize}
\item{PixelLight requires at least Android 2.3 (Gingerbread, \ac{NDK} \ac{API} level 9), previous Android versions had no decent native support}
\item{In case you want to use OpenGL ES 2.0, you'll need a physical device because currently the Android Emulator has no support for OpenGL ES 2.0}
\end{itemize}

Install the usual Android development tools. The following versions are those used when writing this documentation, other versions may work as well:
\begin{itemize}
\item{Android \ac{SDK} Platform Android 2.3, \ac{API} 9\footnote{See \url{http://developer.android.com/guide/appendix/api-levels.html}}}
\item{Android \ac{NDK} (tested with \emph{ndk r6b} and \emph{ndk r7})}
\end{itemize}
Have a look at appendix~\ref{Appendix:AndroidNativeBuildTutorial} for details about setting up the Android development tools.


\paragraph{make (for Windows)}
\begin{itemize}
\item{\emph{Make for Windows}: Make: GNU make utility to maintain groups of programs}
\item{Directly used by the CMake scripts under \ac{MS} Windows when using the \ac{NDK} toolchain}
\item{\emph{cmake/UsedTools/make/make.exe} was downloaded from \url{http://gnuwin32.sourceforge.net/packages/make.htm}}
\end{itemize}
This tool can't be set within a CMake file automatically, there are several options:
\begin{itemize}
\item{Add \emph{\textless PixelLight root path\textgreater /cmake/UsedTools/make} to the \ac{MS} Windows \emph{PATH} environment variable *recommended*}
\item{Use a MinGW installer from e.g. \url{http://www.tdragon.net/recentgcc/} which can set the \emph{PATH} environment variable *overkill because only the 171 KiB \emph{make} is required*}
\item{Use CMake from inside a command prompt by typing for example (\emph{DCMAKE\_TOOLCHAIN\_FILE} is only required when using a toolchain) \\ *not really comfortable when working with it on a regular basis*
\begin{lstlisting}[language=sh]
cmake.exe -G"Unix Makefiles" -DCMAKE\_MAKE\_PROGRAM="<PixelLight root path>/cmake/UsedTools/make/make.exe" -DCMAKE\_TOOLCHAIN\_FILE="<PixelLight root path>/cmake/Toolchains/Toolchain-ndk.cmake"
\end{lstlisting}
}
\end{itemize}




\section{Create Makefiles and Build}
\label{Android:CreateMakefilesAndBuild}
Here's how to compile PixelLight by using the CMake-\ac{GUI}:
\begin{itemize}
\item{Ensure "make" (GNU make utility to maintain groups of programs) can be found by CMake (add for instance "\textless PixelLight root path\textgreater /cmake/UsedTools/make" to the \ac{MS} Windows \emph{PATH} environment variable)}
\item{Start "CMake (cmake-gui)"}
\item{"Where is the source code"-field: e.g. "C:/PixelLight"}
\item{"Where to build the binaries"-field: e.g. "C:/PixelLight/CMakeOutput"}
\item{In case you want to use the Android Emulator instead of a physical Android device: Click on "Add Entry", add a variable named \emph{ARM\_TARGET} of the type \emph{STRING} and assign the value \emph{armeabi} to it (default is \emph{armeabi-v7a} for a physical Android device)}
\item{Press the "Configure"-button}
\item{Choose the generator "Unix Makefiles" and select the radio box "Specify toolchain file for cross-compiling"}
\item{Press the "Next"-button}
\item{"Specify the Toolchain file": e.g. "C:/PixelLight/cmake/Toolchains/Toolchain-ndk.cmake"}
\item{Press the "Generate"-button}
\end{itemize}

The CMake part is done, you can close "CMake (cmake-gui)" now. All required external packages are downloaded automatically, see chapter~\ref{Chapter:ExternalDependencies}.
\begin{itemize}
\item{Open a command prompt and change into e.g. "C:/PixelLight/CMakeOutput" (\ac{MS} Windows: by typing "cd /D C:/PixelLight/CMakeOutput" -> "/D" is only required when changing into another partition)}
\item{Type "make" (example: "make -j 4 -k" will use four \ac{CPU} cores and will keep on going when there are errors)}
\item{(You now should have the ready to be used Android shared library files)}
\end{itemize}




\section{Using the Android Port}
The content within this section was tested using "Ubuntu 11.10 - Oneiric Ocelot". Just mentioned the used version to be on the safe side, other, newer versions may probably work as well. Usually, the build PixelLight version for Android can't be used in the same way as a version build for the host system like Linux. That's the reason this section, showing how to get a PixelLight sample onto your Android device, exists.


\paragraph{\ac{JDK} and \ac{JRE}}
The following is required for building Android \ac{APK}-files.
\begin{itemize}
\item{Install \ac{JDK} ("\textbf{sudo apt-get install default-jdk}") and \ac{JRE} ("\textbf{sudo apt-get install default-jre}")}
\item{Install "ant" ("\textbf{sudo apt-get install ant}"), required to create the \ac{APK} files}
\end{itemize}
To install all packages at once, just use: 
\begin{lstlisting}[language=sh]
apt-get install default-jdk default-jre ant
\end{lstlisting}


\paragraph{Files Overview}
Content of the \emph{project}-directory (name it e.g. \emph{PixelLightAndroid}):
\begin{itemize}
\item{FindPixelLight.cmake}
\item{CMakeLists.txt}
\item{\emph{src}-directory}
\end{itemize}
Content of the \emph{src}-directory:
\begin{itemize}
\item{AndroidMain.cpp}
\item{CMakeLists.txt}
\end{itemize}


\paragraph{FindPixelLight.cmake}
Within the \emph{<PixelLight root path>/Tools} directory is a CMake file named \emph{FindPixelLight.cmake}. This section is using this script file and assumes that it just has been copied into your project directory. It's kind of a link to PixelLight. Have a look at the appendix~\ref{Appendix:FindPixelLight} for more information about this script.


\paragraph{CMakeLists.txt within Project Directory}
Here's an example for a simple \emph{CMakeLists.txt} file within your project directory:
\begin{lstlisting}[language=sh]
cmake_minimum_required(VERSION 2.8.3)

# Add the directory this "CMakeLists.txt" file is in as CMake module path in order to make it possible to find "FindPixelLight.cmake"
set(CMAKE_MODULE_PATH ${CMAKE_MODULE_PATH} ${CMAKE_CURRENT_LIST_DIR})

# Define the project, use the current directory name as project name
get_filename_component(CURRENT_TARGET_NAME ${CMAKE_CURRENT_LIST_DIR} NAME_WE)
project(${CURRENT_TARGET_NAME})

# Add the source code directory
add_subdirectory(src)
\end{lstlisting}


\paragraph{AndroidMain.cpp}
Here's an example for a simple \emph{AndroidMain.cpp} file within your project source code directory:
\begin{lstlisting}[]
#include <PLFrontendOS/FrontendAndroid.h>

/**
*  @brief
*    Program entry point
*
*  @remarks
*    This is the main entry point of a native application that is using
*    android_native_app_glue. It runs in its own thread, with its own
*    event loop for receiving input events and doing other things.
*/
void android_main(struct android_app* state)
{
	// Make sure glue isn't stripped
	app_dummy();

	// Create a frontend instance on the C runtime stack
	PLFrontendOS::FrontendAndroid cFrontendAndroid(*state);

	// Finish the given activity
	ANativeActivity_finish(state->activity);
}
\end{lstlisting}


\paragraph{CMakeLists.txt within Project Source Code Directory}
Here's an example for a simple \emph{CMakeLists.txt} file within your project source code directory:
\begin{lstlisting}[language=sh]
cmake_minimum_required(VERSION 2.8.3)

# Find packages
find_host_package(PixelLight)	# "find_package"-variant provided by "android.toolchain.cmake", required for PixelLight CMake variables like "PL_PLCORE_INCLUDE_DIR"

# Includes
include_directories(${CMAKE_CURRENT_SOURCE_DIR})					# The current source directory
include_directories(${ANDROID_NDK}/sources/android/native_app_glue)	# For the "android_native_app_glue.h"-header include
include_directories(${PL_PLCORE_INCLUDE_DIR})						# PLCore headers
include_directories(${PL_PLFRONTENDOS_INCLUDE_DIR})					# PLFrontendOS headers

# Source codes
set(CURRENT_SRC
	${ANDROID_NDK}/sources/android/native_app_glue/android_native_app_glue.c
	AndroidMain.cpp
)

# Shared libraries
set(CURRENT_SHARED_LIBRARIES
	# Base
	${PL_PLCORE_LIBRARY}
	${PL_PLMATH_LIBRARY}
	${PL_PLGRAPHICS_LIBRARY}
	${PL_PLINPUT_LIBRARY}
	${PL_PLRENDERER_LIBRARY}
	${PL_PLMESH_LIBRARY}
	${PL_PLSCENE_LIBRARY}
	${PL_PLPHYSICS_LIBRARY}
	${PL_PLENGINE_LIBRARY}
	# Plugins
	${PL_PLFRONTENDOS_LIBRARY}			# The PixelLight Android frontend
	${PL_PLRENDEREROPENGLES2_LIBRARY}
	${PL_PLCOMPOSITING_LIBRARY}
	# Application
	"${PL_SAMPLES_BIN_DIR}/lib60Scene.so"
)

# Assets
set(CURRENT_ASSETS
	"${PL_RUNTIME_DATA_DIR}/Standard.zip"
	"${PL_SAMPLES_DATA_DIR}/*.*"		# Copy all sample data
)

# Build
set(CMAKE_CXX_FLAGS "${CMAKE_CXX_FLAGS} -std=c++0x -ffor-scope -fno-rtti -fno-exceptions -pipe -ffunction-sections -fdata-sections -ffast-math -Wnon-virtual-dtor -Wreorder -Wsign-promo -fvisibility=hidden -fvisibility-inlines-hidden -Wstrict-null-sentinel -Os -funroll-all-loops -fpeel-loops -ftree-vectorize")
set(LINKER_FLAGS "${LINKER_FLAGS} -Wl,--as-needed -Wl,--gc-sections -Wl,--no-undefined -Wl,--strip-all -Wl,-rpath-link=${ANDROID_NDK_SYSROOT}/usr/lib/ -L${ANDROID_NDK_SYSROOT}/usr/lib/")
add_library(${CURRENT_TARGET_NAME} SHARED ${CURRENT_SRC})
target_link_libraries(${CURRENT_TARGET_NAME} log android ${CURRENT_SHARED_LIBRARIES})
set_target_properties(${CURRENT_TARGET_NAME} PROPERTIES COMPILE_DEFINITIONS "__STDC_INT64__;LINUX;ANDROID")	# PLCore needs the preprocessor definitions "LINUX" and "ANDROID"

# Post build

# Copy the build shared library as well
set(CURRENT_SHARED_LIBRARIES
	${CURRENT_SHARED_LIBRARIES}
	"${LIBRARY_OUTPUT_PATH}/lib${CURRENT_TARGET_NAME}.so"
)

# Create Android apk file (macro from "<PixelLight>/cmake/Android/Apk.cmake")
android_create_apk("${CURRENT_TARGET_NAME}" "${CMAKE_BINARY_DIR}/apk" "${CURRENT_SHARED_LIBRARIES}" "${CURRENT_ASSETS}" "Data")
\end{lstlisting}


\paragraph{Build}
The build process for this project is the same as described within section~\ref{Android:CreateMakefilesAndBuild}, except that the standalone toolchain \emph{<PixelLight root path>/cmake/Android/android.toolchain.cmake} instead of \emph{<PixelLight root path>/cmake/Toolchains/Toolchain-ndk.cmake} is used. By using the provided macro \emph{android\_create\_apk()}, the project is automatically packed into an \ac{APK}-file, installed on our device and started at once.

At first, this may look a lot - but you usually need to do write this only once per project. During the development you just have to type \emph{make} in order to let your project run on the Android device, without doing everything manually as described within appendix~\ref{Appendix:AndroidNativeBuildTutorial} over and over again.

