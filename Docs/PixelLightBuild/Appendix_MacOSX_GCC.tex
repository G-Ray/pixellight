\chapter{Mac OS X GCC}
\label{Appendix:MacOSX_GCC}
Within a terminal, type
\begin{lstlisting}[language=sh]
gcc -v
\end{lstlisting}
in order to see the currently used \ac{GCC} version. In case you see e.g.
\begin{lstlisting}[language=sh]
gcc version 4.2.1 (Based on Apple Inc. build 5658) (LLVM build 2335.15.00)
\end{lstlisting}
you're using an out-of-date \ac{GCC} version not capably of any C++11 features.

One way to get a new \ac{GCC} version like \ac{GCC} 4.6 is to just compile it yourself. Please note that explaining how to compile \ac{GCC} for Mac OS X is out of the scope of this document. There are many good instructions like \url{http://solarianprogrammer.com/2011/05/28/compiling-gcc-4-6-0-on-mac-osx/} explaining in detail how to compile \ac{GCC} for Mac OS X. The process isn't that complicated, but time consuming. Do not forget to include Objective-C++ support when compiling your \ac{GCC}. When compiling a new \ac{GCC}, the \ac{GCC} of the system will not be touched. In order to use the build \ac{GCC} 4.6 type
\begin{lstlisting}[language=sh]
export CXX=$HOME/my_gcc/bin/g++-4.6.1
export CC=$HOME/my_gcc/bin/gcc-4.6.1
export CPP=$HOME/my_gcc/bin/cpp-4.6.1
\end{lstlisting}
in case you put your build \ac{GCC} into the \emph{\$HOME/my\_gcc/} directory.

In case you don't want to do this over and over again, put this overwritten environment variables into your profile file. To put this into the profile file, type
\begin{lstlisting}[language=sh]
pico $HOME/.profile
\end{lstlisting}
and copy'n'past the export lines above.

When typing
\begin{lstlisting}[language=sh]
gcc -v
\end{lstlisting}
you will still get \emph{4.2.1} as \ac{GCC} because you just set the \ac{GCC} environment variables CMake is using in order to figure out which compiler to use. Type
\begin{lstlisting}[language=sh]
$CXX -v
\end{lstlisting}
and the output will be
\begin{lstlisting}[language=sh]
gcc-Version 4.6.1 (GCC)
\end{lstlisting}
, you're now ready to continue.
