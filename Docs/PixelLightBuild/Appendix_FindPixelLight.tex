\chapter{FindPixelLight.cmake}
\label{Appendix:FindPixelLight}
Within the \emph{<PixelLight root path>/Tools} directory is a CMake file named \emph{FindPixelLight.cmake}. You can use this file in your CMake build scripts in order to find and use PixelLight. You only have to tell this PixelLight find script where it can find the PixelLight runtime.


\paragraph{Windows Registry Key}
You have to add a key to the registry, so that the path to the build PixelLight runtime can be found. This key has to be at "HKEY\_LOCAL\_MACHINE/SOFTWARE/PixelLight/PixelLight-SDK/Runtime" (or at "HKEY\_LOCAL\_MACHINE/SOFTWARE/Wow6432Node/PixelLight/PixelLight-SDK/Runtime" if you are using a \SI{32}{\bit} PixelLight SDK on a \SI{64}{\bit} MS Windows). This "Runtime"-key has e.g. the string value "C:/PixelLight/Bin/Runtime/x86/" (same as the PATH environment variable entry).


\paragraph{Linux}
On Linux, this can be done by setting the environment variable \emph{PL\_RUNTIME} to the \emph{Runtime} directory.

If you're in the root of the source tree, you can use the script \emph{profile} to do this for you:
\begin{lstlisting}[language=sh]
source ./profile
\end{lstlisting}
or
\begin{lstlisting}[language=sh]
. ./profile
\end{lstlisting}
(sets the environment variable \emph{PL\_RUNTIME})

To inspect the content of \emph{PL\_RUNTIME}, call
\begin{lstlisting}[language=sh]
echo $PL_RUNTIME
\end{lstlisting}

To delete the environment variable \emph{PL\_RUNTIME}, call
\begin{lstlisting}[language=sh]
unset PL_RUNTIME
\end{lstlisting}

Of course you can also set this variable e.g. inside your profile or bash-scripts so that they are always available. Within your home (\emph{\textasciitilde}) directory, open the hidden \emph{\textasciitilde /.bashrc} file and add:
\begin{lstlisting}[language=sh]
export PL_RUNTIME="<PixelLight root path>/Bin-Linux-ndk/Runtime/armeabi-v7a"
\end{lstlisting}
This is the most comfortable way, because you only have to do this once, and then you can use the PixelLight runtime even across multiple terminal instances.

