\chapter{Android Native Build Tutorial}
\label{Appendix:AndroidNativeBuildTutorial}
During the implementation of the Android port, a lot of notices were written down. Although there's couple of information about how to do native Android development available, it's often out-of-date, confusing or just not working. A lot of trivial looking information had also be figured out in many frustrating hours of work. So, because there were already a lot of notices written down and the possibility is high that someone using PixelLight for native Android development is requiring the same or quite similar information, those notices were put into this appendix in a revised form.

The good news is, when you build PixelLight for Android, you don't really need to know everything written down in here. The process is heavily automated by using CMake scripts. In case you just find this appendix unnecessary or you hate spoilers and want to figure out all by yourself, ignore this appendix. For all others, this information in here is hopefully useful for you for the first steps in native Android development without the automated CMake build system PixelLight is using.

Please note that this tutorial is using Linux because using \ac{MS} Windows for native Android development is really painful.


\paragraph{Some Hints on how to Read this Tutorial}
The tutorial is written in brief sentences to keep it compact. But this doesn't mean there's no additional information, so, the most important information is marked by using \textbf{bold text}. Additional, nice to know information is marked by using \textrightarrow. If you want to use the fast path, just focus on the \textbf{bold texts} like terminal commands.




\section{Prerequisites}
\begin{itemize}
\item{Used \ac{OS}: "Ubuntu 11.10 - Oneiric Ocelot" (just mentioned the used version to be on the safe side, other, newer versions may probably work as well)}
\item{Install \ac{JDK} ("\textbf{sudo apt-get install default-jdk}") and \ac{JRE} ("\textbf{sudo apt-get install default-jre}")}
\item{Install "ant" ("\textbf{sudo apt-get install ant}"), required to create the \ac{APK} files}
\end{itemize}
To install all packages at once, just use: 
\begin{lstlisting}[language=sh]
apt-get install default-jdk default-jre ant
\end{lstlisting}


\paragraph{Install Android \ac{NDK}}
\begin{itemize}
\item{\textbf{Download} from \url{http://developer.android.com/sdk/ndk/index.html} \textrightarrow{} "android-ndk-r6b-linux-x86.tar.bz2"}
\item{\textbf{Extract} to for example "\textasciitilde /android-ndk-r6b" ("\textasciitilde /" is your home directory)}
\end{itemize}


\paragraph{Android \ac{NDK} (\emph{ndk r6b}) - \ac{MS} Windows}
\begin{itemize}
\item{Extract it and set the \ac{MS} Windows PATH environment variable \emph{ANDROID\_NDK} to the \ac{NDK} root directory}
\item{Set the \ac{MS} Windows \emph{PATH} environment variable \emph{ANDROID\_NDK\_TOOLCHAIN\_ROOT} to the \ac{NDK} toolchain root directory (e.g. "C:/android-ndk-r6b/toolchains/arm-linux-androideabi-4.4.3/prebuilt/windows/arm-linux-androideabi")}
\item{(Those variables can also be added/set within the CMake-\ac{GUI})}
\end{itemize}


\paragraph{Install Android \ac{SDK}}
\begin{itemize}
\item{\textbf{Download} from \url{http://developer.android.com/sdk/index.html} \textrightarrow{} "android-sdk\_r12-linux\_x86.tgz"}
\item{\textbf{Extract} to for example "\textasciitilde /android-sdk-linux\_x86" ("\textasciitilde /" is your home directory)}
\item{Tested with Android \ac{SDK} Tools, revision 12}
\item{Tested with Android \ac{SDK} Platform-tools, revision 6}
\item{Tested with \ac{SDK} Platform Android 2.3, \ac{API} 9\footnote{See \url{http://developer.android.com/guide/appendix/api-levels.html}}}
\end{itemize}


\paragraph{Optional but Highly Recommended for a Decent Workflow}
This example assumes that the data has been extracted directly within the home (\emph{\textasciitilde}) directory. Open hidden "\textasciitilde /.bashrc"-file and add:
\begin{lstlisting}[language=sh]
# Important Android SDK and NDK paths
export ANDROID_SDK=~/android-sdk-linux_x86
export ANDROID_NDK=~/android-ndk-r6b
export PATH=${PATH}:${ANDROID_SDK}/tools:${ANDROID_SDK}/platform-tools:~/${ANDROID_NDK}
\end{lstlisting}
\begin{itemize}
\item{Open a new terminal so the changes from the step above have an effect}
\end{itemize}


\paragraph{Android \ac{SDK} and AVD Manager}
Type "\textbf{android}" to open the "Android SDK and AVD Manager"-\ac{GUI}:
\begin{itemize}
\item{"Available packages" \textrightarrow{} disable the "Display updates only"-checkbox (you may need to enlarge the window to see this checkbox) \textrightarrow{} install at least the following:}
\item{"Android SDK Tools, revision 12"}
\item{"Android SDK Platform-tools, revision 6"}
\item{"SDK Platform Android 2.3.1, \ac{API} 9, revision 2" (it's marked "Obsolete", but within the \ac{NDK} r6b there's only up to \ac{API} level 9 available and we don't want to mix)}
\end{itemize}




\section{Android Emulator and Device}
\begin{itemize}
\item{Android emulator\footnote{General information: \url{http://developer.android.com/guide/developing/devices/emulator.html}} \textrightarrow{} Type "\textbf{android}" to open the "Android SDK and AVD Manager"-\ac{GUI} and do the emulator configuration in here}
\item{Android device\footnote{General information: \url{http://developer.android.com/guide/developing/device.html}} \textrightarrow{} Configuration: Just connect your device to your computer and it should work at once... if you have enabled "USB-Debugging" on your device (launch "Settings", tap "Applications", tap "Development" and enable the checkbox for USB debugging)}
\end{itemize}


\paragraph{Not Required but Nice to Know}
Check available devices: Type "\textbf{adb devices}" \textrightarrow{} You should see at least one entry when your device is connected. The result may look like the following:
\begin{lstlisting}[language=sh]
"List of devices attached
028842074300d157	device
emulator-5554	device"
\end{lstlisting}

\begin{itemize}
\item{The output for each instance is formatted like this: \textrightarrow{} "[serialNumber] [state]"}
\item{In this case "028842074300d157	device" is my connected smartphone}
\item{In this case "emulator-5554	device" is the emulator I created and started}
\end{itemize}

In case you don't see your device, try:
\begin{lstlisting}[language=sh]
adb kill-server
adb start-server
adb devices 
\end{lstlisting}




\section{\ac{NDK} Build System}
The \ac{NDK} build system is not used by PixelLight, but for the first steps it's nice to know how to use it.

Time for a first experiment. Sadly, as on September 2011, it appears that some of the information at \url{http://developer.android.com/sdk/ndk/overview.html} is out-of-date and something like "I just try out native-activity to get the idea" isn't working as "just" as thought. So, here's an updated version with some additional handy information for \ac{MS} Windows users like myself: (yes, there are people out there don't knowing that "\textasciitilde /" is your home directory, so I just mention such stuff)


\paragraph{build.xml}
Run the following command to generate a \emph{build.xml} file:
\begin{itemize}
\item{\textbf{Change} to the "\textasciitilde /android-ndk-r6b/samples/native-activity" \textbf{directory}}
\item{\textrightarrow{} Don't follow the official sample instructions: "android update project -p . -s" \textrightarrow{} "Error: The project either has no target set or the target is invalid. Please provide a \verb+--target+ to the 'android update' command."}
\item{\textrightarrow{} Type \verb+"android --help"+ to see available options}
\item{\textrightarrow{} Type "android list targets" so see available targets (= \ac{API} levels)}
\item{Type "\textbf{android update project -t android-9 -p .}" ("." means "the current directory")}
\item{\textrightarrow{} You now have a "build.xml"-file in the same directory as the "AndroidManifest.xml"-file, this is required for the \ac{APK} creation step}
\end{itemize}


\paragraph{Compile}
Compile the native code using the \emph{ndk-build} command.
\begin{itemize}
\item{"\textbf{ndk-build}"}
\item{\textrightarrow{} You now have "\textasciitilde /android-ndk-r6b/samples/native-activity/libs/armeabi/libnative-activity.so"}
\end{itemize}


\paragraph{\ac{APK}}
Create \ac{APK} file
\begin{itemize}
\item{Type "\textbf{ant debug}" ("debug" is only for developing and testing using easy automatically signing, see http://developer.android.com/guide/publishing/app-signing.html)}
\item{\textrightarrow{} You now have "\textasciitilde /android-ndk-r6b/samples/native-activity/bin" with some files in it}
\item{Start the emulator, or connect your device (ensure that it has Android 2.3 > on it, else the app will just crash)}
\item{Type "\textbf{adb install -r bin/NativeActivity-debug.apk}" ("-r"-option to avoid "Failure [INSTALL\_FAILED\_ALREADY\_EXISTS]"-error when the \ac{APK} is already installed, default destination is "/data/local/tmp/NativeActivity-debug.apk")}
\item{\textrightarrow{} The app is now available and ready to be started within your emulator or on your device}
\item{\textrightarrow{} To uninstall the app, type "\textbf{adb uninstall com.example.native\_activity}"}
\end{itemize}


\paragraph{Release \ac{APK}}
Create release \ac{APK} file\footnote{See \url{http://developer.android.com/guide/publishing/app-signing.html for detailed information}}
\begin{itemize}
\item{Type "\textbf{ant release}" (you now have "bin/NativeActivity-unsigned.apk")}
\item{\textrightarrow{} Don't try "adb install -r bin/NativeActivity-unsigned.apk", this will just result in "Failure [INSTALL\_PARSE\_FAILED\_NO\_CERTIFICATES]"}
\item{\textrightarrow{} The \ac{JDK} tools Jarsigner and Keytool will be used (ensure they're available, if you installed \ac{JRE} they are usually available)}
\item{\textrightarrow{} You need a private key to sign your \ac{APK} file with, if you don't have any: Type e.g. "keytool -genkey -v -keystore my-release-key.keystore -alias myalias -keyalg RSA -keysize 2048 -validity 10000" (you now have a "my-release-key.keystore"-file)}
\item{For signing the \ac{APK} file, type: "\textbf{jarsigner -verbose -keystore my-release-key.keystore bin/NativeActivity-unsigned.apk myalias}" (no new file, it's still "bin/NativeActivity-unsigned.apk" and you may rename it later, but for now we don't touch the name)}
\item{\textrightarrow{} Type "jarsigner -verify bin/NativeActivity-unsigned.apk" to verify that everything went fine and "jarsigner -verify -verbose -certs bin/NativeActivity-unsigned.apk" to get additional information}
\item{Finally, align your \ac{APK} file by typing "\textbf{zipalign -v 4 bin/NativeActivity-unsigned.apk bin/NativeActivity.apk}" (you now have the ready to be released file "bin/NativeActivity.apk")}
\item{To install this file right now, type "\textbf{adb install -r bin/NativeActivity.apk}"}
\item{\textrightarrow{} If you receive a "Failure [INSTALL\_PARSE\_FAILED\_INCONSISTENT\_CERTIFICATES]" you need to remove the previously installed application by typing "adb uninstall com.example.native\_activity"}
\item{\textrightarrow{} To uninstall the app, type "\textbf{adb uninstall com.example.native\_activity}"}
\end{itemize}




\section{CMake Build System}
We want to use the universal CMake build system, not the special \ac{NDK} build system.

\begin{itemize}
\item{\textbf{Download and extract} CMake toolchain "android-cmake" from \url{http://code.google.com/p/android-cmake/} (we're using it's "android.toolchain.cmake"-file) and extract it to e.g. "\textasciitilde /android-cmake"}
\end{itemize}


\paragraph{Optional but Highly recommended for a Decent Workflow}
Open hidden "\textasciitilde /.bashrc"-file and add:
\begin{lstlisting}[language=sh]
# CMake toolchain "android-cmake" from \url{http://code.google.com/p/android-cmake/}
export ANDROID_CMAKE=~/android-cmake
export ANDTOOLCHAIN=$ANDROID_CMAKE/toolchain/android.toolchain.cmake
alias android-cmake='cmake -DCMAKE_TOOLCHAIN_FILE=$ANDTOOLCHAIN '
\end{lstlisting}
\begin{itemize}
\item{Open a new terminal so the changes from the step above have an effect}
\end{itemize}



\subsection{First Experiment}
Time for a first experiment by using "hello-gl2" of \emph{android-cmake}.

\begin{itemize}
\item{\textbf{Change} into the "hello-gl2"-\textbf{directory} of \emph{android-cmake}}
\end{itemize}


\paragraph{Build}
Type:
\begin{itemize}
\item{"\textbf{mkdir build}"}
\item{"\textbf{cd build}"}
\item{"\textbf{android-cmake -DARM\_TARGET=armeabi ..}"}
\item{\textrightarrow{} The Android \ac{SDK} emulator supports only "armeabi", but "android-cmake" has "armeabi-v7a" as default. If you don't change this, the application will just crash when you try to start it within the emulator. For more complex 3D applications, "armeabi-v7a" is highly recommended due to hardware floating point support. So, for more advanced stuff you really need a real device instead of the emulator.}
\item{"\textbf{make}"}
\item{\textrightarrow{} You now have "\textasciitilde hello-gl2/libs/armeabi/libgl2jni.so"}
\end{itemize}


\paragraph{Keystore}
\begin{itemize}
\item{\textbf{Copy "my-release-key.keystore"} from your \ac{NDK} build system experiment into the "hello-gl2"-directory, or keep it within e.g. your home directory and update the entries below}
\end{itemize}


\paragraph{\ac{APK}}
\textbf{Back} to the "hello-gl2"-directory and type:
\begin{itemize}
\item{"\textbf{sh project\_create.sh}" (you may need to open this file first and replace \verb+"android update project --name HelloGL2 --path ."+ through \verb+"android update project -t android-8 --name HelloGL2 --path ."+)}
\item{"\textbf{ant release}" (create the \ac{APK} file)}
\item{"\textbf{jarsigner -verbose -keystore my-release-key.keystore bin/HelloGL2-unsigned.apk myalias}" (sign the \ac{APK} file)}
\item{"\textbf{zipalign -v 4 bin/HelloGL2-unsigned.apk bin/HelloGL2.apk}" (align the \ac{APK} file)}
\item{"\textbf{adb uninstall com.android.gl2jni}" (in order to ensure that we don't get problems when doing the following install step)}
\item{"\textbf{adb install -r bin/HelloGL2.apk}" (install the \ac{APK} file on the device)}
\item{"\textbf{adb shell am start -n com.android.gl2jni/.GL2JNIActivity}" (start the installed \ac{APK} file automatically)}
\end{itemize}



\subsection{Native Activity Experiment}
Another experiment, "native-activity" for a native activity of the \ac{NDK}.

\begin{itemize}
\item{\textbf{Change} into the "native-activity"-\textbf{directory}} of the \ac{NDK}
\end{itemize}


\paragraph{CMakeLists.txt within the "native-activity"-directory}
Within the "native-activity"-directory, create a text file named \emph{CMakeLists.txt} with the following content:
\begin{lstlisting}[language=sh]
cmake_minimum_required(VERSION 2.8)
project(native-activity)
add_subdirectory(jni)
\end{lstlisting}


\paragraph{CMakeLists.txt within the "native-activity/jni"-directory}
Within the "native-activity/jni"-directory, create a text file named \emph{CMakeLists.txt} with the following content:
\begin{lstlisting}[language=sh]
set(CMAKE_C_FLAGS "${CMAKE_C_FLAGS} -fPIC")
include_directories("${ANDROID_NDK}/sources/android/native_app_glue")
include_directories(${CMAKE_CURRENT_SOURCE_DIR})
set(LIBRARY_DEPS log android EGL GLESv1_CM)
set(MY_SRCS
    ${ANDROID_NDK}/sources/android/native_app_glue/android_native_app_glue.c
    main.c
    )
add_library(native-activity SHARED ${MY_SRCS})
target_link_libraries(native-activity ${LIBRARY_DEPS})
\end{lstlisting}


\paragraph{Build}
We're going to use the native activity which was introduced in Android 2.3 (Gingerbread, "android-9" \ac{NDK} \ac{API} level). Within the "native-activity"-directory, type:
\begin{itemize}
\item{"\textbf{mkdir build}"}
\item{"\textbf{cd build}"}
\item{"\textbf{android-cmake -DARM\_TARGET=armeabi -DANDROID\_API\_LEVEL=9 ..}"}
\item{"\textbf{make}"}
\end{itemize}


\paragraph{Nice to Know}
In case you want to write "\textbf{android-cmake -DARM\_TARGET=armeabi ..}" instead of "\textbf{android-cmake -DARM\_TARGET=armeabi -DANDROID\_API\_LEVEL=9 ..}" \textrightarrow{} Open hidden "\textasciitilde /.bashrc"-file and add:
\begin{lstlisting}[language=sh]
export ANDROID_API_LEVEL=9
\end{lstlisting}


\paragraph{Keystore}
\begin{itemize}
\item{\textbf{Copy} "my-release-key.keystore" from your \ac{NDK} build system experiment into the "native-activity"-directory, or keep it within e.g. your home directory and update the entries below}
\end{itemize}


\paragraph{\ac{APK}}
Back to the "native-activity"-directory and type:
\begin{itemize}
\item{\verb+"android update project -t android-9 --name native-activity --path ."+}
\item{"\textbf{ant release}"}
\item{"\textbf{jarsigner -verbose -keystore my-release-key.keystore bin/native-activity-unsigned.apk myalias}"}
\item{"\textbf{zipalign -v 4 bin/native-activity-unsigned.apk bin/native-activity.apk}"}
\item{"\textbf{adb uninstall com.example.native\_activity}"}
\item{"\textbf{adb install -r bin/native-activity.apk}"}
\end{itemize}

All in one single command line row:
\begin{lstlisting}[language=sh]
mkdir build;cd build;android-cmake -DARM_TARGET=armeabi ..;make;cd ..;android update project -t android-9 --name native-activity --path .;ant release;jarsigner -verbose -keystore my-release-key.keystore bin/native-activity-unsigned.apk myalias;rm bin/native-activity.apk;zipalign -v 4 bin/native-activity-unsigned.apk bin/native-activity.apk;adb uninstall com.example.native_activity;adb install -r bin/native-activity.apk
\end{lstlisting}




\section{Glossary}
Glossary (only terms I wasn't really familiar with)
\begin{center}
	\centering
	\begin{tabular}{ | l | l | p{8cm} |}
	\hline
	Short	& Long & Information\\ \hline
	ndk		& Native Development Kit		& \url{http://developer.android.com/sdk/ndk/index.html}\\ \hline
	adb		& Android Debug Bridge			& \url{http://developer.android.com/guide/developing/tools/adb.html}	!important!\\ \hline
	avd		& Android Virtual Device 		& \url{http://developer.android.com/guide/developing/devices/emulator.html}\\ \hline
	aapt	& Android Asset Packaging Tool	& \\ \hline
	JNI		& Java Native Interface			& \\ \hline
	logcat	& 								& Android system central log buffer\\ \hline
	\end{tabular}
\end{center}




\section{Command Glossary}
Command glossary (only terms I wasn't really familiar with)
\begin{center}
	\centering
	\begin{tabular}{ | l | p{10cm} |}
	\hline
	Command							& Result\\ \hline
	android							& Start "Android SDK and AVD Manager" (e.g. to start the emulator)\\ \hline
	adb devices						& See all available emulators/devices\\ \hline
	adb logcat						& Show Android log (called "logcat"), this is realtime, so the command prompt will block and show new upcoming log entries at once, more information: \url{http://developer.android.com/guide/developing/tools/adb.html#logcat}\\ \hline
	adb logcat -s <tag>				& Show only messages with the given tag within the Android log, example: "adb logcat -s PixelLight"\\ \hline
	adb push <local> <remote>		& Copy <local> (file or directory recursively) to the emulator/device to destination <remote>, example: "adb push foo.txt /sdcard/foo.txt"\\ \hline
	adb pull <remote> <local>		& Copy <remove> (file or directory recursively) from the emulator/device to destination <local>, example: "adb pull /sdcard/foo.txt foo.txt"\\ \hline
	adb install <\ac{APK} file>		& Install \ac{APK} file on emulator/device (but doesn't start it automatically)\\ \hline
	adb -d install <\ac{APK} file>	& Install \ac{APK} file on device (but doesn't start it automatically)\\ \hline
	adb -e install <\ac{APK} file>	& Install \ac{APK} file on emulator (but doesn't start it automatically)\\ \hline
	adb uninstall <package>			& Uninstall an \ac{APK}, example: "adb uninstall com.example.native\_activity"\\ \hline
	adb shell am start <app>		& Start an app, example: "adb shell am start -n com.android.gl2jni/.GL2JNIActivity"\\ \hline
	\end{tabular}
\end{center}

\ac{NDK} build system related:
\begin{center}
	\centering
	\begin{tabular}{ | l | p{8cm} |}
	\hline
	Command										& Result\\ \hline
	android update project -t android-9 -p .	& Create/update "build.xml" (required for Ant)\\ \hline
	ndk-build									& Compile native code\\ \hline
	ant debug									& Create debug \ac{APK}\\ \hline
	ant release									& Create release \ac{APK}\\ \hline
	\end{tabular}
\end{center}




\section{Possible issues}


\paragraph{"Android can't load in my shared library"}
\begin{itemize}
\item{Is the build target correct?}
\item{The Android Emulator is only able to deal with "armeabi", not e.g. "armeabi-v7a"}
\item{\textrightarrow{} When using CMake-\ac{GUI}, add the string entry "ARM\_TARGET"="armeabi"}
\end{itemize}


\paragraph{How to Start a Native Activity Automatically?}
I wasn't able to figure this one out. When having a Java file as main program entry point, can can e.g. just type "adb shell am start -n com.android.gl2jni.GL2JNIActivity" and the application starts automatically. Well, a pure native activity without a single Java source file... I have no glue...
