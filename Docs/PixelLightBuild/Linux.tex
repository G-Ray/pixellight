\chapter{Linux}
To build PixelLight on Linux, use the CMake based cross-platform build system of PixelLight.




\section{Prerequisites}
You need to install\footnote{This has to be done as root. Use e.g. \emph{sudo} or \emph{su} for this task.} certain dependencies on your system that are needed to build PixelLight, including build tools and used development libraries.

The following package list applies to the Linux distribution "Ubuntu 11.10 - Oneiric Ocelot". Other required packages will be downloaded automatically by this listed packages. If you use another distribution, please have a look at their package repository to find out which packages you need. In most cases, the package names will not be that different, but we can't cover all Linux distributations, so we focus just on one.


\paragraph{One Line}
In case you don't want to read the following paragraphs, just copy'n'paste the following command line:
\begin{lstlisting}[language=sh]
apt-get install git cmake cmake-qt-gui build-essential libncurses5-dev libxext-dev libxcursor-dev libdbus-1-dev libxxf86vm-dev libxrandr-dev libglu1-mesa-dev doxygen graphviz texlive texlive-science
\end{lstlisting}


\paragraph{Tools}
You might want to install the following tools:
\begin{itemize}
\item{\emph{git} (version control)}
\item{\emph{cmake} (build tool)}
\item{\emph{cmake-qt-gui} (comfortable GUI for CMake, start application via command line by typing \emph{cmake-gui})}
\end{itemize}

To install all packages at once, just use: 
\begin{lstlisting}[language=sh]
apt-get install git cmake cmake-qt-gui
\end{lstlisting}


\paragraph{Compile Binaries}
The following packages are required in order to be able to compile the binaries:
\begin{itemize}
\item{\emph{build-essential} (GNU C++ compiler and other important build tools)}
\item{\emph{libncurses5-dev} (required by PLCore inside the \emph{PLCore::ConsoleLinux}-class)}
\item{\emph{libxext-dev} (required by e.g. PLFrontendOS or PLGui)}
\item{\emph{libxcursor-dev} (required by e.g. PLFrontendOS or PLGui)}
\item{\emph{libdbus-1-dev} (required by PLInput)}
\item{\emph{libxxf86vm-dev} (required by PLRendererOpenGL)}
\item{\emph{libxrandr-dev} (required by PLRendererOpenGL)}
\item{\emph{libglu1-mesa-dev} (required by PLRendererOpenGL)}
\end{itemize}

To install all required packages at once, just use:
\begin{lstlisting}[language=sh]
apt-get install build-essential libncurses5-dev libxext-dev libxcursor-dev libdbus-1-dev libxxf86vm-dev libxrandr-dev libglu1-mesa-dev
\end{lstlisting}


\paragraph{"Could NOT find Qt4 (missing: QT\_QMAKE\_EXECUTABLE")}
In case you get the error message \begin{quote}Could NOT find Qt4 (missing: QT\_QMAKE\_EXECUTABLE)\end{quote}, ensure that the files within the \emph{External/Optional/Qt/Linux\_x86/} directory are executable (file permissions).


\paragraph{"fatal error: dbus/dbus-arch-deps.h: No such file or directory"}
In case you get the error message \begin{quote}fatal error: dbus/dbus-arch-deps.h: No such file or directory\end{quote} and you're using a x86 version of Ubuntu, use the following command in order to copy the missing \emph{dbus-arch-deps.h} file into the correct directory: (This has to be done as root. Use e.g. \emph{sudo} or \emph{su} for this task.)
\begin{lstlisting}[language=sh]
cp /usr/lib/i386-linux-gnu/dbus-1.0/include/dbus/dbus-arch-deps.h /usr/include/dbus-1.0/dbus
\end{lstlisting}


\paragraph{Compile Documentation}
To compile the documentation, the following packages are required:
\begin{itemize}
\item{\emph{doxygen} (required to compile the automatic code documentation, warning: Also downloads \LaTeX{} packages which may take a while)}
\item{\emph{graphviz} (required to create the diagrams of the automatic code documentation)}
\item{\emph{texlive-science} (\LaTeX{} package for compiling the documentation)}
\end{itemize}

To install all required packages at once, just use:
\begin{lstlisting}[language=sh]
apt-get install doxygen graphviz texlive texlive-science
\end{lstlisting}


\paragraph{Compile Using System Libraries}
When using the \emph{maketool} flag \verb+--syslibs+, you also need the following packages:
\begin{itemize}
\item{\emph{libpcre3-dev} (required by PLCore)}
\item{\emph{libjpeg62-dev} (required by PLGraphics)}
\item{\emph{libpng12-dev} (required by PLGraphics)}
\item{\emph{libfreetype6-dev} (required by PLRendererOpenGL and PLRendererOpenGLES)}
\item{\emph{nvidia-cg-toolkit} (required by PLRendererOpenGLCg - >= Cg Toolkit 3.1 - February 2012 required) - \emph{nvidia-cg-toolkit} may be out-of-date, install a newer Cg \ac{SDK}: \url{http://developer.download.nvidia.com/cg/Cg_3.1/Cg-3.1_February2012_x86.deb}}
\item{\emph{libopenal-dev} (required by PLSoundOpenAL)}
\item{\emph{libogg-dev} (PLSoundOpenAL)}
\item{\emph{libvorbis-dev} (PLSoundOpenAL)}
\end{itemize}

To install all required packages at once, just use:
\begin{lstlisting}[language=sh]
apt-get install libpcre3-dev libjpeg62-dev libpng12-dev libfreetype6-dev libopenal-dev libogg-dev libvorbis-dev
\end{lstlisting}





\section{External Packages}
\label{Chapter:Linux_ExternalPackages}
Just like the Windows build, it is necessary to obtain all the external packages used by the engine and install them in the right place for the build. Have a look at the \emph{External} directory, there you find a readme file for every library that describes what files are needed.

You can find the libraries pre-packed in our files-section on our homepage: \url{http://pixellight.sourceforge.net/externals/}. Unfortunately, we can't provide some of those libraries due to their licensing terms. Have a look at the according \emph{Readme.txt} to determine where to obtain those libraries and where to put the resulting files into your source tree.

The library packages need to be unpacked and put at the right position for your specific build type, e.g. on Linux and \SI{32}{\bit}, put everything in the directory \emph{External/\_Linux\_x86\_32}.

The CMake based build will try to download the needed packages automatically from our homepage when you build a project that depends on an external. It will download and unpack all public externals for you in the right directory. The non-public externals must still be installed manually. You should also use the maketool flag \verb+--syslibs+ to use the libraries installed on your system rather than our own external packages whenever that is possible (however, not all externals can be used that way).

For more information about the external dependencies, please have a look at chapter~\ref{Chapter:ExternalDependencies}.




\section{CMake}
\label{Chapter:Linux_CMake}
Here's how to compile PixelLight by using the CMake-\ac{GUI}:
\begin{itemize}
\item{Start "CMake (cmake-gui)"}
\item{"Where is the source code"-field: e.g. "\textasciitilde /PixelLight"}
\item{"Where to build the binaries"-field: e.g. "\textasciitilde /PixelLight/CMakeOutput"}
\item{Press the "Configure"-button}
\item{Choose the generator "Unix Makefiles"}
\item{Press the "Generate"-button}
\end{itemize}

The CMake part is done, you can close "CMake (cmake-gui)" now. All required external packages are downloaded automatically, see chapter~\ref{Chapter:ExternalDependencies}.
\begin{itemize}
\item{Open a terminal and change into e.g. "\textasciitilde /PixelLight/CMakeOutput"}
\item{Type "make" (example: "make -j 4 -k" will use four \ac{CPU} cores and will keep on going when there are errors)}
\end{itemize}




\section{Maketool Script}
\label{Chapter:Linux_Maketool}
In case you don't want to use CMake directly, you can also use the shell script \emph{maketool}, The shell script \emph{maketool} performs the required build steps automatically.

After cloning the Git repository, you first have to call
\begin{lstlisting}[language=sh]
chmod +x maketool
\end{lstlisting}
in order to make the script executable.

To generate the project files just call
\begin{lstlisting}[language=sh]
./maketool setup [--release] [--syslibs]
\end{lstlisting}

On Linux, it is generally recommended to use the flag \verb+--syslibs+, this will cause the build system to use and depend on the libraries found on your system, rather than using our own externals. Although this may cause troubles when distributing an application, you have to ensure that the end user has the required dependencies installed.

After the project files were generated, the project can be compiled by writing
\begin{lstlisting}[language=sh]
./maketool build [--release]
\end{lstlisting}

Here's a list of the most important \emph{maketool}-commands:
\begin{itemize}
\item{"./maketool setup"	- Create project files}
\item{"./maketool build"	- Compile the project}
\item{"./maketool docs"		- Compile the documentation}
\item{"./maketool pack"		- Generate an installable package}
\item{"./maketool clean"	- Delete the build directories}
\end{itemize}

Here's a list of the most important \emph{maketool}-options:
\begin{itemize}
\item{\verb+--debug+								- Create a debug version (default)}
\item{\verb+--release+								- Create a release version}
\item{\verb+--suffix+ \textless suffix\textgreater	- Add a suffix to all library names}
\item{\verb+--externals+							- Repository \ac{URL} were to download the packages with the external dependencies from (e.g. \url{http://pixellight.sourceforge.net/externals/})}
\item{\verb+--username+								- User name for access to restricted packages within the repository}
\item{\verb+--password+								- User password for access to restricted packages within the repository}
\item{\verb+--arch+									- Architecture (e.g. \emph{x86}, \emph{arm})}
\item{\verb+--bitsize+								- Bit size (e.g. \emph{32} or \emph{64})}
\item{\verb+--syslibs+								- Use system libraries}
\item{\verb+--minimal+								- Do only compile the most important projects}
\end{itemize}

To avoid setting the parameters \verb+--externals+, \verb+--username+ and \verb+--password+ over and over again, create a file \emph{pl\_config.cfg} in your home directory (Perl-Script, included and just executed) with the following content:
\begin{lstlisting}[language=sh]
$pl_external_url = "";
$pl_external_user = "";
$pl_external_pass = "";
\end{lstlisting}




\section{Build}
\label{Chapter:Linux_Build}
To build PixelLight, either use maketool to do everything automatically:
\begin{lstlisting}[language=sh]
./maketool build [--release]
\end{lstlisting}

Or change into the build-directory for your configuration (e.g. \emph{build-Debug} or \emph{build-Release}) and run make yourself:
\begin{lstlisting}[language=sh]
cd build-Release
make
\end{lstlisting}

The latter option has the advantage that you can also build individual projects rather than the whole \ac{SDK}. To build projects individually, change into the \emph{build-Debug} or \emph{build-Release} directory and type
\begin{lstlisting}[language=sh]
make <project name>
\end{lstlisting}
(for example: \emph{make PLCore})

In order to use make options, change into the \emph{build-Debug} or \emph{build-Release} directory and type for example
\begin{lstlisting}[language=sh]
make
\end{lstlisting}
In order to make all projects using four \ac{CPU} cores to significantly speed up the make process, add the option "-j 4". You may also add the option \emph{-k} to tell the compiler "Keep on going" if there are errors within the build - useful if you're going for a walk or having some fresh coffee while the build is running. Example:
\begin{lstlisting}[language=sh]
make -j 4 -k
\end{lstlisting}
\clearpage



\section{Create Documentation and Packages}
To create the documentation, build project \emph{Docs}:
\begin{lstlisting}[language=sh]
cd build-Release
make Docs
\end{lstlisting}

Or use maketool to do it for you:
\begin{lstlisting}[language=sh]
./maketool docs
\end{lstlisting}

To create the PixelLight \ac{SDK} and create a Debian installation file, build project \emph{Pack}:
\begin{lstlisting}[language=sh]
cd build-Release
make Pack
\end{lstlisting}

Or use maketool to do it for you:
\begin{lstlisting}[language=sh]
./maketool pack
\end{lstlisting}
\clearpage



\section{Running from a Local Build and Installing}
\label{Chapter:Linux_RunningFromALocalBuildAndInstalling}
Please note that the following information should only be required during development using PixelLight. Release versions of your application should always contain everything so it can be run out-of-the-box. It should not be required to manipulate an \emph{environment variable} on the system of end-users.

Once you have built PixelLight, you may want to run e.g. the sample applications. In order for this to work correctly, PixelLight must know where to search for data files and plugins, which is not always an easy task on Linux systems. You have several options here.

\subsection{1. Changes in Pixellight 1.0.0}
With PixelLight 1.0.0 the name of the executables of the Samples and Tools (which links dynamically to PLCore, currently all Samples, PLViewer, and PLViewerQt) has changed to support an out of the box experience without fiddeling with the system environment (e.g. setting env vars or add additional paths to the ld.so.conf)

The name of the executable has an suffix named \emph{"-bin"} e.g. \emph{PLViewer-bin}. Additional an helper shellscript is generated with the name of the project e.g. \emph{PLViewer}. The shellscript setups the environment so that the app can find all needed libraries and the PixelLight Runtime. The shellscript will not be copied to the Bin-Linux directory. Because the shellscript contains an path which points to
$<CMAKE\_INSTALL\_PREFIX>$/share/pixellight/Runtime/$<ARCH>$ e.g. /usr/local/share/pixellight/Runtime/x86
$\Rightarrow$ the shellscript is only useable when the build is installed via an \emph{make install}

\paragraph{Note:}Due this change the name of some Sample log and configuration files has changed. This affects currently all Samples which number prefix is lower then 30. e.g. 01Application (old $\to$ new):

01Application.log $\to$ 01Application-bin.log

01Application.cfg $\to$ 01Application-bin.cfg
\\
\\In general all applications are affected using PixelLight which doesn't set explicitly the name for the configuration or log file (e.g via using the macros pl\_module\_application or pl\_module\_application\_frontend, FrontendContext::SetName, ApplicationContext::SetLogFilename or ApplicationContext::SetConfigFilename)

\subsection{2. Run from a your Local Source Directory}
This means that everything stays in your source directory. All libraries and applications have already been copied into the directory \emph{Bin-Linux} by means of post-build commands.

Now you only have to tell PixelLight where it can find the runtime and data files. This can be done by setting the environment variable \emph{PL\_RUNTIME} to the \emph{Runtime} directory.

If you're in the root of the source tree, you can use the script \emph{profile} to do this for you:
\begin{lstlisting}[language=sh]
source ./profile
\end{lstlisting}
or
\begin{lstlisting}[language=sh]
. ./profile
\end{lstlisting}
(sets the environment variable \emph{PL\_RUNTIME})

To inspect the content of \emph{PL\_RUNTIME}, call
\begin{lstlisting}[language=sh]
echo $PL_RUNTIME
\end{lstlisting}

To delete the environment variable \emph{PL\_RUNTIME}, call
\begin{lstlisting}[language=sh]
unset PL_RUNTIME
\end{lstlisting}

Of course you can also set this variable e.g. inside your profile or bash-scripts so that they are always available. Within your home (\emph{\textasciitilde}) directory, open the hidden \emph{\textasciitilde /.bashrc} file and add:
\begin{lstlisting}[language=sh]
export PL_RUNTIME="<PixelLight root path>/Bin-Linux/Runtime/x86"
\end{lstlisting}
This is the most comfortable way, because you only have to do this once, and then you can use the PixelLight runtime even across multiple terminal instances.


\paragraph{LD\_LIBRARY\_PATH}
You may also want to have a look into the \emph{LD\_LIBRARY\_PATH} environment variable of Linux. It's comparable to the \emph{PATH} environment variable under Windows and enables the \ac{OS} to find, for example, the \emph{PLCore} shared library if one is requesting it by just using the library name.




\subsection{3. Install the PixelLight \ac{SDK} Locally}
Installing means copying the files built by the project into your local Linux system so that the libraries and applications can be found there. The CMake build script therefore provides you with an \emph{install} target that installs everything on your local machine into the \emph{/usr/local} directory (default) or any path to which the cmake var \emph{CMAKE\_INSTALL\_PREFIX} was set to.

Change into the build directory, e.g.:
\begin{lstlisting}[language=sh]
cd build-Release
\end{lstlisting}

Install: (when \emph{CMAKE\_INSTALL\_PREFIX} was set to an path to which an normal user has no right to write into then this has to be done as root. Use e.g. \emph{sudo} or \emph{su} for this task.)
\begin{lstlisting}[language=sh]
make install
\end{lstlisting}

Update libraries: (This has to be done as root. Use e.g. \emph{sudo} or \emph{su} for this task. With PixelLight 1.0.0 not needed)
\begin{lstlisting}[language=sh]
ldconfig /usr/local/lib
\end{lstlisting}
(if this is not done, new dynamic libs may not be found correctly)

If you have installed Pixellight, you will find the runtime in $<CMAKE\_INSTALL\_PREFIX>$/share/pixellight/Runtime,

and the samples in $<CMAKE\_INSTALL\_PREFIX>$/share/pixellight/Samples

e.g \emph{/usr/local/share/pixellight/Runtime} and \emph{/usr/local/share/pixellight/Samples}.

Now you should be able to run the applications built by the PixelLight project, e.g. run one of the samples:
\begin{lstlisting}[language=sh]
cd /usr/local/share/pixellight/Samples/Bin/x86
./50RendererTriangle
\end{lstlisting}

What also works (due the helper shellscript since PxielLight 1.0.0 ):
\begin{lstlisting}[language=sh]
/usr/local/share/pixellight/Samples/Bin/x86/50RendererTriangle
\end{lstlisting}
