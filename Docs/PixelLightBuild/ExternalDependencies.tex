\chapter{External Dependencies}
\label{Chapter:ExternalDependencies}
In order to build the PixelLight engine for instance with \ac{MSVC}, all the external packages used by the engine need to be in the right place for the build. When using the project files generated by CMake and there's an Internet connection, those external packages are downloaded and extracted automatically during the build. So, this chapter is only in case this automatic process is not working for you our you don't use the CMake way at all. Have a look at the \emph{External}-directory inside the Git repository. There you can find a readme file for every library, that describes what files are needed.




\section{Dependencies}
When writing this documentation in November 2011, there are 37 external dependencies. Please note that this doesn't mean that you have to take care of 37 external libraries before you can start building and using PixelLight. In the minimal build, PixelLight only depends on PCRE used within PLCore for regular expressions. zlib for zip support within PLCore is highly recommended, but can be deactivated. The same is for jpg and png support within PLGraphics. Nice to have but no must have, especially when you just want to do your first steps with building PixelLight. As you can see, the 12 base projects of PixelLight don't have a lot of external dependencies.

Everything else is completely optional, most of them are just dynamically loaded plugins. If you need no Cg shaders, don't use Cg. If you need no Newton physics, don't use it. For scripting, you only need to compile PLCore and e.g. the Lua scripting plugin, that's it. For developing new plugins, you even don't need to compile PixelLight itself, just start a new plugin project and PixelLight will be able to use the new dynamic library providing e.g. \ac{RTTI} classes for new script languages.

This means that the base of PixelLight is really slim when it comes to external dependencies. Those 37 external libraries only mean that there are a lot of optional plugins available for you so that you don't have to write them by yourself if you need them. It's a gift from us, to you.




\section{External Packages}
You can find the libraries pre-packed in the files-section on our homepage: \url{http://pixellight.sourceforge.net/externals/}. When opening this \ac{URL} within a web browser, you'll see a file structure like
\begin{itemize}
\item{Linux-ndk\_armeabi-v7a\_32/}
\item{Linux-ndk\_armeabi\_32/}
\item{Linux\_x86\_32/}
\item{Linux\_x86\_64/}
\item{MacOSX\_x86\_32/}
\item{MacOSX\_x86\_64/}
\item{Windows\_x86\_32/}
\item{Windows\_x86\_64/}
\end{itemize}
There's one directory containing pre-packed external dependencies per platform variation. By default, the CMake based build process will download the packages from this location automatically. Within the CMake-GUI it's possible to change this external packages \ac{URL}. You can also download the packages manually and extract them individually at the right place. Either way, it's always the same principle and data. Don't get confused.


\paragraph{Package examples}
Here's are a few package examples so you're able to get the idea:
\begin{itemize}
\item{zlib package for \ac{MS} Windows \SI{32}{\bit}: \url{http://pixellight.sourceforge.net/externals/Windows_x86_32/zlib.tar.gz}}
\item{zlib package for Linux \SI{32}{\bit}: \url{http://pixellight.sourceforge.net/externals/Linux_x86_32/zlib.tar.gz}}
\end{itemize}

There are also downloads available containing all public external packages. Examples:
\begin{itemize}
\item{All public packages for \ac{MS} Windows \SI{32}{\bit}: \url{http://sourceforge.net/projects/pixellight/files/PixelLight%20v0.9/0.9.10-R1/PixelLight-0.9.10-R1-Externals-Windows_x86_32.tar.gz/download}}
\item{All public packages for Linux \SI{32}{\bit}: \url{http://sourceforge.net/projects/pixellight/files/PixelLight%20v0.9/0.9.10-R1/PixelLight-0.9.10-R1-Externals-Linux_x86_32.tar.gz/download}}
\end{itemize}


\paragraph{None Public Packages}
Unfortunately, we can't provide some of those third party libraries due to their licensing terms. By convention, projects using such proprietary libraries are only optional, not mandatory in order to be able use PixelLight in the first place. Have a look at the according \emph{Readme.txt} of each external dependency to determine where to obtain those libraries and where to put the resulting files in your source tree. When using the CMake based build process, projects depending on those private packages are excluded from the build by default.

In case you're inside the PixelLight development team, just enter your username and password within the CMake-GUI and the build system will download those packages as well. Please note that this service is for the development team, only. In order to avoid legal issues, we just can't give access to other persons. We're sorry.

The library packages must be unpacked and need to be at the right position for your specific build type, e.g. on Windows and \SI{32}{\bit}, put everything into the directory \emph{\textless PixelLight root directory\textgreater /External/\_Windows\_x86\_32}. Examples:
\begin{itemize}
\item{\emph{C:/PixelLight/External/\_Windows\_x86\_32/zlib/include}}
\item{\emph{C:/PixelLight/External/\_Windows\_x86\_32/zlib/lib}}
\item{\emph{C:/PixelLight/External/\_Windows\_x86\_32/libpcre/include}}
\item{\emph{C:/PixelLight/External/\_Windows\_x86\_32/libpcre/lib}}
\end{itemize}


\paragraph{Conclusion}
The easiest way to install at least the public packages is to use the CMake based build and make the project \emph{External}. This will download and unpack all public externals for you in the right directory. The non-public externals must still be installed manually.

These packages are for the latest PixelLight version within the Git repository. So, if you're using an official PixelLight release, you might want to use the packed files from \url{http://sourceforge.net/projects/pixellight/files/} for the used release version in order to avoid compatibility issues.