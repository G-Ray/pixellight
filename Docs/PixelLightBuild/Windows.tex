\chapter{Windows}
If you want to build PixelLight on Windows, you have two choices:
\begin{itemize}
\item{Use the provided Microsoft Visual Studio (MSVC) solutions to build the engine}
\item{Use the CMake based build system}
\end{itemize}

If you want to create SDK packages or documentation or intend to e.g. use the mingw compiler or something like that, stick to the CMake path.

If you want to build only a local version of PixelLight but do not intend to create documentations or SDK packes, you can just use the MSVC solutions files which are much easier to use. Note however, that you need to unpack the needed external packages first (see chapter~\ref{Chapter:ExternalDependencies}).





\section{MSVC Solutions}



\subsection{Build}
Open the solution \emph{PixelLight.sln} with Microsoft Visual Studio, which includes all projects of the PixelLight framework. Then start \emph{Build}. Done.




\section{CMake}



\subsection{Prerequisites}
Here's a list of required programs that you need to fully build the SDK on Windows.


\paragraph{CMake}
\begin{itemize}
\item{Cross platform build tool used to build the SDK}
\item{Download CMake at \url{http://www.cmake.org}}
\item{Tested with \emph{cmake-2.8.5-win32-x86.exe}}
\end{itemize}


\paragraph{NSIS}
\begin{itemize}
\item{Used to create the Windows installer}
\item{Download NSIS (Nullsoft Scriptable Install System) at \url{http://nsis.sourceforge.net/}}
\item{Tested with \emph{nsis-2.46-setup.exe}}
\end{itemize}


\paragraph{Doxygen}
\begin{itemize}
\item{Used to create the code documentations}
\item{Download Doxygen at \url{http://www.doxygen.org}}
\item{Tested with \emph{doxygen-1.7.5-setup.exe}}
\end{itemize}


\paragraph{Graphviz}
\begin{itemize}
\item{Used from Doxygen to create diagrams}
\item{Download Graphviz at \emph{http://www.graphviz.org}}
\item{Tested with \emph{graphviz-2.28.0.msi}}
\item{Ensure that the Graphviz binaries directory is correctly set in the \emph{PATH} and/or \emph{DOT\_PATH} environment variables, otherwise Doxygen can't find Graphviz and as a result there will be no graphs within the generated document}
\end{itemize}


\paragraph{MiKTeX}
\begin{itemize}
\item{Used to create the \LaTeX{} documentations like the one you're currently reading}
\item{Download MiKTeX at \url{http://miktex.org/}}
\item{Tested with \emph{setup-2.9.4222.exe}}
\end{itemize}


\paragraph{Microsoft HTML Help Compiler}
\begin{itemize}
\item{Used to create chm documentations from the html help files created by Doxygen}
\item{The CMake system automatically searches for an installed Microsoft HTML Help Compiler (\emph{hhc.exe})}
\item{Download MS HTML Help Workshop at \url{http://msdn.microsoft.com/en-us/library/ms669985} if it's not yet on your system}
\end{itemize}


\paragraph{Swiss File Knife}
\begin{itemize}
\item{\emph{Swiss File Knife}: file management, search, text processing}
\item{Directly used by the CMake scripts under MS Windows}
\item{\emph{cmake/UsedTools/sfk/sfk.exe} is already within the Git repository and was downloaded from \url{http://sourceforge.net/projects/swissfileknife/}}
\end{itemize}


\paragraph{Diff tools}
\begin{itemize}
\item{Some diff binaries directly used by the CMake scripts under MS Windows}
\item{The \emph{cmake/UsedTools/diff} directory is already within the Git repository}
\end{itemize}




\subsection{MSVC: Create Solutions and Build}
Here's how to compile PixelLight by using the CMake-GUI:
\begin{itemize}
\item{Start "CMake (cmake-gui)"}
\item{"Where is the source code"-field: e.g. "C:/PixelLight"}
\item{"Where to build the binaries"-field: e.g. "C:/PixelLight/CMakeOutput"}
\item{Press the "Configure"-button}
\item{Choose the generator, for instance "Visual Studio 10"}
\item{Press the "Generate"-button}
\item{(The CMake part is done, you can close "CMake (cmake-gui)" now)}
\item{(Ensure you have all required external packages, see chapter~\ref{Chapter:ExternalDependencies})}
\item{Open "CMakeOutput/PixelLight.sln" with Microsoft Visual Studio}
\item{To create a PixelLight SDK, choose "Pack-SDK", please note that you are free to compile projects individually as well}
\item{To build the documentation, choose "Docs"}
\end{itemize}
