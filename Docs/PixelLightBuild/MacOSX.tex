\chapter{Mac OS X}
This chapter explains how to compile PixelLight for \emph{Mac OS X 10.6}.


\paragraph{Under Construction}
The Mac OS X port is currently under construction.




\section{\ac{GCC} 4.6}
Within a terminal, type
\begin{lstlisting}[language=sh]
gcc -v
\end{lstlisting}
in order to see the currently used \ac{GCC} version. In case you see e.g.
\begin{lstlisting}[language=sh]
gcc version 4.2.1 (Based on Apple Inc. build 5658) (LLVM build 2335.15.00)
\end{lstlisting}
you're using an out-of-date \ac{GCC} version not capably of any C++11 features. You will be able to compile \emph{PLCore}, but the singletons are using \emph{extern templates} in order to avoid template instantiation in other modules. \emph{Extern templates} is a feature which has been supported by \ac{GCC} even before C++11, if it's missing you will get linker error messages like the following:
\begin{lstlisting}[language=sh]
Linking CXX shared library libPLMath.dylib
Undefined symbols for architecture i386:
  "PLCore::Singleton<PLCore::ClassManager>::GetInstance()", referenced from:
      _PLGetPluginInfo in PLMath.cpp.o
      ModuleInfo::Register()      in PLMath.cpp.o
      PLMath::GraphPathLoaderPL::_Class::_Class()in GraphPathLoaderPL.cpp.o
      PLMath::GraphLoaderPL::_Class::_Class()in GraphLoaderPL.cpp.o
      PLMath::GraphPathLoader::_Class::_Class()in GraphPathLoader.cpp.o
      PLMath::GraphLoader::_Class::_Class()in GraphLoader.cpp.o
\end{lstlisting}

One way to get a new \ac{GCC} version like \ac{GCC} 4.6 is to just compile it yourself. Please note that explaining how to compile \ac{GCC} for Mac OS X is out of the scope of this document. There are many good instructions like \url{http://solarianprogrammer.com/2011/05/28/compiling-gcc-4-6-0-on-mac-osx/} explaining in detail how to compile \ac{GCC} for Mac OS X. The process isn't that complicated. When compiling a new \ac{GCC}, the \ac{GCC} of the system will not be touched. In order to use the build \ac{GCC} 4.6 type
\begin{lstlisting}[language=sh]
export CXX=$HOME/my_gcc/bin/g++-4.6.1
export CC=$HOME/my_gcc/bin/gcc-4.6.1
export CPP=$HOME/my_gcc/bin/cpp-4.6.1
\end{lstlisting}
in case you put your build \ac{GCC} into the \emph{\$HOME/my\_gcc/} directory.

In case you don't want to do this over and over again, put this overwritten environment variables into your profile file. To put this into the profile file, type
\begin{lstlisting}[language=sh]
pico $HOME/.profile
\end{lstlisting}
and copy'n'past the export lines above.

When typing
\begin{lstlisting}[language=sh]
gcc -v
\end{lstlisting}
you will still get \emph{4.2.1} as \ac{GCC} because you just set the \ac{GCC} environment variables CMake is using in order to figure out which compiler to use. Type
\begin{lstlisting}[language=sh]
$CXX -v
\end{lstlisting}
and the output will be
\begin{lstlisting}[language=sh]
gcc-Version 4.6.1 (GCC)
\end{lstlisting}
, you're now ready to continue.



\section{Build and Run}
On the terminal level, Mac OS X is nearly identical to Linux. So, there's no point in copy'n'past the Linux part. Instead, this section will refer to the relevant Linux sections.


\paragraph{Maketool}
The maketool script as described in \ref{Chapter:Linux_Maketool} can be used in the same way under Mac OS X.


\paragraph{External Packages}
The management of external packages is the same on all platforms. Refer to to the Linux section~\ref{Chapter:Linux_ExternalPackages} or directly to chapter~\ref{Chapter:ExternalDependencies}. On Mac OS X and \SI{32}{\bit}, put everything in the directory \emph{External/\_MacOSX\_x86\_32}.


\paragraph{CMake}
CMake as described in \ref{Chapter:Linux_CMake} can be used in the same way under Mac OS X.


\paragraph{Build and Run}
You may also want to have a look at the Linux sections \ref{Chapter:Linux_Build} and \ref{Chapter:Linux_RunningFromALocalBuildAndInstalling}.
