\chapter{Mac OS X}
This chapter explains how to compile PixelLight for \emph{Mac OS X 10.6}.


\paragraph{Under Construction}
The Mac OS X port is currently under construction.


\paragraph{Compiler Version}
When porting PixelLight to Mac OS X in 2011, there were several issues with the \ac{GCC} 4.2.1 used on the available system. During the port process, Xcode 4.2 was released introducing C++11 support. At the time of writing this document this new version was not yet successfully installed on the used \emph{Mac OS X 10.6} system. PixelLight also compiles without C++11 features, but you might want to use a more up-to-date compiler like \ac{GCC} 4.6. Have a look at appendix~\ref{Appendix:MacOSX_GCC} for more details about this topic.


\paragraph{Linux}
On the terminal level, Mac OS X is nearly identical to Linux. So, there's no point in copy'n'past the Linux part. Instead, this section will refer to the relevant Linux sections.


\paragraph{External Packages}
The management of external packages is the same on all platforms. Refer to to the Linux section~\ref{Chapter:Linux_ExternalPackages} or directly to chapter~\ref{Chapter:ExternalDependencies}. On Mac OS X and \SI{32}{\bit}, put everything in the directory \emph{External/\_MacOSX\_x86\_32}.


\paragraph{CMake}
CMake as described in \ref{Chapter:Linux_CMake} can be used in the same way under Mac OS X.


\paragraph{Maketool}
The maketool script as described in \ref{Chapter:Linux_Maketool} can be used in the same way under Mac OS X.


\paragraph{Build and Run}
You may also want to have a look at the Linux sections \ref{Chapter:Linux_Build} and \ref{Chapter:Linux_RunningFromALocalBuildAndInstalling}.
