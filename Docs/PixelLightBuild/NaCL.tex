\chapter{NaCL}
In principle, there's no difference in building PixelLight for \ac{NaCL} on a \ac{MS} Windows or Linux host. So, the following description is e.g. not \ac{MS} Windows only.

In case you never ever did \ac{NaCL} development before, you may have a look at \url{https://developers.google.com/native-client/devguide/tutorial}. Under Linux, in case you're using environment variables as described within the appendix, don't forget that you have to start CMake-GUI out of a terminal. If you don't, the CMake-GUI doesn't know anything about your environment variables.


\paragraph{Static Library}
For \ac{NaCL}, everything has to be build as a static library. The final application is within a single \emph{nexe}-module for each hardware plattform.


\paragraph{Under Construction}
The \ac{NaCL} port is currently under construction.




\section{Prerequisites}
\begin{itemize}
\item{Installed \ac{NaCL} \ac{SDK}}
\end{itemize}


\paragraph{Path to the \ac{NaCL} \ac{SDK} - Windows}
Add a new environment variable \emph{NACL\_SDK} and set it to \emph{<NaCL directory>/pepper\_18} (example: \emph{C:/nacl\_sdk/pepper\_18}).


\paragraph{Path to the \ac{NaCL} \ac{SDK} - Linux}
This example assumes that the data has been extracted directly within the home (\emph{\textasciitilde}) directory. Open hidden "\textasciitilde /.bashrc"-file and add:
\begin{lstlisting}[language=sh]
# Important NaCL SDK path
export NACL_SDK=~/nacl_sdk/pepper_18
\end{lstlisting}
\begin{itemize}
\item{Open a new terminal so the changes from the step above have an effect}
\end{itemize}




\paragraph{make (for Windows)}
\begin{itemize}
\item{\emph{Make for Windows}: Make: GNU make utility to maintain groups of programs}
\item{Directly used by the CMake scripts under \ac{MS} Windows when using the \ac{NaCL} toolchain}
\item{\emph{cmake/UsedTools/make/make.exe} was downloaded from \url{http://gnuwin32.sourceforge.net/packages/make.htm}}
\end{itemize}
This tool can't be set within a CMake file automatically, there are several options:
\begin{itemize}
\item{Add \emph{\textless PixelLight root path\textgreater /cmake/UsedTools/make} to the \ac{MS} Windows \emph{PATH} environment variable *recommended*}
\item{Use a MinGW installer from e.g. \url{http://www.tdragon.net/recentgcc/} which can set the \emph{PATH} environment variable *overkill because only the 171 KiB \emph{make} is required*}
\item{Use CMake from inside a command prompt by typing for example (\emph{DCMAKE\_TOOLCHAIN\_FILE} is only required when using a toolchain) \\ *not really comfortable when working with it on a regular basis*
\begin{lstlisting}[language=sh]
cmake.exe -G"Unix Makefiles" -DCMAKE\_MAKE\_PROGRAM="<PixelLight root path>/cmake/UsedTools/make/make.exe" -DCMAKE\_TOOLCHAIN\_FILE="<PixelLight root path>/cmake/Toolchains/Toolchain-nacl.cmake"
\end{lstlisting}
}
\end{itemize}




\section{Create Makefiles and Build}
\label{NaCL:CreateMakefilesAndBuild}
Here's how to compile PixelLight by using the CMake-\ac{GUI}:
\begin{itemize}
\item{Ensure "make" (GNU make utility to maintain groups of programs) can be found by CMake (add for instance "\textless PixelLight root path\textgreater /cmake/UsedTools/make" to the \ac{MS} Windows \emph{PATH} environment variable)}
\item{Start "CMake (cmake-gui)"}
\item{"Where is the source code"-field: e.g. "C:/PixelLight"}
\item{"Where to build the binaries"-field: e.g. "C:/PixelLight/CMakeOutput"}
\item{Press the "Configure"-button}
\item{Choose the generator "Unix Makefiles" and select the radio box "Specify toolchain file for cross-compiling"}
\item{Press the "Next"-button}
\item{"Specify the Toolchain file": e.g. "C:/PixelLight/cmake/Toolchains/Toolchain-nacl.cmake"}
\item{Press the "Generate"-button}
\end{itemize}

The CMake part is done, you can close "CMake (cmake-gui)" now. All required external packages are downloaded automatically, see chapter~\ref{Chapter:ExternalDependencies}.
\begin{itemize}
\item{Open a command prompt and change into e.g. "C:/PixelLight/CMakeOutput" (\ac{MS} Windows: by typing "cd /D C:/PixelLight/CMakeOutput" -> "/D" is only required when changing into another partition)}
\item{Type "make" (example: "make -j 4 -k" will use four \ac{CPU} cores and will keep on going when there are errors)}
\item{(You now should have the ready to be used \ac{NaCL} static library files)}
\end{itemize}
