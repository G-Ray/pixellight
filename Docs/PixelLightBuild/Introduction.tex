\chapter{Introduction}
\dictum[The Hitchhiker's Guide to the Galaxy by Douglas Adams]{DON'T PANIC}


\paragraph{Target Audience}
This document is meant for programmers. Please note that also trivial stuff will be mentioned. A lot.


\paragraph{Motivation}
This document describes how to build PixelLight from the sources. Don't be shocked when looking at the size of this document, this doesn't imply that it's highly complicated or near impossible to build the project. Right from the beginning, one of our goals was, that it should be as easy as possible to build the technology. A lot of efforts were and are put into this goal as one may see when looking at this document or the fact that such a document exists.

Sadly, across all the supported platforms and external dependencies there may be some pitfalls - especially for users without much experience within a certain target platform or tool set. The goal of this document is to provide as much information as possible to minimize the frustration of building PixelLight. Please note that this document can't avoid frustration completely, especially if you plan to compile PixelLight with a not officially supported compiler or for a new, untested platform.

Due to the really small development team compared to the dimension of the project, we can't support every compiler or \ac{IDE} existing out there (meaning adding support and especially maintaining it). We have to focus on the mainstream, or what we consider as mainstream. Currently we're using the following compilers and compiler versions:
\begin{itemize}
\item{\ac{MS} Windows: Microsoft Visual Studio 10}
\item{Linux: \ac{GCC} 4.6}
\item{Linux: Clang 3.0}
\item{Mac OS X: \ac{GCC} 4.2.1}
\item{Android \ac{NDK} Toolchain}
\item{Google \ac{NaCL} Toolchain}
\end{itemize}
If you stick to those, you should be on the safe side. Other compilers and compiler versions may work as well, but are untested. There's support for \SI{32}{\bit} and \SI{64}{\bit}.

In general, if you encounter time consuming pitfalls not yet described in this document, please tell us by using e.g. the official forum at \url{http://dev.pixellight.org/forum/} or the bugtracker at \url{http://sourceforge.net/tracker/?group_id=507544&atid=2063682}. If there's no feedback, we can't improve things to make it even easier to use PixelLight in the future.


\paragraph{Linux}
Especially the Linux part is quite detailed due to the fact that it's the base of several other ports like Android and Mac OS X, even when a platform is no direct derivation of Linux. Additionally, this document should also enable \ac{MS} Windows user, without or with just a little experience with Linux, to build PixelLight under Linux.


\paragraph{CMake}
To build PixelLight across all supported platforms in an uniform way, we're using CMake (\url{http://www.cmake.org/}).


\paragraph{Source Codes of Releases}
At \url{http://sourceforge.net/projects/pixellight/files/} are the packed sources codes of every PixelLight release available for download. For example, \emph{PixelLight-1.0.1-R1-SourceCodes.tar.gz} are the packed source codes of the release \emph{1.0.1-R1}. There are also packages containing all public external dependencies used to compile this release version.


\paragraph{Latest Source Codes}
We use Git\footnote{\url{http://www.git-scm.com}} version control to manage our source code repository. This repository contains the main source code of the PixelLight framework. In case you want to access the latest source codes for the next release, you can use this repository. Please note that the latest commit(s) may have broken something or introduced not yet fixed or even recognized new issues. Although we always try keep the version within the Git repository usable, it may even not be compilable for a short time after a breaking commit. So, if you want to be on the safe side, use the sources codes of an official PixelLight release as mentioned in the paragraph above. These versions are tagged within the repository. For anonymous access, it is readable only. To ensure high quality source code, write access is only available to team members of the development team.

To checkout the current source code, use the following Git command line:
\begin{lstlisting}[language=sh]
git clone git://pixellight.git.sourceforge.net/gitroot/pixellight/pixellight
\end{lstlisting}

In case you're a PixelLight team member, you might want to clone by using your SourceForge username by writing
\begin{lstlisting}[language=sh]
git clone ssh://<user name>@pixellight.git.sourceforge.net/gitroot/pixellight/pixellight
\end{lstlisting}
Example:
\begin{lstlisting}[language=sh]
git clone ssh://bob42@pixellight.git.sourceforge.net/gitroot/pixellight/pixellight
\end{lstlisting}
Using the \emph{ssh}-protocol might reduce the risk of firewall problems.

Or use this \ac{URL} to checkout the source code with your favourite Git client:
\begin{lstlisting}[language=sh]
git://pixellight.git.sourceforge.net/gitroot/pixellight/pixellight
\end{lstlisting}

In case you get for instance the following error message during cloning the Git repository
\begin{quotation}
fatal: Unable to look up pixellight.git.sourceforge.net (port 9418) (No such host is known. )
\end{quotation}
you might want to check your firewall settings. Additionally you can browse the Git repository within your web browser by opening  \url{pixellight.git.sourceforge.net/git/gitweb.cgi?p=pixellight/pixellight;a=summary} to ensure that the server is not down (doesn't happen that often).

You can use the following \ac{RSS} feed to get informed automatically about changes within the public Git repository:
\begin{lstlisting}[language=sh]
http://pixellight.git.sourceforge.net/git/gitweb.cgi?p=pixellight/pixellight;a=rss
\end{lstlisting}
